%\chapter{Requirement levels}
\chapter{Уровни требований}
\label{requirementchapter}

%The key words ``must'', ``must not'', ``should'',
%``should not'', ``recommended'', ``may'', and ``optional'' in this
%report are to be interpreted as described in RFC~2119~\cite{mustard}.
%Specifically:
Ключевые слова, ``must'', ``must not'' ``should'', ``should not'', ``recommended'',
``may'', и ``optional'' в данной работе должны интерпретироваться как описано в RFC~2119~\cite
{mustard}. А именно:

\begin{description}
%\item[must]\mainindex{must} This word means that a statement is an absolute
%  requirement of the specification.
\item[must]\mainindex{must} Это слово означает, что инструкция является абсолютным требованием
  спецификации.
%\item[must not]\mainindex{must not} This phrase means that a statement is an absolute
%  prohibition of the specification.
\item[must not]\mainindex{must not} Эта фраза означает, что инструкция является абсолютным запрещением
  спецификации.
%\item[should]\mainindex{should} This word, or the adjective ``recommended'', means that
%  valid reasons may exist in particular circumstances to ignore a
%  statement, but that the implications must be understood and weighed
%  before choosing a different course.
\item[should]\mainindex{should} Это слово, или прилагательное ``recommended'', означают, что
  в особых обстоятельствах могут существовать веские причины для игнорирования инструкции, но перед
  выбором иной линии поведения последствия должны быть осознаны и взвешены.
%\item[should not]\mainindex{should not} This phrase, or the phrase ``not recommended'', means
%  that valid reasons may exist in particular circumstances when the
%  behavior of a statement is acceptable, but that the implications
%  should be understood and weighed before choosing the course described
%  by the statement.
\item[should not]\mainindex{should not} Эта фраза, или фраза ``not recommended'', означают, что
  в особых обстоятельствах могут существовать веские причины для принятия функционирования
  инструкции, но перед выбором описываемой инструкцией линии поведения последствия
  должны быть осознаны и взвешены.
%\item[may]\mainindex{may} This word, or the adjective ``optional'', means that an item
%  is truly optional.
\item[may]\mainindex{may} Это слово, или прилагательное ``optional'' означают, что
  элемент является действительно дополнительным.
\end{description}

%In particular, this report occasionally uses ``should'' to designate
%circumstances that are outside the specification of this report, but
%cannot be practically detected by an implementation; see
%section~\ref{argumentcheckingsection}.  In such circumstances, a
%particular implementation may allow the programmer to ignore the
%recommendation of the report and even exhibit reasonable behavior.
%However, as the report does not specify the behavior,
%these programs may be unportable, that is, their execution might
%produce different results on different implementations.
В частности, ``should'' в данной работе иногда используется для обозначения обстоятельств,
лежащих вне спецификации данной работы, но не обнаруживаемых реализацией на практике;
см. секцию~\ref{argumentcheckingsection}. При таких обстоятельствах конкретная реализация может
позволить программисту игнорировать рекомендацию в данной работе и даже демонстрировать
адекватное функционирование. Однако, поскольку данная работа не специфицирует функционирование,
такие программы могут быть непортируемыми, то есть, их выполнение может привести к различным
результатам в разных реализациях.

\newpage

%Moreover, this report occasionally uses the phrase ``not required'' to note the
%absence of an absolute requirement.
Кроме того, в данной работе иногда используется фраза ``not required'' для обозначения
отсутствия абсолютного требования.

%%% Local Variables:
%%% mode: latex
%%% TeX-master: "r6rs"
%%% End:
