\documentclass[twoside,twocolumn]{algol60}
%\documentclass[twoside]{algol60}

% XeLaTeX:
\RequirePackage{fontspec}
\RequirePackage{xunicode}
\RequirePackage{xltxtra}
\RequirePackage{color}
\RequirePackage[titles]{tocloft}
\defaultfontfeatures{Mapping=tex-text}
%%%%%%%%%%%%%%%%%%%%%%%%%%%%%%%%%%%%%%

% Polyglossia:
\RequirePackage{polyglossia}
\setdefaultlanguage{russian}
\setotherlanguage{english}
\newfontfamily\cyrillicfont{Times New Roman}
\newfontfamily\cyrillicfonttt{Courier New}
\newfontfamily\cyrillicfontsf{Arial}
%%%%%%%%%%%%%%%%%%%%%%%%%%%%%%%%%%%%%%%%%%%

\renewcommand{\baselinestretch}{0.95}


\pagestyle{headings}
\showboxdepth=0
\makeindex
\input{commands}
\input{semantics-commands}

\texonly
\externaldocument[lib:]{r6rs-lib}
\endtexonly

%\def\headertitle{Revised$^{\rnrsrevision}$ Scheme}
\def\headertitle{Редакция$^{\rnrsrevision}$ Scheme}
%\def\TZPtitle{Revised^\rnrsrevision{} Report on the Algorithmic Language Scheme}
\def\TZPtitle{Редакция^\rnrsrevision{} Стандарта Алгоритмического Языка Scheme}

\begin{document}

\thispagestyle{empty}

\topnewpage[{
\begin{center}   {\huge\bf
        Редакция{\Huge$^{\mathbf{\htmlonly\tiny\endhtmlonly{}\rnrsrevision}}$} Стандарта Алгоритмического Языка \\
                              \vskip 3pt
                                Scheme}

\vskip 1ex
$$
\begin{tabular}{l@{\extracolsep{.5in}}lll}
\multicolumn{4}{c}{M\authorsc{ICHAEL} S\authorsc{PERBER}}
\\
\multicolumn{4}{c}{R.\ K\authorsc{ENT} D\authorsc{YBVIG},
  M\authorsc{ATTHEW} F\authorsc{LATT},
  A\authorsc{NTON} \authorsc{VAN} S\authorsc{TRAATEN}}
\\
%\multicolumn{4}{c}{(\textit{Editors})} \\
\multicolumn{4}{c}{(\textit{Редакторы})} \\
\multicolumn{4}{c}{
  R\authorsc{ICHARD} K\authorsc{ELSEY}, W\authorsc{ILLIAM} C\authorsc{LINGER},
  J\authorsc{ONATHAN} R\authorsc{EES}} \\
%\multicolumn{4}{c}{(\textit{Editors, Revised\itspace{}$^5$ Report on the
%    Algorithmic Language Scheme})} \\
\multicolumn{4}{c}{(\textit{Редакторы, Редакция\itspace{}$^5$ Стандарт
    Алгоритмического Языка Scheme})} \\
\multicolumn{4}{c}{
  R\authorsc{OBERT} B\authorsc{RUCE} F\authorsc{INDLER}, J\authorsc{ACOB} M\authorsc{ATTHEWS}} \\
%\multicolumn{4}{c}{(\textit{Authors, formal semantics})} \\[1ex]
\multicolumn{4}{c}{(\textit{Авторы, формальная семантика})} \\[1ex]
\multicolumn{4}{c}{\bf \rnrsrevisiondate}
\end{tabular}
$$



\end{center}

%\chapter*{Summary}
\chapter*{Предисловие}
\medskip

{\parskip 1ex
%The report gives a defining description of the programming language
%Scheme.  Scheme is a statically scoped and properly tail-recursive
%dialect of the Lisp programming language invented by Guy Lewis
%Steele~Jr.\ and Gerald Jay~Sussman.  It was designed to have an
%exceptionally clear and simple semantics and few different ways to
%form expressions.  A wide variety of programming paradigms, including
%functional, imperative, and message passing styles, find convenient
%expression in Scheme.
Стандарт предоставляет описание определения языка программирования Scheme. Scheme -- это чистый
хвост-рекурсивный диалект языка программирования Lisp со статическими областями видимости,
созданный Guy Lewis Steele~Jr.\ и Gerald Jay~Sussman. Он был разработан для использования
исключительно ясной и простой семантики и небольшого количества способов формирования
выражений. Большое разнообразие парадигм программирования, включая функциональный и императивный
стили, а также стиль обмена сообщениями, получили в Scheme удобное представление.

%This report is accompanied by a report describing standard
%libraries~\cite{R6RS-libraries}; references to this document are
%identified by designations such as ``library section'' or ``library
%chapter''.  It is also accompanied by a report containing
%non-normative appendices~\cite{R6RS-appendices}.  A fourth report gives
%some historical background and rationales for many aspects of the
%language and its libraries~\cite{R6RS-rationale}.
К данному стандарту прилагается стандарт, описывающий стандартные
библиотеки~\cite{R6RS-libraries}; ссылки на этот документ идентифицированы обозначениями,
такими, как ``библиотечная секция'' или ``библиотечная глава''. К нему также прилагается
стандарт, содержащий ненормативные приложения~\cite{R6RS-appendices}. Четвертый стандарт
предоставляет несколько исторических справок и логическое обоснование многих аспектов языка и
его библиотек~\cite{R6RS-rationale}.

\medskip

%The individuals listed above are not the sole authors of the text of
%the report.  Over the years, the following individuals were involved
%in discussions contributing to the design of the Scheme language, and
%were listed as authors of prior reports:
Перечисленные выше люди не являются единственными авторами текста стандарта. За все эти годы
следующие люди участвовали в обсуждениях, содействующих разработке языка Scheme, и перечислялись
в качестве авторов предыдущих стандартов:

%Hal Abelson, Norman Adams, David Bartley, Gary Brooks, William
%Clinger, R.\ Kent Dybvig, Daniel Friedman, Robert Halstead, Chris
%Hanson, Christopher Haynes, Eugene Kohlbecker, Don Oxley, Kent Pitman,
%Jonathan Rees, Guillermo Rozas, Guy L.\ Steele Jr., Gerald Jay Sussman, and
%Mitchell Wand.
Hal Abelson, Norman Adams, David Bartley, Gary Brooks, William
Clinger, R.\ Kent Dybvig, Daniel Friedman, Robert Halstead, Chris
Hanson, Christopher Haynes, Eugene Kohlbecker, Don Oxley, Kent Pitman,
Jonathan Rees, Guillermo Rozas, Guy L.\ Steele Jr., Gerald Jay Sussman и
Mitchell Wand.

%In order to highlight recent contributions, they are not listed as
%authors of this version of the report.  However, their contribution
%and service is gratefully acknowledged.
Чтобы подчеркнуть последние дополнения, они не перечислены в качестве авторов данной версии
стандарта. Однако их вклад и работа с благодарностью принимается.

\medskip

%We intend this report to belong to the entire Scheme community, and so
%we grant permission to copy it in whole or in part without fee.  In
%particular, we encourage implementors of Scheme to use this report as
%a starting point for manuals and other documentation, modifying it as
%necessary.
Мы подразумеваем, что данный стандарт принадлежит всему сообществу Scheme, и таким образом, мы
официально даём разрешение на его безвозмездное полное или частичное копирование. В частности,
мы настоятельно рекомендуем разработчикам реализаций Scheme использовать данный стандарт в
качестве отправной точки для руководств и другой документации, изменяя его по мере
необходимости.
}

\bigskip

\input{status}
}]

\texonly\clearpage\endtexonly

%\chapter*{Contents}
\chapter*{Содержание}
\addvspace{3.5pt}                  % don't shrink this gap
\renewcommand{\tocshrink}{-4.0pt}  % value determined experimentally
{
\tableofcontents
}

\vfill
\eject


%\clearextrapart{Introduction}
\clearextrapart{Введение}

\label{historysection}

%Programming languages should be designed not by piling feature on top of
%feature, but by removing the weaknesses and restrictions that make additional
%features appear necessary.  Scheme demonstrates that a very small number
%of rules for forming expressions, with no restrictions on how they are
%composed, suffice to form a practical and efficient programming language
%that is flexible enough to support most of the major programming
%paradigms in use today.
Языки программирования должны разрабатываться не путём накопления функциональных возможностей, а
путём удаления слабостей и ограничений, приводящих к мнимой необходимости дополнительных
средств. Scheme демонстрирует, что очень небольшого количества правил формирования выражений
без ограничений на их составление достаточно для формирования практического и эффективного языка
программирования, достаточно гибкого для поддержки большинства основных парадигм
программирования, используемых в настоящее время.

%Scheme
%was one of the first programming languages to incorporate first-class
%procedures as in the lambda calculus, thereby proving the usefulness of
%static scope rules and block structure in a dynamically typed language.
%Scheme was the first major dialect of Lisp to distinguish procedures
%from lambda expressions and symbols, to use a single lexical
%environment for all variables, and to evaluate the operator position
%of a procedure call in the same way as an operand position.  By relying
%entirely on procedure calls to express iteration, Scheme emphasized the
%fact that tail-recursive procedure calls are essentially gotos that
%pass arguments.  Scheme was the first widely used programming language to
%embrace first-class escape procedures, from which all previously known
%sequential control structures can be synthesized.  A subsequent
%version of Scheme introduced the concept of exact and inexact number objects,
%an extension of Common Lisp's generic arithmetic.
%More recently, Scheme became the first programming language to support
%hygienic macros, which permit the syntax of a block-structured language
%to be extended in a consistent and reliable manner.
Scheme был одним из первых языков программирования, включающим полноправные процедуры, как в
лямбда-исчислении, тем самым доказывая полноценность статических правил области видимости и
блочной структуры в языке с динамической типизацией. Scheme был первым основным диалектом Lisp,
различающим процедуры из лямбда-выражений и символы, использующий единственное лексическое
окружение для всех переменных и вычисляющий позицию оператора вызова процедуры таким же образом,
как и позицию операнда. Полностью основыванный на вызовах процедур для отражения итерации,
Scheme подчеркивает факт, что хвост-рекурсивные вызовы процедур являются по существу командами
перехода, передающими аргументы. Scheme был первым широко применяемым языком программирования,
использующим полноправные управляющие процедуры, из которых могут синтезироваться все известные
ранее последовательные управляющие структуры. В последующую версию Scheme введено понятие точных
и неточных числовых объектов, расширение обобщённой арифметики Common Lisp. Позже Scheme стал
первым языком программирования, поддерживающим гигиенические макросы, предоставляющие
возможность расширения синтаксиса языка с блочной структурой логичным и безопасным способом.\vspace{-3mm}

%\subsection*{Guiding principles}
\subsection*{Руководящие принципы}

%To help guide the standardization effort, the editors have adopted a
%set of principles, presented below.
%Like the Scheme language defined in \rrs{5}~\cite{R5RS}, the language described
%in this report is intended to:
Для содействия в проведении работы по стандартизации редакторы приняли ряд принципов, представленных
ниже. Как и язык Scheme, определённый в \rrs{5}~\cite{R5RS}, язык, описанный в данном стандарте,
предназначен для:\vspace{-1mm}

\begin{itemize}
%\item allow programmers to read each other's code, and allow
%  development of portable programs that can be executed in any
%  conforming implementation of Scheme;
\item предоставления программистам возможности чтения кода друг друга и разработки
  переносимых программ, которые могут быть выполнены в любой соответствующей реализации Scheme;

%\item derive its power from simplicity, a small number of generally
%  useful core syntactic forms and procedures, and no unnecessary
%  restrictions on how they are composed;
\item извлечения пользы из его эффективности за счёт простоты, небольшого количества основных
  синтаксических форм и процедур общего применения и отсутствия ненужных ограничений на их
  составление;

%\item allow programs to define new procedures and new hygienic
%  syntactic forms;
\item предоставления возможности определения в программах новых процедур и гигиенических
синтаксических форм;

%\item support the representation of program source code as data;
\item поддержки представления исходного текста программы в качестве данных;

%\item make procedure calls powerful enough to express any form of
%  sequential control, and allow programs to perform non-local control
%  operations without the use of global program transformations;
\item достижения эффективности вызовов процедур, достаточной для выражения любой формы
  последовательного управления и предоставления возможности выполнения в программах нелокальных
  управляющих операций без использования глобальных преобразований программы;\vspace{-1.2mm}

%\item allow interesting, purely functional programs to run indefinitely
%  without terminating or running out of memory on finite-memory
%  machines;
\item предоставления возможности неопределённо долгой работы интересных, чисто функциональных
  программ без остановки или выхода за границы памяти на машинах с конечной памятью;\vspace{-1.2mm}

%\item allow educators to use the language to teach programming
%  effectively, at various levels and with a variety of pedagogical
%  approaches; and
\item предоставления педагогам возможности применения языка для эффективного преподавания
  программирования на различных уровнях и с разнообразными педагогическими подходами; и\vspace{-1.2mm}

%\item allow researchers to use the language to explore the design,
%  implementation, and semantics of programming languages.
\item предоставления научным работникам возможности применения языка для изучения разработки,
  реализации и семантики языков программирования.
\end{itemize}\vspace{-1.2mm}

%In addition, this report is intended to:
Кроме того, данный стандарт предназначен для:\vspace{-1.2mm}

\begin{itemize}
%\item allow programmers to create and distribute substantial programs
%  and libraries, e.g., implementations of Scheme Requests for
%  Implementation, that run without
%  modification in a variety of Scheme implementations;
\item предоставления программистам возможности создания и распространения фундаментальных
  программ и библиотек, например, реализаций Scheme Requests for Implementation, выполняемых без
  модификации в разнообразных реализациях Scheme;\vspace{-1.2mm}

%\item support procedural, syntactic, and data abstraction more fully
%  by allowing programs to define hygiene-bending and hygiene-breaking
%  syntactic abstractions and new unique datatypes along with
%  procedures and hygienic macros in any scope;
\item максимально полной поддержки абстракций -- процедурных, синтаксических и данных, что
  позволяет определять в программах (сгибающие?) гигиену и (нарушающие?) гигиену синтаксические
  абстракции и новые уникальные типы данных наряду с процедурами и гигиеническими макросами в
  любых областях видимости;\vspace{-1.2mm}

%\item allow programmers to rely on a level of automatic run-time type
%  and bounds checking sufficient to ensure type safety; and
\item предоставления программистам возможности руководствоваться уровнем
  автоматической проверки типов и границ этапа выполнения,
  достаточным для гарантии безопасности типов; и\vspace{-1.2mm}

%\item allow implementations to generate efficient code, without
%  requiring programmers to use implementation-specific operators or
%  declarations.
\item предоставления реализациям возможности генерации эффективного кода, не требуя от
  программистов применения операторов или деклараций конкретной реализации.

\end{itemize}\vspace{-1.2mm}

%While it was possible to write portable programs in Scheme as
%described in \rrs{5}, and indeed portable Scheme programs were written
%prior to this report, many Scheme programs were not, primarily because
%of the lack of substantial standardized libraries and the
%proliferation of implementation-specific language additions.
В то время как существовала возможность написания переносимых программ на Scheme согласно
\rrs{5}, и безусловно переносимые программы Scheme были написаны до данного стандарта, множество
программ не было написано на Scheme прежде всего из-за нехватки фундаментальных
стандартизированных библиотек и чрезмерного количества зависящих от реализации языковых
дополнений.

%In general, Scheme should include building blocks that allow a wide
%variety of libraries to be written, include commonly used user-level
%features to enhance portability and readability of library and
%application code, and exclude features that are less commonly used and
%easily implemented in separate libraries.
В общем случае Scheme должен содержать стандартные блоки, позволяющие создавать широкий спектр
библиотек, включая универсальные средства пользовательского уровня для улучшения переносимости и
удобочитаемости библиотек и прикладного кода, и исключать средства, реже используемые и легко
реализуемые в отдельных библиотеках.

%The language described in this report is intended to also be backward
%compatible with programs written in Scheme as described in \rrs{5} to
%the extent possible without compromising the above principles and
%future viability of the language.  With respect to future viability,
%the editors have operated under the assumption that many more Scheme
%programs will be written in the future than exist in the present, so
%the future programs are those with which we should be most concerned.
Язык, описанный в данном стандарте, также подразумевает обратную совместимость и с программами,
написанными на Scheme согласно \rrs{5} при условии отсутствия компромисса между вышеописанными
основными положениями и будущей эффективностью языка. Принимая во внимание будущую эффективность,
редакторы исходили из предположения, что в будущем на Scheme будет написано больше программ,
чем существует в настоящее время, таким образом, больше всего мы должны быть
заинтересованы в будущих программах.\vspace{3mm}

%\subsection*{Acknowledgements}
\subsection*{Благодарности}\vspace{4mm}

%Many people contributed significant help to this revision of the
%report.  Specifically, we thank Aziz Ghuloum and Andr\'e van Tonder for
%contributing reference implementations of the library system.  We
%thank Alan Bawden, John Cowan, Sebastian Egner, Aubrey Jaffer, Shiro
%Kawai, Bradley Lucier, and Andr\'e van Tonder for contributing insights on
%language design.  Marc Feeley, Martin Gasbichler, Aubrey Jaffer, Lars T Hansen,
%Richard Kelsey, Olin Shivers, and Andr\'e van Tonder wrote SRFIs that
%served as direct input to the report.  Marcus Crestani, David Frese,
%Aziz Ghuloum, Arthur A.\ Gleckler, Eric Knauel, Jonathan Rees, and Andr\'e
%van Tonder thoroughly proofread early versions of the report.
Множество людей внесли существенный вклад в данную редакцию стандарта. В связи с этим мы
благодарим Aziz Ghuloum и Andr\'e van Tonder за предоставление эталонной реализации системы
библиотек. Мы благодарим Alan Bawden, John Cowan, Sebastian Egner, Aubrey Jaffer, Shiro Kawai,
Bradley Lucier и Andr\'e van Tonder за предоставление уникальной информации о разработке языка.
Marc Feeley, Martin Gasbichler, Aubrey Jaffer, Lars T Hansen, Richard Kelsey, Olin Shivers и
Andr\'e van Tonder писали SRFI, которые помогали непосредственным вкладом в стандарт. Marcus
Crestani, David Frese, Aziz Ghuloum, Arthur A.\ Gleckler, Eric Knauel, Jonathan Rees и Andr\'e
van Tonder тщательно вычитывали ранние версии стандарта.\vspace{4mm}

%We would also like to thank the following people for their
%help in creating this report: Lauri Alanko,
%Eli Barzilay, Alan Bawden, Brian C.\ Barnes, Per Bothner, Trent Buck,
%Thomas Bushnell, Taylor Campbell, Ludovic Court\`es, Pascal Costanza,
%John Cowan, Ray Dillinger, Jed Davis, J.A.\ ``Biep'' Durieux, Carl Eastlund,
%Sebastian Egner, Tom Emerson, Marc Feeley, Matthias Felleisen, Andy
%Freeman, Ken Friedenbach, Martin Gasbichler, Arthur A.\ Gleckler, Aziz
%Ghuloum, Dave Gurnell, Lars T Hansen, Ben Harris, Sven Hartrumpf, Dave
%Herman, Nils M.\ Holm, Stanislav Ievlev, James Jackson, Aubrey Jaffer,
%Shiro Kawai, Alexander Kjeldaas, Eric Knauel, Michael Lenaghan, Felix Klock,
%Donovan Kolbly, Marcin Kowalczyk, Thomas Lord, Bradley Lucier, Paulo
%J.\ Matos, Dan Muresan, Ryan Newton, Jason Orendorff, Erich Rast, Jeff
%Read, Jonathan Rees, Jorgen Sch\"afer, Paul Schlie, Manuel Serrano,
%Olin Shivers, Jonathan Shapiro, Jens Axel S\o{}gaard, Jay Sulzberger,
%Pinku Surana, Mikael Tillenius, Sam Tobin-Hochstadt, David Van Horn,
%Andr\'e van Tonder, Reinder Verlinde, Alan Watson, Andrew Wilcox, Jon
%Wilson, Lynn Winebarger, Keith Wright, and Chongkai Zhu.
Мы также хотели бы поблагодарить следующих людей за их помощь в создании данного стандарта: Lauri Alanko,
Eli Barzilay, Alan Bawden, Brian C.\ Barnes, Per Bothner, Trent Buck, Thomas Bushnell, Taylor
Campbell, Ludovic Court\`es, Pascal Costanza, John Cowan, Ray Dillinger, Jed Davis,
J.A.\ ``Biep'' Durieux, Carl Eastlund, Sebastian Egner, Tom Emerson, Marc Feeley, Matthias
Felleisen, Andy Freeman, Ken Friedenbach, Martin Gasbichler, Arthur A.\ Gleckler, Aziz Ghuloum,
Dave Gurnell, Lars T Hansen, Ben Harris, Sven Hartrumpf, Dave Herman, Nils M.\ Holm, Stanislav
Ievlev, James Jackson, Aubrey Jaffer, Shiro Kawai, Alexander Kjeldaas, Eric Knauel, Michael
Lenaghan, Felix Klock, Donovan Kolbly, Marcin Kowalczyk, Thomas Lord, Bradley Lucier, Paulo
J.\ Matos, Dan Muresan, Ryan Newton, Jason Orendorff, Erich Rast, Jeff Read, Jonathan Rees,
Jorgen Sch\"afer, Paul Schlie, Manuel Serrano, Olin Shivers, Jonathan Shapiro, Jens Axel
S\o{}gaard, Jay Sulzberger, Pinku Surana, Mikael Tillenius, Sam Tobin-Hochstadt, David Van Horn,
Andr\'e van Tonder, Reinder Verlinde, Alan Watson, Andrew Wilcox, Jon Wilson, Lynn Winebarger,
Keith Wright, and Chongkai Zhu.\vspace{4mm}

%We would like to thank the following people for their help in creating
%the previous revisions of this report: Alan Bawden, Michael
%Blair, George Carrette, Andy Cromarty, Pavel Curtis, Jeff Dalton, Olivier Danvy,
%Ken Dickey, Bruce Duba, Marc Feeley,
%Andy Freeman, Richard Gabriel, Yekta G\"ursel, Ken Haase, Robert
%Hieb, Paul Hudak, Morry Katz, Chris Lindblad, Mark Meyer, Jim Miller, Jim Philbin,
%John Ramsdell, Mike Shaff, Jonathan Shapiro, Julie Sussman,
%Perry Wagle, Daniel Weise, Henry Wu, and Ozan Yigit.
Мы хотели бы благодарить следующих людей за их помощь в создании предыдущих редакций данного
стандарта: Alan Bawden, Michael Blair, George Carrette, Andy Cromarty, Pavel Curtis, Jeff
Dalton, Olivier Danvy, Ken Dickey, Bruce Duba, Marc Feeley, Andy Freeman, Richard Gabriel, Yekta
G\"ursel, Ken Haase, Robert Hieb, Paul Hudak, Morry Katz, Chris Lindblad, Mark Meyer, Jim
Miller, Jim Philbin, John Ramsdell, Mike Shaff, Jonathan Shapiro, Julie Sussman, Perry Wagle,
Daniel Weise, Henry Wu и Ozan Yigit.\vspace{4mm}

%We thank Carol Fessenden, Daniel
%Friedman, and Christopher Haynes for permission to use text from the Scheme 311
%version 4 reference manual.  We thank Texas Instruments, Inc.~for permission to
%use text from the {\em TI Scheme Language Reference Manual}~\cite{TImanual85}.
%We gladly acknowledge the influence of manuals for MIT Scheme~\cite{MITScheme},
%T~\cite{Rees84}, Scheme 84~\cite{Scheme84}, Common Lisp~\cite{CLtL},
%Chez Scheme~\cite{csug7}, PLT~Scheme~\cite{mzscheme352},
%and Algol 60~\cite{Naur63}.
Мы благодарим Carol Fessenden, Daniel
Friedman и Christopher Haynes за разрешение использовать текст из Scheme 311
version 4 reference manual.  Мы благодарим Texas Instruments, Inc.~за разрешение
использовать текст из {\em TI Scheme Language Reference Manual}~\cite{TImanual85}.
Мы охотно признаём влияние руководств MIT Scheme~\cite{MITScheme},
T~\cite{Rees84}, Scheme 84~\cite{Scheme84}, Common Lisp~\cite{CLtL},
Chez Scheme~\cite{csug7}, PLT~Scheme~\cite{mzscheme352},
и Algol 60~\cite{Naur63}.\vspace{4mm}

%\vest We also thank Betty Dexter for the extreme effort she put into
%setting this report in \TeX, and Donald Knuth for designing the program
%that caused her troubles.
\vest Мы также благодарим Betty Dexter за титанические усилия, которые она вложила при
вводе данного стандарта в \TeX и Donald Knuth за разработку программы,
вызвавшей эти трудности.\vspace{4mm}

\vest The Artificial Intelligence Laboratory of the
Massachusetts Institute of Technology, the Computer Science
Department of Indiana University, the Computer and Information
Sciences Department of the University of Oregon, and the NEC Research
Institute supported the preparation of this report.  Support for the MIT
work was provided in part by
the Advanced Research Projects Agency of the Department of Defense under Office
of Naval Research contract N00014-80-C-0505.  Support for the Indiana
University work was provided by NSF grants NCS 83-04567 and NCS
83-03325.


%%% Local Variables:
%%% mode: latex
%%% TeX-master: "r6rs"
%%% End:
   \par
\vskip 2ex
%\clearchaptergroupstar{Description of the language} %\unskip\vskip -2ex
\clearchaptergroupstar{Описание языка} %\unskip\vskip -2ex
%\chapter{Overview of Scheme}
\chapter{Введение в Scheme}
\label{semanticchapter}

%This chapter gives an overview of Scheme's semantics.
%The purpose of this overview is to explain
%enough about the basic concepts of the language to facilitate
%understanding of the subsequent chapters of the report, which are
%organized as a reference manual.  Consequently, this overview is
%not a complete introduction to the language, nor is it precise
%in all respects or normative in any way.
В данной главе описано введение в семантику Scheme. Целью данного введения является объяснение
фундаментальных концепций языка, достаточных для облегчения понимания последующих глав работы,
организованных в виде справочного руководства. Поэтому данный краткий обзор не является полным
введением в язык, и при этом он не всегда точен в некоторых аспектах или не всегда нормативен.

%\vest Following Algol, Scheme is a statically scoped programming
%language.  Each use of a variable is associated with a lexically
%apparent binding of that variable.
После Algol Scheme является языком программирования со статическими областями
видимости. Каждое использование переменной ассоциировано с лексически явным
связыванием этой переменной.

%\vest Scheme has latent as opposed to manifest types
%\cite{WaiteGoos}.  Types
%are associated with objects\mainindex{object} (also called values) rather than
%with variables.  (Some authors refer to languages with latent types as
%untyped, weakly typed or dynamically typed languages.)  Other languages with
%latent types are Python, Ruby, Smalltalk, and other dialects of Lisp.  Languages
%with manifest types (sometimes referred to as strongly typed or
%statically typed languages) include Algol 60, C, C\#, Java, Haskell, and ML.
\vest Scheme имеет неявные, в отличие от декларативных, типы\cite{WaiteGoos}. Типы связаны с
объектами\mainindex{object} (также называемыми значениями), а не с переменными. (Некоторые
авторы называют языки с неявными типами нетипизированными, слабо типизированными или динамически
типизированными языками.) Другими языками с неявными типами являются Python, Ruby, Smalltalk, а
также другие диалекты Lisp. Языки с декларативными типами (иногда называемые строго
типизированными или статически типизированными языками) включают Algol 60, C, C\#, Java, Haskell
и ML.

%\vest All objects created in the course of a Scheme computation, including
%procedures and continuations, have unlimited extent.
%No Scheme object is ever destroyed.  The reason that
%implementations of Scheme do not (usually!)\ run out of storage is that
%they are permitted to reclaim the storage occupied by an object if
%they can prove that the object cannot possibly matter to any future
%computation.  Other languages in which most objects have unlimited
%extent include C\#, Java, Haskell, most Lisp dialects, ML, Python,
%Ruby, and Smalltalk.
\vest Все объекты, созданные в процессе вычисления Scheme, включая процедуры и продолжения,
имеют неограниченный экстент. Ни один объект Scheme никогда не разрушается. Причина того, что
реализации Scheme полностью не расходуют (обычно!)\ память, состоит в том, что им разрешается
возвращать память, занятую объектом, если они смогут выяснить, что объект не имеет возможности
влиять ни на одно будущее вычисление. Другие языки, в которых большинство объектов имеет
неограниченный экстент, включают C\#, Java, Haskell, большинство диалектов Lisp, ML, Python,
Ruby и Smalltalk.

%Implementations of Scheme must be properly tail-recursive.
%This allows the execution of an iterative computation in constant space,
%even if the iterative computation is described by a syntactically
%recursive procedure.  Thus with a properly tail-recursive implementation,
%iteration can be expressed using the ordinary procedure-call
%mechanics, so that special iteration constructs are useful only as
%syntactic sugar.
Реализации Scheme должны подчиняться правилам хвостовой рекурсии. Это даёт возможность
выполнения итеративного вычисления в постоянном пространстве, даже если итеративное вычисление
описано синтаксически рекурсивной процедурой. Таким образом, при реализации
поддержки хвостовой рекурсии, итерация может быть выражена с помощью обычной техники вызова
процедуры, так, чтобы специальные итеративные конструкции были применимы только в качестве
синтаксического сахара.

%\vest Scheme was one of the first languages to support procedures as
%objects in their own right.  Procedures can be created dynamically,
%stored in data structures, returned as results of procedures, and so
%on.  Other languages with these properties include Common Lisp,
%Haskell, ML, Ruby, and Smalltalk.
\vest Scheme был одним из первых языков, поддерживающих процедуры как объекты сами по себе.
Процедуры могут динамически создаваться, сохраняться в структурах данных, возвращаться как
результаты процедур, и так далее. Другие языки с этими свойствами включают Common Lisp,
Haskell, ML, Ruby и Smalltalk.

%\vest One distinguishing feature of Scheme is that continuations, which
%in most other languages only operate behind the scenes, also have
%``first-class'' status.  First-class continuations are useful for implementing a
%wide variety of advanced control constructs, including non-local exits,
%backtracking, and coroutines.
\vest Одной из отличительных особенностей Scheme является то, что продолжения, которые в
большинстве других языков функционируют только внутренне, также имеют "первоклассный"
статус. Первоклассные продолжения полезны для реализации широкого разнообразия передовых
управляющих конструкций, включая нелокальные выходы, бектрекинг и сопрограммы.

%\newpage

%In Scheme, the argument expressions of a procedure call are evaluated
%before the procedure gains control, whether the procedure needs the
%result of the evaluation or not.  C, C\#, Common Lisp, Python,
%Ruby, and Smalltalk are other languages that always evaluate argument
%expressions before invoking a procedure.  This is distinct from the
%lazy-evaluation semantics of Haskell, or the call-by-name semantics of
%Algol 60, where an argument expression is not evaluated unless its
%value is needed by the procedure.
Выражения аргументов вызова процедуры в Scheme вычисляются перед получением процедурой
управления независимо от необходимости для процедуры результата вычисления.
C, C\#, Common Lisp, Python, Ruby и Smalltalk - другие языки, которые всегда вычисляют выражения
аргументов перед вызовом процедуры. Это отличается от семантики ленивого вычисления Haskell или
семантики вызова по имени Algol 60, где выражение аргумента не вычисляется, если процедура не
нуждается в его значении.\vspace{1mm}

%Scheme's model of arithmetic provides a rich set of numerical types
%and operations on them.  Furthermore, it distinguishes \textit{exact}
%and \textit{inexact} number objects: Essentially, an exact number
%object corresponds to a number exactly, and an inexact number object
%is the result of a computation that involved rounding or other errors.
Арифметическая модель Scheme предоставляет богатый набор численных типов и операций с
ними. Кроме того, она различает \textit{точные} и \textit{неточные} числовые объекты: По
существу, точный числовой объект точно соответствует числу, а неточный числовой объект является
результатом вычисления, которое повлекло за собой округление или другие ошибки.\vspace{1mm}

%\section{Basic types}
\section{Основные типы}\vspace{1mm}

%Scheme programs manipulate \textit{objects}, which are also referred
%to as \textit{values}.
%Scheme objects are organized into sets of values called \textit{types}.
%This section gives an overview of the fundamentally important types of the
%Scheme language.  More types are described in later chapters.
Программы Scheme оперируют \textit{объектами}, которые также называются
\textit{значениями}. Объекты Scheme организованы в наборы значений, которые называются
\textit{типами}. В этой секции дан краткий обзор существенно важных типов языка Scheme. Больше
типов описано в последующих главах.\vspace{1mm}

\begin{note}
  %As Scheme is latently typed, the use of the term \textit{type} in
  %this report differs from the use of the term in the context of other
  %languages, particularly those with manifest typing.
  Поскольку Scheme латентно типизирован, использование термина \textit{тип} в данной работе
  отличается от использования термина в контексте других языков, особенно с явно объявляемой
  типизацией.
\end{note}

%\paragraph{Booleans}
\paragraph{Булевые}\vspace{1mm}

%\mainindex{boolean}A boolean is a truth value, and can be either
%true or false.  In Scheme, the object for ``false'' is written
%\schfalse{}.  The object for ``true'' is written \schtrue{}.  In
%most places where a truth value is expected, however, any object different from
%\schfalse{} counts as true.
\mainindex{boolean}Булевый тип является истинностным значением и может быть true или
false. Объект ``false'' в Scheme записывается \schfalse{}.  Объект ``true'' записывается
\schtrue{}. В большинстве мест, где ожидается истинностное значение, однако, любой объект,
отличный от \schfalse{} интерпретитуется как true.\vspace{1mm}

%\paragraph{Numbers}
\paragraph{Числовые}\vspace{1mm}

%\mainindex{number}Scheme supports a rich variety of numerical data types, including
%objects representing integers of arbitrary precision, rational numbers, complex numbers, and
%inexact numbers of various kinds.  Chapter~\ref{numbertypeschapter} gives an
%overview of the structure of Scheme's numerical tower
\mainindex{number}Scheme поддерживает множество числовых типов данных, включая
объекты, представляющие целые числа произвольной точности, рациональные числа, сложные числа и
приближённые числа различных видов. В Главе~\ref{numbertypeschapter} дан краткий обзор структуры
башни численных типов Scheme.\vspace{1mm}

%\paragraph{Characters}
\paragraph{Символьные}\vspace{1mm}

%\mainindex{character}Scheme characters mostly correspond to textual characters.
%More precisely, they are isomorphic to the \textit{scalar values} of
%the Unicode standard.
Символы Scheme по большей части соответствуют текстовым символам. Более точно,
они изоморфны к скалярным величинам стандарта Unicode.

%\paragraph{Strings}
\paragraph{Строковые}

%\mainindex{string}Strings are finite sequences of characters with fixed length and thus
%represent arbitrary Unicode texts.
\mainindex{string}Строки являются конечными последовательностями символов фиксированной длины и,
таким образом, представляют произвольные тексты Unicode.

%\newpage

%\paragraph{Symbols}
\paragraph{Символьные}

%\mainindex{symbol}A symbol is an object representing a string,
%the symbol's \textit{name}.
%Unlike strings, two symbols whose names are spelled the same
%way are never distinguishable.  Symbols are useful for many applications;
%for instance, they may be used the way enumerated values are used in
%other languages.
Символ является объектом, представляющим строку, \textit{имя} символа. В отличие от строки, два символа,
имена которых записаны в том же порядке, никоим образом не различаются. Символы полезны для
многих применений; например, они могут использоваться, перечислимые значения пути
используются на других языках.

%\paragraph{Pairs and lists}
\paragraph{Пары и списки}

\mainindex{pair}\mainindex{list}
%A pair is a data structure with two components.  The most common use
%of pairs is to represent (singly linked) lists, where the first
%component (the ``car'') represents the first element of the list, and
%the second component (the ``cdr'') the rest of the list.  Scheme also
%has a distinguished empty list, which is the last cdr in a chain of
%pairs that form a list.
Пара является структурой данных с двумя компонентами. Наиболее общее применение пар относится
к представлению (отдельно связанных) списков, где первый компонент ("car") представляет первый
элемент списка, а второй компонент ("cdr") - остальную часть списка. Scheme также имеет
отдельный пустой список, который является последним cdr в цепочке пар, формирующих
список.

%\paragraph{Vectors}
\paragraph{Векторы}

%\mainindex{vector}Vectors, like lists, are linear data structures
%representing finite sequences of arbitrary objects.
%Whereas the elements of a list are accessed
%sequentially through the chain of pairs representing it,
%the elements of a vector are addressed by integer indices.
%Thus, vectors are more appropriate than
%lists for random access to elements.
\mainindex{vector}Векторы, как и списки, являются линейными структурами данных, представляющими конечные
последовательности произвольных объектов. Поскольку к элементам списка получают
доступ последовательно через цепь пар, представляющих его, к элементам вектора обращаются
целыми индексами. Таким образом, векторы предпочтительнее списков для
произвольного доступа к элементам.

%\paragraph{Procedures}
\paragraph{Процедуры}

%\mainindex{procedure}Procedures are values in Scheme.
\mainindex{procedure}В Scheme процедуры являются значениями.

%\section{Expressions}
\section{Выражения}

%The most important elements of Scheme code are
%\mainindex{expression}\textit{expressions}.  Expressions can be
%\textit{evaluated}, producing a \textit{value}.  (Actually, any number
%of values---see section~\ref{multiplereturnvaluessection}.)  The most
%fundamental expressions are literal expressions:
Важнейшими элементами кода Scheme являются \mainindex{expression}\textit{выражения}. Выражения
могут быть \textit{вычислены}, порождая \textit{значение}. (Фактически, любое количество
значений---см. раздел~\ref{multiplereturnvaluessection}.) Самые фундаментальные выражения -
литеральные выражения:\vspace{1mm}

\begin{scheme}
\bfseries{\schtrue{}} \ev \bfseries{\schtrue}
\bfseries{23} \ev \bfseries{23}%
\end{scheme}\vspace{1mm}

%This notation means that the expression \schtrue{} evaluates to
%\schtrue{}, that is, the value for ``true'',  and that the expression
%{\cf 23} evaluates to a number object representing the number 23.
Эта форма записи означает, что выражение \schtrue {} вычисляется как
\schtrue {}, то есть, значение для ``true'', и что выражение {\cf 23} вычисляется к
численному объекту, представляющему число 23.\vspace{1mm}

%Compound expressions are formed by placing parentheses around their
%subexpressions.  The first subexpression identifies an operation; the
%remaining subexpressions are operands to the operation:
Составные выражения формируются помещением круглых скобок вокруг своих подвыражений. Первое
подвыражение идентифицирует операцию; остальные подвыражения являются операндами
операции:\vspace{1mm}
%
\begin{scheme}
\bfseries{(+ 23 42)} \ev \bfseries{65}
\bfseries{(+ 14 (* 23 42))} \ev \bfseries{980}%
\end{scheme}\vspace{1mm}
%
%In the first of these examples, {\cf +} is the name of
%the built-in operation for addition, and {\cf 23} and {\cf 42} are the
%operands.  The expression {\cf (+ 23 42)} reads as ``the sum of 23 and
%42''.  Compound expressions can be nested---the second example reads
%as ``the sum of 14 and the product of 23 and 42''.
В начале этого примера {\bfseries\cf +} является именем встроенной операции сложения, а
{\bfseries\cf 23} и {\bfseries\cf 42} - операндами. Выражение {\bfseries\cf (+ 23 42)}
читается как ``сумма 23 и 42''. Составные выражения могут вкладываться---следующий пример
читается как ``сумма 14 и произведения 23 и 42''.\vspace{1mm}

%As these examples indicate, compound expressions in Scheme are always
%written using the same prefix notation\mainindex{prefix notation}.  As
%a consequence, the parentheses are needed to indicate structure.
%Consequently, ``superfluous'' parentheses, which are often permissible in
%mathematical notation and also in many programming languages, are not
%allowed in Scheme.
Как показывают эти примеры, составные выражения в Scheme всегда записываются с помощью одной и
той же префиксной нотации\mainindex{prefix notation}. Следствием является необходимость круглых
скобок для указания структуры. Следовательно, ``лишние'' круглые скобки, которые часто
допускаются в математической нотации, а также во многих языках программирования, в Scheme не
позволяются.

%As in many other languages, whitespace (including line endings) is not
%significant when it separates subexpressions of an expression, and
%can be used to indicate structure.
Как и во многих других языках, пробельные символы (включая конец строки) являются не значащими,
когда они отделяют подвыражения выражений, и могут использоваться для указания структуры.

%\section{Variables and binding}
\section{Переменные и привязка}

%\mainindex{variable}\mainindex{binding}\mainindex{identifier}Scheme
%allows identifiers to stand for locations containing values.
%These identifiers are called variables.  In many cases, specifically
%when the location's value is never modified after its creation, it is
%useful to think of the variable as standing for the value directly.
\mainindex{variable}\mainindex{binding}\mainindex{identifier}Scheme позволяет привязывать
идентификаторы к содержащим значения ячейкам памяти. Такие идентификаторы называются
переменными. Во многих случаях, определенно когда значение ячейки памяти никогда не изменяется
после её создания, полезно думать о переменной как привязанной к значению непосредственно.\vspace{1mm}

\begin{scheme}
\bfseries(let ((x 23)
\bfseries      (y 42))
\bfseries  (+ x y)) \ev \textbf{65}%
\end{scheme}\vspace{1mm}

%In this case, the expression starting with {\cf let} is a binding
%construct.  The parenthesized structure following the {\cf let} lists
%variables alongside expressions: the variable {\cf x} alongside {\cf
%  23}, and the variable {\cf y} alongside {\cf 42}.  The {\cf let}
%expression binds {\cf x} to 23, and {\cf y} to 42.  These bindings are
%available in the \textit{body} of the {\cf let} expression, {\cf (+ x
%  y)}, and only there.
В данном случае выражение, начинающееся с {\cf\bfseries let} является конструкцией привязки. В
окружённой скобками структуре после {\cf\bfseries let} перечислены переменные совместно с выражениями:
переменная {\cf\bfseries x} совместно с {\cf\bfseries 23} и переменная {\cf\bfseries y}
совместно с {\cf\bfseries 42}. Выражение {\cf\bfseries let} связывает {\cf\bfseries x} с 23 и
{\cf\bfseries y} c 42. Эти привязки доступны в \textit{теле} выражения {\cf\bfseries let},
{\cf\bfseries (+ x y)}, и только там.

%\section{Definitions}
\section{Определения}\vspace{1mm}

%\index{definition}The variables bound by a {\cf let} expression
%are \textit{local}, because their bindings are visible only in the
%{\cf let}'s body.  Scheme also allows creating top-level bindings for
%identifiers as follows:
\index{definition}Переменные, связанные выражением {\cf\bfseries let}, являются
\textit{локальными}, так как их связывания видимы только в теле {\cf\bfseries let}. Scheme также
позволяет создавать связывания верхнего уровня для идентификаторов следующим образом:\vspace{1mm}

\begin{scheme}
\bfseries(define x 23)
\bfseries(define y 42)
\bfseries(+ x y) \ev \textbf{65}%
\end{scheme}\vspace{1mm}

%(These are actually ``top-level'' in the body of a top-level program or library;
%see section~\ref{librariesintrosection} below.)
(Они фактически являются "верхним уровнем" в теле программы верхнего уровня или библиотеки; см. секцию
~\ref {librariesintrosection} ниже.)

%The first two parenthesized structures are \textit{definitions}; they
%create top-level bindings, binding {\cf x} to 23 and {\cf y} to 42.
%Definitions are not expressions, and cannot appear in all places
%where an expression can occur.  Moreover, a definition has no value.
Первые две заключенные в скобки структуры являются \textit {определениями}; они создают
привязки верхнего уровня, связывая {\cf x} с 23, а {\cf y} с 42. Определения не являются
выражениями и не могут находиться там, тде где может находиться выражение.
Кроме того, определение не имеет значения.

%Bindings follow the lexical structure of the program:  When several
%bindings with the same name exist, a variable refers to the binding
%that is closest to it, starting with its occurrence in the program
%and going from inside to outside, and referring to a top-level
%binding if no
%local binding can be found along the way:
Привязки подчиняются лексической структуре программы: при наличии нескольких одноименных
привязок переменная соотносится с ближайшей к ней на пути изнутри снаружу привязкой, начиная от
её появления в программе, и с привязкой верхнего уровня, если локальная привязка не может быть
найдена на этом пути:\vspace{1mm}

\begin{scheme}
\bfseries(define x 23)
\bfseries(define y 42)
\bfseries(let ((y 43))
%\end{scheme}
%
%\newpage
%
%\begin{scheme}
\bfseries  (+ x y)) \ev \textbf{66}

\bfseries(let ((y 43))
\bfseries  (let ((y 44))
\bfseries    (+ x y))) \ev \textbf{67}%
\end{scheme}

%\section{Forms}
\section{Формы}

%While definitions are not expressions, compound expressions and
%definitions exhibit similar syntactic structure:
Хотя определения не являются выражениями, составные выражения и определения
имеют схожую синтаксическую структуру:
%
\begin{scheme}
\bfseries(define x 23)
\bfseries(* x 2)%
\end{scheme}
%
%While the first line contains a definition, and the second an
%expression, this distinction depends on the bindings for {\cf define}
%and {\cf *}.  At the purely syntactical level, both are
%\textit{forms}\index{form}, and \textit{form} is the general name for
%a syntactic part of a Scheme program.  In particular, {\cf 23} is a
%\textit{subform}\index{subform} of the form {\cf (define x 23)}.
При этом первая линия содержит определение, а следующая - выражение, данное различие
основывается на связывании {\cf\bfseries define} и {\cf\bfseries *}. На чисто синтаксическом
уровне обе являются \textit{формами}\index{form}, а \textit{форма} является обобщённым названием
синтаксической части программы Scheme. В частности, {\cf\bfseries 23} является
\textit{подформой} \index{subform} формы {\cf\bfseries (define x 23)}.

%\section{Procedures}
\section{Процедуры}
\label{proceduressection}

%\index{procedure}Definitions can also be used to define
%procedures:
\index{procedure}Определения могут также использоваться для определения процедур.

\begin{scheme}
\bfseries(define (f x)
\bfseries  (+ x 42))

\bfseries(f 23) \ev \textbf{65}%
\end{scheme}

%A procedure is, slightly simplified, an abstraction of an
%expression over objects.  In the example, the first definition defines a procedure
%called {\cf f}.  (Note the parentheses around {\cf f x}, which
%indicate that this is a procedure definition.)  The expression {\cf (f
%  23)} is a \index{procedure call}procedure call, meaning,
%roughly, ``evaluate {\cf (+ x 42)} (the body of the procedure) with
%{\cf x} bound to 23''.
Процедура, несколько упрощённо, является абстракцией выражения посредством объектов. В первом определении
примера определена процедура, названная {\cf\bfseries f}. (Обратите внимание на круглые скобки
вокруг {\cf\bfseries f x}, обозначающие, что это - определение процедуры.) Выражение
{\cf\bfseries (f 23)} является \index{procedure call} вызовом процедуры, приблизительно означающим
"вычислить {\cf\bfseries (+ x 42)} (тело процедуры) с {\cf\bfseries x}, привязанным к \textbf{23}".

%As procedures are objects, they can be passed to other
%procedures:
Поскольку процедуры являются объектами, их можно передавать в другие процедуры:

%
\begin{scheme}
\bfseries(define (f x)
\bfseries  (+ x 42))

\bfseries(define (g p x)
\bfseries  (p x))

\bfseries(g f 23) \ev \textbf{65}%
\end{scheme}

%In this example, the body of {\cf g} is evaluated with {\cf p}
%bound to {\cf f} and {\cf x} bound to 23, which is equivalent
%to {\cf (f 23)}, which evaluates to 65.
В этом примере тело {\cf\bfseries g} вычисляется с {\cf\bfseries p}, привязанным к {\cf\bfseries
  f}, и {\cf\bfseries x}, привязанным к \textbf{23}, что эквивалентно {\cf\bfseries (f 23)} и
вычисляется как \textbf{65}.

%In fact, many predefined operations of Scheme are provided not by
%syntax, but by variables whose values are procedures.
%The {\cf +} operation, for example, which receives
%special syntactic treatment in many other languages, is just a regular
%identifier in Scheme, bound to a procedure that adds number objects.  The
%same holds for {\cf *} and many others:
Фактически многие предопределённые операции Scheme обеспечиваются не синтаксисом, а
переменными, значениями которых являются процедуры. Операция {\cf\bfseries +}, например, приобретающая
специальную синтаксическую трактовку во многих других языках, в Scheme является всего лишь
регулярным идентификатором, связанным с процедурой, складывающей числовые объекты.
То же самое касается и {\cf\bfseries *}, и многих других:

\begin{scheme}
\bfseries(define (h op x y)
\bfseries  (op x y))

\bfseries(h + 23 42) \ev \textbf{65}
\bfseries(h * 23 42) \ev \textbf{966}%
\end{scheme}

%Procedure definitions are not the only way to create procedures.  A
%{\cf lambda} expression creates a new procedure as an object, with no
%need to specify a name:
Определения процедур - не единственный способ создания процедур. {\cf\bfseries lambda}-выражение
создаёт новую процедуру в качестве объекта без необходимости указания имени:

\begin{scheme}
\bfseries((lambda(x)(+ x 42))23) \ev \textbf{65}%
\end{scheme}

%The entire expression in this example is a procedure call; {\cf
%  (lambda (x) (+ x 42))}, evaluates to a procedure that takes a single
%number object and adds 42 to it.
Всё выражение в этом примере является вызовом процедуры; {\cf\bfseries (lambda (x) (+ x 42))}
вычисляется как процедура, принимающая одиночный числовой объект и добавляющая к нему 42.

%\section{Procedure calls and syntactic keywords}
\section{Вызовы процедур и синтаксические ключевые слова}

%Whereas {\cf (+ 23 42)}, {\cf (f 23)}, and {\cf ((lambda (x) (+ x 42))
%  23)} are all examples of procedure calls, {\cf lambda} and {\cf
%  let} expressions are not.  This is because {\cf let}, even though
%it is an identifier, is not a variable, but is instead a \textit{syntactic
%  keyword}\index{syntactic keyword}.  A form that has a
%syntactic keyword as its first subexpression obeys special rules determined by
%the keyword.  The {\cf define} identifier in a definition is also a
%syntactic keyword.  Hence, definitions are also not procedure calls.
Хотя и {\cf\bfseries (+ 23 42)}, и {\cf\bfseries (f 23)}, и {\cf\bfseries ((lambda (x) (+ x 42))
  23)} являются примерами вызовов процедур, выражения {\cf\bfseries lambda} и {\cf\bfseries let} -
нет. Это потому что {\cf\bfseries let}, хоть и идентификатор, но
не переменная, а \textit{синтаксическое ключевое слово}\index{syntactic
  keyword}. Форма, содежащая синтаксическое ключевое слово в качестве своего первого подвыражения,
подчиняется специальным правилам, определяемым ключевым словом. Идентификатор {\cf\bfseries
  define} в определении также является синтаксическим ключевым словом. Следовательно,
определения также не являются вызовами процедур.

%The rules for the {\cf lambda} keyword specify that the first
%subform is a list of parameters, and the remaining subforms are the body of
%the procedure.  In {\cf let} expressions, the first subform is a list
%of binding specifications, and the remaining subforms constitute a body of
%expressions.
Правилами для ключевого слова {\cf\bfseries lambda} определено, что первая подформа является
списком параметров, а остальные подформы - телом процедуры. В выражении {\cf\bfseries let}
первая подформа является списком спецификаций привязки, а остальные подформы образовывают тело
выражений.

%Procedure calls can generally be distinguished from these
%\textit{special forms}\mainindex{special form} by
%looking for a syntactic keyword in the first position of an
%form: if the first position does not contain a syntactic keyword, the expression
%is a procedure call.
%(So-called \textit{identifier macros} allow creating other kinds of
%special forms, but are comparatively rare.)
%The set of syntactic keywords of Scheme is
%fairly small, which usually makes this task fairly simple.
%It is possible, however, to create new bindings for syntactic keywords; see
%section~\ref{macrosintrosection} below.
Обычно вызовы процедур можно отличить от таких \textit{специальных форм} \mainindex{special
  form}, с помощью поиска синтаксического ключевого слова в первом положении формы: если
в первое положении не содержится синтаксического ключевого слова, выражение является вызовом
процедуры. (Так называемый \textit {макрос идентификатора} позволяет создавать
другие виды специальных форм, но сравнительно редко.) Набор синтаксических ключевых слов Scheme
является довольно небольшим, что обычно делает эту задачу довольно простой. Возможно,
однако, создание новых привязок для синтаксических ключевых слов;
см. секцию~\ref{macrosintrosection} ниже.

%\section{Assignment}
\section{Присваивание}

%Scheme variables bound by definitions or {\cf let} or {\cf lambda}
%expressions are not actually bound directly to the objects specified in the
%respective bindings, but to locations containing these objects.  The
%contents of these locations can subsequently be modified destructively
%via \textit{assignment}\index{assignment}:
Переменные Scheme, связанные с определениями или с выражениями {\cf\bfseries let} или
{\cf\bfseries lambda}, привязываются фактически не непосредственно к объектам, определённым в
соответствующих привязках, а к адресам памяти, содержащим эти объекты. Содержимое по этим
адресам впоследствии может быть деструктивно изменено с помощью \textit{присваивания}
\index{assignment}:
%
\begin{scheme}
\bfseries(let ((x 23))
\bfseries  (set! x 42)
\bfseries  x) \ev \textbf{42}%
\end{scheme}

%\newpage

%In this case, the body of the {\cf let} expression consists of two
%expressions which are evaluated sequentially, with the value of the
%final expression becoming the value of the entire {\cf let}
%expression.  The expression {\cf (set! x 42)} is an assignment, saying
%``replace the object in the location referenced by {\cf x} with 42''.
%Thus, the previous value of {\cf x}, 23, is replaced by 42.
В данном случае тело выражения {\cf\bfseries let} состоит из двух вычисляемых последовательно
выражений со значением финального выражения, принимающего значение всего выражения {\cf\bfseries
  let}. Выражение {\cf\bfseries (set! x 42),} является присваиванием, указывающим "заменить объект по
адресу, на который указывает {\cf\bfseries x}, на 42". Таким образом, предыдущее значение
{\cf\bfseries x} - 23 изменяется на 42.

%\section{Derived forms and macros}
\section{Производные формы и макросы}
\label{macrosintrosection}

%Many of the special forms specified in this report
%can be translated into more basic special forms.
%For example, a {\cf let} expression can be translated
%into a procedure call and a {\cf lambda} expression.  The following two
%expressions are equivalent:
Большинство специальных форм, определённых в данной работе, могут быть приведены к более
простым специальным формам. Например, выражение {\cf\bfseries let} может быть приведено к вызову
процедуры и выражению {\cf\bfseries lambda}. Следующие два выражения эквивалентны:
%
\begin{scheme}
\bfseries(let ((x 23)
\bfseries      (y 42))
\bfseries  (+ x y)) \ev \textbf{65}

\bfseries((lambda (x y) (+ x y)) 23 42) \lev \textbf{65}%
\end{scheme}

%Special forms like {\cf let} expressions are called \textit{derived
%  forms}\index{derived form} because their semantics can be
%derived from that of other kinds of forms by a syntactic
%transformation.  Some procedure definitions are also derived forms.  The
%following two definitions are equivalent:
Специальные формы, такие, как выражения {\cf\bfseries let}, называются \textit{производными
  формами}\index{derived form}, так как их семантика может быть получена из того или иного вида
форм синтаксическим преобразованием. Некоторые определения процедур также являются производными
формами. Следующие два определения эквивалентны:

\begin{scheme}
\bfseries(define (f x)
\bfseries  (+ x 42))

\bfseries(define f
\bfseries  (lambda (x)
\bfseries    (+ x 42)))%
\end{scheme}

%In Scheme, it is possible for a program to create its own derived
%forms by binding syntactic keywords to macros\index{macro}:
В программе Scheme имеется возможность создания своих собственных производных форм путём связывания
синтаксических ключевых слов с макросами \index{macro}:

\begin{scheme}
\bfseries(define-syntax def
\bfseries  (syntax-rules ()
\bfseries    ((def f (p ...) body)
\bfseries     (define (f p ...)
\bfseries       body))))

\bfseries(def f (x)
\bfseries  (+ x 42))%
\end{scheme}

%The {\cf define-syntax} construct specifies that a parenthesized
%structure matching the pattern {\cf (def f (p ...) body)}, where {\cf
%  f}, {\cf p}, and {\cf body} are pattern variables, is translated to
%{\cf (define (f p ...) body)}.  Thus, the {\cf def} form appearing in
%the example gets translated to:
Конструкция {\cf\bfseries define-syntax} определяет, что заключенная в скобки структура,
соответствующая шаблону {\cf\bfseries (def f (p ...) body)}, где {\cf\bfseries f}, {\cf\bfseries
  p} и {\cf\bfseries body} - переменные шаблона, приводится к {\cf\bfseries (define (f p ...)
  body)}. Таким образом, форма {\cf\bfseries def}, находящаяся в примере, приводится к:

\begin{scheme}
\bfseries(define (f x)
\bfseries  (+ x 42))%
\end{scheme}

%The ability to create new syntactic keywords makes Scheme extremely
%flexible and expressive, allowing many of the features
%built into other languages to be derived forms in Scheme.
Возможность создания новых синтаксических ключевых слов делает Scheme чрезвычайно гибким и
выразительным, что позволяет большинству особенностей, встроенных в другие языки, быть
производными формами в Scheme.

%\section{Syntactic data and datum values}
\section{Синтаксические данные и значения datum}

%A subset of the Scheme objects is called \textit{datum
%  values}\index{datum value}.
%These include booleans, number objects, characters, symbols,
%and strings as well as lists and vectors whose elements are data.  Each
%datum value may be represented in textual form as a
%\textit{syntactic datum}\index{syntactic datum}, which can be written out
%and read back in without loss of information.
%A datum value may be represented by several different syntactic data.
%Moreover, each datum value
%can be trivially translated to a literal expression in a program by
%prepending a {\cf\singlequote} to a corresponding syntactic datum:
Подмножество объектов Scheme называется \textit{значениями datum}\index{datum value}.
Оно включает булевы, численные объекты, знаки, символы и строки, равно как списки и векторы,
элементы которых являются данными. Каждое значение datum может быть представлено в текстовом
виде как \textit{синтаксический datum}\index{syntactic datum}, который может записываться и
считываться без потери информации. Значение datum может быть представлено различными
синтаксическими данными. Кроме того, каждое базисное значение может быть тривиально приведено к
литеральному выражению в программе путём добавления {\cf\singlequote} к соответствующему
синтаксическому datum:

\begin{scheme}
\bfseries'23 \ev \textbf{23}
\bfseries'\schtrue{} \ev \bfseries\schtrue{}
\bfseries'foo \ev \textbf{foo}
\bfseries'(1 2 3) \ev \textbf{(1 2 3)}
\bfseries'\#(1 2 3) \ev \textbf{\#(1 2 3)}%
\end{scheme}

%The {\cf\singlequote} shown in the previous examples
%is not needed for representations of number objects or booleans.
%The syntactic datum {\cf foo} represents a
%symbol with name ``foo'', and {\cf 'foo} is a literal expression with
%that symbol as its value.  {\cf (1 2 3)} is a syntactic datum that
%represents a list with elements 1, 2, and 3, and {\cf '(1 2 3)} is a literal
%expression with this list as its value.  Likewise, {\cf \#(1 2 3)}
%is a syntactic datum that represents a vector with elements 1, 2 and 3, and
%{\cf '\#(1 2 3)} is the corresponding literal.
{\cf\singlequote}, показанный в предыдущих примерах, не нужен для представлений численных
объектов или булевых переменных. Синтаксический базис {\cf\bfseries foo} представляет символ с
именем ``foo'', а {\cf\bfseries 'foo} - литеральное выражение с этим символом в качестве его
значения. {\cf\bfseries (1 2 3)} - синтаксический базис, представляющий список с
элементами 1, 2 и 3, а {\cf\bfseries '(1 2 3)} - литеральное выражение с этим списком в качестве
его значения. Аналогично, {\cf\bfseries \#(1 2 3)} - синтаксический базис, представляющий
вектор с элементами 1, 2 и 3, а {\cf\bfseries '\#(1 2 3)} - соответствующий литерал.


%The syntactic data are a superset of the Scheme forms.  Thus, data
%can be used to represent Scheme forms as data objects.  In
%particular, symbols can be used to represent identifiers.
Синтаксические данные являются расширенным набором форм Scheme. Таким образом, данные могут
использоваться, для представления форм Scheme в виде объектов данных. В частности, символы могут
использоваться для представления идентификаторов.

\begin{scheme}
\bfseries'(+ 23 42) \ev \bfseries(+ 23 42)
\bfseries'(define (f x) (+ x 42)) \lev \bfseries(define (f x) (+ x 42))%
\end{scheme}

%This facilitates writing programs that operate on Scheme source code,
%in particular interpreters and program transformers.
Это облегчает написание программ, работающих с исходным кодом Scheme, в частности,
интерпретаторов и программных преобразователей.\vspace{-2mm}

%\section{Continuations}
\section{Продолжения}

%Whenever a Scheme expression is evaluated there is a
%\textit{continuation}\index{continuation} wanting the result of the
%expression.  The continuation represents an entire (default) future
%for the computation.  For example, informally the continuation of {\cf 3}
%in the expression
Всякий раз, когда выражение Scheme оценено есть \textit{продолжение}\index{continuation}, ожидая
результат выражения. Продолжение представляет все будущее (по умолчанию) для
вычисления. Например, произвольно, продолжение {\cf\bfseries 3} в выражении
%
\begin{scheme}
\bfseries(+ 1 3)%
\end{scheme}
%
%adds 1 to it.  Normally these ubiquitous continuations are hidden
%behind the scenes and programmers do not think much about them.  On
%rare occasions, however, a programmer may need to deal with
%continuations explicitly.  The {\cf call-with-current-continuation}
%procedure (see section~\ref{call-with-current-continuation}) allows
%Scheme programmers to do that by creating a procedure that reinstates
%the current continuation.  The {\cf call-with-current-continuation}
%procedure accepts a procedure, calls it immediately with an argument
%that is an \textit{escape procedure}\index{escape procedure}.  This
%escape procedure can then be called with an argument that becomes the
%result of the call to {\cf call-with-current-continuation}.  That is,
%the escape procedure abandons its own continuation, and reinstates the
%continuation of the call to {\cf call-with-current-continuation}.
добавляет 1 к нему. Обычно эти встречающиеся повсюду продолжения скрыты на заднем плане, и
программисты особо не заботятся о них. В редких случаях, однако, программист, возможно, должен
иметь дело с продолжениями явно. Процедура {\cf\bfseries call-with-current-continuation}
(см. секцию ~\ref{call-with-current-continuation}) позволяет программистам Scheme делать это,
создавая процедуру, которая восстанавливает текущее продолжение. Процедура {\cf\bfseries
  call-with-current-continuation} принимает процедуру, вызывает её немедленно с аргументом,
который является \textit{аварийной процедурой}\index{escape procedure}. Эту аварийную процедуру
можно тогда вызвать с аргументом, который становится результатом вызова {\cf\bfseries
  call-with-current-continuation}. Таким образом, аварийная процедура оставляет ее собственное
продолжение, и восстанавливает продолжение вызова к {\cf\bfseries call-with-current-continuation}.

%In the following example, an escape procedure representing the
%continuation that adds 1 to its argument is bound to {\cf escape}, and
%then called with 3 as an argument.  The continuation of the call to
%{\cf escape} is abandoned, and instead the 3 is passed to the
%continuation that adds 1:
В следующем примере, аварийная процедура, представляющая продолжение, которое добавляет 1 к его
аргументу, привязана к {\cf\bfseries escape}, и затем вызывается с 3 как аргумент. Продолжение
вызова {\cf\bfseries escape} оставлено, и вместо этого эти 3 передают к продолжению, которое добавляет
1:
%
\begin{scheme}
\bfseries(+ 1 (call-with-current-continuation
\bfseries       (lambda (escape)
\bfseries         (+ 2 (escape 3))))) \lev \bfseries 4%
\end{scheme}
%
%An escape procedure has unlimited extent: It can be called after the
%continuation it captured has been invoked, and it can be called
%multiple times.  This makes {\cf call-with-current-continuation}
%significantly more powerful than typical non-local control constructs
%such as exceptions in other languages.
Аварийная процедура имеет неограниченный экстент: Она может быть вызвана после вызова
захватившего его продолжения, и это можно вызвать несколько раз. Это делает {\cf\bfseries
  call-with-current-continuation} значительно более мощным чем типичные нелокальные управляющие
конструкции, типа исключений в других языках.

%\section{Libraries}
\section{Библиотеки}
\label{librariesintrosection}

%Scheme code can be organized in components called
%\textit{libraries}\index{library}.  Each library contains
%definitions and expressions.  It can import definitions
%from other libraries and export definitions to other libraries.
Код Scheme может быть организован в компонентах, которые называются
\textit{библиотеками}\index{library}. Каждая библиотека содержит определения и выражения. Она
может импортировать определения из других библиотек и экспортировать определения в другие
библиотеки.

%The following library called {\cf (hello)} exports a definition called
%{\cf hello-world},  and imports the base library (see
%chapter~\ref{baselibrarychapter}) and the simple I/O library (see
%library section~\extref{lib:simpleiosection}{Simple I/O}).  The {\cf
%  hello-world} export is a procedure that displays {\cf Hello World}
%on a separate line:
Следующая библиотека называется {\cf\bfseries (hello)}, экспортирует определение, которое
называется {\cf\bfseries hello-world}, и импортирует основную библиотеку (см. главу
~\ref{baselibrarychapter}), и простая библиотека I/O (см. библиотечную
секцию~\extref{lib:simpleiosection} {Простой ввод/вывод}). Экспорт {\cf\bfseries hello-world} является
процедурой, которая показывает {\cf\bfseries Hello World} на отдельной линии:
%
\begin{scheme}
\bfseries(library (hello)
\bfseries  (export hello-world)
\bfseries  (import (rnrs base)
\bfseries          (rnrs io simple))
\bfseries  (define (hello-world)
\bfseries    (display "Hello World")
\bfseries    (newline)))%
\end{scheme}

%\section{Top-level programs}
\section{Программы верхнего уровня}

%A Scheme program is invoked via a \textit{top-level
%  program}\index{top-level program}.  Like a library, a top-level
%program contains imports, definitions and expressions, and specifies
%an entry point for execution.  Thus a top-level program defines, via
%the transitive closure of the libraries it imports, a Scheme program.
Программа Scheme активизируется посредством \textit{программы верхнего уровня}\index{top-level
  program}. Как и библиотека, программа верхнего уровня содержит импорт, определения и
выражения, и определяет точку входа для выполнения. Таким образом, программа верхнего уровня
определяет, через замкнутое выражение библиотек, которые это импортирует, программу Scheme.

%The following top-level program obtains the first argument from the command line
%via the {\cf command-line} procedure from the \rsixlibrary{programs}
%library (see library chapter~\extref{lib:programlibchapter}{Command-line
%  access and exit values}).  It then opens the file using {\cf
%  open-file-input-port} (see library section~\extref{lib:portsiosection}),
%yielding a \textit{port}, i.e.\ a connection to the file as a data
%source, and calls the {\cf get-bytes-all} procedure to obtain the
%contents of the file as binary data.  It then uses {\cf put-bytes} to
%output the contents of the file to standard output:
Следующая программа верхнего уровня получает первый аргумент из командной строки через процедуру
{\cf\bfseries command-line} из библиотеки {\bfseries\rsixlibrary{programs}} (см. библиотечную
главу~\extref{lib:programlibchapter}{Доступ к командной строке и выходные значения}). Это тогда
открывает файл с помощью {\cf\bfseries open-file-input-port} (см. секцию
~\extref{lib:portsiosection}), приводя к \textit{порт}, т.е.\ к связи с файлом как источником
данных, и вызывает процедуру {\cf\bfseries get-bytes-all} получения содержимого файла в виде
двоичных данных. Это тогда использует {\cf\bfseries put-bytes} для вывода содержимого файла в
стандартный вывод:\vspace{3mm}
%
\begin{scheme}
\bfseries\#!r6rs
\bfseries(import (rnrs base)
\bfseries        (rnrs io ports)
\bfseries        (rnrs programs))
\bfseries(put-bytes (standard-output-port)
\bfseries           (call-with-port
\bfseries               (open-file-input-port
\bfseries                 (cadr (command-line)))
\bfseries             get-bytes-all))%
\end{scheme}

%%% Local Variables:
%%% mode: latex
%%% TeX-master: "r6rs"
%%% End:
  \par
%\chapter{Requirement levels}
\chapter{Уровни требований}\vspace{5mm}
\label{requirementchapter}

%The key words ``must'', ``must not'', ``should'',
%``should not'', ``recommended'', ``may'', and ``optional'' in this
%report are to be interpreted as described in RFC~2119~\cite{mustard}.
%Specifically:
Ключевые слова, ``must'', ``must not'' ``should'', ``should not'', ``recommended'',
``may'', и ``optional'' в данной работе должны интерпретироваться как описано в RFC~2119~\cite
{mustard}. А именно:\vspace{1mm}

\begin{description}
%\item[must]\mainindex{must} This word means that a statement is an absolute
%  requirement of the specification.
\item[must]\mainindex{must} Это слово означает, что инструкция является абсолютным требованием
  спецификации.
%\item[must not]\mainindex{must not} This phrase means that a statement is an absolute
%  prohibition of the specification.
\item[must not]\mainindex{must not} Эта фраза означает, что инструкция является абсолютным запрещением
  спецификации.
%\item[should]\mainindex{should} This word, or the adjective ``recommended'', means that
%  valid reasons may exist in particular circumstances to ignore a
%  statement, but that the implications must be understood and weighed
%  before choosing a different course.
\item[should]\mainindex{should} Это слово, или прилагательное ``recommended'', означают, что
  в особых обстоятельствах могут существовать веские причины для игнорирования инструкции, но перед
  выбором иной линии поведения последствия должны быть осознаны и взвешены.
%\item[should not]\mainindex{should not} This phrase, or the phrase ``not recommended'', means
%  that valid reasons may exist in particular circumstances when the
%  behavior of a statement is acceptable, but that the implications
%  should be understood and weighed before choosing the course described
%  by the statement.
\item[should not]\mainindex{should not} Эта фраза, или фраза ``not recommended'', означают, что
  в особых обстоятельствах могут существовать веские причины для принятия функционирования
  инструкции, но перед выбором описываемой инструкцией линии поведения последствия
  должны быть осознаны и взвешены.
%\item[may]\mainindex{may} This word, or the adjective ``optional'', means that an item
%  is truly optional.
\item[may]\mainindex{may} Это слово, или прилагательное ``optional'' означают, что
  элемент является действительно дополнительным.
\end{description}\vspace{1mm}

%In particular, this report occasionally uses ``should'' to designate
%circumstances that are outside the specification of this report, but
%cannot be practically detected by an implementation; see
%section~\ref{argumentcheckingsection}.  In such circumstances, a
%particular implementation may allow the programmer to ignore the
%recommendation of the report and even exhibit reasonable behavior.
%However, as the report does not specify the behavior,
%these programs may be unportable, that is, their execution might
%produce different results on different implementations.
В частности, ``should'' в данной работе иногда используется для обозначения обстоятельств,
лежащих вне спецификации данной работы, но не обнаруживаемых реализацией на практике;
см. секцию~\ref{argumentcheckingsection}. При таких обстоятельствах конкретная реализация может
позволить программисту игнорировать рекомендацию в данной работе и даже демонстрировать
адекватное функционирование. Однако, поскольку данная работа не специфицирует функционирование,
такие программы могут быть непортируемыми, то есть, их выполнение может привести к различным
результатам в разных реализациях.

%\newpage

%Moreover, this report occasionally uses the phrase ``not required'' to note the
%absence of an absolute requirement.
Кроме того, иногда используемая в данной работе фраза ``not required'' означает отсутствие
абсолютного требования.

%%% Local Variables:
%%% mode: latex
%%% TeX-master: "r6rs"
%%% End:
 \par
%\chapter{Numbers}
\chapter{Числа}
\label{numbertypeschapter}
\mainindex{number}

%This chapter describes Scheme's model for numbers.  It is important to
%distinguish between the mathematical numbers, the Scheme objects that
%attempt to model them, the machine representations used to implement
%the numbers, and notations used to write numbers.  In this report, the
%term \textit{number} refers to a mathematical number, and the term
%\textit{number object} refers to a Scheme object representing a
%number.  This report uses the types \type{complex}, \type{real},
%\type{rational}, and \type{integer} to refer to both mathematical
%numbers and number objects.  The \type{fixnum} and \type{flonum} types
%refer to special subsets of the number objects, as determined by
%common machine representations, as explained below.
В данной главе описывается модель чисел Scheme. Важно различать математические числа,
представляющие их объекты Scheme, машинные представления, используемые для реализации чисел, и
используемые для записи чисел нотации. В данном стандарте термин \textit{число} относится к
математическому числу, а термин \textit{числовой объект} -- к объекту Scheme, представляющему
число. В данном стандарте используются типы \type{complex}, \type{real}, \type{rational} и
\type{integer} для обращения и к математическим числам, и к числовым объектам. Типы
\type{fixnum} и \type{flonum} относятся к специальным подмножествам числовых объектов, как
определяется общими машинными представлениями и объясняется ниже.

%\section{Numerical tower}
\section{Числовая башня}
\label{numericaltypes}
\index{numerical types}

%Numbers may be arranged into a tower of subsets in which each level
%is a subset of the level above it:
Числа могут быть систематизированы башней подмножеств, каждый уровень которой является
подмножеством вышележащего уровня:
\begin{tabbing}
\ \ \ \ \ \ \ \ \ \=\tupe{number} \\
\> \tupe{complex} \\
\> \tupe{real} \\
\> \tupe{rational} \\
\> \tupe{integer}
\end{tabbing}

%For example, 5 is an integer.  Therefore 5 is also a rational,
%a real, and a complex.  The same is true of the number objects
%that model 5.
Например, 5 является целым числом. Поэтому 5 является также и рациональным, и действительным, и
комплексным. Это справедливо и для числовых объектов, представляющих 5.

%Number objects are organized as a corresponding tower of subtypes
%defined by the predicates {\cf number?}, {\cf complex?}, {\cf real?},
%{\cf rational?}, and {\cf integer?}; see section~\ref{number?}.
%Integer number objects are also called \textit{integer
%  objects}\mainindex{integer object}.
Числовые объекты организованы в виде соответствующей башни подтипов, определяемых предикатами
{\cf\bfseries number?}, {\cf\bfseries complex?}, {\cf\bfseries real?}, {\cf\bfseries racional?},
и {\cf\bfseries integer?}; см. секцию~\ref{number?}. Объекты целых чисел также
называются \textit{целыми объектами}\mainindex{integer objects}.

%There is no simple relationship between the subset that contains a
%number and its representation inside a computer.  For example, the
%integer 5 may have several representations.  Scheme's numerical
%operations treat number objects as abstract data, as independent of
%their representation as possible.  Although an implementation of
%Scheme may use many different representations for numbers, this should
%not be apparent to a casual programmer writing simple programs.
Не существует очевидной взаимосвязи между подмножеством, содержащим число, и его представлением
в компьютере. Например, целое число 5 может иметь несколько представлений. Операции с числами в
Scheme интерпретируют числовые объекты как абстрактные данные, настолько независимые от их
представления, насколько это возможно. Хотя реализация Scheme может использовать множество
различных представлений чисел, это не должно быть очевидно программисту, изредка пишущему
простые программы.

%\section{Exactness}
\section{Точность}
\label{exactly}

%\mainindex{exactness}It is useful to distinguish between number objects
%that are known to correspond to a number exactly, and those number
%objects whose computation involved rounding or other errors.  For
%example, index operations into data structures may need to know the index
%exactly, as may some operations on polynomial coefficients in a symbolic algebra
%system.  On the other hand, the results of measurements are inherently
%inexact, and irrational numbers may be approximated by rational and
%therefore inexact approximations.  In order to catch uses of numbers
%known only inexactly where exact numbers are required, Scheme
%explicitly distinguishes \defining{exact} from \defining{inexact} number objects.  This
%distinction is orthogonal to the dimension of type.
\mainindex{exactness}Следует различать числовые объекты, гарантированно
соответствующие числу, и числовые объекты, вычисление которых повлекло за собой округление или
другие ошибки. Например, при операциях индексации структур данных может потребоваться знание точного
индекса, как и при некоторых операциях с коэффициентами полинома в системе символьной алгебры. С
другой стороны, результаты измерений являются неточными по определению, и иррациональные
числа могут быть апроксимированы рациональной, и поэтому неточной, апроксимацией. Для обнаружения
применения чисел, известных только приблизительно, там, где требуются точные числа,
Scheme явно различает \defining{точные} числовые объекты от \defining{неточных}. Это
различие не зависит от размера типа.\vspace{0.6mm}

%A
%number object is exact if it is the value of an exact numerical
%literal or was derived from exact number objects using only exact
%operations.  Exact number objects correspond to mathematical numbers
%in the obvious way.
Числовой объект является точным, если он является значением точного числового литерала или был
получен из точных числовых объектов с помощью только точных операций. Точные числовые объекты
соответствуют математическим числам явным образом.\vspace{0.6mm}

%Conversely, a number object is inexact if it is the value of an
%inexact numerical literal, or was derived from inexact number objects,
%or was derived using inexact operations.  Thus inexactness is
%contagious.
В свою очередь, числовой объект является неточным, если он является значением неточного
числового литерала, был получен из неточных числовых объектов или с помощью неточных
операций. Таким образом, неточность инфекционна.\vspace{0.6mm}

%Exact arithmetic is reliable in the following sense: If exact number
%objects are passed to any of the arithmetic procedures described in
%section~\ref{propagationsection}, and an exact number object is
%returned, then the result is mathematically correct.  This is
%generally not true of computations involving inexact number objects
%because approximate methods such as floating-point arithmetic may be
%used, but it is the duty of each implementation to make the result as
%close as practical to the mathematically ideal result.
Точная арифметика безопасна в следующем смысле: Если точные числовые объекты передаются любой из
арифметических процедур, описанных в секции~\ref{propagationsection}, и возвращается точный
числовой объект, результат математически корректен. Это в общем случае не верно для вычислений с
использованием неточных числовых объектов, так как могут применяться методы аппроксимации,
типа арифметики с плавающей точкой, но обязанностью каждой реализации является
выдача результата, наиболее близкого к математически идеальному.\vspace{0.6mm}

\section{Fixnum и flonum}\vspace{0.6mm}

%A \defining{fixnum} is an exact integer object that lies
%within a certain implementation-dependent subrange of the
%exact integer objects. (Library section \extref{lib:fixnumssection}{Fixnums} describes a
%library for computing with fixnums.)
%Likewise, every implementation must
%designate a subset of its inexact real number objects as \defining{flonum}s, and
%to convert certain external representations into flonums.
%(Library section \extref{lib:flonumssection}{Flonums} describes a library for
%computing with flonums.)  Note that
%this does not imply that an implementation must use
%floating-point representations.
\defining{fixnum} является точным целым объектом, принадлежащим конкретному, зависимому от
реализации поддиапазону точных целых объектов. (В секции \extref{lib:fixnumssection}{Fixnums}
описана библиотека для вычислений с fixnum.) Аналогично, каждая реализация должно обозначать
подмножество своих неточных вещественных числовых объектов как \defining{flonum} и
преобразовывать конкретные внешние представления в flonum. (В секции
\extref{lib:flonumssection}{Flonums} описана библиотека для вычислений с flonum.) Следует
отметить, что не подразумевается обязанность использования реализацией представлений с плавающей
точкой.\vspace{0.6mm}

%\section{Implementation requirements}
\section{Требования к реализациям}\vspace{0.6mm}

\index{implementation restriction}\label{restrictions}

%Implementations of Scheme must support number objects for
%the entire tower of subtypes given in section~\ref{numericaltypes}.
%Moreover, implementations must support exact integer
%objects and exact rational number objects of practically unlimited
%size and precision, and to implement certain procedures (listed in
%\ref{propagationsection}) so they always return exact results when
%given exact arguments.  (``Practically unlimited'' means that the size
%and precision of these numbers should only be limited by the size of
%the available memory.)
Реализации Scheme должны поддерживать числовые объекты для всей башни подтипов, приведённых в
секции~\ref{numericaltypes}. Кроме того, реализации должны поддерживать точные целые объекты
и точные рациональные числовые объекты практически неограниченного размера и точности, и так
реализовывать конкретные процедуры (перечисленные в \ref{propagationsection}), чтобы
они всегда возвращали точные результаты при предоставлении точных аргументов. (``Практически
неограниченный'' означает, что размер и точность таких чисел должны
ограничиваться только размером доступной памяти.)

%\newpage

%Implementations may support only a limited range of inexact number
%objects of any type, subject to the requirements of this section.  For
%example, an implementation may limit the range of the inexact real
%number objects (and therefore the range of inexact integer and
%rational number objects) to the dynamic range of the flonum format.
%Furthermore the gaps between the inexact integer objects and
%rationals are likely to be very large in such an implementation as the
%limits of this range are approached.
Реализации могут поддерживать только ограниченный диапазон неточных числовых объектов любого
типа, подчининяясь требованиям данной секции. Например, реализация может ограничить диапазон
неточных вещественных числовых объектов (и, следовательно, диапазон неточных целых и
рациональных числовых объектов) динамическим диапазоном формата flonum. При этом зазоры
между неточными целыми объектами и рациональными числами на границах этого диапазона в такой
реализации, очевидно, будут очень большими.

%An implementation may use floating point and other approximate
%representation strategies for \tupe{inexact} numbers.
%This report recommends, but does not require, that the IEEE
%floating-point standards be followed by implementations that use
%floating-point representations, and that implementations using
%other representations should match or exceed the precision achievable
%using these floating-point standards~\cite{IEEE}.
В реализации может применяться плавающая точка и другие приближённые способы представления
\tupe{неточных} чисел. В данном стандарте рекомендуется, но не требуется, чтобы реализации,
использующие представления с плавающей точкой, следовали стандартам плавающей точки IEEE, а
реализации, использующие другие представления, имели точность, соответствующую или превышающую
достижимую при применении стандартов с плавающей точкой~\cite{IEEE}.

%In particular, implementations that use floating-point representations
%must follow these rules: A floating-point result must be represented
%with at least as much precision as is used to express any of the
%inexact arguments to that operation.
%Potentially inexact operations such as {\cf sqrt}, when
%applied to exact arguments, should produce exact answers whenever possible
%(for example the square root of an exact 4 ought to be an exact 2).
%However, this is not required.
%If, on the other hand, an exact number object is operated upon so as to produce an
%inexact result (as by {\cf sqrt}), and if the result is represented in
%floating point, then the most precise floating-point format available
%must be used; but if the result is represented in some other way then
%the representation must have at least as much precision as the most
%precise floating-point format available.
В частности, реализации, использующие представления с плавающей точкой, должны соблюдать
следующие правила: результат с плавающей точкой должен представляться с точностью, не меньшей ,
чем используемая для выражения любого неточного аргумента в данной операции. При применении
потенциально неточных операций, таких, как {\cf\bfseries sqrt}, к точным аргументам, всегда,
когда это возможно, должны выдаваться точные ответы (например, квадратный корень точного 4
должен быть точным 2). Однако это не является требованием. С другой стороны, при операции,
применяемой к точному числовому объекту с целью получения неточного результата (как
{\cf\bfseries sqrt}) и представлении результата с плавающей точкой, должен использоваться самый
точный доступный формат с плавающей точкой; но если результат представлен неким иным способом,
представление должно иметь точность, не меньщую доступной для самого точного формата с
плавающей точкой.

%It is the programmer's responsibility to avoid using inexact number
%objects with magnitude or significand too large to be represented in
%the implementation.
Отказ от применения неточных числовых объектов со слишком большим для представления в
реализации значением или значащей частью является обязанностью программистов.\vspace{-2mm}

%\section{Infinities and NaNs}
\section{Бесконечность и не числа}

%Some Scheme implementations, specifically those that follow the IEEE
%floating-point standards, distinguish special number objects called
%\mainindex{infinity}\defining{positive infinity}, \defining{negative
%  infinity}, and \defining{NaN}.
Некоторые реализации Scheme, например, следующие стандартам IEEE плавающей
точки, различают специальные числовые объекты, называемые
\mainindex{infinity}\defining{положительная бесконечность}, \defining {отрицательная
  бесконечность} и \defining{не число}.

%Positive infinity is regarded as an inexact real (but not rational) number
%object that represents an indeterminate number greater than the
%numbers represented by all rational number objects.  Negative infinity
%is regarded as an inexact real (but not rational) number object that represents
%an indeterminate number less than the numbers represented by all
%rational numbers.
Положительная бесконечность рассматривается как неточный вещественный (но не рациональный)
числовой объект, представляющий неопределённое число, большее, чем числа, представляемые
всеми рациональными числовыми объектами. Отрицательная бесконечность рассматривается как
неточный вещественный (но не рациональный) числовой объект, представляющий неизвестное
число, меньшее, чем числа, представляемые всеми рациональными числами.

%A NaN is regarded as an inexact real (but not rational) number object so
%indeterminate that it might represent any real number, including
%positive or negative infinity, and might even be greater than positive
%infinity or less than negative infinity.
Не число рассматривается как неточный вещественный (но не рациональный) числовой объект, настолько
неопределённый, что он может представлять любое вещественное число, включая положительную или
отрицательную бесконечность, и даже может быть больше положительной бесконечности или
меньше отрицательной бесконечности.\vspace{-5mm}

%\section{Distinguished -0.0}
\section{Распознавание -0.0}\vspace{-2mm}

%\index{-0.0}
%Some Scheme implementations, specifically those that follow the IEEE
%floating-point standards, distinguish between number objects for $0.0$
%and $-0.0$, i.e., positive and negative inexact zero.  This report
%will sometimes specify the behavior of certain arithmetic operations
%on these number objects.  These specifications are marked with ``if
%$-0.0$ is distinguished'' or ``implementations that distinguish
%$-0.0$''.
\index{-0.0} Некоторые реализации Scheme, например, следующие стандартам IEEE
плавающей точки, распознают числовые объекты $0.0$ и $-0.0$, то есть, положительный и
отрицательный неточный ноль. В данном стандарте в ряде случаев специфицируется функционирование
конкретных арифметических операций с такими числовыми объектами. Такие спецификации помечаются
выражениями ``если $-0.0$ распознан'' или ``реализация, распознающая $-0.0$''.\vspace{-4mm}

%%% Local Variables:
%%% mode: latex
%%% TeX-master: "r6rs"
%%% End:
 \par
% Lexical structure
\hyphenation{white-space}
%%\vfill\eject
%\chapter{Lexical syntax and datum syntax}
\chapter{Лексический синтаксис и datum-синтаксис}
\label{readsyntaxchapter}

%The syntax of Scheme code is organized in three levels:
Синтаксис кода Scheme организован на трёх уровнях:\vspace{-3mm}
%
\begin{enumerate}
%\item the \textit{lexical syntax} that describes how a program text is split
%  into a sequence of lexemes,
\item \textit{Лексический синтаксис}, описывающий, как текст программы разбивается на
  последовательность лексем,
%\item the \textit{datum syntax}, formulated in terms of the lexical
%  syntax, that structures the lexeme sequence as a sequence of
%  \textit{syntactic data\mainindex{datum}\mainindex{syntactic
%      datum}}, where a syntactic datum is
%    a recursively structured entity,
\item \textit{Datum-синтаксис}, сформулированный в терминах лексического синтаксиса,
структурирующий последовательность лексем в виде последовательности \textit{синтаксических
  данных\mainindex{datum}\mainindex{syntactic datum}}, где
синтаксический datum является рекурсивно структурированным элементом,
%\item the \textit{program syntax} formulated in terms of the read
%  syntax, imposing further structure and assigning meaning to
%  syntactic data.
\item \textit{Программный синтаксис}, сформулированный в терминах синтаксиса считывания, задающий
  дальнейшую структуру и наполняющий смысловым содержанием синтаксические данные.
\end{enumerate}\vspace{-3mm}
%
%Syntactic data (also called \textit{external
%  representations\index{external representation}}) double
%as a notation for objects, and Scheme's \rsixlibrary{io ports} library
%(library section~\extref{lib:portsiosection}{Port I/O})
%provides the {\cf get-datum} and {\cf put-datum} procedures
%for reading and writing syntactic data, converting between their
%textual representation and the corresponding objects.
%Each syntactic datum represents a corresponding \defining{datum value}.
%A syntactic datum can be used in a program to obtain the corresponding
%datum value using {\cf quote} (see section~\ref{quote}).
Синтаксические данные (называемые также \textit{внешними представлениями\index{external
    representation}}) одновременно служат как формой записи объектов, так и библиотекой Scheme
{\bfseries\rsixlibrary{io ports}} (секция библиотек~\extref{lib:portsiosection}{Port I/O}),
предоставляющей процедуры {\cf\bfseries get-datum} и {\cf\bfseries put-datum} для чтения и
записи синтаксических данных, преобразовывая их из текстового представления в
соответствующие объекты, и наоборот. Каждый синтаксический datum представляет соответствующее
\defining{datum-значение}. Синтаксический datum может использоваться в программе для
получения соответствующего datum-значения с помощью {\cf\bfseries quote}
(см. секцию~\ref{quote}).

%Scheme source code consists of syntactic data and (non-significant) comments.
%Syntactic data in Scheme source code are called
%\textit{forms}\mainindex{form}.
%(A form nested inside another form is
%called a \defining{subform}.)
%Consequently, Scheme's syntax has the property that any sequence of
%characters that is a form is also a syntactic datum representing
%some object.  This can lead to confusion, since it may not be obvious
%out of context whether a given sequence of characters is intended to
%be a representation of objects or the text of a program.
%It is also a source of power, since it
%facilitates writing programs such as interpreters or compilers that
%treat programs as objects (or vice versa).
Исходный текст Scheme состоит из синтаксических данных и (незначащих)
комментариев. Синтаксические данные в исходном тексте Scheme называются
\textit{формами}\mainindex{form}. (Форма, вложенная в другую форму, называется
\defining{подформой}.) Следовательно, синтаксис Scheme обладает таким свойством, что любая
последовательность символов, являющаяся формой, является также и синтаксическим datum,
представляющим некоторый объект. Это может привести к путанице, так как из контекста может
быть не ясно, предназначена ли данная последовательность знаков для представления объектов или
текста программы. Это также и мощное вычислительное средство, так как облегчает написание
программ, таких, как интерпретаторы или компилятороы, интерпретирующих программы как объекты
(или наоборот).

%A datum value may have several different external representations.
%For example, both ``{\tt \#e28.000}'' and
%``{\tt\#x1c}'' are syntactic data representing the exact integer
%object 28, and the syntactic data ``{\tt(8 13)}'', ``{\tt( 08 13 )}'', ``{\tt(8 .\
%  (13 .\ ()))}''
%all represent a list containing the exact integer objects 8 and 13.
%Syntactic data that represent equal objects (in the sense of {\cf
%  equal?}; see section~\ref{equal?}) are always equivalent
%as forms of a program.
Datum-значение может иметь несколько различных внешних представлений. Например, и ``{\tt\bfseries
  \#e28.000}'', и ``{\tt\bfseries\#x1c}'' являются синтаксическими данными, представляющими
точный целый объект 28, а синтаксические данные ``{\tt\bfseries(8 13)}'', ``{\tt\bfseries( 08 13
  )}'', ``{\tt\bfseries(8 .\ (13 .\ ()))}'' представляют список, содержащий точные целые
объекты 8 и 13. Синтаксические данные, представляющие равные объекты (в смысле {\cf\bfseries
  equal?}; см. секцию~\ref{equal?}), всегда эквивалентны как формы программы.

%Because of the close correspondence between syntactic data and datum
%values, this report sometimes uses the term \defining{datum} for
%either a syntactic datum or a datum value when the exact meaning
%is apparent from the context.
Вследствие близкого соответствия синтаксических данных и datum-значений в данном стандарте термин
\defining{datum} иногда используется и для синтаксического datum, и для datum-значения, если
точный смысл очевиден из контекста.

%An implementation must not extend the lexical or datum syntax in
%any way, with one exception: it need not treat the syntax
%{\cf \sharpsign{}!\meta{identifier}}, for any \meta{identifier} (see
%section~\ref{identifiersection}) that is not {\cf r6rs}, as a syntax
%violation, and it may use specific {\cf \sharpsign{}!}-prefixed
%identifiers as flags indicating that subsequent input contains extensions
%to the standard lexical or datum syntax.
%The syntax {\cf \sharpsign{}!r6rs} may be used to signify that
%the input afterward is written with the lexical syntax and
%datum syntax described by
%this report.
%{\cf \sharpsign{}!r6rs} is otherwise treated as a comment; see section~\ref{whitespaceandcomments}.
Реализации запрещено расширять лексический или datum-синтаксис, за одним исключением: синтаксис
{\cf{\bfseries \sharpsign{}!}\meta{identifier}} при любом \meta{identifier} (см. секцию~\ref
{identifiersection}), отличном от {\cf\bfseries r6rs}, не должен считаться синтаксическим
нарушением, а специфические идентификаторы с префиксом {\cf\bfseries \sharpsign {}!} могут
использоваться в качестве флагов, указывающих на наличие в последующем вводе расширений
лексического стандарта или datum-синтаксиса. Синтаксис {\cf \sharpsign \bfseries{}!r6rs} может
применяться для указания, что последующий ввод записан с лексическим и datum-синтаксисом,
описанным в данном стандарте. В противном случае {\cf \sharpsign\bfseries {}!r6rs} считается
комментарием; см. секцию~\ref{whitespaceandcomments}.\vspace{-1mm}

%\section{Notation}
\section{Нотация}
\label{BNF}

%The formal syntax for Scheme is written in an extended BNF.
%Non-terminals are written using angle brackets.  Case is insignificant
%for non-terminal names.
Формальный синтаксис Scheme записан в расширенной BNF. Нетерминальные символы заключены в
угловые скобки. Для нетерминальных имён регистр не имеет значения.

%All spaces in the grammar are for legibility.
%\meta{Empty} stands for the empty string.
Все пробелы в грамматике применяются для удобочитаемости. \meta{Empty} обозначает пустую строку.

%The following extensions to BNF are used to make the description more
%concise:  \arbno{\meta{thing}} means zero or more occurrences of
%\meta{thing}, and \atleastone{\meta{thing}} means at least one
%\meta{thing}.
Для более лаконичного описания используются следующие расширения BNF: \arbno{\meta{thing}} означает
ноль или более вхождений \meta{thing}, а \atleastone{\meta{thing}} означает не менее
одного \meta{thing}.

%Some non-terminal names refer to the Unicode scalar values of the same
%name: \meta{character tabulation} (U+0009), \meta{linefeed} (U+000A),
%\meta{carriage return} (U+000D), \meta{line tabulation} (U+000B),
%\meta{form feed} (U+000C), \meta{carriage return} (U+000D),
%\meta{space} (U+0020), \meta{next line} (U+0085), \meta{line
%  separator} (U+2028), and \meta{paragraph separator} (U+2029).
Некоторые нетерминальные имена относятся к одноимённым скалярным значениям Unicode:
\meta{character tabulation} (U+0009), \meta{linefeed} (U+000A), \meta{carriage return} (U+000D),
\meta{line tabulation} (U+000B), \meta{form feed} (U+000C), \meta{carriage return} (U+000D),
\meta{space (U+0020)}, \meta{next line} (U+0085), \meta{line separator} (U+2028) и
\meta{paragraph separator} (U+2029).\vspace{-1mm}

%\section{Lexical syntax}
\section{Лексический синтаксис}
\label{lexicalsyntaxsection}

%The lexical syntax determines how a character sequence is split into a
%sequence of lexemes\index{lexeme}, omitting non-significant portions
%such as comments and whitespace.  The character sequence is assumed to
%be text according to the Unicode standard~\cite{Unicode}.  Some of
%the lexemes, such as identifiers, representations of number objects, strings etc., of the lexical
%syntax are syntactic data in the datum syntax, and thus represent objects.
%Besides the formal account of the syntax, this section also describes
%what datum values are represented by these syntactic data.
Лексический синтаксис определяет, как символьная последовательность разбивается на
последовательность лексем\index{lexeme} с пропуском незначащих частей, таких, как комментарии и
пробелы. Полагается, что символьная последовательность является текстом согласно стандарту
Unicode~\cite{Unicode}. Некоторые лексемы, такие, как идентификаторы, представления числовых
объектов, строки и т.д. лексического синтаксиса являются синтаксическими данными в
datum-синтаксисе и, таким образом, представляют объекты. Помимо формального описания
синтаксиса, в этой секции также описывается, какие datum-значения представляют эти синтаксические
данные.

%The lexical syntax, in the description of comments, contains
%a forward reference to \meta{datum}, which is described as part of the
%datum syntax.  Being comments, however, these \meta{datum}s do not play
%a significant role in the syntax.
Лексический синтаксис в описании комментариев содержит прямое обращение к \meta{datum},
описанное как часть datum-синтаксиса. Будучи комментариями, однако,
такие \meta{datum} не играют существенную роль в синтаксисе.

%Case is significant except in representations of booleans, number objects, and
%in hexadecimal numbers specifying Unicode scalar values.  For example, {\cf \#x1A}
%and {\cf \#X1a} are equivalent.  The identifier {\cf Foo} is, however,
%distinct from the identifier {\cf FOO}.
Регистр является значащим, за исключением представлений булевых выражений, числовых объектов и
шестнадцатеричных чисел, определяющих скалярные значения Unicode. Например, {\cf\bfseries \#x1A}
и {\cf\bfseries \#X1a} эквивалентны. А вот идентификатор {\cf\bfseries Foo} отличается от
идентификатора {\cf\bfseries FOO}.

%\subsection{Formal account}
\subsection{Формальное описание}
\label{lexicalgrammarsection}

%\meta{Interlexeme space} may occur on either side of any lexeme, but not
%within a lexeme.
\meta{Interlexeme space} может находиться с любой стороны лексемы, но не внутри лексемы

%\hyper{Identifier}s, {\cf .}, \hyper{number}s, \hyper{character}s, and
%\hyper{boolean}s, must be terminated by a \meta{delimiter} or by the
%end of the input.
\hyper{Identifier}, {\cf .}, \hyper{number}, \hyper{character} и
\hyper{boolean} могут заканчиваться \meta{delimiter} или концом ввода.

%The following two characters are reserved for future extensions to the
%language: {\tt \verb"{" \verb"}"}
Следующие два знака зарезервированы для будущих расширений языка:
{\tt\bfseries \verb"{" \verb"}"}

{%
\renewcommand{\baselinestretch}{1.08}
\selectfont
\begin{grammar}%
\meta{lexeme} \: \meta{identifier} \| \meta{boolean} \| \meta{number}\index{identifier}
\>  \| \meta{character} \| \meta{string}
\>  \| \textbf{(} \| \textbf{)} \| \textbf{\openbracket{}} \| \textbf{\closedbracket{}} \| %
\textbf{\sharpsign(} \| \textbf{\sharpsign{}vu8(} | \textbf{\singlequote{}} \| %
\textbf{\backquote{}} \| \textbf{,} \| \textbf{,@} \| {\bf\bfseries.}
\>  \| \textbf{\sharpsign\singlequote{}} \| \textbf{\sharpsign\backquote{}} \| %
\textbf{\sharpsign,} \| \textbf{\sharpsign,@}
\meta{delimiter} \: \textbf{(} \| \textbf{)} \| \textbf{\openbracket{}} \| %
\textbf{\closedbracket{}} \| \textbf{"} \| \textbf{;} \| \textbf{\sharpsign{}}
\>  \| \meta{whitespace}
\meta{whitespace} \: \meta{character tabulation}
\> \| \meta{linefeed} \| \meta{line tabulation} \| \meta{form feed}
\> \| \meta{carriage return} \| \meta{next line}
\> \| \meta{any character whose category is Zs, Zl, or Zp}
\meta{line ending} \: \meta{linefeed} \| \meta{carriage return}
\> \| \meta{carriage return} \meta{linefeed} \| \meta{next line}
\> \| \meta{carriage return} \meta{next line} \| \meta{line separator}
\meta{comment} \: ; \= $\langle$\rm all subsequent characters up to a
                    \>\ \rm \meta{line ending} or \meta{paragraph separator}$\rangle$\index{comment}
\qquad \= \| \meta{nested comment}
\> \| \textbf{\#}; \meta{interlexeme space} \meta{datum}
\> \| \textbf{\#!r6rs}
\meta{nested comment} \: \textbf{\#|} \= \meta{comment text}
\> \arbno{\meta{comment cont}} \textbf{|\#}
\meta{comment text} \: \= $\langle$\rm character sequence not containing
\>\ \rm {\tt\bfseries \#|} or {\tt\bfseries |\#}$\rangle$
\meta{comment cont} \: \meta{nested comment} \meta{comment text}
\meta{atmosphere} \: \meta{whitespace} \| \meta{comment}
\meta{interlexeme space} \: \arbno{\meta{atmosphere}}%
\end{grammar}

\label{extendedalphas}
\label{identifiersyntax}

% This is a kludge, but \multicolumn doesn't work in tabbing environments.
\setbox0\hbox{\cf\meta{variable} \goesto{} $\langle$}

\begin{grammar}%
\meta{identifier} \: \meta{initial} \arbno{\meta{subsequent}}
 \>  \| \meta{peculiar identifier}
\meta{initial} \: \meta{constituent} \| \meta{special initial}
 \> \| \meta{inline hex escape}
\meta{letter} \:  \textbf{a} \| \textbf{b} \| \textbf{c} \| ... \| \textbf{z}
\> \| \textbf{A} \| \textbf{B} \| \textbf{C} \| ... \| \textbf{Z}
\meta{constituent} \: \meta{letter}
 \> \| $\langle${\rm any character whose Unicode scalar value is greater than}
 \> \quad {\rm 127, and whose category is Lu, Ll, Lt, Lm, Lo, Mn,}
 \> \quad {\rm Nl, No, Pd, Pc, Po, Sc, Sm, Sk, So, or Co}$\rangle$
%\end{grammar}
%
%\newpage
%
%\begin{grammar}
\meta{special initial} \: \textbf{!} \| \textbf{\$} \| \textbf{\%} \| {\bfseries\verb"&"} \| %
\textbf{*} \| \textbf{/} \| \textbf{:} \| \textbf{<} \| \textbf{=}
 \>  \| \textbf{>} \| \textbf{?} \| {\bfseries\verb"^"} \| {\bfseries\verb"_"} \| {\bfseries\verb"~"}
\meta{subsequent} \: \meta{initial} \| \meta{digit}
 \>  \| \meta{any character whose category is Nd, Mc, or Me}
 \>  \| \meta{special subsequent}
\meta{digit} \: \textbf{0} \| \textbf{1} \| \textbf{2} \| \textbf{3} \| \textbf{4} \| \textbf{5} %
\| \textbf{6} \| \textbf{7} \| \textbf{8} \| \textbf{9}
\meta{hex digit} \: \meta{digit}
 \> \| \textbf{a} \| \textbf{A} \| \textbf{b} \| \textbf{B} \| \textbf{c} \| \textbf{C} \| %
 \textbf{d} \| \textbf{D} \| \textbf{e} \| \textbf{E} \| \textbf{f} \| \textbf{F}
\meta{special subsequent} \: \textbf{+} \| \textbf{-} \| \textbf{.}\ \| \textbf{@}
\meta{inline hex escape} \: {\bfseries\backwhack{}x}\meta{hex scalar value};
\meta{hex scalar value} \: \atleastone{\meta{hex digit}}
\meta{peculiar identifier} \: \textbf{+} \| \textbf{-} \| ... \| \textbf{->} \arbno{\meta{subsequent}}
\meta{boolean} \: {\bfseries\schtrue{}} \| \bfseries{\#T} \| {\bfseries\schfalse{}} \| {\bfseries\#F}
\meta{character} \: {\bfseries\#\backwhack{}}\meta{any character}
 \>  \| {\bfseries\#\backwhack{}}\meta{character name}
 \>  \| {\bfseries\#\backwhack{}x}\meta{hex scalar value}
\meta{character name} \: \textbf{nul} \| \textbf{alarm} \| \textbf{backspace} \| \textbf{tab}
\> \| \textbf{linefeed} \| \textbf{newline} \| \textbf{vtab} \| \textbf{page} \| \textbf{return}
\> \| \textbf{esc} \| \textbf{space} \| \textbf{delete}
\meta{string} \: " \arbno{\meta{string element}} "
\meta{string element} \: \meta{any character other than {\bfseries\doublequote{}} or {\bfseries\backwhack}}
 \> \| {\bfseries\backwhack{}a} \| {\bfseries\backwhack{}b} \| {\bfseries\backwhack{}t} \| %
 {\bfseries\backwhack{}n} \| {\bfseries\backwhack{}v} \| {\bfseries\backwhack{}f} \| {\bfseries\backwhack{}r}
 \>  \| {\bfseries\backwhack\doublequote{}} \| {\bfseries\backwhack\backwhack}
 \>  \| {\bfseries\backwhack}\meta{intraline whitespace}\meta{line ending}
 \>  \hspace*{4em}\meta{intraline whitespace}
 \>  \| \meta{inline hex escape}
\meta{intraline whitespace} \: \meta{character tabulation}
\> \| \meta{any character whose category is Zs}%
\end{grammar}

}

%A \meta{hex scalar value} represents a Unicode scalar value
%between 0 and \sharpsign{}x10FFFF, excluding the range
%$\left[\sharpsign{}x\textrm{D800}, \sharpsign{}x\textrm{DFFF}\right]$.
\meta{hex scalar value} представляет собой скалярное значение Unicode
между 0 и \sharpsign{}x10FFFF, за исключением диапазона
$\left[\sharpsign{}x\textrm{D800}, \sharpsign{}x\textrm{DFFF}\right]$.

\label{numbersyntax}%
%The rules for \meta{num $R$}, \meta{complex $R$}, \meta{real
%$R$}, \meta{ureal $R$}, \meta{uinteger $R$}, and \meta{prefix $R$} below
%should be replicated for \hbox{$R = 2, 8, 10,$}
%and $16$.  There are no rules for \meta{decimal $2$}, \meta{decimal
%$8$}, and \meta{decimal $16$}, which means that number representations containing
%decimal points or exponents must be in decimal radix.
Правила для \meta{num $R$}, \meta{complex $R$}, \meta{real $R$}, \meta{ureal $R$},
\meta{uinteger $R$} и \meta{prefix $R$} ниже должны быть дублированы для \hbox{$R = 2, 8, 10$}
и $16$.  Не существует правил для \meta{decimal $2$}, \meta{decimal $8$} и \meta{decimal $16$},
что означает, что представления чисел, содержащих десятичные точки или экспоненты
должны быть с десятичным основанием.

\begin{grammar}%
\meta{number} \: \meta{num $2$} \| \meta{num $8$}
   \>  \| \meta{num $10$} \| \meta{num $16$}
\meta{num $R$} \: \meta{prefix $R$} \meta{complex $R$}
\meta{complex $R$} \: %
         \meta{real $R$} %
      \| \meta{real $R$} \textbf{@} \meta{real $R$}
   \> \| \meta{real $R$} \textbf{+} \meta{ureal $R$} \textbf{i} %
      \| \meta{real $R$} \textbf{-} \meta{ureal $R$} \textbf{i}
   \> \| \meta{real $R$} \textbf{+} \meta{naninf} \textbf{i} %
      \| \meta{real $R$} \textbf{-} \meta{naninf} \textbf{i}
   \> \| \meta{real $R$} \textbf{+ i} %
      \| \meta{real $R$} \textbf{- i}
   \> \| \textbf{+} \meta{ureal $R$} \textbf{i} %
      \| \textbf{-} \meta{ureal $R$} \textbf{i}
   \> \| \textbf{+} \meta{naninf} \textbf{i} %
      \| \textbf{-} \meta{naninf} \textbf{i}
   \> \| \textbf{+ i} %
      \| \textbf{- i}
\meta{real $R$} \: \meta{sign} \meta{ureal $R$}
  \> \| \textbf{+} \meta{naninf} \| \textbf{-} \meta{naninf}
\meta{naninf} \: \textbf{nan.0} \| \textbf{inf.0}
\meta{ureal $R$} \: %
         \meta{uinteger $R$}
   \> \| \meta{uinteger $R$} / \meta{uinteger $R$}
   \> \| \meta{decimal $R$} \meta{mantissa width}
\meta{decimal $10$} \: %
         \meta{uinteger $10$} \meta{suffix}
   \> \| \textbf{.} \atleastone{\meta{digit $10$}} \meta{suffix}
   \> \| \atleastone{\meta{digit $10$}} \textbf{.} \arbno{\meta{digit $10$}} \meta{suffix}
   \> \| \atleastone{\meta{digit $10$}} \textbf{.} \meta{suffix}
\meta{uinteger $R$} \: \atleastone{\meta{digit $R$}}
\meta{prefix $R$} \: %
         \meta{radix $R$} \meta{exactness}
   \> \| \meta{exactness} \meta{radix $R$}
\end{grammar}

\begin{grammar}%
\meta{suffix} \: \meta{empty}
   \> \| \meta{exponent marker} \meta{sign} \atleastone{\meta{digit $10$}}
\meta{exponent marker} \: \textbf{e} \| \textbf{E} \| \textbf{s} \| \textbf{S} \| \textbf{f} \| \textbf{F}
   \> \| \textbf{d} \| \textbf{D} \| \textbf{l} \| \textbf{L}
\meta{mantissa width} \: \meta{empty}
   \> \| \textbf{|} \atleastone{\meta{digit 10}}
\meta{sign} \: \meta{empty}  \| \textbf{+} \|  \textbf{-}
\meta{exactness} \: \meta{empty}
   \> \| {\bfseries\#i\sharpindex{i}} \| {\bfseries\#I} \| {\bfseries\#e\sharpindex{e}} \| {\bfseries\#E}
\meta{radix 2} \: {\bfseries\#b\sharpindex{b}} \| {\bfseries\#B}
\meta{radix 8} \: {\bfseries\#o\sharpindex{o}} \| {\bfseries\#O}
\meta{radix 10} \: \meta{empty} \| {\bfseries\#d} \| {\bfseries\#D}
\meta{radix 16} \: {\bfseries\#x\sharpindex{x}} \| {\bfseries\#X}
\meta{digit 2} \: \textbf{0} \| \textbf{1}
\meta{digit 8} \: \textbf{0} \| \textbf{1} \| \textbf{2} \| \textbf{3} \| \textbf{4} \| %
\textbf{5} \| \textbf{6} \| \textbf{7}
\meta{digit 10} \: \meta{digit}
\meta{digit 16} \: \meta{hex digit}
\end{grammar}

%\subsection{Line endings}
\subsection{Окончания строк}
\label{lineendings}

%Line endings are significant in Scheme in single-line comments (see
%section~\ref{whitespaceandcomments}) and within string literals.  In
%Scheme source code, any of the line endings in \meta{line ending}
%marks the end of a line.  Moreover, the two-character line endings
%\meta{carriage return} \meta{linefeed} and \meta{carriage return}
%\meta{next line} each count as a single line ending.
В Scheme окончания строк являются значащими в однострочных комментариях
(см. секцию~\ref{whitespaceandcomments}) и внутри строковых литералов. В исходном тексте Scheme
любые окончания строк в \meta{line ending} означают конец строки. Кроме того, двухсимвольные
окончания строк \meta{carriage return} \meta{linefeed} и \meta{carriage return} \meta{next
  line} считаются одним окончанием строки.

%In a string literal, a \hyper{line ending} not preceded by a {\cf\backwhack}
%stands for a linefeed character, which is the standard line-ending
%character of Scheme.
В строковом литерале \hyper{line ending} без {\cf\backwhack} перед ним обозначает
символ перевода строки -- стандартный символ конца строки Scheme.

%\subsection{Whitespace and comments}
\subsection{Пробельные символы и комментарии}
\label{whitespaceandcomments}

%\defining{Whitespace} characters are spaces, linefeeds,
%carriage returns, character tabulations, form feeds, line tabulations,
%and any other character whose category is Zs, Zl, or Zp.
%Whitespace is used for improved readability and
%as necessary to separate lexemes from each other.  Whitespace may
%occur between any two lexemes,
%but not within a lexeme.  Whitespace may also occur inside a string,
%where it is significant.
\defining{Пробельными} символами являются пробелы, обратные переводы строк, переводы каретки, символы
табуляции, переводы страницы, линейные табуляции и любой другой символ, категорией которого является Zs,
Zl или Zp. Пробельные символы используются для улучшения читаемости и при необходимости
отделения лексем друг от друга. Пробельные символы могут находиться между любыми двумя
лексемами, но не внутри лексемы. Пробельный символ может также находиться в строке, где он
является значащим.

%The lexical syntax includes several comment forms. In all cases,
%comments are invisible to Scheme, except that they act as delimiters,
%so, for example, a comment cannot appear in the middle of an
%identifier or representation of a number object.
Лексический синтаксис включает несколько форм комментариев. В Sheme комментарии всегда
невидимы, за исключением того, что они действуют как разделители, так, например, комментарий не
может находиться в середине идентификатора или представления числового объекта.

%\newpage

%A semicolon ({\tt;}) indicates the start of a line
%comment.\mainindex{comment}\mainschindex{;} The comment continues to
%the end of the line on which the semicolon appears.
Точка с запятой ({\tt;}) задаёт начало строки комментария.\mainindex{comment}\mainschindex{;}
Комментарий продолжается до конца строки, на которой находится точка с запятой.\vspace{1mm}

%Another way to indicate a comment is to prefix a \hyper{datum}
%(cf.\ section~\ref{datumsyntax}) with {\tt \#;}\sharpindex{;}, possibly with
%\meta{interlexeme space} before the \hyper{datum}.  The comment consists of
%the comment prefix {\tt \#;} and the \hyper{datum} together.  This
%notation is useful for ``commenting out'' sections of code.
Другим способом задания комментария является добавление к \hyper{datum} (см.\ секцию~\ref
{datumsyntax}) префикса {\tt\bfseries \#;}\sharpindex{;}, возможно, с \meta{interlexeme space} перед
\hyper{datum}. Комментарий состоит из приставки комментария {\tt\bfseries \#;} и \hyper{datum}.
Такая нотация применяется для ``комментирования'' секций кода.\vspace{1mm}

%Block comments may be indicated with properly nested {\tt
%  \#|}\index{#"|@\texttt{\sharpsign\verticalbar}}\index{"|#@\texttt{\verticalbar\sharpsign}}
%and {\tt |\#} pairs.
Блоковые комментарии могут задаваться согласовано вложенными парами {\tt\bfseries \#|}
\index{#"|'@\texttt{\sharpsign\verticalbar}}\index{"|#@\texttt{\verticalbar\sharpsign}} и {\tt\bfseries |\#}.\vspace{1mm}

\begin{scheme}
\bfseries\#|
\bfseries   The FACT procedure computes the factorial
\bfseries   of a non-negative integer.
\bfseries|\#
\bfseries(define fact
\bfseries  (lambda (n)
\bfseries    ;; base case
\bfseries    (if (= n 0)
\bfseries        \#;(= n 1)
\bfseries        1       ; identity of *
\bfseries        (* n (fact (- n 1))))))%
\end{scheme}\vspace{1mm}

%The lexeme {\cf \sharpsign{}!r6rs}, which signifies that the program text
%that follows is written with the lexical and datum syntax described in this
%report, is also otherwise treated as a comment.
Лексема {\cf\bfseries \sharpsign{}!r6rs}, означающая, что текст следующей далее программы
записан с лексическим и datum-синтаксисом, описанным в данном стандарте, в других случаях
также интерпретируется как комментарий.\vspace{1mm}

%\subsection{Identifiers}
\subsection{Идентификаторы}\vspace{1mm}
\label{identifiersection}

%Most identifiers\mainindex{identifier} allowed by other programming
%languages are also acceptable to Scheme.  In general,
%a sequence of letters, digits, and ``extended alphabetic
%characters'' is
%an identifier when it begins with a character that cannot begin a
%representation of a number object.
%In addition, \ide{+}, \ide{-}, and \ide{...} are identifiers, as is
%a sequence of letters, digits, and extended alphabetic
%characters that begins with the two-character sequence \ide{->}.
%Here are some examples of identifiers:
Большинство идентификаторов\mainindex{identifier}, допустимых в других языках программирования,
также допустимы и в Scheme. В общем случае последовательность букв, цифр и ``расширенных алфавитных
символов'' является идентификатором, если она начинается с символа, с которого не может начинаться
представление числового объекта. Кроме того, идентификаторами являются \ide{+}, \ide{-} и \ide{...},
равно как и последовательность букв, цифр и расширенных алфавитных символов, начинающаяся
с двухсимвольной последовательности \ide{->}. Несколько примеров идентификаторов:\vspace{1mm}

\begin{scheme}
\bfseries lambda         q                soup
\bfseries list->vector   {+}                V17a
\bfseries <=             a34kTMNs         ->-
\bfseries the-word-recursion-has-many-meanings%
\end{scheme}\vspace{1mm}

%Extended alphabetic characters may be used within identifiers as if
%they were letters.  The following are extended alphabetic characters:
Расширенные алфавитные символы могут использоваться внутри идентификаторов, как если бы они
были буквами. Расширенными алфавитными символами являются:\vspace{1mm}

\begin{scheme}
\bfseries !\ \$ \% \verb"&" * + - . / :\ < = > ? @ \verb"^" \verb"_" \verb"~" %
\end{scheme}\vspace{1mm}

%Moreover, all characters whose Unicode scalar values are greater than 127 and
%whose Unicode category is Lu, Ll, Lt, Lm, Lo, Mn, Mc, Me, Nd, Nl, No, Pd,
%Pc, Po, Sc, Sm, Sk, So, or Co can be used within identifiers.
%In addition, any character can be used within an identifier
%when specified via an \meta{inline hex escape}.  For example, the
%identifier \verb|H\x65;llo| is the same as the identifier
%\verb|Hello|, and the identifier \verb|\x3BB;| is the same as the
%identifier $\lambda$.
Кроме того, в идентификаторах могут использоваться все символы, скалярные значения Unicode
которых больше 127, и с категорией Unicode Lu, ~Ll, ~Lt, ~Lm, ~Lo, ~Mn, ~Mc, ~Me, ~Nd, ~Nl, ~No, ~Pd, ~Pc,
~Po, ~Sc, ~Sm, ~Sk, ~So или ~Co. Кроме того, в идентификаторе может использоваться любой символ, если
он специфицирован посредством \meta{inline hex escape}. Например, идентификаторы
{\bfseries\verb|H\x65;llo|} и {\bfseries\verb|Hello|} аналогичны, также аналогичны
идентификаторы {\bfseries\verb|\x3BB;|} и {\bfseries
  $\lambda$}.~~~~~~~~~~ ~~~~~~~~~~ ~~~~~~~~~~ ~~~~~~~~~~ ~~~~~~~~~~ ~~~~~~~~~~ ~~~~~~~~~~
~~~~~~~~~~ ~~~~~~~~~~ ~~~~~~~~~~ ~~~~~~~~~~ ~~~~~~~~~~ ~~~~~~~~~~ ~~~~~~~~~~ ~~~~~~~~~~
~~~~~~~~~~

%Any identifier may be used as a variable\index{variable} or as a
%syntactic keyword\index{syntactic keyword} (see
%sections~\ref{variablesection} and~\ref{macrosection}) in a Scheme
%program.
%Any identifier may also be used as a syntactic datum, in which case it
%represents a \textit{symbol}\index{symbol} (see section~\ref{symbolsection}).
В программе Scheme любой идентификатор может использоваться как переменная\index{variable} или
как синтаксическое ключевое слово\index{syntactic keyword} (см. секции~\ref{variablesection}
и~\ref{macrosection}). Любой идентификатор может также использоваться как синтаксический datum,
в этом случае он представляет собой \textit{символ}\index{symbol} (см. секцию~\ref{symbolsection}).

\subsection{Булевы значения}

%The standard boolean objects for true and false have external representations
%\schtrue{} and \schfalse.\sharpindex{t}\sharpindex{f}
Стандартные булевые объекты для true и false имеют внешние представления {\bfseries\schtrue{}} и
{\bfseries\schfalse}.\sharpindex{t}\sharpindex{f}

\subsection{Символы}

%Characters are represented using the notation
%\sharpsign\backwhack\hyper{character}\index{#\@\texttt{\sharpsign\backwhack}} or
%\sharpsign\backwhack\hyper{character name} or
%\sharpsign\backwhack{}x\meta{hex scalar value}.
Символы представляются с помощью нотации
{\bfseries\sharpsign\backwhack}\hyper{character}\index{#\@\texttt{\sharpsign\backwhack}} или
{\bfseries\sharpsign\backwhack}\hyper{character name} или
{\bfseries\sharpsign\backwhack{}x}\meta{hex scalar value}.

%For example:
Например:

\texonly
\newcommand{\extab}{\>}
{%
\renewcommand{\baselinestretch}{1.05}
\selectfont
\begin{tabbing}
{\cf\#\backwhack{}x0000000000}\=\kill
\endtexonly
\htmlonly
\newcommand{\extab}{&}
\begin{tabular}{ll}
\endhtmlonly
{\cf\bfseries\#\backwhack{}a}          \extab \textrm{lower case letter a}\\
{\cf\bfseries\#\backwhack{}A}          \extab \textrm{upper case letter A}\\
{\cf\bfseries\#\backwhack{}(}          \extab \textrm{left parenthesis}\\
{\cf\bfseries\#\backwhack{}}           \extab \textrm{space character}\\
{\cf\bfseries\#\backwhack{}nul}        \extab \textrm{U+0000}\\
{\cf\bfseries\#\backwhack{}alarm}      \extab \textrm{U+0007}\\
{\cf\bfseries\#\backwhack{}backspace}  \extab \textrm{U+0008}\\
{\cf\bfseries\#\backwhack{}tab}        \extab \textrm{U+0009}\\
{\cf\bfseries\#\backwhack{}linefeed}   \extab \textrm{U+000A}\\
{\cf\bfseries\#\backwhack{}newline}   \extab \textrm{U+000A}\\
{\cf\bfseries\#\backwhack{}vtab}       \extab \textrm{U+000B}\\
{\cf\bfseries\#\backwhack{}page}       \extab \textrm{U+000C}\\
{\cf\bfseries\#\backwhack{}return}     \extab \textrm{U+000D}\\
{\cf\bfseries\#\backwhack{}esc}        \extab \textrm{U+001B}\\
{\cf\bfseries\#\backwhack{}space}      \extab \textrm{U+0020}\\
 \extab preferred way to write a space\\
{\cf\bfseries\#\backwhack{}delete}     \extab \textrm{U+007F}\\[1ex]
{\cf\bfseries\#\backwhack{}xFF}        \extab \textrm{U+00FF}\\
{\cf\bfseries\#\backwhack{}x03BB}      \extab \textrm{U+03BB}\\
{\cf\bfseries\#\backwhack{}x00006587}  \extab \textrm{U+6587}\\
{\cf\bfseries\#\backwhack{}\(\lambda\)} \extab \textrm{U+03BB}\\[1ex]
{\cf\bfseries\#\backwhack{}x0001z}     \extab \exception{\bfseries\&lexical}\\
{\cf\bfseries\#\backwhack{}\(\lambda\)x}         \extab \exception{\bfseries\&lexical}\\
{\cf\bfseries\#\backwhack{}alarmx}     \extab \exception{\bfseries\&lexical}\\
{\cf\bfseries\#\backwhack{}alarm x}    \extab \textrm{U+0007}\\
 \extab followed by {\cf{}x}\\
{\cf\bfseries\#\backwhack{}Alarm}      \extab \exception{\bfseries\&lexical}\\
{\cf\bfseries\#\backwhack{}alert}      \extab \exception{\bfseries\&lexical}\\
{\cf\bfseries\#\backwhack{}xA}         \extab \textrm{U+000A}\\
{\cf\bfseries\#\backwhack{}xFF}        \extab \textrm{U+00FF}\\
{\cf\bfseries\#\backwhack{}xff}        \extab \textrm{U+00FF}\\
{\cf\bfseries\#\backwhack{}x ff}       \extab \textrm{U+0078}\\
 \extab followed by another datum, {\bfseries\cf{}ff}\\%%%%%%%%%%%%%%%%%%%[34mm]%%%%%%%%%%%%%%%%%%%%%%%%%%%%%%%%%%%%
{\cf\bfseries\#\backwhack{}x(ff)}      \extab \textrm{U+0078}\\
 \extab followed by another datum,\\
 \extab a parenthesized {\bfseries\cf{}ff}\\
{\cf\bfseries\#\backwhack{}(x)}        \extab \exception{\bfseries\&lexical}\\
{\cf\bfseries\#\backwhack{}(x}         \extab \exception{\bfseries\&lexical}\\
{\cf\bfseries\#\backwhack{}((x)}       \extab \textrm{U+0028}\\
 \extab followed by another datum,\\
 \extab parenthesized {\bfseries\cf{}x}\\
{\cf\bfseries\#\backwhack{}x00110000}  \extab \exception{\bfseries\&lexical}\\
 \extab out of range\\
{\cf\bfseries\#\backwhack{}x000000001} \extab \textrm{U+0001}  \\
{\cf\bfseries\#\backwhack{}xD800}      \extab \exception{\bfseries\&lexical}\\
 \extab in excluded range
\htmlonly
\end{tabular}
\endhtmlonly
\texonly
\end{tabbing}

}
\endtexonly



%(The notation \exception{\&lexical} means that the line in question is
%a lexical syntax violation.)
(Нотация \exception{\bfseries\&lexical} означает, что рассматриваемая строка является
лексическим нарушением синтаксиса.)

%Case is significant in \sharpsign\backwhack\hyper{character}, and in
%\sharpsign\backwhack{\rm$\langle$character name$\rangle$}, % \hyper doesn't allow a linebreak
%but not in {\cf\sharpsign\backwhack{}x}\meta{hex scalar value}.
%A \meta{character} must be followed by a \meta{delimiter} or by the end of the input.
%This rule resolves various ambiguous cases involving named characters,
%requiring, for
%example, the sequence of characters ``{\tt\sharpsign\backwhack space}''
%to be interpreted as the space character rather than as
%the character ``{\tt\sharpsign\backwhack s}'' followed
%by the identifier ``{\tt pace}''.
Регистр является значащим в {\bfseries\sharpsign\backwhack}\hyper{character} и в
{\bfseries\sharpsign\backwhack} {\rm$\langle$character name$\rangle$}, но не в
{\bfseries\cf\sharpsign\backwhack{}x}\meta{hex scalar value}. За \meta{character} должен находиться
\meta{delimeter} или конец ввода. Это правило разрешает различные неоднозначные ситуации с
именованными символами, оно требует, например, чтобы последовательность символов
``{\bfseries\tt\sharpsign\backwhack space}'' интерпретировалась как символ пробела, а не как символ
``{\bfseries\tt\sharpsign\backwhack s}''и назодящийся за ним идентификатор ``{\bfseries\tt pace}''.

\begin{note}
  %The {\cf\sharpsign\backwhack{}newline} notation is retained for
  %backward compatibility.  Its use is deprecated;
  %{\cf\sharpsign\backwhack{}linefeed} should be used instead.
  Нотация {\bfseries\cf\sharpsign\backwhack{}newline} сохранена с целью обратной совместимости. Её
  использование устарело; вместо неё должна использоваться
  {\bfseries\cf\sharpsign\backwhack{}linefeed}.
\end{note}

%\subsection{Strings}
\subsection{Строки}

%\vest String are represented by sequences of characters enclosed within doublequotes
%({\cf "}).  Within a string literal, various escape
%sequences\mainindex{escape sequence} represent characters other than
%themselves.  Escape sequences always start with a backslash (\backwhack{}):
\vest Строка представляет собой последовательность символов, окружённых двойными кавычками
({\bfseries\cf "}). Внутри стрового литерала различные управляющие
последовательности\mainindex{escape sequence} представляют собой символы, отличные от них
самих. Управляющие последовательности всегда начинаются с обратного слеша
({\bfseries\backwhack{}}):

\begin{itemize}
\item{\bfseries\cf\backwhack{}a} : alarm, U+0007
\item{\bfseries\cf\backwhack{}b} : backspace, U+0008
\item{\bfseries\cf\backwhack{}t} : character tabulation, U+0009
\item{\bfseries\cf\backwhack{}n} : linefeed, U+000A
\item{\bfseries\cf\backwhack{}v} : line tabulation, U+000B
\item{\bfseries\cf\backwhack{}f} : formfeed, U+000C
\item{\bfseries\cf\backwhack{}r} : return, U+000D
\item{\bfseries\cf\backwhack{}}\verb|"| : doublequote, U+0022
\item{\bfseries\cf\backwhack{}\backwhack{}} : backslash, U+005C
\item{\bfseries\cf\backwhack{}}\hyper{intraline whitespace}\hyper{line ending}\\\hspace*{2em}\hyper{intraline whitespace} : nothing
\item{{\bfseries\cf\backwhack{}x}\meta{hex scalar value};} : specified character (note the
  terminating semi-colon).
\end{itemize}

%These escape sequences are case-sensitive, except that the alphabetic
%digits of a \meta{hex scalar value} can be uppercase or lowercase.
Эти управляющие последовательности регистрочувствительны, с одним исключением -- алфавитные
цифры \meta{hex scalar value} могут быть заглавными или строчными.

%Any other character in a string after a backslash is a syntax violation. Except
%for a line ending, any
%character outside of an escape sequence and not a doublequote stands
%for itself in the string literal. For example the single-character
%string literal {\tt "$\lambda$"} (doublequote, a lower case lambda, doublequote)
%represents the same string as {\tt "\backwhack{}x03bb;"}.
%A line ending that does not follow a backslash stands for a linefeed character.
Любой другой символ в строке после обратного слеша является нарушением синтаксиса. За
исключением окончания строки, любой символ в строковом литерале вне управляющей
последовательностии, не являющийся двойными кавычками, обозначает самого себя. Например,
односимвольный строковый литерал {\tt\bfseries "$\lambda$"} (двойные кавычки, строчная lambda, двойные
кавычки) представляет собой ту же строку, что и {\tt\bfseries "\backwhack{}x03bb;"}. Окончание строки,
находящееся не после обратного слеша, обозначает символ перевода строки.

%Examples:
Примеры:

\texonly
\begin{tabbing}
{\cf "\backwhack{}x0000000000;"} \=\kill
\endtexonly
\htmlonly
\begin{tabular}{ll}
\endhtmlonly
{\bfseries\cf "abc"} \extab  \textrm{U+0061, U+0062, U+0063}\\
{\bfseries\cf "\backwhack{}x41;bc"} \extab  {\bfseries\cf "Abc"} ; \textrm{U+0041, U+0062, U+0063}\\
{\bfseries\cf "\backwhack{}x41; bc"} \extab {\bfseries\cf "A bc"}\\
 \extab U+0041, U+0020, U+0062, U+0063\\
{\bfseries\cf "\backwhack{}x41bc;"} \extab  \textrm{U+41BC}\\
{\bfseries\cf "\backwhack{}x41"} \extab \exception{\bfseries\&lexical}\\
{\bfseries\cf "\backwhack{}x;"} \extab \exception{\bfseries\&lexical}\\
{\bfseries\cf "\backwhack{}x41bx;"} \extab \exception{\bfseries\&lexical}\\
{\bfseries\cf "\backwhack{}x00000041;"} \extab  {\bfseries\cf "A"} ; \textrm{U+0041}\\
{\bfseries\cf "\backwhack{}x0010FFFF;"} \extab \textrm{U+10FFFF}\\
{\bfseries\cf "\backwhack{}x00110000;"} \extab  \exception{\bfseries\&lexical}\\
 \extab out of range\\
{\bfseries\cf "\backwhack{}x000000001;"} \extab \textrm{U+0001}\\
{\bfseries\cf "\backwhack{}xD800;"} \extab \exception{\bfseries\&lexical}\\
 \extab in excluded range\\
{\bfseries\cf "A}\\
{\bfseries\cf bc"} \extab \textrm{U+0041, U+000A, U+0062, U+0063}\\
 \extab if no space occurs after the {\bfseries\cf{}A}
\htmlonly
\end{tabular}
\endhtmlonly
\texonly
\end{tabbing}
\endtexonly

%\subsection{Numbers}
\subsection{Числа}
\label{numbernotations}

%The syntax of external representations for number objects is described
%formally by the \meta{number} rule in the formal grammar.
%Case is not significant in external representations of number objects.
Синтаксис внешних представлений числовых объектов формально описывается правилом \meta{number}
формальной грамматики. Во внешних представлениях числовых объектов регистр является незначащим.

%A representation of a number object may be written in binary, octal, decimal, or
%hexadecimal by the use of a radix prefix.  The radix prefixes are {\cf
%\#b}\sharpindex{b} (binary), {\cf \#o}\sharpindex{o} (octal), {\cf
%\#d}\sharpindex{d} (decimal), and {\cf \#x}\sharpindex{x} (hexadecimal).  With
%no radix prefix, a representation of a number object is assumed to be expressed in decimal.
Представление числового объекта может быть записано в двоичном, восьмеричном, десятичном или
шестнадцатеричном виде при помощи приставки основания. Приставками основания являются
{\bfseries\cf \#b}\sharpindex{b} (двоичная), {\bfseries\cf \#o}\sharpindex{o} (восьмеричная),
{\bfseries\cf \#d}\sharpindex{d} (десятичная) и {\bfseries\cf \#x}\sharpindex{x}
(шестнадцатеричная). При отсутствии приставки основания представление числового объекта
полагается десятичным.

%A representation of a number object may be specified to be either exact or
%inexact by a prefix.  The prefixes are {\cf \#e}\sharpindex{e}
%for exact, and {\cf \#i}\sharpindex{i} for inexact.  An exactness
%prefix may appear before or after any radix prefix that is used.  If
%the representation of a number object has no exactness prefix, the
%constant is
%inexact if it contains a decimal point, an
%exponent, or
%a nonempty mantissa width;
%otherwise it is exact.
Представление числового объекта может указываться точным или неточным с помощью
приставки. Приставками являются {\bfseries\cf \#e}\sharpindex{e} в случае точного, и {\bfseries\cf
  \#i}\sharpindex{i} в случае неточного представления. Приставка точности может находиться до
или после любой используемой приставки основания. При отсутствии в представлении числового объекта
приставки точности, константа является неточной, если она содержит десятичную точку, экспоненту или
непустую ширину мантиссы; в противном случае она является точной.

%In systems with inexact number objects
%of varying precisions, it may be useful to specify
%the precision of a constant.  For this purpose, representations of
%number objects
%may be written with an exponent marker that indicates the
%desired precision of the inexact
%representation.  The letters {\cf s}, {\cf f},
%{\cf d}, and {\cf l} specify the use of \var{short}, \var{single},
%\var{double}, and \var{long} precision, respectively.  (When fewer
%than four internal
%inexact
%representations exist, the four size
%specifications are mapped onto those available.  For example, an
%implementation with two internal representations may map short and
%single together and long and double together.)  In addition, the
%exponent marker {\cf e} specifies the default precision for the
%implementation.  The default precision has at least as much precision
%as \var{double}, but
%implementations may wish to allow this default to be set by the user.
В системах с неточными числовыми объектами переменной точности может применяться спецификация
точности константы. Для этого представления числовых объектов могут записываться с
экспоненциальным маркером, задающим желаемую точность неточного представления. Буквы
{\bfseries\cf s}, {\bfseries\cf f}, {\bfseries\cf d} и {\bfseries\cf l} специфицируют использование
точности \var{short}, \var{single}, \var {double} и \var{long} соответственно. (При наличии
менее четырёх внутренних неточных представлений четыре спецификации размера преобразовываются в
доступные. Например, реализация с двумя внутренними представлениями может совместно преобразовывать
данные short и single, а также long и double) Кроме того, экспоненциальный маркер
{\bfseries\cf e} указывает точность реализации по умолчанию. Точность по умолчанию имеет
значение не менее точности \var{double}, но реализации могут разрешать пользователю
устанавливать данное значение по умолчанию.

\begin{scheme}
\bfseries 3.1415926535898F0
       {\rm{}Round to single, perhaps} {\bfseries 3.141593}
\bfseries 0.6L0
       {\rm{}Extend to long, perhaps} {\bfseries .600000000000000}%
\end{scheme}

%A representation of a number object with nonempty mantissa width,
%{\cf \var{x}|\var{p}}, represents the best binary
%floating-point approximation of \var{x} using a \var{p}-bit significand.
%For example, {\cf 1.1|53} is a
%representation of the best approximation of 1.1 in IEEE double
%precision.
%If \var{x} is an external representation of an inexact real number object
%that contains no vertical bar, then its numerical value should be computed
%as though it had a mantissa width of 53 or more.
Представление числового объекта с непустой шириной мантиссы {\cf \var{x}|\var{p}} представляет
собой наилучшую двоичную аппроксимацию плавающей точки \var{x} с помощью \var{p}-битной значащей
части. Например, {\cf\bfseries 1.1|53} является представлением наилучшей аппроксимации 1.1 в
IEEE двойной точности. Если \var{x} является внешним представлением неточного действительного
числового объекта, не содержащим вертикальной черты, его численное значение должно вычисляться
так, как если бы оно имело ширину мантиссы 53 или более.

%Implementations that use binary floating-point representations
%of real number objects should represent {\cf \var{x}|\var{p}}
%using a \var{p}-bit significand if practical, or by a greater
%precision if a \var{p}-bit significand is not practical, or
%by the largest available precision if \var{p} or more bits
%of significand are not practical within the implementation.
Реализации, использующие двоичные представления с плавающей точкой действительных числовых
объектов, должны представлять {\cf \var{x}|\var{p}} с помощью \var{p}-битной значащей части,
если это удобно, или с большей точностью, если \var{p}-битная значащая часть не
удобна, или с наибольшей доступной точностью, если \var{p} или более бит
значащей части не удобны в реализации.

\begin{note}
%The precision of a significand should not be confused with the
%number of bits used to represent the significand.  In the IEEE
%floating-point standards, for example, the significand's most
%significant bit is implicit in single and double precision but
%is explicit in extended precision.  Whether that bit is implicit
%or explicit does not affect the mathematical precision.
%In implementations that use binary floating point, the default
%precision can be calculated by calling the following procedure:
Точность значащей части не следует путать с количеством бит, используемых для представления
значащей части. В стандартах IEEE с плавающей точкой, например, старший значащий бит значащей
части является неявным при одинарной и двойной точности, и явным при расширенной точности. На
математическую точность не влияет, является ли этот бит явным или неявным. В реализациях,
использующих двоичную плавающую точку, точность по умолчанию может быть вычислена путём вызова
следующей процедуры:

\begin{scheme}
\bfseries (define (precision)
\bfseries   (do ((n 0 (+ n 1))
\bfseries        (x 1.0 (/ x 2.0)))
\bfseries     ((= 1.0 (+ 1.0 x)) n)))
\end{scheme}
\end{note}

\begin{note}
%When the underlying floating-point representation is IEEE double
%precision, the {\cf |\var{p}} suffix should not always be omitted:
%Denormalized floating-point numbers have diminished precision,
%and therefore their external representations should
%carry a {\cf |\var{p}} suffix with the actual width of the
%significand.
Если основным представлением плавающей точки является двойная точность IEEE, суффикс {\cf |\var{p}}
не обязан всегда пропускаться: Денормализованные числа с плавающей точкой имеют пониженную
точность, и поэтому их внешние представления должны содержать суффикс {\cf |\var{p}} с
фактической шириной значащей части.
\end{note}

%The literals {\cf +inf.0} and {\cf -inf.0} represent positive and
%negative infinity, respectively.  The {\cf +nan.0}
%literal represents the NaN that is the result of {\cf (/ 0.0 0.0)},
%and may represent other NaNs as well.
Литералы {\cf\bfseries +inf.0} и {\cf\bfseries -inf.0} представляют положительную и
отрицательную бесконечность соответственно. Литерал {\cf\bfseries +nan.0} представляет не число,
являющееся результатом {\cf\bfseries(/ 0.0 0.0)}, и может также представлять другие не числа.

%If \var{x} is an external representation of an inexact real number
%object and
%contains no vertical bar and no exponent marker
%other than {\cf e}, the inexact real number object it represents is a flonum
%(see library section~\extref{lib:flonumssection}{Flonums}).
%Some or all of the other external representations of
%inexact real number objects may also represent flonums, but that is not required by
%this report.
Если \var{x} является внешним представлением неточного действительного числового объекта, а
также не содержит вертикальной черты и иного экспоненциального маркера, нежели {\cf\bfseries e},
неточным действительным числовым объектом, представляющим его, является flonum (см. библиотечную
секцию~\extref{lib:flonumssection}{Flonums}). Некоторые или все другие внешние представления
неточных действительных числовых объектов также могут представлять flonum, но это не является
требованием данного стандарта.

%\section{Datum syntax}
\section{Datum-синтаксис}
\label{datumsyntaxsection}

%The datum syntax describes the syntax of
%syntactic data\mainindex{syntactic datum} in terms of a sequence of
%\meta{lexeme}s, as defined in the lexical syntax.
Datum-синтаксис описывает синтаксис синтаксических данных\mainindex{syntactic datum}
в терминах последовательности \meta{lexeme}, как определено в лексическом
синтаксисе.

%Syntactic data include the lexeme data described in the
%previous section as well as the following constructs for forming
%compound data:
Синтаксические данные включают данные лексем, описанные в предыдущих секциях, а также
следующие конструкции для формирования составных данных:
%
\begin{itemize}
%\item pairs and lists, enclosed by \verb|( )| or \verb|[ ]| (see
%  section~\ref{pairlistsyntax})
\item пары и списки, заключённые в {\bfseries\verb|( )|} или {\bfseries\verb|[ ]|} (см.
  секцию~\ref{pairlistsyntax})
%\item vectors (see section~\ref{vectorsyntax})
\item векторы (см. секцию~\ref{vectorsyntax})
%\item bytevectors (see section~\ref{bytevectorsyntax})
\item байтовые векторы (см. секцию~\ref{bytevectorsyntax})
\end{itemize}

%\subsection{Formal account}
\subsection{Формальное описание}
\label{datumsyntax}

%The following grammar describes the syntax of syntactic data in terms
%of various kinds of lexemes defined in the grammar in
%section~\ref{lexicalsyntaxsection}:
Следующая грамматика описывает синтаксис синтаксических данных в терминах лексем различных видов,
определённых в грамматике в секции ~\ref{lexicalsyntaxsection}:

{%
\renewcommand{\baselinestretch}{1.05}
\selectfont
\begin{grammar}%
\meta{datum} \: \meta{lexeme datum}
\>  \| \meta{compound datum}
\meta{lexeme datum} \: \meta{boolean} \| \meta{number}
\>  \| \meta{character} \| \meta{string} \|  \meta{symbol}
\meta{symbol} \: \meta{identifier}
\meta{compound datum} \: \meta{list} \| \meta{vector} \| \meta{bytevector}
\meta{list} \: \textbf{(}\arbno{\meta{datum}}\textbf{)} \| \textbf{[}\arbno{\meta{datum}}\textbf{]}
\>    \| \textbf{(}\atleastone{\meta{datum}} .\ \meta{datum}\textbf{)} \| \textbf{[}\atleastone{\meta{datum}} .\ \meta{datum}\textbf{]}
\>    \| \meta{abbreviation}
\meta{abbreviation} \: \meta{abbrev prefix} \meta{datum}
\meta{abbrev prefix} \: \textbf{'} \| \textbf{`} \| \textbf{,} \| \textbf{,@}
\>    \| \textbf{\#'} | \textbf{\#`} | \textbf{\#,} | \textbf{\#,@}
\meta{vector} \: \textbf{\#(}\arbno{\meta{datum}}\textbf{)}
\meta{bytevector} \: \textbf{\#vu8(}\arbno{\meta{u8}}\textbf{)}
\meta{u8} \: $\langle${\rm any \meta{number} representing an exact}
 \>\>\quad\quad {\rm integer in $\{0, \ldots, 255\}$}$\rangle$%
\end{grammar}

}

%\subsection{Pairs and lists}
\subsection{Пары и списки}
\label{pairlistsyntax}

%List and pair data, representing pairs and lists of values
%(see section~\ref{listsection}) are represented using parentheses or brackets.
%Matching pairs of brackets that occur in the rules of \meta{list} are
%equivalent to matching pairs of parentheses.
Данные списков и пар, представляющие значения пар и списков (см. секцию~\ref{listsection}),
представляются с помощью круглых или квадратных скобок. Соответствие пар квадратных скобок,
находящихся в правилах \meta{list}, эквивалентно соответствию пар круглых
скобок.

%\newpage

%The most general notation for Scheme pairs as syntactic data is
%the ``dotted'' notation \hbox{\cf (\hyperi{datum} .\ \hyperii{datum})} where
%\hyperi{datum} is the representation of the value of the car field and
%\hyperii{datum} is the representation of the value of the
%cdr field.  For example {\cf (4 .\ 5)} is a pair whose car is 4 and whose
%cdr is 5.
Наиболее общей нотацией для пар Scheme как синтаксических данных является ``точечная'' нотация
\hbox{\cf (\hyperi{datum} .\ \hyperii{datum})}, где \hyperi{datum} является представлением
значения поля car, а \hyperii{datum} -- значения поля cdr. Например, {\cf\bfseries (4 .\ 5)} является
парой, car которой -- 4, а cdr -- 5.

%A more streamlined notation can be used for lists: the elements of the
%list are simply enclosed in parentheses and separated by spaces.  The
%empty list\index{empty list} is represented by {\tt()} .  For example,
Для списков может использоваться упрощённая нотация: элементы списка просто заключаются в
круглые скобки и разделяются пробелами. Пустой список\index{empty list} представляется, как
{\bfseries\tt()}. Например,

\begin{scheme}
\bfseries(a b c d e)%
\end{scheme}

%and
и

\begin{scheme}
\bfseries (a . (b . (c . (d . (e . ())))))%
\end{scheme}

%are equivalent notations for a list of symbols.
являются эквивалентными формами записи списка символов.

%The general rule is that, if a dot is followed by an open parenthesis,
%the dot, open parenthesis, and matching closing parenthesis
%can be omitted in the external representation.
Согласно общему правилу, если за точкой следует открывающая круглая скобка, то точка,
открывающая круглая скобка и соответствующая ей закрывающая круглая скобка во внешнем
представлении могут быть пропущены.

%The sequence of characters ``{\cf (4 .\ 5)}'' is the external representation of a
%pair, not an expression that evaluates to a pair.
%Similarly, the sequence of characters ``{\tt(+ 2 6)}'' is {\em not} an
%external representation of the integer 8, even though it {\em is} an
%expression (in the language of the \rsixlibrary{base} library)
%evaluating to the integer 8; rather, it is a
%syntactic datum representing a three-element list, the elements of which
%are the symbol {\tt +} and the integers 2 and 6.
Последовательность символов ``{\cf\bfseries (4 .\ 5)}'' является внешним представлением пары, а не
выражением, которое вычисляется как пара. Аналогично, последовательность символов ``{\tt\bfseries (+ 2 6)}''
{\em не} является внешним представлением целого числа 8, даже при том, что она {\em является} выраженим
(на языке библиотеки {\bfseries\rsixlibrary{base}}), вычисляемым как целое число 8; напротив, это
синтаксический datum, представляющий трёхэлементный список, элементами которого являются
символ {\tt\bfseries +} и целые числа 2 и 6.

%\subsection{Vectors}
\subsection{Векторы}
\label{vectorsyntax}

%Vector data, representing vectors of objects (see
%section~\ref{vectorsection}), are represented using the notation
%{\tt\#(\hyper{datum} \dotsfoo)}.  For example, a vector of length 3
%containing the number object for zero in element 0, the list {\cf(2 2 2 2)} in
%element 1, and the string {\cf "Anna"} in element 2 can be represented as
%follows:
Данные векторов, представляюшие вектора объектов (см. секцию~\ref{vectorsection}),
представляются с помощью нотации {\tt{\bfseries\#(}\hyper{datum}
  \dotsfoo{\bfseries)}}. Например, вектор с длиной 3, содержащий числовой объект для нуля в
элементе 0, список {\cf\bfseries (2 2 2 2)} в элементе 1 и строку {\cf\bfseries "Anna"} в
элементе 2, может быть представлен следующим образом:

\begin{scheme}
\bfseries \#(0 (2 2 2 2) "Anna")%
\end{scheme}

%This is the external representation of a vector, not an
%expression that evaluates to a vector.
Это внешнее представление вектора, а не выражение, вычисляемое как вектор.

%\subsection{Bytevectors}
\subsection{Байтовые векторы}
\label{bytevectorsyntax}

%Bytevector data, representing bytevectors (see
%library chapter~\extref{lib:bytevectorschapter}{Bytevectors}), are represented using the notation
%{\tt\#vu8(\hyper{u8} \dotsfoo)}, where the \hyper{u8}s represent the octets of
%the bytevector.  For example, a bytevector of length 3 containing the
%octets 2, 24, and 123 can be represented as follows:
Данные байтовых векторов, представляющие байтовые вектора (см. библиотечную
главу~\extref{lib:bytevectorschapter}{Bytevectors}), представляются с помощью нотации
{\tt{\bfseries\#vu8(}\hyper{u8} \dotsfoo{\bfseries )}}, где \hyper{u8} представляет октет
байтового вектора. Например, байтовый вектор с длиной 3, содержащий октеты 2, 24 и 123, может быть
представлен следующим образом:

\begin{scheme}
\bfseries \#vu8(2 24 123)%
\end{scheme}

%This is the external representation of a bytevector, and also an
%expression that evaluates to a bytevector.
Это внешнее представление байтового вектора, а не выражение, вычисляемое как байтовый вектор.

%\subsection{Abbreviations}\unsection
\subsection{Сокращения}\unsection
\label{abbreviationsection}

{%
\renewcommand{\baselinestretch}{1.05}
\selectfont
\begin{entry}{%
\pproto{\textbf{\singlequote}\hyper{datum}}{}
\pproto{\textbf{\backquote}\hyper{datum}}{}
\pproto{\textbf{,}\hyper{datum}}{}
\pproto{\textbf{,\atsign}\hyper{datum}}{}
\pproto{\textbf{\#'}\hyper{datum}}{}
\pproto{\textbf{\#\backquote}\hyper{datum}}{}
\pproto{\textbf{\#,}\hyper{datum}}{}
\pproto{\textbf{\#,@}\hyper{datum}}{}
}\vspace{-1mm}

%Each of these is an abbreviation:
Каждый из них является сокращением:
%\\\quad\schindex{'}\singlequote\hyper{datum}
%for {\cf (quote \hyper{datum})},
%\\\quad\schindex{`}\backquote\hyper{datum}
%for {\cf (quasiquote \hyper{datum})},
%\\\quad\schindex{,}{\cf,}\hyper{datum}
%for {\cf (unquote \hyper{datum})},
%\\\quad\index{,@\texttt{,\atsign}}{\cf,}\atsign\hyper{datum}
%for {\cf (unquote-splicing \hyper{datum})},
%\\\quad\sharpindex{'}{\cf\#'}\hyper{datum}
%for {\cf (syntax \hyper{datum})},
%\\\quad\sharpindex{`}{\cf\#`}\hyper{datum}
%for {\cf (quasisyntax \hyper{datum})},
%\\\quad\sharpindex{,}{\cf\#,}\hyper{datum}
%for {\cf (unsyntax \hyper{datum})}, and
%\\\quad\index{#,@\texttt{\#,\atsign}}{\cf\#,@}\hyper{datum}
%for {\cf (unsyntax-splicing \hyper{datum})}.
\\\quad\schindex{'}{\bfseries\singlequote}\hyper{datum}
для {\cf \textbf{(quote} \hyper{datum}\textbf{)}},
\\\quad\schindex{`}{\bfseries\backquote}\hyper{datum}
для {\cf \textbf{(quasiquote} \hyper{datum}\textbf{)}},
\\\quad\schindex{\bfseries ,}{\cf\bfseries ,}\hyper{datum}
для {\cf \textbf{(unquote} \hyper{datum}\textbf{)}},
\\\quad\index{,@\texttt{,\atsign}}{\cf,}{\bfseries\atsign}\hyper{datum}
для {\cf \textbf{(unquote-splicing} \hyper{datum}\textbf{)}},
\\\quad\sharpindex{'}{\bfseries\cf\#'}\hyper{datum}
для {\cf \textbf{(syntax} \hyper{datum}\textbf{)}},
\\\quad\sharpindex{`}{\bfseries\cf\#`}\hyper{datum}
для {\cf \textbf{(quasisyntax} \hyper{datum}\textbf{)}},
\\\quad\sharpindex{,}{\bfseries\cf\#,}\hyper{datum}
для {\cf \textbf{(unsyntax} \hyper{datum}\textbf{)}}, and
\\\quad\index{#,@\texttt{\#,\atsign}}{\bfseries\cf\#,@}\hyper{datum}
для {\cf \textbf{(unsyntax-splicing} \hyper{datum}\textbf{)}}.
\end{entry}\vspace{-1mm}

}

%%% Local Variables:
%%% mode: latex
%%% TeX-master: "r6rs"
%%% End:
     \par
%\vfill\eject
%\chapter{Semantic concepts}
\chapter{Семантические концепции}
\label{basicchapter}

%\section{Programs and libraries}
\section{Программы и библиотеки}

%A Scheme program consists of a \textit{top-level program\index{top-level program}}
%together with a set of \textit{libraries\index{library}}, each
%of which defines a part of the program connected to the others through
%explicitly specified exports and imports.  A library consists of a set
%of export and import specifications and a body, which consists of
%definitions, and expressions.
%A top-level program is similar to a library, but
%has no export specifications.
%Chapters~\ref{librarychapter} and \ref{programchapter}
%describe the syntax and semantics of libraries and top-level programs,
%respectively.
%Chapter~\ref{baselibrarychapter} describes a base
%library that defines many of the constructs traditionally associated with
%Scheme.
%A separate report~\cite{R6RS-libraries}
%describes the various \textit{standard libraries}\index{standard
%  library} provided by a Scheme system.
Программа Scheme состоит из \textit{программы верхнего уровня\index{top-level program}} совместно с
набором \textit{библиотек\index{library}}, каждая из которых определяет часть программы, связанную
с другими частями посредством явно указываемых экспорта и импорта. Библиотека состоит из ряда
спецификаций экспорта и импорта, а также тела, состоящего из определений и выражений. Программа
верхнего уровня похожа на библиотеку, но не имеет спецификаций
экспорта. В главах~\ref{librarychapter} и \ref{programchapter} описаны синтаксис и семантика
библиотек и программ верхнего уровня соответственно. В главе~\ref {baselibrarychapter} описана
основная библиотека, в которой определено большинство конструкций, традиционно ассоциированных со
Scheme. В отдельной работе~\cite{R6RS-libraries} описаны различные \textit{стандартные
библиотеки}, \index{standard library} предоставляемые системой Scheme.

%The division between the base library and the other standard libraries is
%based on use, not on construction.  In particular, some facilities
%that are typically implemented as ``primitives'' by a compiler or the
%run-time system rather than in terms of other standard procedures
% or syntactic forms are not part of the base library, but are defined in
%separate libraries.  Examples include the fixnums and flonums libraries,
%the exceptions and conditions libraries, and the libraries for
%records.
Деление на основную библиотеку и прочие стандартные библиотеки является прикладным, а не
конструктивным. В частности, некоторые средства, обычно реализуемые как ``примитивы''
компилятором или системой во время выполнения, а не в терминах других стандартных процедур или
синтаксических форм, не являются частью основной библиотеки, а определены в отдельных
библиотеках. Примеры включают библиотеки fixnums и flonums, библиотеки исключений и условий, а
также библиотеки для записей.

%\section{Variables, keywords, and regions}
\section{Переменные, ключевые слова и регионы}
\label{specialformsection}
\label{variablesection}

%Within the body of a library or top-level program,
%an identifier\index{identifier} may name a kind of syntax, or it may name
%a location where a value can be stored.  An identifier that names a kind
%of syntax is called a {\em keyword}\mainindex{keyword}, or {\em syntactic keyword}\mainindex{syntactic keyword},
%and is said to be {\em bound} to that kind of syntax (or, in the case of a
%syntactic abstraction, a {\em transformer} that translates the syntax into more
%primitive forms; see section~\ref{macrosection}).  An identifier that names a
%location is called a {\em variable}\mainindex{variable} and is said to be
%{\em bound} to that location.
%At each point within a top-level program or a library, a specific, fixed set
%of identifiers is bound.  The set of these identifiers, the set of \textit{visible
%bindings}\mainindex{binding}, is
%known as the {\em environment} in effect at that point.
В теле библиотеки или программы верхнего уровня идентификатор\index{identifier} может именовать
или вид синтаксиса, или ячейку памяти, где может храниться
значение. Идентификатор, именующий вид синтаксиса, называется {\em ключевым
  словом}\mainindex{keyword}, или {\em синтаксическим ключевым словом}\mainindex{syntactic
  keyword}, и считается {\em привязанным} к этому виду синтаксиса (или, в случае
синтаксической абстракции, {\em преобразователем}, транслирующим синтаксис в более
примитивные формы; см. секцию~\ref{macrosection}). Идентификатор, именующий ячейку памяти,
называется {\em переменной}\mainindex{variable} и считается {\em привязанным} к этой
ячейке памяти. К каждой точке программы верхнего уровня или библиотеке привязан конкретный,
постоянный набор идентификаторов. Набор таких идентификаторов, набор
\textit{видимого связывания}\mainindex{binding}, называется {\em окружением}, действующим в
данной точке.

%Certain forms are used to create syntactic abstractions
%and to bind keywords to transformers for those new syntactic abstractions, while other
%forms create new locations and bind variables to those
%locations.  Collectively, these forms are called {\em binding
%  constructs}.\mainindex{binding construct}
%Some binding constructs take the form of
%\textit{definitions}\index{definition}, while others are
%expressions.
%With the exception of exported library bindings, a binding created
%by a definition is visible only within the body in which the
%definition appears, e.g., the body of a library, top-level program,
%or {\cf lambda} expression.
%Exported library bindings are also visible within the bodies of
%the libraries and top-level programs that import them (see
%chapter~\ref{librarychapter}).
Одни формы используются для создания синтаксических абстракций и связывания ключевых
слов с преобразователями для этих новых синтаксических абстракций, в то время как другие формы
создают новые ячейки памяти и связывают переменные с этими ячейками. Все эти
формы обобщённо называются {\em конструкциями привязки}.\mainindex{binding construct}, Некоторые
конструкции привязки принимают форму \textit{определений}\index{definition}, в то время как
другие являются выражениями. За исключением экспортируемых библиотечных привязок, привязка,
созданная определением, видима только внутри тела, в котором находится определение,
например, в теле библиотеки, в программе верхнего уровня или в выражении {\cf\bfseries
  lambda}. Экспортируемые библиотечные привязки также видимы внутри тел библиотек и программ
верхнего уровня, импортирующих их (см. главу~\ref{librarychapter}).

%Expressions that bind variables include the {\cf lambda},
%{\cf let}, {\cf let*}, {\cf letrec}, {\cf letrec*}, {\cf let-values},
%and {\cf let*-values} forms from the base library (see
%sections~\ref{lambda}, \ref{letrec}).
%Of these, {\cf lambda} is the most fundamental.
%Variable definitions appearing within the body of
%such an expression, or within the bodies of a library or top-level
%program, are treated as a set of
%{\cf letrec*} bindings.
%In addition, for library bodies,
%the variables exported from the library can be referenced by
%importing libraries and top-level programs.
Выражения, связывающие переменные, включают формы {\cf\bfseries lambda}, {\cf\bfseries let},
{\cf\bfseries let*}, {\cf\bfseries letrec}, {\cf\bfseries letrec*}, {\cf\bfseries let-values}, и
{\cf\bfseries let*-values} из основной библиотеки (см. секции~\ref{lambda}, \ref {letrec}). Из
них {\cf\bfseries lambda} является самой фундаментальной. Определения переменных, находящиеся внутри
тела такого выражения, или внутри тела библиотеки или программы верхнего уровня,
интерпретируются как набор привязок {\cf\bfseries letrec*}. Кроме того, для тел библиотеки, к переменным,
экспортируемым из библиотеки, можно обратиться, импортируя программы верхнего уровня и
библиотеки.

%Expressions that bind keywords include the {\cf
%  let-syntax} and {\cf letrec-syntax} forms (see
%section~\ref{bindsyntax}).  A {\cf define} form (see section~\ref{define}) is a
%definition that creates a variable binding (see
%section~\ref{defines}), and a {\cf define-syntax} form is
%a definition that creates a keyword binding (see
%section~\ref{syntaxdefinitionsection}).
Выражения, привязывающие ключевые слова, включают формы {\cf\bfseries let-syntax} и
{\cf\bfseries letrec-syntax} (см. секцию~\ref{bindsyntax}). Форма {\cf\bfseries define}
(см. секцию~\ref{define}) является определением, создающим привязку переменной
(см. секцию~\ref{define}), а форма {\cf\bfseries define-syntax} - определением, создающим
привязку ключевого слова (см. секцию~\ref{syntaxdefinitionsection}).

%\vest Scheme is a statically scoped language with
%block structure.  To each place in a top-level program or library body where an identifier is bound
%there corresponds a \defining{region} of code within which
%the binding is visible.  The region is determined by the particular
%binding construct that establishes the binding; if the binding is
%established by a {\cf lambda} expression, for example, then its region
%is the entire {\cf lambda} expression.  Every mention of an identifier
%refers to the binding of the identifier that establishes the
%innermost of the regions containing the use.  If a use of an
%identifier appears in a place where none of the surrounding expressions
%contains a binding for the identifier, the use may refer to a
%binding established by a definition or import at the top of the
%enclosing library or top-level program
%(see chapter~\ref{librarychapter}).
%If there is no binding for the identifier,
%it is said to be \defining{unbound}.\mainindex{bound}
\vest Scheme -- язык со статическими областями видимости и блочной структурой. Каждой позиции в
программе верхнего уровня или тела библиотеки, где привязан идентификатор, соответствует
\defining{регион} кода, внутри которого видима привязка. Регион определяется конкретной
конструкцией привязки, устанавливающей привязку; если привязка установлена, например, выражением
{\cf\bfseries lambda}, её регионом является всё выражение {\cf\bfseries lambda}. Каждое
упоминание идентификатора обращается к привязке идентификатора, установленной самым внутренним
из регионов, содержащих её применение. Если применение идентификатора находится в позиции, где
ни одно из близлежащих выражений не содержит привязки идентификатора, применение может
обратиться к привязке, установленной определением или импортом из верхнего уровня окружающей
библиотеки, или программой верхнего уровня (см. главу~\ref {librarychapter}). Если для
идентификатора не существует привязки, он называется \defining{несвязанным}.\mainindex{bound}

%\section{Exceptional situations}
\section{Исключительные ситуации}
\label{exceptionalsituationsection}

%\mainindex{exceptional situation}A variety of exceptional situations
%are distinguished in this report, among them violations of syntax,
%violations of a procedure's specification, violations of
%implementation restrictions, and exceptional situations in the
%environment.  When an exceptional situation is detected by the
%implementation, an \textit{exception is raised}\mainindex{raise},
%which means that a special procedure called the \textit{current
%  exception handler} is called.  A program can also raise an
%exception, and override the current exception handler; see
%library section~\extref{lib:exceptionssection}{Exceptions}.
\mainindex{exceptional situation}В данной работе рассматривается разнообразие исключительных
ситуаций, среди которых нарушения синтаксиса, нарушения спецификаций процедур, нарушения
ограничений реализации и исключительные ситуации в окружении. При обнаружении реализацией
исключительной ситуации \textit{возбуждается исключение}\mainindex{raise}, что
означает вызов специальной процедуры, называемой \textit{текущим обработчиком исключений}.
Программа может также возбудить исключение и переопределить текущий обработчик
исключений; см. библиотечную секцию~\extref{lib:exceptionssection}{Exceptions}.

%When an exception is raised, an object is provided that
%describes the nature of the exceptional situation.  The report uses
%the condition system described in library section~\extref{lib:conditionssection}{Conditions} to
%describe exceptional situations, classifying them by condition types.
При возбуждении исключения порождается объект, описывающий характер исключительной ситуации. В
работе используется система состояний, описанная в библиотечной
секции~\extref{lib:conditionssection}{Conditions}, для описания исключительных ситуаций
классификацией их типами состояний.

%Some exceptional situations allow continuing the program if the
%exception handler takes appropriate action.  The corresponding
%exceptions are called \textit{continuable}\index{continuable exception}.
%For most of the exceptional situations described in this report,
%portable programs cannot rely upon the exception being continuable
%at the place where the situation was detected.
%For those exceptions, the exception handler that is invoked by the
%exception should not return.
%In some cases, however, continuing is permissible, and the
%handler may return.  See library section~\extref{lib:exceptionssection}{Exceptions}.
Некоторые исключительные ситуации допускают продолжение программы, если обработчик исключений
предпримет соответствующее действие. Соответствующие исключения называются
\textit{продолжаемыми}\index{continuable exception}. В большинстве исключительных ситуаций,
описанных в данной работе, переносимые программы не могут зависеть от продолжаемых в месте
обнаружения ситуации исключений. Для таких исключений обработчик исключений, вызванный
исключением, не должен возвращаться. В некоторых случаях, однако, продолжение допустимо, и
обработчик может возвращаться. См. библиотечную
секцию~\extref{lib:exceptionssection}{Exceptions}.

%Implementations must raise an exception
%when they are unable to continue correct execution
%of a correct program due to some \defining{implementation restriction}.  For
%example, an implementation that does not support infinities
%must raise an exception with condition type
%{\cf\&implementation-restriction} when it evaluates an expression
%whose result would be an infinity.
Реализации должны возбудить исключение в случае невозможности продолжения корректного выполнения
корректной программы из-за некоторого \defining{ограничения реализации}. Например, реализация,
не поддерживающая бесконечность, должна возбудить исключение с типом состояния
{\cf\bfseries\&implementation-restriction}, при вычислениии выражения, результатом которого
может быть бесконечность.

%Some possible implementation restrictions such as the lack of
%representations for NaNs and infinities (see
%section~\ref{infinitiesnanssection}) are anticipated by this report,
%and implementations typically must raise an exception of the
%appropriate condition type if they encounter such a situation.
Некоторые возможные ограничения реализации, типа нехватки представлений для NaN и
бесконечностей (см. секцию~\ref{infinitiesnanssection}) предугадываются в соответствии с данной
работой, и реализации, как правило, должны возбуждать исключение соответствующего типа
состояния, если они сталкиваются с такой ситуацией.

%This report uses the phrase ``an exception is raised'' synonymously
%with ``an exception must be raised''.
%This report uses the phrase ``an exception with condition type \var{t}''
%to indicate that the object provided with the
%exception is a condition object of the specified type.
%The phrase ``a continuable exception is raised'' indicates an
%exceptional situation that permits the exception handler to return.
В данной работе применение фразы ``исключение возбуждено'' синонимично с "исключение должно быть
возбуждено". В данной работе используется фраза ``исключение с типом состояния \var{t}'', для указания,
что объект, порождаемый с исключением, является объектом состояния указанного типа. Фраза
``продолжаемое исключение возбуждено'', указывает исключительную ситуацию, разрешающую возврат обработчика
исключений.

%\section{Argument checking}
\section{Проверка аргументов}
\label{argumentcheckingsection}

\mainindex{argument checking}
%Many procedures specified in this report or as part of a
%standard library restrict the arguments they accept.
%Typically, a procedure accepts only specific numbers and types of arguments.
%Many syntactic forms similarly restrict the values to which one or
%more of their subforms can evaluate.
%These restrictions imply responsibilities\mainindex{responsibility} for
%both the programmer and the implementation.
%Specifically, the programmer is responsible for ensuring
%that the values indeed adhere to the restrictions described
%in the specification.  The implementation must check
%that the restrictions in the specification are indeed met, to the
%extent that it is reasonable, possible, and necessary to allow the
%specified operation to complete successfully.  The implementation's
%responsibilities are specified in more detail in
%chapter~\ref{entryformatchapter} and throughout the report.
Многие процедуры, определённые в данной работе или в качестве части стандартной библиотеки,
накладывают ограничения на принимаемые ими аргументы. Обычно процедура принимает только
конкретное количество и конкретные типы аргументов. Аналогично, многие синтаксические формы
накладывают ограничения на значения результатов вычислений одной или более своих подформ. Эти
ограничения подразумевают обязанности\mainindex{responsibility} и программиста, и реализации. А
именно, программист обязан гарантировать, что значения действительно удовлетворяют описанным в
спецификации ограничениям. Реализация должна проверить, что ограничения спецификации
действительно выполнены в том смысле, что предоставление возможности успешного завершения
указанной операции является разумным, возможным и необходимым. Обязанности реализации определены
более подробно в главе~\ref{entryformatchapter} и повсюду в данной работе.

%Note that it is not always possible for an implementation to completely check
%the restrictions set forth in a specification.  For example, if an
%operation is specified to accept a procedure with specific properties,
%checking of these properties is undecidable in general.  Similarly,
%some operations accept both lists and procedures that are
%called by these operations.  Since lists can be mutated by the procedures
%through the \rsixlibrary{mutable-pairs} library (see library
%chapter~\extref{lib:pairmutationchapter}{Mutable pairs}), an argument that is a list
%when the operation starts may become a non-list during the execution of the operation.
%Also, the procedure might escape to a different continuation,
%preventing the operation from performing more checks.
%Requiring the operation to check that the argument is a list after
%each call to such a procedure would be impractical.  Furthermore, some
%operations that accept lists only need to traverse these lists
%partially to perform their function; requiring the implementation to
%traverse the remainder of the list to verify that all specified
%restrictions have been met might
%violate reasonable performance assumptions.  For these reasons, the
%programmer's obligations may exceed the checking obligations of the
%implementation.
Необходимо отметить, что у реализации не всегда имеется возможность полной проверки
сформулированных в спецификации ограничений. Например, если операция определена для приёма
процедуры с конкретными свойствами, проверка этих свойств неразрешима в принципе. Аналогично,
некоторые операции принимают как списки, так и процедуры, которые вызывают эти операции. Так как
списки могут быть видоизменены процедурами с помощью библиотеки \textbf{\rsixlibrary{mutable-pairs}}
(см. библиотечную главу~\extref{lib:pairmutationchapter}{Изменяемые пары}), аргумент, являвшийся
списком при запуске операции, может перестать быть списком во время выполнения операции. К тому
же, процедура может перейти к другому продолжению, что не позволит операции произвести больше
проверок. Требовать от операции проверки, что аргумент является списком после каждого вызова
такой процедуры было бы непрактично. Кроме того, некоторым операциям, принимающим только
списки, для выполнения своей функции необходимо частичное прохождение по данным спискам; требование
реализации проходить остаток списка для проверки выполнения всех указанных ограничений
может нарушить разумные условия производительности. По этим причинам обязанности программистов
могут превышать обязанности проверки реализацией.

%When an implementation detects a violation of a restriction for an
%argument, it must raise an exception with condition type
%{\cf\&assertion} in a way consistent with the safety of execution as
%described in section~\ref{safetysection}.
При обнаружении реализацией нарушения ограничения аргумента она должна возбудить исключение
с типом состояния {\cf\&assertion} совместимым с безопасностью выполнения образом, как описано в
секции ~\ref{safetysection}.\vspace{3mm}

%\section{Syntax violations}
\section{Нарушения синтаксиса}\vspace{3mm}

%The subforms of a special form usually need to obey certain syntactic
%restrictions.  As forms may be subject to macro expansion, which may
%not terminate, the question of whether they obey the specified
%restrictions is undecidable in general.
Подформы специальной формы обычно должны подчиняться некоторым синтаксическим
ограничениям. Поскольку формы могут быть представлены макро-разворачиванием, возможно, не
завершённым, вопрос их подчинения указанным ограничениям неразрешим в принципе.

%When macro expansion terminates, however, implementations must detect
%violations of the syntax.  A \defining{syntax violation} is an error
%with respect to the syntax of library bodies, top-level bodies,
%or the ``\exprtype'' entries in the
%specification of the base library or the standard libraries.
%Moreover, attempting to assign to an immutable variable (i.e., the
%variables exported by a library; see
%section~\ref{importsareimmutablesection}) is also
%considered a syntax violation.
При окончании макро-разворачивания, однако, реализации должны обнаруживать нарушения
синтаксиса. \defining{Нарушение синтаксиса} - ошибка относительно синтаксиса библиотечных тел,
тел верхнего уровня, или ``\exprtype'' записей в спецификации базовой библиотеки или
стандартных библиотек. Кроме того, попытка присваивания неизменяемой переменной (то есть,
переменным, экспортируемым библиотекой; см. секцию~\ref{importsareimmutablesection}) также
считается нарушением синтаксиса.

\newpage

%If a top-level or library form in a program is not syntactically
%correct, then the implementation must raise an exception with
%condition type {\cf\&syntax}, and execution of that top-level program
%or library must not be allowed to begin.
Если форма верхнего уровня или библиотеки в программе не является синтаксически корректной,
реализация должна возбудить исключение с типом состояния {\bfseries\cf\&syntax}, и не должна
позволить начать выполнение такой программы или библиотеки верхнего уровня.\vspace{-2mm}

%\section{Safety}
\section{Безопасность}\vspace{-1mm}
\label{safetysection}

%The standard libraries whose exports are described by this document
%are said to be \defining{safe libraries}.  Libraries and top-level
%programs that import only from safe libraries are also said to be safe.
Стандартные библиотеки, экспорт которых описан в данном документе, называются
\defining{безопасными библиотеками}. Библиотеки и программы верхнего уровня,
импортирующие только из безопасных библиотек, также называются безопасными.

%As defined by this document, the Scheme programming language
%is safe in the following sense:
%The execution of a safe top-level program
%cannot go so badly wrong as to crash or to continue to
%execute while behaving in ways that are
%inconsistent with the semantics described in this document,
%unless an exception is raised.
Как определено данным документом, язык программирования Scheme безопасен в
следующем смысле: выполнение безопасной программы верхнего уровня не может происходить
настолько неверно, чтобы привести к аварийному завершению или продолжению выполнения,
функционирование которого противоречит семантике, описанной в данном документе, если исключение
не возбуждено.

%Violations of an implementation restriction must raise an
%exception with condition type {\cf\&implementation-\hp{}restriction},
%as must all
%violations and errors that would otherwise threaten system
%integrity in ways that might result in execution that is
%inconsistent with the semantics described in this document.
Нарушения ограничений реализации должны возбуждать исключение с типом состояния
{\bfseries\cf\&implementation-\hp{}restriction}, как должны все нарушения и ошибки, которые в
противном случае угрожали бы целостности системы таким образом, что это могло бы привести к
выполнению, противоречащему семантике, описанной в данном документе.

%The above safety properties are guaranteed only for top-level programs
%and libraries that are said to be safe.  In particular,
%implementations may provide access to unsafe libraries in ways that
%cannot guarantee safety.
Вышеупомянутые свойства безопасности гарантируются только для программ верхнего уровня и
библиотек, которые называются безопасными. В частности, реализации могут обеспечивать
доступ к опасным библиотекам способами, которые не могут гарантировать безопасность.\vspace{-2mm}

%\section{Boolean values}
\section{Булевы значения}\vspace{-1mm}
\label{booleanvaluessection}

%Although there is a separate boolean type, any Scheme value can be
%used as a boolean value for the purpose of a conditional test.  In a
%conditional test, all values count as true in such a test except for
%\schfalse{}.  This report uses the word ``true'' to refer to any
%Scheme value except \schfalse{}, and the word ``false'' to refer to
%\schfalse{}. \mainindex{true} \mainindex{false}
Хотя существует отдельный булевый тип, любое значение Scheme может использоваться в качестве
булевого для проверки условия. В проверке условия все значения считаются истинными в таком тесте,
за исключением {\bfseries\schfalse{}}. В данной работе используется слово ``true'' для обращения
к любому значению Scheme, кроме {\bfseries\schfalse{}}, и слово ``false'' для обращения к
{\bfseries\schfalse{}}. \mainindex{true}\mainindex{false}\vspace{-2mm}

%\section{Multiple return values}
\section{Несколько возвращаемых значений}\vspace{-1mm}
\label{multiplereturnvaluessection}

%A Scheme expression can evaluate to an arbitrary finite number of
%values.  These values are passed to the expression's continuation.
Выражение Scheme может вычисляться, как произвольное конечное количество значений. Эти значения
передаются продолжению выражения.

%Not all continuations accept any number of values. For example, a continuation that
%accepts the argument to a procedure call is guaranteed to accept
%exactly one value.  The effect of passing some other number of values
%to such a continuation is unspecified.  The {\cf call-with-values}
%procedure
%described in section~\ref{controlsection} makes it possible to create
%continuations that accept specified numbers of return values.
%If the number of
%return values passed to a continuation created by a call to
%{\cf call-with-values} is not accepted by its consumer
%that was passed in that call, then an exception is raised.
%A more complete description of the number of values accepted by
%different continuations and the consequences of passing an unexpected
%number of values is given in the description of the {\cf values}
%procedure in section~\ref{values}.
Не все продолжения принимают несколько значений. Например, продолжение, принимающее аргумент
вызова процедуры, гарантированно примет ровно одно значение. Результат передачи любого другого
количества значений такому продолжению не определён. Процедура {\bfseries\cf call-with-values},
описанная в секции~\ref{controlsection}, позволяет создать продолжения, принимающие указанное
количество возвращаемых значений. Если количество возвращаемых значений, переданных продолжению,
созданному вызовом {\bfseries\cf call-with-values}, не принято его потребителем, переданным в
этом вызове, возбуждается исключение. Более полное описание количества значений, принимаемых
различными продолжениями и последствия передачи неожидаемого количества значений приведено в
описании процедуры {\bfseries\cf values} в секции~\ref{values}.

%A number of forms in the base library have sequences of expressions
%as subforms that are evaluated sequentially, with the return values of
%all but the last expression being discarded.  The continuations
%discarding these values accept any number of values.
Во множестве форм основной библиотеки содержатся последовательности выражений в качестве подформ,
вычисляемых последовательно, со всеми возвращаемыми ими значениями, кроме последнего, не учитывемого
выражения. Продолжения, не учитывающие эти значения, принимают любое количество значений.\vspace{-4mm}

%\section{Unspecified behavior}
\section{Неопределённое поведение}\vspace{-2mm}

%\vest If an expression is said to ``return unspecified values'',
%then the expression must evaluate without raising an exception, but
%the values returned depend on the implementation; this report
%explicitly does not say how many or what values should be returned.
%Programmers should not rely on a specific number of return values or
%the specific values themselves.
%\mainindex{unspecified behavior}\mainindex{unspecified values}
\vest Если сообщается, что выражение ``возвращает неопределённые значения'', выражение должно
вычисляться без возбуждения исключения, но возвращаемые значения зависят от реализации; в данной
работе явно не оговаривается, сколько или какие значения должны возвращаться. Программисты не
должны полагаться на конкретное количество возвращаемых значений или непосредственно конкретных
значений. \mainindex{unspecified behavior}\mainindex{unspecified values}\vspace{-4mm}

%\section{Storage model}
\section{Модель памяти}\vspace{-2mm}
\label{storagemodel}

%Variables and objects such as pairs, vectors, bytevectors, strings,
%hashtables, and records implicitly
%refer to locations\mainindex{location} or sequences of locations.  A string, for
%example, contains as many locations as there are characters in the string.
%(These locations need not correspond to a full machine word.) A new value may be
%stored into one of these locations using the {\tt string-set!} procedure, but
%the string contains the same locations as before.
Переменные и объекты типа пар, векторов, байтовых векторов, строк, хэш-таблиц и записей, неявно
адресуются к областям памяти\mainindex{location} или последовательностям областей памяти. Строка,
например, содержит столько областей памяти, сколько символов в строке. (Эти области памяти не
обязаны соответствовать полному машинному слову.) Новое значение может быть сохранено в одной из
этих областей памяти с помощью процедуры {\bfseries\tt string-set!}, но строка содержит те же
области памяти, как и прежде.

%An object fetched from a location, by a variable reference or by
%a procedure such as {\cf car}, {\cf vector-ref}, or {\cf string-ref}, is
%equivalent in the sense of \ide{eqv?} % and \ide{eq?} ??
%(section~\ref{equivalencesection})
%to the object last stored in the location before the fetch.
Объект, считанный из области памяти обращением к переменной или процедурой типа {\bfseries\cf
  car}, {\bfseries\cf vector-ref} или {\bfseries\cf string-ref}, эквивалентен в смысле
\textbf{\ide{eqv?}} (секция~\ref{equivalencesection}) последнему сохранённому в области памяти перед
считыванием объекту.

%Every location is marked to show whether it is in use.
%No variable or object ever refers to a location that is not in use.
%Whenever this report speaks of storage being allocated for a variable
%or object, what is meant is that an appropriate number of locations are
%chosen from the set of locations that are not in use, and the chosen
%locations are marked to indicate that they are now in use before the variable
%or object is made to refer to them.
Каждая область памяти помечается признаком её использования. Ни переменная, ни объект, никогда
не адресуются к неиспользуемой области памяти. При каждом упоминании в данной работе памяти,
выделяемой для переменной или объекта, имеется в виду, что соответствующее количество областей
памяти выбрано из множества неиспользуемых областей памяти, и выбранные области памяти
помечаются признаком их использования перед обращением к ним переменной или объекта.

%It is desirable for constants\index{constant} (i.e. the values of
%literal expressions) to reside in read-only memory.  To express this,
%it is convenient to imagine that every object that refers to locations
%is associated with a flag telling whether that object is
%mutable\index{mutable} or immutable\index{immutable}.  Literal
%constants, the strings returned by \ide{symbol->string}, records with
%no mutable fields, and other values explicitly designated as immutable
%are immutable objects, while all objects created by the other
%procedures listed in this report are mutable.  An attempt to store a
%new value into a location referred to by an immutable object
%should raise an exception with condition type {\cf\&assertion}.
Хранить константы\index{constant} (то есть значения литеральных выражений) целесообразно в
памяти только для чтения. Чтобы выразить это, удобно представить, что каждый объект,
адресующийся к области памяти, связан с флагом, указывающим, является ли объект
изменяемым\index{mutable} или неизменяемым\index{immutable}. Литеральные константы, строки,
возвращаемые {\bfseries\ide{symbol->string}}, записи без изменяемых полей и другие значения,
явно определённые неизменяемыми являются неизменяемыми объектами, в то время как все объекты,
созданные другими процедурами, перечисленными в данной работе - изменяемыми. Попытка сохранить
новое значение в области памяти, адресуемое неизменяемым объектом, должна возбудить исключение с
типом состояния {\bfseries\cf\&assertion}.\vspace{-2mm}

%\section{Proper tail recursion}
\section{Чистая хвостовая рекурсия}\vspace{-1mm}
\label{proper tail recursion}

%Implementations of Scheme must be
%{\em properly tail-recursive}\mainindex{proper tail recursion}.
%Procedure calls that occur in certain syntactic
%contexts called \textit{tail contexts}\index{tail context}
%are \textit{tail calls}\mainindex{tail call}.  A Scheme implementation is
%properly tail-recursive if it supports an unbounded number of active
%tail calls.  A call is {\em active} if the called procedure may still
%return.  Note that this includes regular returns as well as returns
%through continuations captured earlier by
%{\cf call-with-current-continuation} that are later invoked.
%In the absence of captured continuations, calls could
%return at most once and the active calls would be those that had not
%yet returned.
%A formal definition of proper tail recursion can be found
%in Clinger's paper~\cite{propertailrecursion}.  The rules for identifying tail calls
%in constructs from the \rsixlibrary{base} library are described in
%section~\ref{basetailcontextsection}.
Реализации Scheme должны быть {\em чистыми хвост-рекурсивными}\mainindex{proper tail
  recursion}. Вызовы процедур, находящихся в определённых синтаксических контекстах, называемых
\textit{хвостовыми контекстами}\index{tail context}, являются \textit{хвостовыми
  вызовами}\mainindex{tail call}. Реализация Scheme обладает свойством чистой хвостовой
рекурсии, если она поддерживает неограниченное количество активных хвостовых вызовов. Вызов
является {\em активным}, если вызываемая процедура может все еще возвращаться. Отметьте, что это
включает регулярные возвращения так же как возвращения через продолжения, захваченные ранее
{\bfseries\cf call-with-current-continuation}, вызываемые позже. В отсутствии захваченных
продолжений, вызовы могли возвратиться максимум один раз, и активными будут вызовы, которые еще
не возвратились. Формальное определение чистой хвостовой рекурсии может быть найдено в газете
Клайнджера~\cite{propertailrecursion}. Правила идентификации хвостовых вызовов в конструкциях из
библиотеки \textbf{\rsixlibrary{base}} описаны в секции ~\ref{basetailcontextsection}.\vspace{-2mm}

%\section{Dynamic extent and the dynamic environment}
\section{Динамический экстент и динамическое окружение}\vspace{-1mm}
\label{dynamicenvironmentsection}

%For a procedure call, the time between when it is initiated and when
%it returns is called its \defining{dynamic extent}.  In Scheme, {\cf
%  call-with-current-continuation}
%(section~\ref{call-with-current-continuation}) allows reentering a
%dynamic extent after its procedure call has returned.  Thus, the
%dynamic extent of a call may not be a single, connected time period.
Для вызова процедуры, интервал между её запуском и возвращением, называется её \defining
{динамическим экстентом}. В Scheme, {\cf\bfseries call-with-current-continuation}
(секция~\ref{call-with-current-continuation}) позволяет ввести заново динамический экстент после
возвращения вызова её процедуры. Таким образом, динамический экстент вызова может не быть
единственным, связанным периодом времени.

%Some operations described in the report acquire information in
%addition to their explicit arguments from the \defining{dynamic
%  environment}.  For example, {\cf call-\hp{}with-\hp{}current-\hp{}continuation}
%accesses an implicit context established
%by {\cf dynamic-wind} (section~\ref{dynamic-wind}), and the {\cf
%  raise} procedure (library
%section~\extref{lib:exceptionssection}{Exceptions}) accesses the
%current exception handler.  The operations that modify the dynamic
%environment do so dynamically, for the dynamic extent of a call to a
%procedure like {\cf dynamic-wind} or {\cf with-exception-handler}.
%When such a call returns, the previous dynamic environment is
%restored.  The dynamic environment can be thought of as part of the
%dynamic extent of a call.  Consequently, it is captured by {\cf
%  call-with-current-continuation}, and restored by invoking the escape
%procedure it creates.
Некоторые операции, описанные в работе, помимо своих явных аргументов, получают информацию
из \defining{динамического окружения}. Например, {\bfseries\cf
  call-\hp{}with-\hp{}current-\hp{}continuation} получает доступ к неявному контексту,
установленному {\bfseries\cf dynamic-wind} (секция~\ref{dynamic-wind}), а процедура {\bfseries\cf raise}
(библиотечная секция~\extref{lib:exceptionssection}{Exceptions}) получает доступ к текущему
обработчику исключения. Операции, изменяющие динамическое окружение, делают
это динамически, для динамического экстента вызова процедуры типа {\bfseries\cf dynamic-wind} или
{\bfseries\cf with-exception-handler}. При возвращении такого вызова восстанавливается предыдущее динамическое
окружение. Динамическое окружение можно считать частью
динамического экстента вызова. Следовательно, оно захватывается {\bfseries\cf call-with-current-continuation},
и восстанавливается вызывом создающей его управляющей процедуры.\vspace{-2mm}


%%% Local Variables:
%%% mode: latex
%%% TeX-master: "r6rs"
%%% End:
   \par
%\vfill\eject
%\chapter{Entry format}
\chapter{Формат записи}
\label{entryformatchapter}

%The chapters that describe bindings in the base library and the standard
%libraries are organized
%into entries.  Each entry describes one language feature or a group of
%related features, where a feature is either a syntactic construct or a
%built-in procedure.  An entry begins with one or more header lines of the form
Главы, описывающие привязки в основной и стандартных библиотеках,
упорядочены записями. Каждая запись описывает одну языковую характеристику или группу
связанных характеристик, где характеристика является или синтаксической конструкцией, или
встроенной процедурой. Запись начинается с одной или более строк заголовка вида

%\newpage

\noindent\pproto{\var{template}}{\var{category}}\unpenalty

%The \var{category} defines the kind of binding described by the entry,
%typically either ``\exprtype'' or ``procedure''.
%An entry may specify various restrictions on subforms or arguments.
%For background on this, see section~\ref{argumentcheckingsection}.
\var{category} определяет вид привязки, описываемой записью, обычно или ``\exprtype'', или
``procedure''. В записи могут указываться различные ограничения подформ или аргументов. Вводную
информацию о них см. в секции~\ref{argumentcheckingsection}.\vspace{2mm}

%\section{Syntax entries}
\section{Запись синтаксиса}\vspace{2mm}

%If \var{category} is ``\exprtype'', the entry describes a
%special syntactic construct, and the template gives the syntax of the
%forms of the construct.
%The template is written in a notation similar to a right-hand
%side of the BNF rules in chapter~\ref{readsyntaxchapter}, and describes
%the set of forms equivalent to the forms matching the
%template as syntactic data.  Some ``\exprtype'' entries carry a
%suffix ({\cf expand}), specifying that the syntactic keyword of the
%construct is exported with level
%$1$.  Otherwise, the syntactic keyword is exported with level $0$; see
%section~\ref{phasessection}.
Если \var{category} -- ``\exprtype'', запись описывает специальную синтаксическую
конструкцию, а шаблон демонстрирует синтаксис форм конструкции. Шаблон записывается в
нотации, аналогичной правой части правил BNF в главе~\ref{readsyntaxchapter}, и описывает
набор форм, эквивалентных формам, соответствующим шаблону в качестве синтаксических
данных. Некоторые записи ``\exprtype'' содержат суффикс ({\bfseries\cf expand}), что указывает
на экспорт синтаксического ключевого слова конструкции с уровнем $1$. В противном случае
синтаксическое ключевое слово экспортируется с уровнем $0$; см. секцию~\ref{phasessection}.\vspace{2mm}

%Components of the form described by a template are designated
%by syntactic variables, which are written using angle brackets, for
%example, \hyper{expression}, \hyper{variable}.  Case is insignificant
%in syntactic variables.  Syntactic variables
%stand for other forms, or
%sequences of them.  A syntactic variable may refer to a non-terminal
%in the grammar for syntactic data (see section~\ref{datumsyntax}),
%in which case only forms matching
%that non-terminal are permissible in that position.
%For example, \hyper{identifier} stands for a form which must be an
%identifier.
%Also,
%\hyper{expression} stands for any form which is a
%syntactically valid expression.  Other non-terminals that are used in
%templates are defined as part of the specification.
Компоненты описываемой шаблоном формы обозначаются синтаксическими переменными, заключёнными
в угловые скобки, например, \hyper{expression}, \hyper{variable}. В синтаксических переменных
регистр является незначащим. Синтаксические переменные представляют другие формы или их
последовательности. Синтаксическая переменная может обращаться к нетерминальному символу в
грамматике синтаксических данных (см. секцию~\ref{datumsyntax}) только в случае, если формы,
соответствующие нетерминальному символу, допустимы в этой позиции. Например, \hyper{identifier}
представляет форму, которая должна быть идентификатором. Кроме того, \hyper{expression}
представляет любую форму, являющуюся синтаксически верным выражением. Другие используемые
в шаблонах нетерминальные символы определены как часть спецификации.\vspace{2mm}

%The notation
Нотация
\begin{tabbing}
\qquad \hyperi{thing} $\ldots$
\end{tabbing}
%indicates zero or more occurrences of a \hyper{thing}, and
означает ноль или более вхождений \hyper{thing}, а
\begin{tabbing}
\qquad \hyperi{thing} \hyperii{thing} $\ldots$
\end{tabbing}
%indicates one or more occurrences of a \hyper{thing}.
означает одно или более вхождение \hyper{thing}\vspace{2mm}

%It is the programmer's responsibility to ensure that each component of
%a form has the shape specified by a template.  Descriptions of syntax
%may express other restrictions on the components of a form.
%Typically, such a restriction is formulated as a phrase of the form
%``\hyper{x} must be\mainindex{must be} a \ldots''.  Again, these
%specify the programmer's responsibility.  It is the implementation's
%responsibility to check that these restrictions are satisfied, as long
%as the macro transformers involved in expanding the form terminate.
%If the implementation detects that a component does not meet the
%restriction, an exception with condition type {\cf\&syntax} is raised.
Обеспечение каждого компонента формы шейпом, определённым в шаблоне, является обязанностью
программиста. В описании синтаксиса могут формулироваться и другие ограничения на компоненты формы.
Обычно такое ограничение формулируется фразой вида ``\hyper{x} должен быть\mainindex{must be}
\ldots''. Опять же, это подразумевает обязанности программистов. Обязанностью реализации является
проверка удовлетворения этих ограничений при условии завершения работы участвующих в
разворачивании формы макротрансформеров. При обнаружении реализацией невыполнения компонентом
ограничения генерируется исключение с типом состояния {\bfseries\cf\&syntax}.\vspace{2mm}

%\newpage

%\section{Procedure entries}
\section{Запись процедур}

%If \var{category} is ``procedure'', then the entry describes a procedure, and
%the header line gives a template for a call to the procedure.  Parameter
%names in the template are \var{italicized}.  Thus the header line
Если \var{category} -- ``procedure'', запись описывает процедуру, а строка заголовка
демонстрирует шаблон вызова процедуры. Имена параметров в шаблоне выделены \var{курсивом}.
Таким образом, строка заголовка\vspace{1mm}

%\noindent\pproto{(vector-ref \var{vector} \var{k})}{procedure}\unpenalty
\noindent\pproto{\textbf{(vector-ref} \var{vector} \var{k}\textbf{)}}{procedure}\unpenalty

%indicates that the built-in procedure {\tt vector-ref} takes
%two arguments, a vector \var{vector} and an exact non-negative integer
%object \var{k} (see below).  The header lines
указывает, что встроенная процедура {\bfseries\tt vector-ref} принимает два аргумента, вектор
\var{vector} и точный неотрицательный целый объект \var{k} (см. ниже). Строки заголовка

\noindent%
%\pproto{(make-vector \var{k})}{procedure}
%\pproto{(make-vector \var{k} \var{fill})}{procedure}\unpenalty
\pproto{\textbf{(make-vector} \var{k}\textbf{)}}{procedure}
\pproto{\textbf{(make-vector} \var{k} \var{fill}\textbf{)}}{procedure}\unpenalty

%indicate that the {\tt make-vector} procedure takes
%either one or two arguments.  The parameter names are
%case-insensitive: \var{Vector} is the same as \var{vector}.
указывает, что процедура {\bfseries\tt make-vector} принимает один или два
аргумента. Имена параметров нечувствительны к регистру: \var{Vector} тождественен
\var{vector}.

%As with syntax templates, an ellipsis \dotsfoo{} at the end of a header
%line, as in
Как и в случае синтаксических шаблонов, многоточие \dotsfoo{} в конце строки заголовка, как например

\noindent\pproto{\textbf{(=} \vari{z} \varii{z} \variii{z} \dotsfoo\textbf{)}}{procedure}\unpenalty

%indicates that the procedure takes arbitrarily many arguments of the
%same type as specified for the last parameter name.  In this case,
%{\cf =} accepts two or more arguments that must all be complex
%number objects.
указывает, что процедура принимает произвольное количество аргументов того же типа, который
указан для последнего имени параметра. В данном случае {\bfseries\cf =} принимает два или более
аргумента, которые должны быть комплексными числовыми объектами.\vspace{1mm}

\label{typeconventions}
%A procedure that detects an argument that it is not specified to
%handle must raise an exception with condition type
%{\cf\&assertion}.  Also, the argument specifications are exhaustive: if the
%number of arguments provided in a procedure call does not match
%any number of arguments accepted by the procedure, an exception with
%condition type {\cf\&assertion} must be raised.
Процедура, обнаружившая не определённый для обработки аргумент, должна возбудить
исключение с типом состояния {\bfseries\cf\&assertion}. Кроме того, спецификации аргументов
являются полными: если количество аргументов, предоставляемых в вызове процедуры, не
соответствует любому принимаемому процедурой количеству аргументов, должно быть возбуждено
исключение с типом состояния {\bfseries\cf\&assertion}.\vspace{1mm}

%For succinctness, the report follows the convention
%that if a parameter name is also the name of a type, then the corresponding argument must be of the named type.
%For example, the header line for {\tt vector-ref} given above dictates that the
%first argument to {\tt vector-ref} must be a vector.  The following naming
%conventions imply type restrictions:
Для краткости в стандарте выполняется соглашение, что если имя параметра является также и
именем типа, соответствующий аргумент должен иметь именованный тип. Например, строка
заголовка для {\bfseries\tt vector-ref}, приведённая выше, предписывает, что первый аргумент
{\bfseries\tt vector-ref} должен быть вектором. Следующие соглашения именования
подразумевают ограничения типов:\vspace{1mm}

%\texonly\begin{center}\endtexonly
%  \begin{tabular}{ll}
%    \var{obj}&any object\\
%    \var{z}&complex number object\\
%    \var{x}&real number object\\
%    \var{y}&real number object\\
%    \var{q}&rational number object\\
%    \var{n}&integer object\\
%    \var{k}&exact non-negative integer object\\
%    \var{bool}&boolean (\schfalse{} or \schtrue{})\\
%    \var{octet}&exact integer object in $\{0, \ldots, 255\}$\\
%    \var{byte}&exact integer object in $\{-128, \ldots, 127\}$\\
%    \var{char}&character (see section~\ref{charactersection})\\
%    \var{pair}&pair (see section~\ref{listsection})\\
%    \var{vector}&vector (see section~\ref{vectorsection})\\
%    \var{string}&string (see section~\ref{stringsection})\\
%    \var{condition}&condition (see library section~\extref{lib:conditionssection}{Conditions})\\
%    \var{bytevector}&bytevector (see library chapter~\extref{lib:bytevectorschapter}{Bytevectors})\\
%    \var{proc}&procedure (see section~\ref{proceduressection})
%  \end{tabular}
%\texonly\end{center}\endtexonly
\texonly\begin{center}\endtexonly
  \begin{tabular}{ll}
    \var{obj}&любой объект\\
    \var{z}&комплексный числовой объект\\
    \var{x}&действительный числовой объект\\
    \var{y}&действительный числовой объект\\
    \var{q}&рациональный числовой объект\\
    \var{n}&целый объект\\
    \var{k}&точный неотрицательный целый объект\\
    \var{bool}&булево значение ({\bfseries\schfalse{}} или \bfseries\schtrue{})\\
    \var{octet}&точный целый объект диапазона $\{0, \ldots, 255\}$\\
    \var{byte}&точный целый объект диапазона $\{-128, \ldots, 127\}$\\
    \var{char}&символ (см. секцию~\ref{charactersection})\\
    \var{pair}&пара (см. секцию~\ref{listsection})\\
    \var{vector}&вектор (см. секцию~\ref{vectorsection})\\
    \var{string}&строка (см. секцию~\ref{stringsection})\\
    \var{condition}&условие (см. библиотечную секцию~\extref{lib:conditionssection}{Conditions})\\
    \var{bytevector}&байт-вектор (см. библиотецный раздел~\extref{lib:bytevectorschapter}{Bytevectors})\\
    \var{proc}&процедура (см. секцию~\ref{proceduressection})
  \end{tabular}
\texonly\end{center}\endtexonly

%Other type restrictions are expressed through parameter-naming
%conventions that are described in specific chapters.  For example,
%library chapter~\extref{lib:numberchapter}{Arithmetic} uses a number of special
%parameter variables for the various subsets of the numbers.
Другие ограничения типов выражаются посредством соглашений именований параметров, описанных в
конкретных главах. Например, в библиотечной главе~\extref{lib:numberchapter}{Arithmetic}
используется множество специальных параметрических переменных для различных подмножеств
чисел.\vspace{2.1mm}

%With the listed type restrictions, it is the programmer's responsibility to
%ensure that the corresponding argument is of the specified type.
%It is the implementation's responsibility to check for
%that type.
При перечисленных ограничениях типа обязанностью программиста является гарантия, что
соответствующий аргумент имеет указанный тип. Проверка этого типа является обязанностью
реализации.\vspace{2.1mm}

%A parameter called \var{list} means that it is the
%programmer's responsibility to pass an argument that is a list (see
%section~\ref{listsection}).  It is the implementation's responsibility
%to check that the argument is appropriately structured for the
%operation to perform its function, to the extent that this is possible
%and reasonable.  The implementation must at least check that the
%argument is either an empty list or a pair.
Параметр, называемый \var{list}, означает, что обязанностью программиста является передача
аргумента, который является списком (см. секцию~\ref{listsection}). Обязанностью реализации
является проверка того, что аргумент так структуирован для выполнения функции операции, что это
возможно и разумно. Реализация должна проверить, что аргумент является по крайней мере или
пустым списком, или парой.\vspace{2.1mm}

%Descriptions of procedures may express other restrictions on the
%arguments of a procedure.  Typically, such a restriction is formulated
%as a phrase of the form ``\var{x} must be a \ldots'' (or otherwise
%using the word ``must'').
Описания процедур могут выражать другие ограничения аргументов процедур. Обычно такое
ограничение формулируется фразой вида ``\var{x} должна быть \ldots'' (или иначе
с помощью слова ``должна'').\vspace{2.1mm}

%\section{Implementation responsibilities}
\section{Обязанности реализации}\vspace{2.1mm}

%In addition to the restrictions implied by naming conventions, an
%entry may list additional explicit restrictions.
%These explicit restrictions usually describe both the
%programmer's responsibilities, who must ensure that the subforms of a
%form are appropriate, or that an appropriate
%argument is passed, and the implementation's responsibilities, which
%must check that subform adheres to the specified restrictions (if
%macro expansion terminates), or if the argument is appropriate.  A description
%may explicitly list the implementation's responsibilities for some
%arguments or subforms in a paragraph labeled ``\textit{Implementation
%  responsibilities}''.  In this case, the responsibilities specified
%for these subforms or arguments in the rest of the description are only for the
%programmer.  A paragraph describing implementation responsibility does not
%affect the implementation's responsibilities for checking subforms or arguments not
%mentioned in the paragraph.
Кроме ограничений, подразумеваемых соглашениями именования, в записи могут перечисляться
дополнительные явные ограничения. Эти явные ограничения обычно описывают и обязанности
программиста, который должен гарантировать, что подформы формы являются соответствующими, или
что передаётся соответствующий аргумент, и обязанности реализаций, которые должны проверить, что
подформа соблюдает указанные ограничения (если разворачивание макросов завершено), или если
аргумент является соответствующим. В описании могут явно перечисляться обязанности реализации
для некоторых аргументов или подформ в параграфе, помеченном ``\textit {Обязанности
  реализации}''. В этом случае обязанности, указанные для этих подформ или аргументов в
остальной части описания, касаются только программиста. Параграф, описывающий обязанность
реализации, не затрагивает обязанности реализации по проверке подформ или аргументов, не
упомянутых в параграфе.\vspace{2.1mm}

%\section{Other kinds of entries}
\section{Другие виды записей}\vspace{2.1mm}

%If \var{category} is something other than ``syntax'' and
%``procedure'', then the entry describes a non-procedural value, and
%the \var{category} describes the type of that value.  The header line
Если \var{category} -- нечто иное, чем ``syntax'' и ``procedure'', запись описывает
непроцедурное значение, и \var{category} описывает тип этого значения. Строка
заголовка\vspace{2.1mm}

\noindent\rvproto{\bfseries\&who}{condition type}\\
%indicates that {\cf\&who} is a condition type.  The header line
указывает, что {\bfseries\cf\&who} является типом состояния. Строка заголовка

\noindent\rvproto{\bfseries unquote}{auxiliary syntax}\\
%indicates that {\cf unquote} is a syntax binding that may occur
%only as part of specific surrounding expressions.  Any use as an
%independent syntactic construct or identifier is a syntax violation.
%As with ``\exprtype'' entries, some ``auxiliary syntax'' entries  carry a
%suffix ({\cf expand}), specifying that the syntactic keyword of the
%construct is exported with level $1$.
%\section{Equivalent entries}
указывает, что {\bfseries\cf unquote} является синтаксической привязкой, которая может
фигурировать только как часть конкретных окружающих выражений. Любое применение
в качестве независимой синтаксической конструкции или идентификатора является нарушением
синтаксиса. Как и в случае с записями ``\exprtype'', некоторые записи ``вспомогательного синтаксиса''
содержат суффикс ({\bfseries\cf expand}), что указывает на экспорт синтаксического ключевого слова
конструкции с уровнем $1$.\vspace{1mm}

\section {Эквивалентные записи}\vspace{1mm}

%The description of an entry occasionally states that it is \textit{the
%  same} as another entry.  This means that both entries are
%equivalent.  Specifically, it means that if both entries have the same
%name and are thus exported from different libraries, the entries from
%both libraries can be imported under the same name without conflict.
В описании записи иногда утверждается, что она \textit{аналогична} другой записи. Это
означает, что обе записи эквивалентны. Реально это означает, что, если обе записи имеют
одинаковые имена и при этом экспортируются из различных библиотек, записи
могут импортироваться из обеих библиотек под одинаковыми именами без конфликта.\vspace{1mm}

%\section{Evaluation examples}
\section{Примеры вычислений}\vspace{1mm}

%The symbol ``\evalsto'' used in program examples can be read
%``evaluates to''.  For example,
Символ ``\evalsto'' применяется в примерах программ и может быть прочитан ``вычисляется
как''. Например,\vspace{1mm}

\begin{scheme}
\bfseries(* 5 8)      \ev\bfseries  40%
\end{scheme}\vspace{1mm}

%means that the expression {\tt(* 5 8)} evaluates to the object {\tt 40}.
%Or, more precisely:  the expression given by the sequence of characters
%``{\tt(* 5 8)}'' evaluates, in an environment that imports the relevant library, to an object
%that may be represented externally by the sequence of characters ``{\tt
%40}''.  See section~\ref{datumsyntaxsection} for a discussion of external
%representations of objects.
означает, что выражение {\bfseries\tt (* 5 8)} вычисляется как объект {\bfseries\tt 40}. Или,
точнее: заданное последовательностью символов выражение ``{\bfseries\tt (* 5 8)}'' в окружении,
импортирующем релевантную библиотеку, вычисляется как объект, который может быть представлен
внешне последовательностью символов ``{\bfseries\tt 40}''. См. секцию ~\ref{datumsyntaxsection},
где рассматриваются внешние представления объектов.\vspace{1mm}

%The ``\evalsto'' symbol is also used when the evaluation of an
%expression causes a violation.  For example,
Если вычисление выражения приводит к нарушению, также применяется символ ``\evalsto''.
Например,\vspace{1mm}

\begin{scheme}
\bfseries(integer->char \sharpsign{}xD800) \xev \exception{\bfseries\&assertion}%
\end{scheme}\vspace{1mm}
%
%means that the evaluation of the expression {\cf (integer->char
%  \sharpsign{}xD800)} must raise an exception with condition type
%{\cf\&assertion}.
означает, что вычисление выражения {\bfseries\cf (integer->char \sharpsign{}xD800)}
должно возбудить исключение с типом состояния {\bfseries\cf\&assertion}.\vspace{1mm}

%Moreover, the ``\evalsto'' symbol is also used to explicitly say that
%the value of an expression in unspecified.  For example:
Кроме того, символ ``\evalsto'' также используется для явного указания, что значение выражения
не определено. Например:\vspace{1mm}
%
\begin{scheme}
\bfseries(eqv? "" "")             \ev  \unspecified%
\end{scheme}\vspace{1mm}

%Mostly, examples merely illustrate the behavior specified in the
%entry.  In some cases, however, they disambiguate otherwise ambiguous
%specifications and are thus normative.  Note that, in some cases,
%specifically in the case of inexact number objects, the return value is only
%specified conditionally or approximately.  For example:
В основном примеры просто демонстрируют определённую записью функциональность. Однако в
некоторых случаях они устраняют неоднозначность двусмысленных в ином случае спецификаций и,
таким образом, нормативны. Необходимо отметить, что в некоторых случаях, как правило в случае
неточных числовых объектов, возвращаемое значение указывается только условно или
приблизительно. Например:\vspace{1mm}
%
\begin{scheme}
\bfseries(atan -inf.0)                  \lev \textbf{-1.5707963267948965} ; \textrm{approximately}%
\end{scheme}

%\newpage

%\section{Naming conventions}
\section{Соглашения по именованию}

%By convention, the names of procedures that store values into previously
%allocated locations (see section~\ref{storagemodel}) usually end in
%``\ide{!}''.
В соответствии с соглашением имена процедур, сохраняющих значения в ранее выделенных
областях памяти (см. секцию ~\ref{storagemodel}), обычно завершаются ``\ide{\bfseries !}''.

%By convention, ``\ide{->}'' appears within the names of procedures that
%take an object of one type and return an analogous object of another type.
%For example, {\cf list->vector} takes a list and returns a vector whose
%elements are the same as those of the list.
В соответствии с соглашением ``\ide{\bfseries ->}'' присутствует внутри имён процедур,
принимающих объект одного типа, и возвращающих аналогичный объект другого типа. Например,
{\cf\bfseries list->vector} принимает список и возвращает вектор, элементы которого совпадают с
таковыми из списка.

%By convention, the names of predicates---procedures that always return
%a boolean value---end in ``\ide{?}'' when the name contains any
%letters; otherwise, the predicate's name does not end with a question
%mark.
В соответствии с соглашением имена предикатов --- процедур, всегда возвращающих
булево значение --- оканчиваются ``\ide{\bfseries ?}'', если имя содержит произвольные буквы; в
противном случае имя предиката не оканчивается вопросительным знаком.

%By convention, the components of compound names are separated by ``\ide{-}''
%In particular, prefixes that are actual words or can be pronounced as
%though they were actual words are followed by a hyphen, except when
%the first character following the hyphen would be something other than
%a letter, in which case the hyphen is omitted.  Short,
%unpronounceable prefixes (``\ide{fx}'' and ``\ide{fl}'') are not
%followed by a hyphen.
В соответствии с соглашением компоненты составных имён разделяются ``\ide{\bfseries -}''. В
частности, после приставок, являющихся фактическими словами или которые могут быть произнесены,
как если бы они были фактическими словами, ставится дефис, кроме тех случаев, когда первый
символ после дефиса не является буквой, в этом случае дефис не ставится. После коротких,
труднопроизносимых приставок (``\ide{\bfseries fx}'' и ``\ide{\bfseries fl}'') дефис не
ставится.

%By convention, the names of condition types start with
В соответствии с соглашением имена типов состояния начинаются с
``{\bfseries\cf\&}''\index{&@\texttt{\&}}.

%%% Local Variables:
%%% mode: latex
%%% TeX-master: "r6rs"
%%% End:
 \par
%\chapter{Libraries}
\chapter{Библиотеки}
\label{librarychapter}
\mainindex{library}
%Libraries are parts of a program that can be distributed
%independently.
%The library system supports macro definitions within libraries,
%macro exports, and distinguishes the phases in which definitions
%and imports are needed.  This chapter defines the notation for
%libraries and a semantics for library expansion and execution.
Библиотеки являются частями программы, которые могут поставляться независимо. Система библиотек
поддерживает макроопределения внутри библиотек, макроэкспорт, а также различает фазы, в которых
необходимы определения и импорт. В данной главе описана нотация библиотек и семантика расширения
и реализации библиотек.

%\section{Library form}
\section{Библиотечная форма}
\label{librarysyntaxsection}

%A library definition must have the following form:\mainschindex{library}\mainschindex{import}\mainschindex{export}
Определение библиотеки имеет следующую форму:\mainschindex{library}\mainschindex{import}\mainschindex{export}

\begin{scheme}
(\textbf{library} \hyper{library~name}
  (\textbf{export} \hyper{export~spec} \ldots)
  (\textbf{import} \hyper{import~spec} \ldots)
  \hyper{library~body})%
\end{scheme}

%A library declaration contains the following elements:
Объявление библиотеки содержит следующие элементы:

\begin{itemize}
%\item The \hyper{library~name} specifies the name of the library
%  (possibly with version).
\item \hyper{library~name} указывает имя библиотеки (возможно с версией).
%\item The {\cf export} subform specifies a list of exports, which name
%  a subset of the bindings defined within or imported into the
%  library.
\item Подформа {\cf\bfseries export} указывает список экспорта, именующий подмножество
  привязок, определённых внутри или импортированных в библиотеку.
%\item The {\cf import} subform specifies the imported bindings as a
%  list of import dependencies, where each dependency specifies:
\item Подформа {\cf\bfseries import} указывает импортированные привязки в виде списка
  зависимостей импорта, где каждая зависимость указывает:
\begin{itemize}
%\item the imported library's name, and, optionally, constraints on its
%  version,
\item имя импортируемой библиотеки и, при необходимости, ограничения её версии,
%\item the relevant levels, e.g., expand or run time (see
%  section~\ref{phasessection}, and
\item релевантные уровни, например, время разворачивания или выполнения (см
  секцию~\ref{phasessection}, и
%\item the subset of the library's exports to make available within the
%      importing library, and the local names to use within the importing
%      library for each of the library's exports.
\item подмножество библиотечных экспортов, чтобы сделать доступный внутри библиотеки
  импортирования, и локальных имён, для использования внутри библиотеки
  импортирования для каждого из экспортов библиотеки.
\end{itemize}
%\item The \hyper{library body} is the library body, consisting of a
%  sequence of definitions followed by a sequence of expressions.  The
%  definitions may be both for local (unexported) and exported
%  bindings, and the expressions are initialization expressions to be evaluated
%  for their effects.
\item \hyper{library body} - тело библиотеки, состоящее из последовательности определений
  и следующей за ней последовательности выражений. Определения могут быть как для локальных
  (неэкспортируемых), так и для экспортируемых связываний, а выражения являются
  инициализирующими выражениями, которые будут вычисляться для их эффектов.
\end{itemize}

%An identifier can be imported with the same local name from two or
%more libraries or for two levels from the same library only if the
%binding exported by each library is the same (i.e., the binding is
%defined in one library, and it arrives through the imports only by
%exporting and re-exporting).  Otherwise, no identifier can be imported
%multiple times, defined multiple times, or both defined and imported.
%No identifiers are visible within a library except for those
%explicitly imported into the library or defined within the library.
Идентификатор может импортироваться с тем же локальным именем из двух или более библиотек, или
для двух уровней из той же библиотеки только в случае, если привязки, экспортируемые каждой
библиотекой, тождественны (то есть, связывание определено в одной библиотеке, и оно поступает
только через импорт, экспортируя и реэкспортируя). В противном случае идентификатор не может
быть импортирован несколько раз, определён несколько раз, или определён и
импортирован. Идентификаторы не видимы внутри библиотеки, за исключением явно импортированных в
библиотеку или определённых внутри библиотеки.

%A \hyper{library name} uniquely identifies a library within an
%implementation, and is globally visible in the {\cf import} clauses
%(see below) of all other libraries within an implementation.
%A \hyper{library name} has the following form:
\hyper{library name} однозначно идентифицирует библиотеку внутри реализации и является глобально
видимым в разделах {\cf\bfseries import} (см. ниже) всех других библиотек внутри
реализации. \hyper{library name} имеет следующую форму:

\begin{scheme}
(\hyperi{identifier} \hyperii{identifier} \ldots \hyper{version})%
\end{scheme}

%where \hyper{version} is empty or has the following form:
где \hyper{version} является пустым или принимает следующую форму:
%
\begin{scheme}
(\hyper{sub-version} \ldots)%
\end{scheme}

%Each \hyper{sub-version} must represent an exact nonnegative integer object.
%An empty \hyper{version} is equivalent to {\cf ()}.
Каждый \hyper{sub-version} должен представлять точный неотрицательный целый объект.
Пустой \hyper{version} эквивалентен {\cf ()}.

%An \hyper{export~spec} names a set of imported and locally defined bindings to
%be exported, possibly with different
%external names.  An \hyper{export~spec} must have one of the
%following forms:
\hyper{export~spec} именует совокупность импортированных и локально определённых привязок
для экспортирования, возможно с другими внешними именами. \hyper{export~spec} должен иметь
одну из следующих форм:

\begin{scheme}
\hyper{identifier}
(\textbf{rename} (\hyperi{identifier} \hyperii{identifier}) \ldots)%
\end{scheme}

%In an \hyper{export~spec}, an \hyper{identifier} names a single binding defined
%within or imported into the library, where the external name for the export is
%the same as the name of the binding within the library.
%A {\cf rename} spec exports the binding named by
%\hyperi{identifier} in each {\cf (\hyperi{identifier}
%  \hyperii{identifier})} pairing, using \hyperii{identifier} as the
%external name.
В \hyper{export~spec}, \hyper{identifier} именует одиночную привязку, определённую внутри или
импортированную в библиотеку, причём внешнее имя экспорта совпадает с именем привязки внутри
библиотеки. Спецификация {\cf\bfseries rename} экспортирует привязку с именем
\hyperi{identifier} в каждой паре {\cf (\hyperi{identifier} \hyperii{identifier})},
используя \hyperii{identifier} в качестве внешнего имени.

%Each \hyper{import~spec} specifies a set of bindings to be imported into
%the library, the levels at which they are to be available, and the local
%names by which they are to be known.  An \hyper{import spec} must
%be one of the following:
Каждый \hyper{import~spec} указывает множество привязок, импортируемых в
библиотеку, уровни, на которых они должны быть доступны и локальные имена, которыми
они должны быть названы. \hyper{import spec} должен быть одним из следующего:

\newpage

\begin{scheme}
\hyper{import set}
(\textbf{for} \hyper{import~set} \hyper{import~level} \ldots)%
\end{scheme}

%An \hyper{import level}  is one of the following:
\hyper{import level} является одним из следующего:
\begin{scheme}
\textbf{run}
\textbf{expand}
\textbf{(meta} \hyper{level}\textbf{)}%
\end{scheme}

%where \hyper{level} represents an exact integer object.
Где \hyper{level} представляет точный целый объект.

%As an \hyper{import level}, {\cf run} is an abbreviation for {\cf
%  (meta 0)}, and {\cf expand} is an abbreviation for {\cf (meta 1)}.
%Levels and phases are discussed in section~\ref{phasessection}.
Как \hyper{import level}, {\cf \bfseries run} является сокращением для {\cf\bfseries (meta 0)},
а {\cf\bfseries expand} - сокращением для {\cf\bfseries (meta 1)}. Уровни и фазы описаны в
секции~\ref{phasessection}.

%An \hyper{import~set} names a set of bindings from another library and
%possibly specifies local names for the imported bindings.  It must be
%one of the following:
\hyper{import~set} именует множество привязок из другой библиотеки и, возможно, определяет
локальные имена для импортированных привязок. Он должен быть одним из следующего:

\begin{scheme}
\hyper{library~reference}
\textbf{(library} \hyper{library~reference}\textbf{)}
\textbf{(only} \hyper{import~set} \hyper{identifier} \ldots\textbf{)}
\textbf{(except} \hyper{import~set} \hyper{identifier} \ldots\textbf{)}
\textbf{(prefix} \hyper{import~set} \hyper{identifier}\textbf{)}
\textbf{(rename} \hyper{import~set} \textbf{(}\hyperi{identifier} \hyperii{identifier}\textbf{)} \ldots\textbf{)}%
\end{scheme}

%A \hyper{library~reference} identifies a library by its
%name and optionally by its version.  It has one of the following forms:
\hyper{library~reference} идентифицирует библиотеку её именем и,
произвольно, её версией. Он имеет одну из следующих форм:

\begin{scheme}
\textbf{(}\hyperi{identifier} \hyperii{identifier} \ldots\textbf{)}
\textbf{(}\hyperi{identifier} \hyperii{identifier} \ldots \hyper{version~reference}\textbf{)}%
\end{scheme}

%A \hyper{library~reference} whose first \hyper{identifier} is
%{\cf for}, {\cf library}, {\cf only}, {\cf except}, {\cf prefix}, or {\cf rename} is
%permitted only within a {\cf library} \hyper{import~set}.
%The \hyper{import~set} {\cf (library \hyper{library~reference})} is
%otherwise equivalent to \hyper{library~reference}.
\hyper{library~reference}, чьим первым \hyper{identifier} является {\cf\bfseries for}, {\cf\bfseries
library}, {\cf\bfseries only}, {\cf\bfseries except}, {\cf\bfseries prefix} или {\cf\bfseries
rename}, разрешается только внутри {\cf\bfseries library}
\hyper{import~set}. \hyper{import~set} {\cf\textbf{(library}
  \hyper{library~reference}\textbf{)}} в противном случае
эквивалентен \hyper{library~reference}.

%A \hyper{library~reference} with no \hyper{version~reference}
%(first form above) is equivalent to a \hyper{library~reference} with a
%\hyper{version~reference} of {\cf ()}.
\hyper{library~reference} без \hyper{version~reference} (первая форма выше) эквивалентен
\hyper{library~reference} с \hyper{version~reference} {\cf\bfseries ()}.

%A \hyper{version~reference} specifies a set of \hyper{version}s that
%it matches.  The \hyper{library~reference} identifies all libraries of
%the same name and whose version is matched by the
%\hyper{version~reference}.  A \hyper{version~reference} has
%the following form:
\hyper{version~reference} указывает множество \hyper{version}, которым он
соответствует. \hyper{library~reference} идентифицирует все библиотеки того же самого
имени и чья версия соответствует \hyper{version~reference}. \hyper{version~reference}
имеет следующую форму:
%
\begin{scheme}
\textbf{(}\hyperi{sub-version reference} \ldots \hypern{sub-version reference}\textbf{)}
\textbf{(and} \hyper{version reference} \ldots\textbf{)}
\textbf{(or} \hyper{version reference} \ldots\textbf{)}
\textbf{(not} \hyper{version reference}\textbf{)}%
\end{scheme}
%
%A \hyper{version reference} of the first form matches a \hyper{version}
%with at least $n$ elements, whose \hyper{sub-version reference}s match
%the corresponding \hyper{sub-version}s.  An {\cf and} \hyper{version
%  reference} matches a version if all \hyper{version references}
%following the {\cf and} match it.  Correspondingly, an {\cf
%  or} \hyper{version reference} matches a version if one of
%\hyper{version references} following the {\cf or} matches it,
%and a {\cf not} \hyper{version reference} matches a version if the
%\hyper{version reference} following it does not match it.
\hyper{version reference} первой формы соответствует \hyper{version} с по крайней мере $n$
элементы, \hyper{sub-version reference} которых соответствует
соответств. \hyper{sub-version}. {\cf\bfseries and} \hyper{version reference} соответствует
версии, если все \hyper{version references} после {\cf\bfseries and} соответствуют
этому. Соответственно, {\cf\bfseries or} \hyper{version reference} соответствует версии, если
один из \hyper{version references} после {\cf\bfseries or} соответствуют этому, и {\cf\bfseries
  not} \hyper{version reference} соответствует версии, если \hyper{version references} после
этого не соответствует этому.

%A \hyper{sub-version reference} has one of the following forms:
\hyper{sub-version reference} имеет одну из следующих форм:

\begin{scheme}
\hyper{sub-version}
\textbf{(>=} \hyper{sub-version}\textbf{)}
\textbf{(<=} \hyper{sub-version}\textbf{)}
\textbf{(and} \hyper{sub-version~reference} \ldots\textbf{)}
\textbf{(or} \hyper{sub-version~reference} \ldots\textbf{)}
\textbf{(not} \hyper{sub-version~reference}\textbf{)}%
\end{scheme}

%A \hyper{sub-version reference} of the first form matches a
%\hyper{sub-version} if it is equal to it.  A {\cf >=}
%\hyper{sub-version reference} of the first form matches a sub-version
%if it is greater or equal to the \hyper{sub-version} following it;
%analogously for {\cf <=}.  An {\cf and} \hyper{sub-version reference}
%matches a sub-version if all of the subsequent \hyper{sub-version
%  reference}s match it.  Correspondingly, an {\cf or}
%\hyper{sub-version reference} matches a sub-version if one of the
%subsequent \hyper{sub-version reference}s matches it, and a {\cf not}
%\hyper{sub-version reference} matches a sub-version if the subsequent
%\hyper{sub-version reference} does not match it.
\hyper{sub-version reference} первой формы соответствует \hyper{sub-version}, если они равны.
{\cf\bfseries >=} \hyper{sub-version reference} первой формы соответствует sub-version, если оно
больше или равно \hyper{sub-version} после него; аналогично для {\cf\bfseries
  <=}. {\cf\bfseries and} \hyper{sub-version reference} соответствует sub-version, если все
последующие \hyper{sub-version reference} соответствуют ему. Соответственно, {\cf\bfseries or}
\hyper{sub-version reference} соответствует sub-version, если один из последующих \hyper{sub-version
reference} соответствует ему, и {\cf\bfseries not} \hyper{sub-version reference} соответствует
sub-version, если последующий \hyper{sub-version reference} не соответствует ему.

%Examples:
Примеры:

\texonly\begin{center}\endtexonly
  \begin{tabular}{lll}
    version reference & version & match?
    \\
    {\cf ()} & {\cf (1)} & yes\\
    {\cf (1)} & {\cf (1)} & yes\\
    {\cf (1)} & {\cf (2)} & no\\
    {\cf (2 3)} & {\cf (2)} & no\\
    {\cf (2 3)} & {\cf (2 3)} & yes\\
    {\cf (2 3)} & {\cf (2 3 5)} & yes\\
    {\cf (or (1 (>= 1)) (2))} & {\cf (2)} & yes\\
    {\cf (or (1 (>= 1)) (2))} & {\cf (1 1)} & yes\\
    {\cf (or (1 (>= 1)) (2))} & {\cf (1 0)} & no\\
    {\cf ((or 1 2 3))} & {\cf (1)} & yes\\
    {\cf ((or 1 2 3))} & {\cf (2)} & yes\\
    {\cf ((or 1 2 3))} & {\cf (3)} & yes\\
    {\cf ((or 1 2 3))} & {\cf (4)} & no
  \end{tabular}
\texonly\end{center}\endtexonly

%When more than one library is identified by a library reference, the
%choice of libraries is determined in some implementation-dependent manner.
Если при обращении к библиотеке идентифицировано более одной библиотеки, выбор библиотеки
определяется неким зависимым от реализации методом.

%To avoid problems such as incompatible types and replicated state,
%implementations should prohibit the two libraries whose library names
%consist of the same sequence of identifiers but whose versions do not
%match to co-exist in the same program.
Во избежании проблем, таких, как несовместимость типов и дублирование состояний, реализация
должна запретить сосуществование в одной программе двух библиотек, библиотечные имена которых
состоят из одинаковой последовательности идентификаторов, но версии которых являются
несоответствующими.

%By default, all of an imported library's exported bindings are made
%visible within an importing library using the names given to the bindings
%by the imported library.
%The precise set of bindings to be imported and the names of those
%bindings can be adjusted with the {\cf only}, {\cf except},
%{\cf prefix}, and {\cf rename} forms as described below.
По умолчанию все экспортируемые привязки из импортируемой библиотеки предполагаются видимыми
внутри импортированной библиотеки с именами, присвоенными привязкам в импортируемой
библиотеке. Точный набор импортитуемых привязок и имён этих привязок может быть установлен с
помощью форм {\cf\bfseries only}, {\cf\bfseries except}, {\cf\bfseries prefix} и {\cf\bfseries
rename} как описано ниже.

\begin{itemize}
%\item An {\cf only} form produces a subset of the bindings from another
%\hyper{import~set}, including only the listed
%\hyper{identifier}s.
%The included \hyper{identifier}s must be in
%the original \hyper{import~set}.
\item Форма {\cf\bfseries only} порождает подмножество привязок из иного \hyper{import~set},
  содержащее только перечисленные \hyper{identifier}. Включенные \hyper{identifier} должны
  существовать в исходном \hyper{import~set}.
\item An {\cf except} form produces a subset of the bindings from another
\hyper{import~set}, including all but the listed
\hyper{identifier}s.
All of the excluded \hyper{identifier}s must be in
the original \hyper{import~set}.
\item A {\cf prefix} form adds the \hyper{identifier} prefix to each
name from another \hyper{import~set}.
\item A {\cf rename} form, {\cf (rename (\hyperi{identifier} \hyperii{identifier}) \ldots)},
removes the bindings for {\cf \hyperi{identifier} \ldots} to form an
intermediate \hyper{import~set}, then adds the bindings back for the
corresponding {\cf \hyperii{identifier} \ldots} to form the final
\hyper{import~set}.
Each \hyperi{identifier} must be in the original \hyper{import~set},
each \hyperii{identifier} must not be in the intermediate \hyper{import~set},
and the \hyperii{identifier}s must be distinct.
\end{itemize}
It is a syntax violation if a constraint given above is not met.

\label{librarybodysection}
The \hyper{library~body} of a {\cf library} form consists of forms
that are classified as
\textit{definitions}\mainindex{definition} or
\textit{expressions}\mainindex{expression}.  Which forms belong to
which class depends on the imported libraries and the result of
expansion---see chapter~\ref{expansionchapter}.  Generally, forms that
are not
definitions (see section~\ref{defines} for definitions available
through the base library) are expressions.

A \hyper{library~body} is like a \hyper{body} (see section~\ref{bodiessection}) except that
a \hyper{library~body}s need not include any expressions.  It must
have the following form:

\begin{scheme}
\hyper{definition} \ldots \hyper{expression} \ldots%
\end{scheme}

When {\cf begin}, {\cf let-syntax}, or {\cf letrec-syntax} forms
occur in a top-level body prior to the first
expression, they are spliced into the body; see section~\ref{begin}.
Some or all of the body, including portions wrapped in {\cf begin},
{\cf let-syntax}, or {\cf letrec-syntax}
forms, may be specified by a syntactic abstraction
(see section~\ref{macrosection}).

The transformer expressions and bindings are evaluated and created
from left to right, as described in chapter~\ref{expansionchapter}.
The expressions of variable definitions are evaluated
from left to right, as if in an implicit {\cf letrec*},
and the body expressions are also evaluated from left to right
after the expressions of the variable definitions.
A fresh location is created for each exported variable and initialized
to the value of its local counterpart.
The effect of returning twice to the continuation of the last body
expression is unspecified.

\begin{note}
The names {\cf library}, {\cf export}, {\cf import},
{\cf for}, {\cf run}, {\cf expand}, {\cf meta},
{\cf import}, {\cf export}, {\cf only}, {\cf except}, {\cf
  prefix}, {\cf rename}, {\cf and}, {\cf or}, {\cf not}, {\cf >=}, and {\cf <=}
appearing in the library syntax are part of the
syntax and are not reserved, i.e., the same names can be used for other
purposes within the library or even exported from or imported
into a library with different meanings, without affecting their
use in the {\cf library} form.
\end{note}

Bindings defined with a library are not visible in code
outside of the library, unless the bindings are explicitly exported from the
library.
An exported macro may, however, \emph{implicitly export} an otherwise
unexported identifier defined within or imported into the library.
That is, it may insert a reference to that identifier into the output code
it produces.

\label{importsareimmutablesection}
All explicitly exported variables are immutable in both the
exporting and importing libraries.
It is thus a syntax violation if an
explicitly exported variable appears on the left-hand side of a {\cf set!}
expression, either in the exporting or importing libraries.

All implicitly exported variables are also immutable in both the
exporting and importing libraries.
It is thus a syntax violation if a
variable appears on the left-hand side of a {\cf set!}
expression in any code produced by an exported macro outside of the
library in which the variable is defined.
It is also a syntax violation if a
reference to an assigned variable appears in any code produced by
an exported macro outside of the library in which the variable is defined,
where an assigned variable is one that appears on the left-hand
side of a {\cf set!} expression in the exporting library.

All other variables defined within a library are mutable.

\section{Import and export levels}
\label{phasessection}

Expanding a library may require run-time information from another
library.  For example, if a macro transformer calls a
procedure from library $A$, then the library $A$ must be
instantiated before expanding any use of the macro in library $B$.  Library $A$ may
not be
needed when library $B$ is eventually run as part of a program, or it
may be needed for run time of library $B$, too.  The library
mechanism distinguishes these times by phases, which are explained in
this section.

Every library can be characterized by expand-time information (minimally,
its imported libraries, a list of the exported keywords, a list of the
exported variables, and code to evaluate the transformer expressions) and
run-time information (minimally, code to evaluate the variable definition
right-hand-side expressions, and code to evaluate the body expressions).
The expand-time information must be available to expand references to
any exported binding, and the run-time information must be available to
evaluate references to any exported variable binding.

\mainindex{phase}
%
A \emph{phase} is a time at which the expressions within a library are
evaluated.
Within a library body, top-level expressions and
the right-hand sides of {\cf define} forms are evaluated at run time,
i.e., phase $0$, and the right-hand
sides of {\cf define-syntax} forms are evaluated at expand time, i.e.,
phase $1$.
When {\cf define-syntax},
{\cf let-syntax}, or {\cf letrec-syntax}
forms appear within code evaluated at phase $n$, the right-hand sides
are evaluated at phase $n+1$.

These phases are relative to the phase in which the library itself is
used.
An \defining{instance} of a library corresponds to an evaluation of its
variable definitions and expressions in a particular phase relative to another
library---a process called \defining{instantiation}.
For example, if a top-level expression in a library $B$ refers to
a variable export from another library $A$, then it refers to the export from an
instance of $A$ at phase $0$ (relative to the phase of $B$).
But if a phase $1$ expression within $B$ refers to the same binding from
$A$, then it refers to the export from an instance of $A$ at phase $1$
(relative to the phase of $B$).

A \defining{visit} of a library corresponds to the evaluation of its syntax
definitions in a particular phase relative to another
library---a process called \defining{visiting}.
For example, if a top-level expression in a library $B$ refers to
a macro export from another library $A$, then it refers to the export from a
visit of $A$ at phase $0$ (relative to the phase of $B$), which corresponds
to the evaluation of the macro's transformer expression at phase $1$.


\mainindex{level}\mainindex{import level}
%
A \emph{level} is a lexical property of an identifier that determines
in which phases it can be referenced. The level for each identifier
bound by a definition within a library is $0$; that is, the identifier
can be referenced only at phase $0$ within the library.
The level for each imported binding is determined by the enclosing {\cf
  for} form of the {\cf import} in the importing library, in
addition to the levels of the identifier in the exporting
library. Import and export levels are combined by pairwise addition of
all level combinations.  For example, references to an imported
identifier exported for levels $p_a$ and $p_b$ and imported for levels
$q_a$, $q_b$, and $q_c$ are valid at levels $p_a+q_a$, $p_a+q_b$,
$p_a+q_c$, $p_b+q_a$, $p_b+q_b$, and $p_b+q_c$. An \hyper{import~set}
without an enclosing {\cf for} is equivalent to {\cf (for
  \hyper{import~set} run)}, which is the same as {\cf (for
  \hyper{import~set} (meta 0))}.

The export level of an exported binding is $0$ for all bindings
that are defined within the exporting library. The export levels of a
reexported binding, i.e., an export imported from another library, are the
same as the effective import levels of that binding within the reexporting
library.

For the libraries defined in the library report, the export level is
$0$ for nearly all bindings. The exceptions are {\cf syntax-rules},
{\cf identifier-syntax}, {\cf ...}, and {\cf \_} from the
\rsixlibrary{base} library, which are exported with level $1$, {\cf
  set!} from the \rsixlibrary{base} library, which is exported with
levels $0$ and $1$, and all bindings from the composite
\thersixlibrary{} library (see library
chapter~\extref{lib:complibchapter}{Composite library}), which are
exported with levels $0$ and $1$.

Macro expansion within a library can introduce a reference to an
identifier that is not explicitly imported into the library. In that
case, the phase of the reference must match the identifier's level as
shifted by the difference between the phase of the source library
(i.e., the library that supplied the identifier's lexical context) and
the library that encloses the reference. For example, suppose that
expanding a library invokes a macro transformer, and the evaluation of
the macro transformer refers to an identifier that is exported from
another library (so the phase-$1$ instance of the library is used);
suppose further that the value of the binding is a syntax object
representing an identifier with only a level-$n$ binding; then, the
identifier must be used only at phase $n+1$ in the
library being expanded. This combination of levels and phases is why
negative levels on identifiers can be useful, even though libraries
exist only at non-negative phases.

If any of a library's definitions are referenced at phase $0$ in the
expanded form of a program, then an instance of the referenced library
is created for phase $0$ before the program's definitions and
expressions are evaluated. This rule applies transitively: if the
expanded form of one library references at phase $0$ an identifier
from another library, then before the referencing library is
instantiated at phase $n$, the referenced library must be instantiated
at phase $n$. When an identifier is referenced at any phase $n$
greater than $0$, in contrast, then the defining library is
instantiated at phase $n$ at some unspecified time before the
reference is evaluated. Similarly, when a macro keyword is referenced at
phase $n$ during the expansion of a library, then the
defining library is visited at phase $n$ at some unspecified time
before the reference is evaluated.

An implementation may distinguish instances/visits of a library for
different phases or to use an instance/visit at any phase as an instance/visit at
any other phase. An implementation may further
expand each {\cf library} form with distinct
visits of libraries in any phase and/or instances of
libraries in phases above $0$. An implementation may
create instances/visits of more libraries at more phases than required to
satisfy references. When an identifier appears as an expression in a
phase that is inconsistent with the identifier's level, then an
implementation may raise an exception either at expand time or run
time, or it may allow the reference. Thus, a library whose meaning depends on whether the
instances of a library are distinguished or shared across phases or
{\cf library} expansions may be unportable.

\section{Examples}

Examples for various \hyper{import~spec}s and \hyper{export~spec}s:

\begin{scheme}
(library (stack)
  (export make push! pop! empty!)
  (import (rnrs))

  (define (make) (list '()))
  (define (push! s v) (set-car! s (cons v (car s))))
  (define (pop! s) (let ([v (caar s)])
                     (set-car! s (cdar s))
                     v))
  (define (empty! s) (set-car! s '())))

(library (balloons)
  (export make push pop)
  (import (rnrs))

  (define (make w h) (cons w h))
  (define (push b amt)
    (cons (- (car b) amt) (+ (cdr b) amt)))
  (define (pop b) (display "Boom! ")
                  (display (* (car b) (cdr b)))
                  (newline)))

(library (party)
  ;; Total exports:
  ;; make, push, push!, make-party, pop!
  (export (rename (balloon:make make)
                  (balloon:push push))
          push!
          make-party
          (rename (party-pop! pop!)))
  (import (rnrs)
          (only (stack) make push! pop!) ; not empty!
          (prefix (balloons) balloon:))

  ;; Creates a party as a stack of balloons,
  ;; starting with two balloons
  (define (make-party)
    (let ([s (make)]) ; from stack
      (push! s (balloon:make 10 10))
      (push! s (balloon:make 12 9))
      s))
  (define (party-pop! p)
    (balloon:pop (pop! p))))


(library (main)
  (export)
  (import (rnrs) (party))

  (define p (make-party))
  (pop! p)        ; displays "Boom! 108"
  (push! p (push (make 5 5) 1))
  (pop! p))       ; displays "Boom! 24"%
\end{scheme}

Examples for macros and phases:

\begin{schemenoindent}
(library (my-helpers id-stuff)
  (export find-dup)
  (import (rnrs))

  (define (find-dup l)
    (and (pair? l)
         (let loop ((rest (cdr l)))
           (cond
            [(null? rest) (find-dup (cdr l))]
            [(bound-identifier=? (car l) (car rest))
             (car rest)]
            [else (loop (cdr rest))])))))

(library (my-helpers values-stuff)
  (export mvlet)
  (import (rnrs) (for (my-helpers id-stuff) expand))

  (define-syntax mvlet
    (lambda (stx)
      (syntax-case stx ()
        [(\_ [(id ...) expr] body0 body ...)
         (not (find-dup (syntax (id ...))))
         (syntax
           (call-with-values
               (lambda () expr)
             (lambda (id ...) body0 body ...)))]))))

(library (let-div)
  (export let-div)
  (import (rnrs)
          (my-helpers values-stuff)
          (rnrs r5rs))

  (define (quotient+remainder n d)
    (let ([q (quotient n d)])
      (values q (- n (* q d)))))
  (define-syntax let-div
    (syntax-rules ()
     [(\_ n d (q r) body0 body ...)
      (mvlet [(q r) (quotient+remainder n d)]
        body0 body ...)])))%
\end{schemenoindent}


%%% Local Variables:
%%% mode: latex
%%% TeX-master: "r6rs"
%%% End:
 \par
%\chapter{Top-level programs}
\chapter{Программы верхнего уровня}
\label{programchapter}

%A \defining{top-level program} specifies an entry point for defining and running
%a Scheme program.  A top-level program specifies a set of libraries to import and
%code to run.  Through the imported libraries, whether directly or through the
%transitive closure of importing, a top-level program defines a complete Scheme
%program.
\defining{Программа верхнего уровня} задаёт точку входа к определению и выполнению программы
Scheme. В программе верхнего уровня указывается набор импортируемых библиотек и исполняемый код.
Через импортированные библиотеки непосредственно, или через транзитивное замыкание
импортирования, программа верхнего уровня определяет полную программу Scheme.

%\section{Top-level program syntax}
\section{Синтаксис программ верхнего уровня}
\label{programsyntaxsection}

%A top-level program is a delimited piece of text, typically a file, that
%has the following form:
Программа верхнего уровня является ограниченной частью текста, обычно файлом, имеющим следующую
форму:
%
\begin{scheme}
\hyper{import form} \hyper{top-level body}%
\end{scheme}
%
%An \hyper{import form} has the following form:
\hyper{import form} имеет следующую форму:
%
\begin{scheme}
(\textbf{import} \hyper{import spec} \dotsfoo)%
\end{scheme}
%
%A \hyper{top-level body} has the following form:
\hyper{top-level body} имеет следующую форму:
\begin{scheme}
\hyper{top-level body form} \dotsfoo%
\end{scheme}
%
%A \hyper{top-level body form} is either a \hyper{definition} or an
%\hyper{expression}.
\hyper{top-level body form} является или \hyper{definition} или
\hyper{expression}.

%The \hyper{import form} is identical to the import clause in
%libraries (see section~\ref{librarysyntaxsection}),
%and specifies a set of libraries to import.  A \hyper{top-level
% body} is like a \hyper{library body} (see
%section~\ref{librarybodysection}), except that
%definitions and expressions may occur in any order.  Thus, the syntax
%specified by \hyper{top-level body form} refers to the result of macro
%expansion.
\hyper{import form} идентичен разделу import в библиотеках
(см. секцию~\ref{librarysyntaxsection}) и задаёт набор импортируемых библиотек. \hyper{top-level body}
похож на \hyper{library body} (см. секцию~\ref{librarybodysection}), за исключением того, что
определения и выражения могут находиться в произвольном порядке. Таким образом, синтаксис,
задаваемый \hyper{top-level body form} относится к результату разворачивания макросов.

%When uses of {\cf begin}, {\cf let-syntax}, or {\cf letrec-syntax}
%from the \rsixlibrary{base} library
%occur in a top-level body prior to the first
%expression, they are spliced into the body; see section~\ref{begin}.
%Some or all of the body, including portions wrapped in {\cf begin},
%{\cf let-syntax}, or {\cf letrec-syntax}
%forms, may be specified by a syntactic abstraction
%(see section~\ref{macrosection}).
Если используются {\bfseries\cf begin}, {\bfseries\cf let-syntax} или {\bfseries\cf
  letrec-syntax} из библиотеки \textbf{\rsixlibrary{base}}, находящиеся в теле верхнего уровня
до первого выражения, они соединяются с телом; см. секцию~\ref{begin}. Часть или всё тело,
включая части, обернутые в формы {\bfseries\cf begin}, {\bfseries\cf let-syntax} или
{\bfseries\cf letrec-syntax}, может быть определено синтаксической абстракцией
(см. секцию~\ref{macrosection}).

%\section{Top-level program semantics}
\section{Семантика программы верхнего уровня}\vspace{1mm}

%A top-level program is executed by treating the program similarly to a library, and
%evaluating its definitions and expressions.
%The semantics of a top-level body may be roughly explained by
%a simple translation into a library body:
%Each \hyper{expression} that appears before a
%definition in
%the top-level body is converted into a dummy definition
Программа верхнего уровня выполняется путём обработки программы аналогично библиотеке, и вычисления её
определений и выражений. Семантику тела верхнего уровня можно грубо интерпретировать простым
переводом в тело библиотеки: Каждый \hyper{expression}, находящееся перед определением
в теле верхнего уровня, преобразовывается в фиктивное определение\vspace{2mm}
%
\begin{scheme}
\textbf{(define} \hyper{variable} \textbf{(begin} \hyper{expression} \hyper{unspecified}\textbf{))}%
\end{scheme}\vspace{2mm}
%
%where \hyper{variable} is a fresh identifier and \hyper{unspecified}
%is a side-effect-free expression returning an unspecified value.
%(It is generally impossible to determine which forms are
%definitions and expressions without concurrently expanding the body, so
%the actual translation is somewhat more complicated; see
%chapter~\ref{expansionchapter}.)
где \hyper{variable} -- новый идентификатор, а \hyper{unspecified} -- выражение без
побочного эффекта, возвращающее неопределённое значение. (В ообщем случае невозможно определить, какие формы
являются определениями и выражениями, одновременно не разворачивая тело, таким образом, фактическая
трансляция несколько более сложна; см. главу~\ref{expansionchapter}.)\vspace{1mm}

%On platforms that support it, a top-level program may access its command line
%by calling the {\cf command-line} procedure (see library
%section~\extref{lib:command-line}{Command-line access and exit values}).
На платформах, которые поддерживают это, программа верхнего уровня может получить доступ к её
командной строке вызовом процедуры {\bfseries\cf command-line} (см. библиотечную
секцию~\extref{lib:command-line}{Command-line access and exit values}).


%%% Local Variables:
%%% mode: latex
%%% TeX-master: "r6rs"
%%% End:
 \par
%\chapter{Primitive syntax}
\chapter{Синтаксис примитивов}\vspace{2mm}

%After the {\cf import} form within a {\cf library} form or a top-level
%program, the forms
%that constitute the body of the library or the top-level program
%depend on the libraries that are
%imported. In particular, imported syntactic keywords determine
%the available syntactic abstractions and whether each form is a
%definition or expression. A few form types are
%always available independent of imported libraries, however,
%including constant literals, variable references, procedure calls,
% and macro uses.
После формы {\bfseries\cf import} внутри формы {\bfseries\cf library} или программы верхнего уровня,
формы, составляющие тело библиотеки или программы верхнего уровня, зависят от
импортированных библиотек.
В частности, импортированные синтаксические ключевые слова определяют
доступные синтаксические абстракции и является ли каждая форма определением или
выражением. Некоторые типы форм всегда доступны независимо от импортированных библиотек,
однако, включая константные литералы, обращения к переменным, вызовы
процедур и применения макросов.

%\section{Primitive expression types}
\section{Типы примитивных выражений}
\label{primitiveexpressionsection}

%The entries in this section all describe expressions, which may occur
%in the place of \hyper{expression} syntactic variables.  See
%also section~\ref{expressionsection}.
Все записи в этой секции описывают выражения, которые могут находиться на месте синтаксических
переменных \hyper{expression}. См. также секцию~\ref{expressionsection}.

%\subsection*{Constant literals}\unsection
\subsection*{Константные литералы}\unsection

\begin{entry}{%
\pproto{\hyper{number}}{\exprtype}
\pproto{\hyper{boolean}}{\exprtype}
\pproto{\hyper{character}}{\exprtype}
\pproto{\hyper{string}}{\exprtype}
\pproto{\hyper{bytevector}}{\exprtype}}\mainindex{literal}\vspace{1mm}

%An expression consisting of a representation of a number object, a
%boolean, a character, a string, or a bytevector, evaluates ``to
%itself''.
Выражение, состоящее из представления числового объекта, булевого значения, символа, строки или
байт-вектора, вычисляется ``как есть''.\vspace{1mm}

\begin{scheme}
\bfseries 145932     \ev  \bfseries 145932
\bfseries \schtrue   \ev  \bfseries \schtrue
\bfseries "abc"      \ev  \bfseries "abc"
\bfseries \#vu8(2 24 123) \ev \bfseries\#vu8(2 24 123)%
\end{scheme}\vspace{1mm}

%As noted in section~\ref{storagemodel}, the value of a literal
%expression is immutable.
Как отмечено в секции ~\ref{storagemodel}, значение литерального выражения является неизменяемым.
\end{entry}

%\subsection*{Variable references}\unsection
\subsection*{Обращения к переменным}\unsection
\begin{entry}{%
\pproto{\hyper{variable}}{\exprtype}}\vspace{1mm}

%An expression consisting of a variable\index{variable}
%(section~\ref{variablesection}) is a variable reference if it is not a
%macro use (see below).  The value of
%the variable reference is the value stored in the location to which the
%variable is bound.  It is a syntax violation to reference
%an unbound\index{unbound} variable.
Выражение, состоящее из переменной\index{variable} (секция~\ref{variablesection}), является
обращением к переменной в случае, если оно не является применением макроса (см. ниже). Значением
обращения к переменной является значение,
хранящееся в области памяти, с которой связана переменная.
Обращение к непривязанной \index{unbound} переменной является нарушением синтаксиса.

%The following example examples assumes the base library
%has been imported:
В следующих примерах предполагается, что основная библиотека импортирована:\vspace{1mm}
%
\begin{scheme}
\bfseries(define x 28)
\bfseries x   \ev  \bfseries 28%
\end{scheme}\vspace{1mm}
\end{entry}

%\subsection*{Procedure calls}\unsection
\subsection*{Вызовы процедур}\unsection

\begin{entry}{%
\pproto{\textbf{(}\hyper{operator} \hyperi{operand} \dotsfoo\textbf{)}}{\exprtype}}\vspace{1mm}

%A procedure call consists of expressions for the procedure to be
%called and the arguments to be passed to it, with enclosing
%parentheses.  A form in an expression context is a procedure call if
%\hyper{operator} is not an identifier bound as a syntactic keyword
%(see section~\ref{macrosection} below).
Вызов процедуры состоит из заключённых в круглые скобки выражений вызова процедуры и передаваемых ей
аргументов. Форма в контексте выражения является вызовом процедуры, если \hyper{operator} не
является идентификатором, привязанным к синтаксическому ключевому слову
(см. секцию~\ref{macrosection} ниже).

%When a procedure call is evaluated, the operator and operand
%expressions are evaluated (in an unspecified order) and the resulting
%procedure is passed the resulting
%arguments.\mainindex{call}\mainindex{procedure call}
При вычислении вызова процедуры вычисляются (в произвольном порядке) выражения оператора и операндов
и полученной процедуре передаются полученные аргументы.\mainindex{call}\mainindex{procedure
  call}

%The following examples assume the \rsixlibrary{base} library
%has been imported:
В следующих примерах предполагается, что библиотека \textbf{\rsixlibrary{base}} импортирована.\vspace{1mm}
%
\begin{scheme}%
\bfseries (+ 3 4)                          \ev\bfseries 7
\bfseries ((if \schfalse + *) 3 4)         \ev\bfseries 12%
\end{scheme}\vspace{1mm}
%
%If the value of \hyper{operator} is not a procedure, an exception with
%condition type {\cf\&assertion} is raised.  Also, if \hyper{operator}
%does not accept as many arguments as there are \hyper{operand}s, an
%exception with condition type {\cf\&assertion} is raised.

Если значение \hyper{operator} не является процедурой, возбуждается исключение с типом состояния
{\cf\bfseries\&assertion}. Если количество \hyper{operand} превышает количество принимаемых
\hyper{operand} аргументом, также возбуждается исключение с типом состояния {\cf\bfseries\&assertion}.

\begin{note} %In contrast to other dialects of Lisp, the order of
%evaluation is unspecified, and the operator expression and the operand
%expressions are always evaluated with the same evaluation rules.
В отличие от других диалектов Lisp порядок вычисления не определён, и выражение оператора, и
выражения операнда всегда вычисляются с теми же правилами вычисления.

%Although the order of evaluation is otherwise unspecified, the effect of
%any concurrent evaluation of the operator and operand expressions is
%constrained to be consistent with some sequential order of evaluation.
%The order of evaluation may be chosen differently for each procedure call.
Хотя порядок вычисления иначе не определён, эффект любого параллельного вычисления выражений
оператора и операнда ограничен быть совместимым с некоторым последовательным порядком
вычисления. Порядок вычисления может быть выбран другим при каждом вызове процедуры.
\end{note}

%\newpage

\begin{note} %In many dialects of Lisp, the form {\tt
%()} is a legitimate expression.  In Scheme, expressions written as
%list/pair forms must have at
%least one subexpression, so {\tt ()} is not a syntactically valid
%expression.
Во многих диалектах Lisp форма {\tt\textbf{()}} является допустимым выражением. В Scheme
выражения, записанные как формы списков/пар, должны иметь по крайней мере одно подвыражение, таким
образом, {\tt\textbf{()}} не является синтаксически допустимым выражением.
\end{note}

\end{entry}

%\section{Macros}
\section{Макросы}\vspace{1mm}
\label{macrosection}

%Libraries and top-level programs can define and use new kinds of derived expressions and
%definitions called {\em syntactic abstractions} or
%{\em macros}.\mainindex{syntactic abstraction}\mainindex{macro}
%A syntactic abstraction is created by binding a keyword to a
%{\em macro transformer} or, simply, {\em transformer}.
%\index{macro transformer}\index{transformer}
%The transformer determines
%how a use of the macro (called a \defining{macro use})
%is transcribed into a more primitive form.
В библиотеках и программах верхнего уровня могут определяться и использоваться новые виды
производных выражений и определений, называемых {\em синтаксическими абстракциями} или {\em
  макросами}.\mainindex{syntactic abstraction} \mainindex {macro} Синтаксическая абстракция
создаётся путём привязки ключевого слова к {\em макротрансформеру}, или просто {\em
  трансформеру}.\index{macro transformer}\index{macro transformer} Трансформер
определяет, как применение макроса (называемое \defining{макроприменением}) расшифровывается в
более примитивную форму.\vspace{1mm}

%Most macro uses have the form:
Большинство макросов имеют форму:\vspace{1mm}
\begin{scheme}
\textbf{(}\hyper{keyword} \hyper{datum} \dotsfoo\textbf{)}%
\end{scheme}\vspace{1mm}%
%where \hyper{keyword} is an identifier that uniquely determines the
%kind of form.  This identifier is called the {\em syntactic
%keyword}\index{syntactic keyword}, or simply {\em
%keyword}\index{keyword}, of the macro\index{macro keyword}.
%The number of \hyper{datum}s and the syntax
%of each depends on the syntactic abstraction.
где \hyper{keyword} -- идентификатор, уникально определяющий вид формы. Этот
идентификатор называется {\em синтаксическим ключевым словом}\index{syntactic keyword},
или просто {\em ключевым словом}\index{keyword}, макроса\index{macro keyword}. Количество
\hyper{datum} и синтаксис каждого из них зависит от синтаксической абстракции.\vspace{1mm}

%Macro uses can also take the form of improper lists, singleton
%identifiers, or {\cf set!} forms, where the second subform of the
%{\cf set!} is the keyword (see section~\ref{identifier-syntax})
%library section~\extref{lib:make-variable-transformer}{{\cf make-variable-transformer}}):
Макроприменения могут также принимать форму нестрогих списков, еденичных идентификаторов
или форм {\cf\bfseries set!}, где вторая подформа {\cf\bfseries set!} является
ключевым словом (см. секцию~\ref{identifier-syntax}), библиотечную
секцию~\extref{lib:make-variable-transformer} {{\cf\bfseries make-variable-transformer}}):\vspace{1mm}
\begin{scheme}
\textbf{(}\hyper{keyword} \hyper{datum} \dotsfoo . \hyper{datum}\textbf{)}
\hyper{keyword}
\textbf{(}set! \hyper{keyword} \hyper{datum}\textbf{)}%
\end{scheme}\vspace{1mm}

%The {\cf define-syntax}, {\cf let-syntax} and {\cf letrec-syntax}
%forms, described in sections~\ref{define-syntax} and \ref{let-syntax},
%create bindings for keywords, associate them with macro transformers,
%and control the scope within which they are visible.
Формы {\cf\bfseries define-syntax}, {\cf\bfseries let-syntax} и {\cf\bfseries letrec-syntax},
описанные в секциях~\ref{define-syntax} и \ref{letrec-syntax}, создают привязки для ключевых
слов, связывают их с макротрансформерами, и задают область, внутри которых они видимы.\vspace{1mm}

%The {\cf syntax-rules} and {\cf identifier-syntax} forms, described in
%section~\ref{syntaxrulessection}, create transformers via a pattern
%language.  Moreover, the {\cf syntax-case} form, described in library
%chapter~\extref{lib:syntaxcasechapter}{{\cf syntax-case}},
%allows creating transformers via arbitrary Scheme code.
Формы {\cf\bfseries syntax-rules} и {\cf\bfseries identifier-syntax}, описанные в секции
~\ref{syntaxrulessection}, создают трансформеры посредством языка шаблонов. Кроме того, форма
{\cf\bfseries syntax-case}, описанная в библиотечной главе~\extref{lib:syntaxcasechapter}
{{\cf\bfseries syntax-case}}, позволяет создавать трансформеры посредством произвольного кода
Scheme.\vspace{1mm}

%Keywords occupy the same name space as variables.
%That is, within the same
%scope, an identifier can be bound as a variable or keyword, or neither, but
%not both, and local bindings of either kind may shadow other bindings of
%either kind.
Ключевые слова занимают то же пространство имён, что и переменные. Таким образом,
внутри той же области видимости, идентификатор может привязываться как переменная или
ключевое слово, или ни один, но не оба, и локальные привязки любого вида могут маскировать другие
привязки любого вида.\vspace{1mm}

%Macros defined using {\cf syntax-rules} and {\cf identifier-\hp{}syntax}
%are ``hygienic'' and ``referentially transparent'' and thus preserve
%Scheme's lexical scoping~\cite{Kohlbecker86,
%  hygienic,Bawden88,macrosthatwork,syntacticabstraction}:
%\mainindex{hygienic} \mainindex{referentially transparent}
Макрос, определённый с помощью {\cf\bfseries syntax-rules} и {\cf\bfseries
  identifier-\hp{}syntax}, является ``гигиеничным'' и ``референциально прозрачным'' и, таким
образом, предохраняет лексическую сферу действия
Scheme~\cite{Kohlbecker86,hygienic,Bawden88,macrosthatwork,syntacticabstraction}:\mainindex{hygienic}
\mainindex{referentially transparent}

\begin{itemize}
%\item If a macro transformer inserts a binding for an identifier
%(variable or keyword) not appearing in the macro use, the identifier is in effect renamed
%throughout its scope to avoid conflicts with other identifiers.
\item Если макротрансформер вносит привязку для идентификатора (переменной или ключевого
слова), отсутствующего в макроприменении, идентификатор используется переименованным
по всей своей области видимости для предотвращения конфликтов с другими идентификаторами.

%\item If a macro transformer inserts a free reference to an
%identifier, the reference refers to the binding that was visible
%where the transformer was specified, regardless of any local
%bindings that may surround the use of the macro.
\item Если макротрансформер вносит свободное обращение к идентификатору, обращение относится к
  привязке, которая была видима в месте определения трансформера, независимо от всех
  локальных привязкок, которые могут окружать применение макроса.
\end{itemize}

%Macros defined using the {\cf syntax-case} facility are also
%hygienic unless {\cf datum\coerce{}syntax}
%(see library section~\extref{lib:conversionssection}{Syntax-object and datum conversions}) is
%used.
Макрос, определённый с помощью средства {\cf\bfseries syntax-case},
также является гигиеничным, кроме случаев, когда используется {\cf\bfseries
  datum\coerce{}syntax} (см. библиотечную секцию~\extref{lib:conversionssection}{Syntax-object
  and datum conversions}).

%%% Local Variables:
%%% mode: latex
%%% TeX-master: "r6rs"
%%% End:
 \par
%\chapter{Expansion process}
\chapter{Процесс разворачивания}
\label{expansionchapter}

%Macro uses (see section~\ref{macrosection}) are expanded into \textit{core
%forms}\mainindex{core form} at the start of evaluation (before compilation or interpretation)
%by a syntax \emph{expander}.
%The set of core forms is implementation-dependent, as is the
%representation of these forms in the expander's output.
%If the expander encounters a syntactic abstraction, it invokes
%the associated transformer to expand the syntactic abstraction, then
%repeats the expansion process for the form returned by the transformer.
%If the expander encounters a core form, it recursively
%processes its subforms that are in expression or definition context,
%if any, and reconstructs the form from the
%expanded subforms.
%Information about identifier bindings is maintained during expansion
%to enforce lexical scoping for variables and keywords.
Макроприменения (см. секцию~\ref{macrosection}) разворачиваются в \textit{основные формы}
\mainindex{core form} в начале вычисления (до компиляции или интерпретации)
синтаксическим \emph{экспандером}. Набор основных форм зависит от реализации, как и
представление этих форм на выходе экспандера. Если экспандер
встречает синтаксическую абстракцию, он запускает ассоциированный трансформер для
разворачивания синтаксической абстракции, затем повторяет процесс разворачивания формы, возвращённой
трансформером. Если экспандер встречает основную форму, он рекурсивно
обрабатывает её подформы, находящиеся в выражении или контексте определения, если таковые
имеются, и восстанавливает форму из развёрнутых подформ. Информация о привязках
идентификаторов сохраняется в течение разворачивания для обеспечения лексической сферы действия
переменных и ключевых слов.

%To handle definitions, the expander processes the initial
%forms in a \hyper{body} (see section~\ref{bodiessection}) or
%\hyper{library body} (see section~\ref{librarybodysection})
%from left to
%right.  How the expander processes each form encountered
%depends upon the kind of form.
Для обработки определений экспандер обрабатывает начальные формы в \hyper{body}
(см. секцию~\ref{bodiessection}) или \hyper{library body} (см. секцию~\ref{librarybodysection})
слева направо. То, как экспандер обрабатывает каждую обнаруженную форму, зависит от
вида формы.

\begin{description}
%\item[macro use]
%The expander invokes the associated transformer to transform the macro
%use, then recursively performs whichever of these actions are appropriate
%for the resulting form.
\item[макроприменение]
  Экспандер активизирует ассоциированный трансформер для преобразования макроприменения, затем
  рекурсивно выполняет одно из этих действий, соответствующее полученной форме.

%\item[{\cf define-syntax} form]
%The expander expands and evaluates the right-hand-side expression and binds the
%keyword to the resulting transformer.
\item[{\cf форма define-syntax}]
  Экспандер разворачивает и вычисляет выражение правой части и связывает ключевое слово
  с полученным трансформером.

%\item[{\cf define} form]
%The expander records the fact that the defined identifier is a variable but defers
%expansion of the right-hand-side expression until after all of the
%definitions have been processed.
\item[{\cf форма define}]
  Экспандер фиксирует факт, что определённый идентификатор является переменной, но
  откладывает разворачивание выражения правой части до обработки всех определений.

%\item[{\cf begin} form]
%The expander splices the subforms into the list of body forms it is
%processing.  (See section~\ref{begin}.)
\item[{\cf форма begin}]
  Экспандер соединяет подформы в обрабатываемый список форм тела. (См. секцию~\ref{begin}.)

%\item[{\cf let-syntax} or {\cf letrec-syntax} form]
%The expander splic\-es the inner body forms into the list of (outer) body forms it is
%processing, arranging for the keywords bound by the {\cf let-syntax}
%and {\cf letrec-syntax} to be visible only in the inner body forms.
\item[{\cf формы let-syntax} or {\cf letrec-syntax}]
  Экспандер соединяет внутренние формы тела в обрабатываемый список (внешних) форм тела,
  обеспечивая видимость ключевых слов, привязанных посредством {\cf\bfseries let-syntax} и
  {\cf\bfseries letrec-syntax}, только во внутренних формах тела.

%\newpage

%\item[expression, i.e., nondefinition]
%The expander completes the expansion of the deferred right-hand-side expressions
%and the current and remaining expressions in the body, and then
%creates the equivalent of a {\cf letrec*} form from the defined variables,
%expanded right-hand-side expressions, and expanded body expressions.
\item[выражение, т.е. не определение]
  Экспандер завершает разворачивание отложеннных выражений правой части, а также текущих и
  оставшихся выражений в теле, и затем создает эквивалент формы {\cf\bfseries letrec*} из
  определённых переменных, развёрнутых выражений правой части и развёрнутых выражений тела.
\end{description}\vspace{1mm}

%For the right-hand side of the definition of a variable, expansion is
%deferred until after all of the definitions have been seen.  Consequently,
%each keyword and variable reference within the right-hand side
%resolves to the local binding, if any.
Разворачивание правой части определения переменной откладывается до обнаружения всех
определений. Следовательно, каждое ключевое слово и обращение к переменной внутри правой части
разрешается локальной привязкой, если таковые имеются.\vspace{1mm}

%A definition in the sequence of forms must not define any identifier whose
%binding is used to determine the meaning of the undeferred portions of the
%definition or any definition that precedes it in the sequence of forms.
%For example, the bodies of the following expressions violate this
%restriction.
Определение в последовательности форм не должно определять любой идентификатор, привязка
которого используется для задания смысла неотложенных частей определения или любого
определения, предшествующего ему в последовательности форм. Например, в телах следующих
выражений это ограничение нарушается.\vspace{1mm}

\begin{scheme}
\bfseries (let ()
\bfseries   (define define 17)
\bfseries   (list define))
\bfseries
\bfseries (let-syntax ([def0 (syntax-rules ()
\bfseries                      [(\_ x) (define x 0)])])
\bfseries   (let ([z 3])
\bfseries     (def0 z)
\bfseries     (define def0 list)
\bfseries     (list z)))

\bfseries (let ()
\bfseries   (define-syntax foo
\bfseries     (lambda (e)
\bfseries       (+ 1 2)))
\bfseries   (define + 2)
\bfseries   (foo))%
\end{scheme}\vspace{2mm}

%The following do not violate the restriction.
Ниже ограничение не нарушается.\vspace{1mm}

\begin{scheme}
\bfseries (let ([x 5])
\bfseries   (define lambda list)
\bfseries   (lambda x x))         \ev\bfseries  (5 5)

\bfseries (let-syntax ([def0 (syntax-rules ()
\bfseries                      [(\_ x) (define x 0)])])
\bfseries   (let ([z 3])
\bfseries     (define def0 list)
\bfseries     (def0 z)
\bfseries     (list z)))          \ev\bfseries  (3)

\bfseries (let ()
\bfseries   (define-syntax foo
\bfseries     (lambda (e)
\bfseries       (let ([+ -]) (+ 1 2))))
\bfseries   (define + 2)
\bfseries   (foo))                \ev\bfseries  -1%
\end{scheme}\vspace{2mm}

%The implementation should treat a violation of the restriction as a
%syntax violation.
Реализация должна трактовать нарушение ограничения как нарушение синтаксиса.

% Andre's proposed implementation:
% To detect this violation, the expander can record each
% identifier whose denotation is determined during expansion
% of the body, together with the denotation.
% Before an identifier is bound, its current denotation is compared
% against denotations already used for the same (in the sense of
% bound-identifier=?) identifier in the scope of the intended binding,
% to determine if its current denotation has already been used
% during the expansion of the body.

%Note that this algorithm does not directly reprocess any form.
%It requires a single left-to-right pass over the definitions followed by a
%single pass (in any order) over the body expressions and deferred
%right-hand sides.
Следует отметить, что данный алгоритм непосредственно не обрабатывает повторно какую-либо
форму. Он требует единственного прохода слева направо по определениям и следующего за ним
отдельного прохода (в любом порядке) по выражениям тела и отложенным правым частям.

%Example:
Примеры:\vspace{2.4mm}

\begin{scheme}
\bfseries (lambda (x)
\bfseries   (define-syntax defun
\bfseries     (syntax-rules ()
\bfseries       [(\_ x a e) (define x (lambda a e))]))
\bfseries   (defun even? (n) (or (= n 0) (odd? (- n 1))))
\bfseries   (define-syntax odd?
\bfseries     (syntax-rules () [(\_ n) (not (even? n))]))
\bfseries   (odd? (if (odd? x) (* x x) x)))%
\end{scheme}\vspace{2.4mm}

%In the example, the definition of {\cf defun} is encountered first, and the keyword
%{\cf defun} is associated with the transformer resulting from
%the expansion and evaluation of the corresponding right-hand side.
%A use of {\cf defun} is encountered next and expands into a
%{\cf define} form.
%Expansion of the right-hand side of this define form is deferred.
%The definition of {\cf odd?} is next and results in the association
%of the keyword {\cf odd?} with the transformer resulting from
%expanding and evaluating the corresponding right-hand side.
%A use of {\cf odd?} appears next and is expanded; the resulting
%call to {\cf not} is recognized as an expression
%because {\cf not} is bound as a variable.
%At this point, the expander completes the expansion of the current
%expression (the call to {\cf not}) and the deferred right-hand side of the
%{\cf even?} definition;
%the uses of {\cf odd?} appearing in these expressions are expanded
%using the transformer associated with the keyword {\cf odd?}.
%The final output is the equivalent of
В начале примера обнаруживается определение {\cf\bfseries defun}, и ключевое слово {\cf\bfseries
  defun} ассоциируется с трансформером, являющимся результатом разворачивания и вычисления
соответствующей правой части. Далее обнаруживается применение {\cf\bfseries defun} и
разворачивается в форму {\cf\bfseries define}. Разворачивание правой части этой формы define
откладывается. Определение {\cf\bfseries odd?} является следующим и приводит к ассоциации
ключевого слова {\cf\bfseries odd?} с трансформером, полученным в результате разворачивания и
вычисления соответствующей правой части. Далее обнаруживается и разворачивается применение
{\cf\bfseries odd?}; полученный в результате вызов {\cf\bfseries not} распознаётся как
выражение, так как {\cf\bfseries not} привязан как переменная. В этой точке экспандер
завершает разворачивание текущего выражения (вызов {\cf\bfseries not}) и отложенной правой части
определения {\cf\bfseries even?}; применения {\cf\bfseries odd?}, фигурирующие в этих
выражениях, разворачиваются с помощью трансформера, связанного с ключевым словом {\cf\bfseries
  odd?}. Окончательный вывод эквивалентен\vspace{2.4mm}

\begin{scheme}
\bfseries (lambda (x)
\bfseries   (letrec* ([even?
\bfseries               (lambda (n)
\bfseries                 (or (= n 0)
\bfseries                     (not (even? (- n 1)))))])
\bfseries     (not (even? (if (not (even? x)) (* x x) x)))))%
\end{scheme}\vspace{2.4mm}

%although the structure of the output is implementation-dependent.
хоть и структура выхода зависит от реализации.\vspace{2.4mm}

%Because definitions and expressions can be interleaved in a
%\hyper{top-level body} (see chapter~\ref{programchapter}),
%the expander's processing of a \hyper{top-level body} is somewhat
%more complicated.
%It behaves as described above for a \hyper{body} or
%\hyper{library body} with the following exceptions:
%When the expander finds a nondefinition,
%it defers its expansion and continues scanning for definitions.
%Once it reaches the end of the set of forms, it processes the
%deferred right-hand-side and body expressions, then
%generates the equivalent of a {\cf letrec*} form from the defined variables,
%expanded right-hand-side expressions, and expanded body expressions.
%For each body expression \hyper{expression} that appears before a variable definition
%in the body, a dummy binding is created at the corresponding place within
%the set of {\cf letrec*} bindings, with a fresh temporary variable on the
%left-hand side and the equivalent of {\cf (begin \hyper{expression}
%  \hyper{unspecified})},
%where \hyper{unspecified} is a side-effect-free expression returning
%an unspecified value,
%on the right-hand side, so that
%left-to-right evaluation order is preserved.
%The {\cf begin} wrapper allows \hyper{expression} to evaluate to an
%arbitrary number of values.
Поскольку определения и выражения в \hyper{top-level body} могут чередоваться
(см. главу~\ref{programchapter}), обработка экспандером \hyper{top-level body} несколько
сложнее. Она происходит так, как описано выше для \hyper{body} или \hyper{library body} со
следующими исключениями: При нахождении экспандером неопределения, он откладывает его
разворачивание и продолжает сканировать определения. Как только он достигает конца набора форм,
он обрабатывает отложенные выражения правой части и тела, затем генерирует эквивалент формы
{\cf\bfseries letrec*} из определённых переменных, развёрнутых выражений правой части и
развёрнутых выражений тела. Для каждого выражения тела \hyper{expression}, находящегося в теле
перед определением переменной, создаётся фиктивная привязка в соответствующем месте внутри
набора привязок {\cf\bfseries letrec*}, с новой временной переменной в левой части и
эквивалентом {\cf \textbf{(begin} \hyper{expression} \hyper{unspecified}\textbf{)}}, где
\hyper{unspecified} -- выражение без побочного эффекта, возвращающее неопределённое значение
правой части, так, чтобы порядок вычисления слева направо сохранялся. Обёртка {\cf\bfseries
  begin} позволяет \hyper{expression} вычисляться к произвольному количеству значений.

%%% Local Variables:
%%% mode: latex
%%% TeX-master: "r6rs"
%%% End:
 \par
\input{base}    \par
\clearchaptergroupstar{Appendices}
\appendix
\chapter{Formal semantics}
\label{formalsemanticschapter}
\input{semantics} \par
\input{derived} \par
\input{repository} \par
\input{example} \par
\input{changes} \par
\newpage
\renewcommand{\bibname}{References}

\bibliographystyle{plain}
\bibliography{abbrevs,rrs}

\vfill\eject


\newcommand{\indexheading}{Alphabetic index of definitions of
  concepts, keywords, and procedures}
\texonly
\newcommand{\indexintro}{The index includes entries from the library
  document; the entries are marked with ``(library)''.}
\endtexonly

\printindex

\end{document}
