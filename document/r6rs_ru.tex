\documentclass[twoside,twocolumn]{algol60}
%\documentclass[twoside]{algol60}

% XeLaTeX:
\RequirePackage{fontspec}
\RequirePackage{xunicode}
\RequirePackage{xltxtra}
\RequirePackage{color}
\RequirePackage[titles]{tocloft}
\defaultfontfeatures{Mapping=tex-text}
%%%%%%%%%%%%%%%%%%%%%%%%%%%%%%%%%%%%%%

% Polyglossia:
\RequirePackage{polyglossia}
\setdefaultlanguage{russian}
\setotherlanguage{english}
\newfontfamily\cyrillicfont{Times New Roman}
\newfontfamily\cyrillicfonttt{Courier New}
\newfontfamily\cyrillicfontsf{Arial}
%%%%%%%%%%%%%%%%%%%%%%%%%%%%%%%%%%%%%%%%%%%

\renewcommand{\baselinestretch}{0.95}


\pagestyle{headings}
\showboxdepth=0
\makeindex
% Macros for R^nRS.

% tex2page.sty mucks with in some manner
\let\centerlinesaved=\centerline
\usepackage{tex2page}
\let\centerline=\centerlinesaved

\usepackage{xr-hyper}

\usepackage{makeidx}
\usepackage{hyperref}

% \let\htmlonly=\iffalse
% \let\endhtmlonly=\fi
% \let\texonly=\iftrue
% \let\endtexonly=\fi

\makeatletter

\texonly
\newcommand{\topnewpage}{\@topnewpage}
\endtexonly

\htmlonly
\newcommand{\topnewpage}[0][]{#1}
\endhtmlonly

\newcommand{\authorsc}[1]{{\scriptsize\scshape #1}}

% Chapters, sections, etc.

\newcommand{\extrapart}[1]{
 % \chapter{#1}
  \chapter*{#1}
  \markboth{#1}{#1}
  \vskip 1ex
  \addcontentsline{toc}{chapter}{#1}}

\newcommand{\clearchaptergroupstar}[1]{
  \texonly
  \clearpage
  \addcontentsline{toc}{chaptergroup}{#1}
  \topnewpage[
    \centerline{\large\bf\uppercase{#1}}
    \bigskip]
    \endtexonly
  }

\newcommand{\clearchapterstar}[1]{
  \clearpage
  \topnewpage[
    \centerline{\large\bf\uppercase{#1}}
    \bigskip]}

\newcommand{\clearextrapart}[1]{
  \clearchapterstar{#1}
  \markboth{#1}{#1}
  \addcontentsline{toc}{chapter}{#1}}

\newcommand{\vest}{}
\newcommand{\dotsfoo}{$\ldots\,$}

\newcommand{\sharpfoo}[1]{{\tt\##1}}
\newcommand{\schfalse}{\sharpfoo{f}}
\newcommand{\schtrue}{\sharpfoo{t}}

\newcommand{\ampfoo}[1]{{\tt\&#1}}

\newcommand{\libfoo}[1]{{\tt(#1)}}

\newcommand{\singlequote}{{\tt'}}  %\char19
\newcommand{\doublequote}{{\tt"}}
\newcommand{\backquote}{{\tt`}}%\char18}}
\newcommand{\backwhack}{{\tt\char`\\}}
\newcommand{\comma}{{\tt\char`\@}}
\newcommand{\atsign}{{\tt\char`\@}}
\newcommand{\bang}{{\tt\char`\!}}
\newcommand{\sharpsign}{{\tt\#}}
\newcommand{\verticalbar}{{\tt|}}
\newcommand{\openbracket}{{\tt\char`\[}}
\newcommand{\closedbracket}{{\tt\char`\]}}
\newcommand{\ampersand}{{\tt\char`\&}}

\newcommand{\coerce}{\discretionary{->}{}{->}}

% Knuth's \in sucks big boulders
\def\elem{\hbox{\raise.13ex\hbox{$\scriptstyle\in$}}}

\newcommand{\meta}[1]{{\noindent\hbox{\rm$\langle$#1$\rangle$}}}
\let\hyper=\meta
\newcommand{\hyperi}[1]{\hyper{#1$_1$}}
\newcommand{\hyperii}[1]{\hyper{#1$_2$}}
\newcommand{\hyperiii}[1]{\hyper{#1$_3$}}
\newcommand{\hyperiv}[1]{\hyper{#1$_4$}}
\newcommand{\hyperj}[1]{\hyper{#1$_i$}}
\newcommand{\hypern}[1]{\hyper{#1$_n$}}
\texonly
\newcommand{\var}[1]{\noindent\hbox{\textnormal{\textit{#1}}}}
\endtexonly
\htmlonly
\newcommand{\var}[1]{\textnormal{\textit{#1}}}
\endhtmlonly
\newcommand{\vari}[1]{\var{#1$_1$}}
\newcommand{\varii}[1]{\var{#1$_2$}}
\newcommand{\variii}[1]{\var{#1$_3$}}
\newcommand{\variv}[1]{\var{#1$_4$}}
\newcommand{\varj}[1]{\var{#1$_j$}}
\newcommand{\vark}[1]{\var{#1$_k$}}
\newcommand{\varn}[1]{\var{#1$_n$}}

\newcommand{\vr}[1]{{\noindent\hbox{$#1$\/}}}  % Careful, is \/ always the right thing?
\newcommand{\vri}[1]{\vr{#1_1}}
\newcommand{\vrii}[1]{\vr{#1_2}}
\newcommand{\vriii}[1]{\vr{#1_3}}
\newcommand{\vriv}[1]{\vr{#1_4}}
\newcommand{\vrv}[1]{\vr{#1_5}}
\newcommand{\vrj}[1]{\vr{#1_j}}
\newcommand{\vrn}[1]{\vr{#1_n}}

%%R4%% The excessive use of the code font in the numbers section was
% confusing, somewhat obnoxious, and inconsistent with the rest
% of the report and with parts of the section itself.  I added
% a \tupe no-op, and changed most old uses of \type to \tupe,
% to make it easier to change the fonts back if people object
% to the change.

\newcommand{\type}[1]{{\it#1}}
\newcommand{\tupe}[1]{{#1}}

\newcommand{\defining}[1]{\mainindex{#1}{\em #1}}
\newcommand{\ide}[1]{{\schindex{#1}\frenchspacing\tt{#1}}}

\newcommand{\lambdaexp}{{\cf lambda} expression}

\newcommand{\callcc}{{\tt call-with-current-continuation}}

\newcommand{\mainschindex}[1]{\label{#1}\index{#1@\texttt{#1}}}
\newcommand{\mainindex}[1]{\index{#1}}
\newcommand{\schindex}[1]{\index{#1@\texttt{#1}}}
\newcommand{\sharpindex}[1]{\index{#1@\texttt{\#{}#1}}}
\newcommand{\sharpbangindex}[1]{\index{#1@\texttt{\#!#1}}}
\newcommand{\ampindex}[1]{\index{#1@\texttt{\&{}#1}}}
\newcommand{\libindex}[1]{\index{#1@\texttt{(#1)}}}

\texonly
\newcommand{\extref}[2]{\ref{#1}}
\endtexonly
\htmlonly
\newcommand{\extref}[2]{on ``#2''}
\endhtmlonly

\renewenvironment{theindex}
{\texonly\clearpage\endtexonly
\topnewpage[
    \begin{center}
      \large\bf\MakeUppercase{\indexheading}
    \end{center}
    \vskip 1ex \bigskip]
    \markboth{Index}{Index}
    \addcontentsline{toc}{chapter}{\indexheading}
    \parindent\z@
    \texonly\parskip\z@ plus .1pt\endtexonly\relax\let\item\@idxitem
    \indexintro\par\bigskip}
               {\texonly\clearpage\endtexonly}


\newcommand{\domain}[1]{#1}
\newcommand{\nodomain}[1]{}
%\newcommand{\todo}[1]{{\rm$[\![$!!~#1$]\!]$}}
\newcommand{\todo}[1]{}

% \frobq will make quote and backquote look nicer.
\def\frobqcats{%\catcode`\'=13
\catcode`\`=13{}}
{\frobqcats
\gdef\frobqdefs{%\def'{\singlequote}
\def`{\backquote}}}
\def\frobq{\frobqcats\frobqdefs}

% \cf = code font
% Unfortunately, \cf \cf won't work at all, so don't even attempt to
% next constructions which use them...
\newcommand{\cf}{\frenchspacing\frobq\tt}

\texonly
% Same as \obeycr, but doesn't do a \@gobblecr.
{\catcode`\^^M=13 \gdef\myobeycr{\catcode`\^^M=13 \def^^M{\\}}%
\gdef\restorecr{\catcode`\^^M=5 }}
\endtexonly

{\obeyspaces\gdef {\hbox{\hskip0.5em}}}

\gdef\gobblecr{\@gobblecr}

\def\setupcode{\@makeother\^}

% Scheme example environment
% At 11 points, one column, these are about 56 characters wide.
% That's 32 characters to the left of the => and about 20 to the right.

\newcommand{\exception}[1]{{\cf#1} \textnormal{\textit{exception}}}
\newenvironment{schemenoindent}{
  % Commands for scheme examples
  \newcommand{\ev}{\>\>\evalsto}
  \newcommand{\xev}{\>\>\hspace*{-1em}\evalsto}
  \newcommand{\lev}{\\\>\evalsto}
  \newcommand{\unspecified}{{\em{}unspecified}}
  \newcommand{\theunspecified}{{\em{}unspecified}}
  \setupcode
  \small \cf \obeyspaces \myobeycr
  \begin{tabbing}%
\qquad\=\hspace*{5em}\=\hspace*{9em}\=\evalsto~\=\kill%   was 16em
\gobblecr}{\unskip\end{tabbing}}

%\newenvironment{scheme}{\begin{schemenoindent}\+\kill}{\end{schemenoindent}}
\newenvironment{scheme}{
  % Commands for scheme examples
  \newcommand{\ev}{\>\>\evalsto}
  \newcommand{\xev}{\>\>\hspace*{-1em}\evalsto}
  \newcommand{\lev}{\\\>\evalsto}
  \renewcommand{\em}{\rmfamily\itshape}
  \newcommand{\unspecified}{{\em{}unspecified}}
  \newcommand{\theunspecified}{{\em{}unspecified}}
  \setupcode
  \small \cf \obeyspaces \myobeycr
  \begin{tabbing}%
\qquad\=\hspace*{5em}\=\hspace*{9em}\=\evalsto~\=\+\kill%   was 16em
\gobblecr}{\unskip\end{tabbing}}

\texonly
\newcommand{\evalsto}{$\Longrightarrow$}
\endtexonly
 \htmlonly
\newcommand{\evalsto}{$\Rightarrow$}
\endhtmlonly

% Rationale

\newenvironment{rationale}{%
\bgroup\small\noindent{\em Rationale:}\space}{%
\egroup}

% Notes

\newenvironment{note}{%
\bgroup\small\noindent{\em Note:}\space}{%
\egroup}

% Names of library modules

\newcommand{\library}[1]{{\tt (#1)}}
\newcommand{\deflibrary}[1]{\library{#1}\libindex{#1}}

\newcommand{\rsixlibrary}[1]{\library{rnrs #1 (6)}}
\newcommand{\defrsixlibrary}[1]{\deflibrary{rnrs #1 (6)}}

\newcommand{\thersixlibrary}{\library{rnrs (6)}}
\newcommand{\defthersixlibrary}{\deflibrary{rnrs (6)}}

% Manual entries

\newenvironment{entry}[1]{
  \vspace{3.1ex plus .5ex minus .3ex}\noindent#1%
\unpenalty\nopagebreak}{\vspace{0ex plus 1ex minus 1ex}}

\newcommand{\exprtype}{syntax}

\newcommand{\unspecifiedreturn}{unspecified values}
\newcommand{\isunspecified}{is unspecified}
\newcommand{\areunspecified}{are unspecified}

% Primitive prototype
\newcommand{\pproto}[2]{\unskip%
\hbox{\cf\spaceskip=0.5em#1}\hfill\penalty 0%
\hbox{ }\nobreak\hfill\hbox{\rm #2}\break}

% Parenthesized prototype
\newcommand{\proto}[3]{\pproto{(\mainschindex{#1}\hbox{#1}{\it#2\/})}{#3}}

% Variable prototype
\newcommand{\vproto}[2]{\mainschindex{#1}\pproto{#1}{#2}}

% Condition-type prototype
 \newcommand{\ctproto}[1]{\ampindex{#1}\pproto{\ampfoo{#1}}{condition type}}

% Prototype for literal syntax, no index
\newcommand{\litprotonoindex}[1]{\pproto{#1}{auxiliary syntax}}

% Prototype for literal syntax
\newcommand{\litproto}[1]{\mainschindex{#1}\litprotonoindex{#1}}

% Prototype for literal syntax at level 1, no index
\newcommand{\litprotoexpandnoindex}[1]{\pproto{#1}{auxiliary syntax ({\tt expand)}}}

% Prototype for literal syntax at level 1
\newcommand{\litprotoexpand}[1]{\mainschindex{#1}\litprotoexpandnoindex{#1}}

% Extending an existing definition (\proto without the index entry)
\newcommand{\rproto}[3]{\pproto{(\hbox{#1}{\it#2\/})}{#3}}

% Extending an existing definition, with index entry
\newcommand{\irproto}[3]{\schindex{#1}\rproto{#1}{#2}{#3}}

% Variable prototype
\newcommand{\rvproto}[2]{\pproto{#1}{#2}}

% Grammar environment

\newenvironment{grammar}{
  \def\:{\goesto{}}
  \def\|{$\vert$}
  \cf \myobeycr
  \begin{tabbing}
    %\qquad\quad \=
    \qquad \= $\vert$ \= \kill
  }{\unskip\end{tabbing}}

%\newcommand{\unsection}{\unskip}
\newcommand{\unsection}{{\vskip -2ex}}

% Allow line break after hyphen
\newcommand{\hp}{\linebreak[0]}

\texonly
\newcommand{\itspace}{\hspace{1pt}}
\endtexonly
\htmlonly
\newcommand{\itspace}{}
\endhtmlonly

% Commands for grammars
\newcommand{\arbno}[1]{#1\hbox{\rm*}}
\newcommand{\atleastone}[1]{#1\hbox{$^+$}}

\texonly
\newcommand{\goesto}{$\longrightarrow$}
\endtexonly
 \htmlonly
\newcommand{\goesto}{$\rightarrow$}
\endhtmlonly

\newcommand{\syntax}{{\em Syntax: }}
\newcommand{\semantics}{{\em Semantics: }}
\newcommand{\implresp}{{\em Implementation responsibilities: }}

\newcommand{\rrs}[1]{\textit{Revised$^#1$ Report on the Algorithmic Language Scheme}}

\newcommand{\libindexentry}[1]{#1 (library)}

\makeatother

\def\rnrsrevision{6}
%\def\rnrsrevisiondate{26 September 2007}
\def\rnrsrevisiondate{26 сентября 2007}


%!TEX root = r6rs.tex

\usepackage{latexsym}
\usepackage{mathrsfs}
\usepackage{stmaryrd}

\newcounter{subfig}
\newcommand{\subfigurestart}{\renewcommand{\thefigure}{A.\arabic{figure}\alph{subfig}}\setcounter{subfig}{1}}


% needed for the second thru the nth figure
\newcommand{\subfigureadjust}{\addtocounter{figure}{-1}\addtocounter{subfig}{1}}

\newcommand{\subfigurestop}{\renewcommand{\thefigure}{A.\arabic{figure}}}


\newcommand{\semanticsindex}[2]{\index{#1@{\texttt{#1} (formal semantics)}}}

\newcommand{\pltreducks}{PLT Redex}
\newcommand{\rnrs}{Report}
\newcommand{\rnrslongspace}{\mbox{Revised\ensuremath{\,^{\mbox{\textrm{\scriptsize 5}}}} Report on Scheme}}
\newcommand{\rnrslong}{\mbox{Revised\ensuremath{^{\mbox{\textrm{\scriptsize 5}}}} Report on Scheme}}
\newcommand{\largernrslong}{\mbox{Revised\ensuremath{\,^{\mbox{\textrm{\large 5}}}} Report on Scheme}}

%\newenvironment*{proof}
%{\noindent\textbf{Proof} }
%{$\Box$ \\}

%\newcommand{\either}{*\!{}\!{}\!\!\circ}
\newcommand{\either}{*\!\circ}

\newcommand{\hole}{[~]}
\newcommand{\holes}{\ensuremath{\hole_{\star}}}
\newcommand{\holeone}{\ensuremath{\hole_\circ}}
\newcommand{\holeany}{\ensuremath{\hole_{\either}}}

%% multi-letter nonterminals (one-letter can be done with $_$)
\newcommand{\nt}[1]{\textnormal{\textit{#1}}}

%\newcommand{\sy}[1]{\textnormal{\textbf{#1}}}
%\newcommand{\va}[1]{\textnormal{\textsf{#1}}}

\newcommand{\sy}[1]{{\cf #1}}
\newcommand{\va}[1]{{\cf #1}}


\newcommand{\beginF}{\ensuremath{\textbf{begin}^{\mbox{\textrm{\textbf{\scriptsize F}}}}}}
\newcommand{\Eo}{\ensuremath{E^{\circ}}}
\newcommand{\Estar}{\ensuremath{E^{\star}}}
\newcommand{\Fo}{\ensuremath{F^{\circ}}}
\newcommand{\Fstar}{\ensuremath{F^{\star}}}
\newcommand{\Io}{\ensuremath{I^{\circ}}}
\newcommand{\Istar}{\ensuremath{I^{\star}}}

\imgdef\calP{\ensuremath{\mathcal{P}}}
\imgdef\calS{\ensuremath{\mathcal{S}}}
\imgdef\calR{\ensuremath{\mathcal{R}}}
\imgdef\calRv{\ensuremath{\mathcal{R}_v}}
\imgdef\calA{\ensuremath{\mathcal{A}}}
\imgdef\scrO{\ensuremath{\mathscr{O}}}

\newcommand{\semfalse}{\texttt{\#f}}
\newcommand{\semtrue}{\texttt{\#t}}

\newcommand{\aline}{\noindent\hrulefill\par}

%\def\beginfig{\begin{figure*}[t]{\noindent\hrulefill\par}\small}
%\def\endfig{{\noindent\hrulefill\par}\end{figure*}}

\def\beginfig{\begin{figure*}[tb!]{\noindent\par}\small}
\def\endfig{{\noindent\hrulefill\par}\end{figure*}}

\newcommand{\dom}{\textit{dom}}

\newcommand{\gopen}{{^{\scriptscriptstyle\lceil}\!\!}}
\newcommand{\gclose}{\!\!{}^{\scriptscriptstyle\rceil}}

\newcommand{\mrk}{\diamond}
\newcommand{\umrk}{^\mrk}

\newcommand{\rulename}[1]{\textsf{[#1]}}

\newcommand{\extraspterm}{\\[6pt]}

\newcommand{\twolinerule}[3]{\twolineruleA{#1}{#2}{\rulename{#3}}{\rightarrow}}
\newcommand{\twolinescrule}[4]{\twolinescruleA{#1}{#2}{\rulename{#3}}{#4}{\rightarrow}}
\newcommand{\onelinerule}[3]{\onelineruleA{#1}{#2}{\rulename{#3}}{\rightarrow}}
\newcommand{\onelinescrule}[4]{\onelinescruleA{#1}{#2}{\rulename{#3}}{#4}{\rightarrow}}

\newcommand{\twolineruleA}[4]{
\multicolumn{3}{l}{{#1} {#4}} & {#3}\\ 
\multicolumn{3}{l}{{#2}} & \extraspterm}

\newcommand{\twolinescruleA}[5]{
\multicolumn{3}{l}{{#1} {#5}} & {#3}\\ 
\multicolumn{4}{l}{{#2 ~ ~ ~ {#4}}} \extraspterm}

\newcommand{\twolinescruleB}[5]{
\multicolumn{3}{l}{{#1} {#5}} & {#3}\\ 
\multicolumn{4}{l}{#2} \\
\multicolumn{4}{l}{~ ~ ~ #4} \extraspterm}

\newcommand{\threelinescruleA}[5]{
\multicolumn{3}{l}{{#1} {#5}} & {#4}\\ 
\multicolumn{4}{l}{#2} \\
\multicolumn{4}{l}{#3} \extraspterm}

\newcommand{\threelinescruleB}[6]{
\multicolumn{3}{l}{{#1} {#6}} & {#4}\\ 
\multicolumn{4}{l}{#2} \\
\multicolumn{4}{l}{#3} \\
\multicolumn{4}{l}{~ ~ ~ #5} \extraspterm}


\newcommand{\fourlinescruleB}[7]{
\multicolumn{3}{l}{{#1} {#7}} & {#5}\\ 
\multicolumn{4}{l}{#2} \\
\multicolumn{4}{l}{#3} \\
\multicolumn{4}{l}{#4} \\
\multicolumn{4}{l}{~ ~ ~ #6} \extraspterm}


\newcommand{\onelineruleA}[4]{
\multicolumn{1}{l}{#1} & {#4} ~ & {#2} & {#3} \extraspterm}

\newcommand{\onelinescruleA}[5]{
\multicolumn{1}{l}{#1} & {#5} ~ & {#2} & {#3} \\
& & {#4} \extraspterm}


\texonly
\externaldocument[lib:]{r6rs-lib}
\endtexonly

%\def\headertitle{Revised$^{\rnrsrevision}$ Scheme}
\def\headertitle{Редакция$^{\rnrsrevision}$ Scheme}
%\def\TZPtitle{Revised^\rnrsrevision{} Report on the Algorithmic Language Scheme}
\def\TZPtitle{Редакция^\rnrsrevision{} Стандарта Алгоритмического Языка Scheme}

\begin{document}

\thispagestyle{empty}

\topnewpage[{
\begin{center}   {\huge\bf
        Редакция{\Huge$^{\mathbf{\htmlonly\tiny\endhtmlonly{}\rnrsrevision}}$} Стандарта Алгоритмического Языка \\
                              \vskip 3pt
                                Scheme}

\vskip 1ex
$$
\begin{tabular}{l@{\extracolsep{.5in}}lll}
\multicolumn{4}{c}{M\authorsc{ICHAEL} S\authorsc{PERBER}}
\\
\multicolumn{4}{c}{R.\ K\authorsc{ENT} D\authorsc{YBVIG},
  M\authorsc{ATTHEW} F\authorsc{LATT},
  A\authorsc{NTON} \authorsc{VAN} S\authorsc{TRAATEN}}
\\
%\multicolumn{4}{c}{(\textit{Editors})} \\
\multicolumn{4}{c}{(\textit{Редакторы})} \\
\multicolumn{4}{c}{
  R\authorsc{ICHARD} K\authorsc{ELSEY}, W\authorsc{ILLIAM} C\authorsc{LINGER},
  J\authorsc{ONATHAN} R\authorsc{EES}} \\
%\multicolumn{4}{c}{(\textit{Editors, Revised\itspace{}$^5$ Report on the
%    Algorithmic Language Scheme})} \\
\multicolumn{4}{c}{(\textit{Редакторы, Редакция\itspace{}$^5$ Стандарт
    Алгоритмического Языка Scheme})} \\
\multicolumn{4}{c}{
  R\authorsc{OBERT} B\authorsc{RUCE} F\authorsc{INDLER}, J\authorsc{ACOB} M\authorsc{ATTHEWS}} \\
%\multicolumn{4}{c}{(\textit{Authors, formal semantics})} \\[1ex]
\multicolumn{4}{c}{(\textit{Авторы, формальная семантика})} \\[1ex]
\multicolumn{4}{c}{\bf \rnrsrevisiondate}
\end{tabular}
$$



\end{center}

%\chapter*{Summary}
\chapter*{Предисловие}
\medskip

{\parskip 1ex
%The report gives a defining description of the programming language
%Scheme.  Scheme is a statically scoped and properly tail-recursive
%dialect of the Lisp programming language invented by Guy Lewis
%Steele~Jr.\ and Gerald Jay~Sussman.  It was designed to have an
%exceptionally clear and simple semantics and few different ways to
%form expressions.  A wide variety of programming paradigms, including
%functional, imperative, and message passing styles, find convenient
%expression in Scheme.
Стандарт предоставляет описание определения языка программирования Scheme. Scheme -- это чистый
хвост-рекурсивный диалект языка программирования Lisp со статическими областями видимости,
созданный Guy Lewis Steele~Jr.\ и Gerald Jay~Sussman. Он был разработан для использования
исключительно ясной и простой семантики и небольшого количества способов формирования
выражений. Большое разнообразие парадигм программирования, включая функциональный и императивный
стили, а также стиль обмена сообщениями, получили в Scheme удобное представление.

%This report is accompanied by a report describing standard
%libraries~\cite{R6RS-libraries}; references to this document are
%identified by designations such as ``library section'' or ``library
%chapter''.  It is also accompanied by a report containing
%non-normative appendices~\cite{R6RS-appendices}.  A fourth report gives
%some historical background and rationales for many aspects of the
%language and its libraries~\cite{R6RS-rationale}.
К данному стандарту прилагается стандарт, описывающий стандартные
библиотеки~\cite{R6RS-libraries}; ссылки на этот документ идентифицированы обозначениями,
такими, как ``библиотечная секция'' или ``библиотечная глава''. К нему также прилагается
стандарт, содержащий ненормативные приложения~\cite{R6RS-appendices}. Четвертый стандарт
предоставляет несколько исторических справок и логическое обоснование многих аспектов языка и
его библиотек~\cite{R6RS-rationale}.

\medskip

%The individuals listed above are not the sole authors of the text of
%the report.  Over the years, the following individuals were involved
%in discussions contributing to the design of the Scheme language, and
%were listed as authors of prior reports:
Перечисленные выше люди не являются единственными авторами текста стандарта. За все эти годы
следующие люди участвовали в обсуждениях, содействующих разработке языка Scheme, и перечислялись
в качестве авторов предыдущих стандартов:

%Hal Abelson, Norman Adams, David Bartley, Gary Brooks, William
%Clinger, R.\ Kent Dybvig, Daniel Friedman, Robert Halstead, Chris
%Hanson, Christopher Haynes, Eugene Kohlbecker, Don Oxley, Kent Pitman,
%Jonathan Rees, Guillermo Rozas, Guy L.\ Steele Jr., Gerald Jay Sussman, and
%Mitchell Wand.
Hal Abelson, Norman Adams, David Bartley, Gary Brooks, William
Clinger, R.\ Kent Dybvig, Daniel Friedman, Robert Halstead, Chris
Hanson, Christopher Haynes, Eugene Kohlbecker, Don Oxley, Kent Pitman,
Jonathan Rees, Guillermo Rozas, Guy L.\ Steele Jr., Gerald Jay Sussman и
Mitchell Wand.

%In order to highlight recent contributions, they are not listed as
%authors of this version of the report.  However, their contribution
%and service is gratefully acknowledged.
Чтобы подчеркнуть последние дополнения, они не перечислены в качестве авторов данной версии
стандарта. Однако их вклад и работа с благодарностью принимается.

\medskip

%We intend this report to belong to the entire Scheme community, and so
%we grant permission to copy it in whole or in part without fee.  In
%particular, we encourage implementors of Scheme to use this report as
%a starting point for manuals and other documentation, modifying it as
%necessary.
Мы подразумеваем, что данный стандарт принадлежит всему сообществу Scheme, и таким образом, мы
официально даём разрешение на его безвозмездное полное или частичное копирование. В частности,
мы настоятельно рекомендуем разработчикам реализаций Scheme использовать данный стандарт в
качестве отправной точки для руководств и другой документации, изменяя его по мере
необходимости.
}

\bigskip

% We're done.
}]

\texonly\clearpage\endtexonly

%\chapter*{Contents}
\chapter*{Содержание}
\addvspace{3.5pt}                  % don't shrink this gap
\renewcommand{\tocshrink}{-4.0pt}  % value determined experimentally
{
\tableofcontents
}

\vfill
\eject


%\clearextrapart{Introduction}
\clearextrapart{Введение}

\label{historysection}

%Programming languages should be designed not by piling feature on top of
%feature, but by removing the weaknesses and restrictions that make additional
%features appear necessary.  Scheme demonstrates that a very small number
%of rules for forming expressions, with no restrictions on how they are
%composed, suffice to form a practical and efficient programming language
%that is flexible enough to support most of the major programming
%paradigms in use today.
Языки программирования должны разрабатываться не путём наращивания функциональных возможностей,
а путём удаления слабостей и ограничений, приводящих к мнимой необходимости дополнительных
средств. Scheme наглядно подтверждает, что для создания практичного и эффективного языка
программирования, достаточно гибкого для поддержки большинства основных парадигм
программирования, используемых в настоящее время, достаточно крайне малого количества правил
формирования выражений без ограничений на их составление.

%Scheme
%was one of the first programming languages to incorporate first-class
%procedures as in the lambda calculus, thereby proving the usefulness of
%static scope rules and block structure in a dynamically typed language.
%Scheme was the first major dialect of Lisp to distinguish procedures
%from lambda expressions and symbols, to use a single lexical
%environment for all variables, and to evaluate the operator position
%of a procedure call in the same way as an operand position.  By relying
%entirely on procedure calls to express iteration, Scheme emphasized the
%fact that tail-recursive procedure calls are essentially gotos that
%pass arguments.  Scheme was the first widely used programming language to
%embrace first-class escape procedures, from which all previously known
%sequential control structures can be synthesized.  A subsequent
%version of Scheme introduced the concept of exact and inexact number objects,
%an extension of Common Lisp's generic arithmetic.
%More recently, Scheme became the first programming language to support
%hygienic macros, which permit the syntax of a block-structured language
%to be extended in a consistent and reliable manner.
Scheme был одним из первых языков программирования, включающим полноправные процедуры, как в
лямбда-исчислении, тем самым доказывая практическую значимость статических правил области видимости и
блочной структуры в языке с динамической типизацией. Scheme был первым основным диалектом Lisp,
различающим процедуры из лямбда-выражений и символы, использующий единственное лексическое
окружение для всех переменных и вычисляющий позицию оператора вызова процедуры таким же образом,
как и позицию операнда. Полностью основывываясь на вызовах процедур для выражения итерации,
Scheme подчеркивает факт, что хвост-рекурсивные вызовы процедур являются по существу командами
перехода, передающими аргументы. Scheme был первым широко применяемым языком программирования,
использующим полноправные управляющие процедуры, из которых могут быть синтезированы все известные
ранее последовательные управляющие структуры. В последующую версию Scheme введено понятие точных
и неточных числовых объектов, расширение обобщённой арифметики Common Lisp. Позже Scheme стал
первым языком программирования, поддерживающим гигиенические макросы, предоставляющие
возможность расширения синтаксиса языка с блочной структурой логичным и безопасным способом.\vspace{-3mm}

%\subsection*{Guiding principles}
\subsection*{Руководящие принципы}

%To help guide the standardization effort, the editors have adopted a
%set of principles, presented below.
%Like the Scheme language defined in \rrs{5}~\cite{R5RS}, the language described
%in this report is intended to:
Для содействия в проведении работ по стандартизации редакторы приняли ряд принципов, представленных
ниже. Как и язык Scheme, определённый в \rrs{5}~\cite{R5RS}, язык, описанный в данном стандарте,
предназначен для:\vspace{-1mm}

\begin{itemize}
%\item allow programmers to read each other's code, and allow
%  development of portable programs that can be executed in any
%  conforming implementation of Scheme;
\item предоставления программистам возможности чтения кода друг друга и разработки
  переносимых программ, которые могут быть выполнены в любой соответствующей реализации Scheme;

%\item derive its power from simplicity, a small number of generally
%  useful core syntactic forms and procedures, and no unnecessary
%  restrictions on how they are composed;
\item использования его эффективности за счёт простоты, малого количества основных
  синтаксических форм и процедур общего применения, а также отсутствия ненужных ограничений на их
  составление;

%\item allow programs to define new procedures and new hygienic
%  syntactic forms;
\item предоставления возможности определения в программах новых процедур и гигиенических
синтаксических форм;

%\item support the representation of program source code as data;
\item поддержки представления исходного текста программы в качестве данных;

%\item make procedure calls powerful enough to express any form of
%  sequential control, and allow programs to perform non-local control
%  operations without the use of global program transformations;
\item достижения эффективности вызовов процедур, достаточной для выражения любой формы
  последовательного управления и предоставления возможности выполнения в программах нелокальных
  управляющих операций без использования глобальных преобразований программы;\vspace{-1.2mm}

%\item allow interesting, purely functional programs to run indefinitely
%  without terminating or running out of memory on finite-memory
%  machines;
\item предоставления возможности неопределённо долгой работы интересных, чисто функциональных
  программ без остановки или выхода за границы памяти на машинах с конечной памятью;\vspace{-1.2mm}

%\item allow educators to use the language to teach programming
%  effectively, at various levels and with a variety of pedagogical
%  approaches; and
\item предоставления педагогам возможности применения языка для эффективного преподавания
  программирования на различных уровнях и с разнообразными педагогическими подходами; и\vspace{-1.2mm}

%\item allow researchers to use the language to explore the design,
%  implementation, and semantics of programming languages.
\item предоставления научным работникам возможности применения языка для изучения разработки,
  реализации и семантики языков программирования.
\end{itemize}\vspace{-1.2mm}

%In addition, this report is intended to:
Кроме того, данный стандарт предназначен для:\vspace{-1.2mm}

\begin{itemize}
%\item allow programmers to create and distribute substantial programs
%  and libraries, e.g., implementations of Scheme Requests for
%  Implementation, that run without
%  modification in a variety of Scheme implementations;
\item предоставления программистам возможности создания и распространения фундаментальных
  программ и библиотек, например, реализаций Scheme Requests for Implementation, выполняемых без
  модификации в разнообразных реализациях Scheme;\vspace{-1.2mm}

%\item support procedural, syntactic, and data abstraction more fully
%  by allowing programs to define hygiene-bending and hygiene-breaking
%  syntactic abstractions and new unique datatypes along with
%  procedures and hygienic macros in any scope;
\item максимально полной поддержки абстракций -- процедурных, синтаксических и данных, что
  позволяет определять в программах (сгибающие?) гигиену и (нарушающие?) гигиену синтаксические
  абстракции и новые уникальные типы данных наряду с процедурами и гигиеническими макросами в
  любых областях видимости;\vspace{-1.2mm}

%\item allow programmers to rely on a level of automatic run-time type
%  and bounds checking sufficient to ensure type safety; and
\item предоставления программистам возможности руководствоваться уровнем
  автоматической проверки типов и границ этапа выполнения,
  достаточным для гарантии безопасности типов; и\vspace{-1.2mm}

%\item allow implementations to generate efficient code, without
%  requiring programmers to use implementation-specific operators or
%  declarations.
\item предоставления реализациям возможности генерации эффективного кода, не требуя от
  программистов применения операторов или деклараций конкретной реализации.

\end{itemize}\vspace{-1.2mm}

%While it was possible to write portable programs in Scheme as
%described in \rrs{5}, and indeed portable Scheme programs were written
%prior to this report, many Scheme programs were not, primarily because
%of the lack of substantial standardized libraries and the
%proliferation of implementation-specific language additions.
В то время как существовала возможность написания переносимых программ на Scheme согласно
\rrs{5}, и безусловно переносимые программы Scheme были написаны до данного стандарта, множество
программ не было написано на Scheme прежде всего из-за нехватки фундаментальных
стандартизированных библиотек и чрезмерного количества зависящих от реализации языковых
дополнений.

%In general, Scheme should include building blocks that allow a wide
%variety of libraries to be written, include commonly used user-level
%features to enhance portability and readability of library and
%application code, and exclude features that are less commonly used and
%easily implemented in separate libraries.
В общем случае Scheme должен содержать стандартные блоки, позволяющие создавать широкий спектр
библиотек, включая универсальные средства пользовательского уровня для улучшения переносимости и
удобочитаемости библиотек и прикладного кода, и исключать средства, реже используемые и легко
реализуемые в отдельных библиотеках.

%The language described in this report is intended to also be backward
%compatible with programs written in Scheme as described in \rrs{5} to
%the extent possible without compromising the above principles and
%future viability of the language.  With respect to future viability,
%the editors have operated under the assumption that many more Scheme
%programs will be written in the future than exist in the present, so
%the future programs are those with which we should be most concerned.
Язык, описанный в данном стандарте, также подразумевает обратную совместимость и с программами,
написанными на Scheme согласно \rrs{5} при условии отсутствия компромисса между вышеописанными
основными положениями и будущей эффективностью языка. Принимая во внимание будущую эффективность,
редакторы исходили из предположения, что в будущем на Scheme будет написано больше программ,
чем существует в настоящее время, таким образом, больше всего мы должны быть
заинтересованы в будущих программах.\vspace{3mm}

%\subsection*{Acknowledgements}
\subsection*{Благодарности}\vspace{4mm}

%Many people contributed significant help to this revision of the
%report.  Specifically, we thank Aziz Ghuloum and Andr\'e van Tonder for
%contributing reference implementations of the library system.  We
%thank Alan Bawden, John Cowan, Sebastian Egner, Aubrey Jaffer, Shiro
%Kawai, Bradley Lucier, and Andr\'e van Tonder for contributing insights on
%language design.  Marc Feeley, Martin Gasbichler, Aubrey Jaffer, Lars T Hansen,
%Richard Kelsey, Olin Shivers, and Andr\'e van Tonder wrote SRFIs that
%served as direct input to the report.  Marcus Crestani, David Frese,
%Aziz Ghuloum, Arthur A.\ Gleckler, Eric Knauel, Jonathan Rees, and Andr\'e
%van Tonder thoroughly proofread early versions of the report.
Множество людей внесли существенный вклад в данную редакцию стандарта. В связи с этим мы
благодарим Aziz Ghuloum и Andr\'e van Tonder за предоставление эталонной реализации системы
библиотек. Мы благодарим Alan Bawden, John Cowan, Sebastian Egner, Aubrey Jaffer, Shiro Kawai,
Bradley Lucier и Andr\'e van Tonder за предоставление уникальной информации о разработке языка.
Marc Feeley, Martin Gasbichler, Aubrey Jaffer, Lars T Hansen, Richard Kelsey, Olin Shivers и
Andr\'e van Tonder написали SRFI, служащий исходными данными стандарта. Marcus
Crestani, David Frese, Aziz Ghuloum, Arthur A.\ Gleckler, Eric Knauel, Jonathan Rees и Andr\'e
van Tonder тщательно вычитывали ранние версии стандарта.\vspace{4mm}

%We would also like to thank the following people for their
%help in creating this report: Lauri Alanko,
%Eli Barzilay, Alan Bawden, Brian C.\ Barnes, Per Bothner, Trent Buck,
%Thomas Bushnell, Taylor Campbell, Ludovic Court\`es, Pascal Costanza,
%John Cowan, Ray Dillinger, Jed Davis, J.A.\ ``Biep'' Durieux, Carl Eastlund,
%Sebastian Egner, Tom Emerson, Marc Feeley, Matthias Felleisen, Andy
%Freeman, Ken Friedenbach, Martin Gasbichler, Arthur A.\ Gleckler, Aziz
%Ghuloum, Dave Gurnell, Lars T Hansen, Ben Harris, Sven Hartrumpf, Dave
%Herman, Nils M.\ Holm, Stanislav Ievlev, James Jackson, Aubrey Jaffer,
%Shiro Kawai, Alexander Kjeldaas, Eric Knauel, Michael Lenaghan, Felix Klock,
%Donovan Kolbly, Marcin Kowalczyk, Thomas Lord, Bradley Lucier, Paulo
%J.\ Matos, Dan Muresan, Ryan Newton, Jason Orendorff, Erich Rast, Jeff
%Read, Jonathan Rees, Jorgen Sch\"afer, Paul Schlie, Manuel Serrano,
%Olin Shivers, Jonathan Shapiro, Jens Axel S\o{}gaard, Jay Sulzberger,
%Pinku Surana, Mikael Tillenius, Sam Tobin-Hochstadt, David Van Horn,
%Andr\'e van Tonder, Reinder Verlinde, Alan Watson, Andrew Wilcox, Jon
%Wilson, Lynn Winebarger, Keith Wright, and Chongkai Zhu.
Мы также хотели бы поблагодарить следующих людей за их помощь в создании данного стандарта: Lauri Alanko,
Eli Barzilay, Alan Bawden, Brian C.\ Barnes, Per Bothner, Trent Buck, Thomas Bushnell, Taylor
Campbell, Ludovic Court\`es, Pascal Costanza, John Cowan, Ray Dillinger, Jed Davis,
J.A.\ ``Biep'' Durieux, Carl Eastlund, Sebastian Egner, Tom Emerson, Marc Feeley, Matthias
Felleisen, Andy Freeman, Ken Friedenbach, Martin Gasbichler, Arthur A.\ Gleckler, Aziz Ghuloum,
Dave Gurnell, Lars T Hansen, Ben Harris, Sven Hartrumpf, Dave Herman, Nils M.\ Holm, Stanislav
Ievlev, James Jackson, Aubrey Jaffer, Shiro Kawai, Alexander Kjeldaas, Eric Knauel, Michael
Lenaghan, Felix Klock, Donovan Kolbly, Marcin Kowalczyk, Thomas Lord, Bradley Lucier, Paulo
J.\ Matos, Dan Muresan, Ryan Newton, Jason Orendorff, Erich Rast, Jeff Read, Jonathan Rees,
Jorgen Sch\"afer, Paul Schlie, Manuel Serrano, Olin Shivers, Jonathan Shapiro, Jens Axel
S\o{}gaard, Jay Sulzberger, Pinku Surana, Mikael Tillenius, Sam Tobin-Hochstadt, David Van Horn,
Andr\'e van Tonder, Reinder Verlinde, Alan Watson, Andrew Wilcox, Jon Wilson, Lynn Winebarger,
Keith Wright, and Chongkai Zhu.\vspace{4mm}

%We would like to thank the following people for their help in creating
%the previous revisions of this report: Alan Bawden, Michael
%Blair, George Carrette, Andy Cromarty, Pavel Curtis, Jeff Dalton, Olivier Danvy,
%Ken Dickey, Bruce Duba, Marc Feeley,
%Andy Freeman, Richard Gabriel, Yekta G\"ursel, Ken Haase, Robert
%Hieb, Paul Hudak, Morry Katz, Chris Lindblad, Mark Meyer, Jim Miller, Jim Philbin,
%John Ramsdell, Mike Shaff, Jonathan Shapiro, Julie Sussman,
%Perry Wagle, Daniel Weise, Henry Wu, and Ozan Yigit.
Мы хотели бы благодарить следующих людей за их помощь в создании предыдущих редакций данного
стандарта: Alan Bawden, Michael Blair, George Carrette, Andy Cromarty, Pavel Curtis, Jeff
Dalton, Olivier Danvy, Ken Dickey, Bruce Duba, Marc Feeley, Andy Freeman, Richard Gabriel, Yekta
G\"ursel, Ken Haase, Robert Hieb, Paul Hudak, Morry Katz, Chris Lindblad, Mark Meyer, Jim
Miller, Jim Philbin, John Ramsdell, Mike Shaff, Jonathan Shapiro, Julie Sussman, Perry Wagle,
Daniel Weise, Henry Wu и Ozan Yigit.\vspace{4mm}

%We thank Carol Fessenden, Daniel
%Friedman, and Christopher Haynes for permission to use text from the Scheme 311
%version 4 reference manual.  We thank Texas Instruments, Inc.~for permission to
%use text from the {\em TI Scheme Language Reference Manual}~\cite{TImanual85}.
%We gladly acknowledge the influence of manuals for MIT Scheme~\cite{MITScheme},
%T~\cite{Rees84}, Scheme 84~\cite{Scheme84}, Common Lisp~\cite{CLtL},
%Chez Scheme~\cite{csug7}, PLT~Scheme~\cite{mzscheme352},
%and Algol 60~\cite{Naur63}.
Мы благодарим Carol Fessenden, Daniel
Friedman и Christopher Haynes за разрешение использовать текст из Scheme 311
version 4 reference manual.  Мы благодарим Texas Instruments, Inc.~за разрешение
использовать текст из {\em TI Scheme Language Reference Manual}~\cite{TImanual85}.
Мы охотно признаём влияние руководств MIT Scheme~\cite{MITScheme},
T~\cite{Rees84}, Scheme 84~\cite{Scheme84}, Common Lisp~\cite{CLtL},
Chez Scheme~\cite{csug7}, PLT~Scheme~\cite{mzscheme352},
и Algol 60~\cite{Naur63}.\vspace{4mm}

%\vest We also thank Betty Dexter for the extreme effort she put into
%setting this report in \TeX, and Donald Knuth for designing the program
%that caused her troubles.
\vest Мы также благодарим Betty Dexter за титанические усилия, приложенные ей при
вводе данного стандарта в \TeX и Donald Knuth за разработку программы,
вызвавшей эти трудности.\vspace{4mm}

%\vest The Artificial Intelligence Laboratory of the
%Massachusetts Institute of Technology, the Computer Science
%Department of Indiana University, the Computer and Information
%Sciences Department of the University of Oregon, and the NEC Research
%Institute supported the preparation of this report.  Support for the MIT
%work was provided in part by
%the Advanced Research Projects Agency of the Department of Defense under Office
%of Naval Research contract N00014-80-C-0505.  Support for the Indiana
%University work was provided by NSF grants NCS 83-04567 and NCS
%83-03325.
\vest The Artificial Intelligence Laboratory of the
Massachusetts Institute of Technology, the Computer Science
Department of Indiana University, the Computer and Information
Sciences Department of the University of Oregon и the NEC Research
Institute способствовали подготовке данного стандарта.  Поддержка работы MIT
частично обеспечивалась
the Advanced Research Projects Agency of the Department of Defense under Office
of Naval Research contract N00014-80-C-0505.  Support for the Indiana
University work was provided by NSF grants NCS 83-04567 and NCS
83-03325.


%%% Local Variables:
%%% mode: latex
%%% TeX-master: "r6rs"
%%% End:
   \par
\vskip 2ex
%\clearchaptergroupstar{Description of the language} %\unskip\vskip -2ex
\clearchaptergroupstar{Описание языка} %\unskip\vskip -2ex
%\chapter{Overview of Scheme}
\chapter{Введение в Scheme}
\label{semanticchapter}

%This chapter gives an overview of Scheme's semantics.
%The purpose of this overview is to explain
%enough about the basic concepts of the language to facilitate
%understanding of the subsequent chapters of the report, which are
%organized as a reference manual.  Consequently, this overview is
%not a complete introduction to the language, nor is it precise
%in all respects or normative in any way.
В данной главе описано введение в семантику Scheme. Целью данного введения является объяснение
фундаментальных концепций языка, достаточных для облегчения понимания последующих глав работы,
организованных в виде справочного руководства. Поэтому данный краткий обзор не является полным
введением в язык, и при этом он не всегда точен в некоторых аспектах или не всегда нормативен.

%\vest Following Algol, Scheme is a statically scoped programming
%language.  Each use of a variable is associated with a lexically
%apparent binding of that variable.
После Algol Scheme является языком программирования со статическими областями
видимости. Каждое использование переменной ассоциировано с лексически явным
связыванием этой переменной.

%\vest Scheme has latent as opposed to manifest types
%\cite{WaiteGoos}.  Types
%are associated with objects\mainindex{object} (also called values) rather than
%with variables.  (Some authors refer to languages with latent types as
%untyped, weakly typed or dynamically typed languages.)  Other languages with
%latent types are Python, Ruby, Smalltalk, and other dialects of Lisp.  Languages
%with manifest types (sometimes referred to as strongly typed or
%statically typed languages) include Algol 60, C, C\#, Java, Haskell, and ML.
\vest Scheme имеет неявные, в отличие от декларативных, типы\cite{WaiteGoos}. Типы связаны с
объектами\mainindex{object} (также называемыми значениями), а не с переменными. (Некоторые
авторы называют языки с неявными типами нетипизированными, слабо типизированными или динамически
типизированными языками.) Другими языками с неявными типами являются Python, Ruby, Smalltalk, а
также другие диалекты Lisp. Языки с декларативными типами (иногда называемые строго
типизированными или статически типизированными языками) включают Algol 60, C, C\#, Java, Haskell
и ML.

%\vest All objects created in the course of a Scheme computation, including
%procedures and continuations, have unlimited extent.
%No Scheme object is ever destroyed.  The reason that
%implementations of Scheme do not (usually!)\ run out of storage is that
%they are permitted to reclaim the storage occupied by an object if
%they can prove that the object cannot possibly matter to any future
%computation.  Other languages in which most objects have unlimited
%extent include C\#, Java, Haskell, most Lisp dialects, ML, Python,
%Ruby, and Smalltalk.
\vest Все объекты, созданные в процессе вычисления Scheme, включая процедуры и продолжения,
имеют неограниченный экстент. Ни один объект Scheme никогда не разрушается. Причина того, что
реализации Scheme полностью не расходуют (обычно!)\ память, состоит в том, что им разрешается
возвращать память, занятую объектом, если они смогут выяснить, что объект не имеет возможности
влиять ни на одно будущее вычисление. Другие языки, в которых большинство объектов имеет
неограниченный экстент, включают C\#, Java, Haskell, большинство диалектов Lisp, ML, Python,
Ruby и Smalltalk.

%Implementations of Scheme must be properly tail-recursive.
%This allows the execution of an iterative computation in constant space,
%even if the iterative computation is described by a syntactically
%recursive procedure.  Thus with a properly tail-recursive implementation,
%iteration can be expressed using the ordinary procedure-call
%mechanics, so that special iteration constructs are useful only as
%syntactic sugar.
Реализации Scheme должны подчиняться правилам хвостовой рекурсии. Это даёт возможность
выполнения итеративного вычисления в постоянном пространстве, даже если итеративное вычисление
описано синтаксически рекурсивной процедурой. Таким образом, при реализации
поддержки хвостовой рекурсии, итерация может быть выражена с помощью обычной техники вызова
процедуры, так, чтобы специальные итеративные конструкции были применимы только в качестве
синтаксического сахара.

%\vest Scheme was one of the first languages to support procedures as
%objects in their own right.  Procedures can be created dynamically,
%stored in data structures, returned as results of procedures, and so
%on.  Other languages with these properties include Common Lisp,
%Haskell, ML, Ruby, and Smalltalk.
\vest Scheme был одним из первых языков, поддерживающих процедуры как объекты сами по себе.
Процедуры могут динамически создаваться, сохраняться в структурах данных, возвращаться как
результаты процедур, и так далее. Другие языки с этими свойствами включают Common Lisp,
Haskell, ML, Ruby и Smalltalk.

%\vest One distinguishing feature of Scheme is that continuations, which
%in most other languages only operate behind the scenes, also have
%``first-class'' status.  First-class continuations are useful for implementing a
%wide variety of advanced control constructs, including non-local exits,
%backtracking, and coroutines.
\vest Одной из отличительных особенностей Scheme является то, что продолжения, которые в
большинстве других языков функционируют только внутренне, также имеют "первоклассный"
статус. Первоклассные продолжения полезны для реализации широкого разнообразия передовых
управляющих конструкций, включая нелокальные выходы, бектрекинг и сопрограммы.

%In Scheme, the argument expressions of a procedure call are evaluated
%before the procedure gains control, whether the procedure needs the
%result of the evaluation or not.  C, C\#, Common Lisp, Python,
%Ruby, and Smalltalk are other languages that always evaluate argument
%expressions before invoking a procedure.  This is distinct from the
%lazy-evaluation semantics of Haskell, or the call-by-name semantics of
%Algol 60, where an argument expression is not evaluated unless its
%value is needed by the procedure.
Выражения аргументов вызова процедуры в Scheme вычисляются перед получением процедурой
управления независимо от необходимости для процедуры результата вычисления.
C, C\#, Common Lisp, Python, Ruby и Smalltalk - другие языки, которые всегда вычисляют выражения
аргументов перед вызовом процедуры. Это отличается от семантики ленивого вычисления Haskell или
семантики вызова по имени Algol 60, где выражение аргумента не вычисляется, если процедура не
нуждается в его значении.

%Scheme's model of arithmetic provides a rich set of numerical types
%and operations on them.  Furthermore, it distinguishes \textit{exact}
%and \textit{inexact} number objects: Essentially, an exact number
%object corresponds to a number exactly, and an inexact number object
%is the result of a computation that involved rounding or other errors.
Арифметическая модель Scheme предоставляет богатый набор численных типов и операций с
ними. Кроме того, она различает \textit{точные} и \textit{неточные} числовые объекты: По
существу, точный числовой объект точно соответствует числу, а неточный числовой объект является
результатом вычисления, которое повлекло за собой округление или другие ошибки.

%\section{Basic types}
\section{Основные типы}

%Scheme programs manipulate \textit{objects}, which are also referred
%to as \textit{values}.
%Scheme objects are organized into sets of values called \textit{types}.
%This section gives an overview of the fundamentally important types of the
%Scheme language.  More types are described in later chapters.
Программы Scheme оперируют \textit{объектами}, которые также называются
\textit{значениями}. Объекты Scheme организованы в наборы значений, которые называются
\textit{типами}. В этой секции дан краткий обзор существенно важных типов языка Scheme. Больше
типов описано в последующих главах.

\begin{note}
  %As Scheme is latently typed, the use of the term \textit{type} in
  %this report differs from the use of the term in the context of other
  %languages, particularly those with manifest typing.
  Поскольку Scheme латентно типизирован, использование термина \textit{тип} в данной работе
  отличается от использования термина в контексте других языков, особенно с явно объявляемой
  типизацией.
\end{note}

%\paragraph{Booleans}
\paragraph{Булевые}

%\mainindex{boolean}A boolean is a truth value, and can be either
%true or false.  In Scheme, the object for ``false'' is written
%\schfalse{}.  The object for ``true'' is written \schtrue{}.  In
%most places where a truth value is expected, however, any object different from
%\schfalse{} counts as true.
\mainindex{boolean}Булевый тип является истинностным значением и может быть true или
false. Объект ``false'' в Scheme записывается \schfalse{}.  Объект ``true'' записывается
\schtrue{}. В большинстве мест, где ожидается истинностное значение, однако, любой объект,
отличный от \schfalse{} интерпретитуется как true.

%\paragraph{Numbers}
\paragraph{Числовые}

%\mainindex{number}Scheme supports a rich variety of numerical data types, including
%objects representing integers of arbitrary precision, rational numbers, complex numbers, and
%inexact numbers of various kinds.  Chapter~\ref{numbertypeschapter} gives an
%overview of the structure of Scheme's numerical tower
\mainindex{number}Scheme поддерживает множество числовых типов данных, включая
объекты, представляющие целые числа произвольной точности, рациональные числа, сложные числа и
приближённые числа различных видов. В Главе~\ref{numbertypeschapter} дан краткий обзор структуры
башни численных типов Scheme.

%\paragraph{Characters}
\paragraph{Знаковые}

%\mainindex{character}Scheme characters mostly correspond to textual characters.
%More precisely, they are isomorphic to the \textit{scalar values} of
%the Unicode standard.
Символы Scheme по большей части соответствуют текстовым символам. Более точно,
они изоморфны к скалярным величинам стандарта Unicode.

%\paragraph{Strings}
\paragraph{Строковые}

%\mainindex{string}Strings are finite sequences of characters with fixed length and thus
%represent arbitrary Unicode texts.
\mainindex{string}Строки являются конечными последовательностями символов фиксированной длины и,
таким образом, представляют произвольные тексты Unicode.

\newpage

%\paragraph{Symbols}
\paragraph{Символьные}

%\mainindex{symbol}A symbol is an object representing a string,
%the symbol's \textit{name}.
%Unlike strings, two symbols whose names are spelled the same
%way are never distinguishable.  Symbols are useful for many applications;
%for instance, they may be used the way enumerated values are used in
%other languages.
Символ является объектом, представляющим строку, \textit{имя} символа. В отличие от строки, два символа,
имена которых записаны в том же порядке, никоим образом не различаются. Символы полезны для
многих применений; например, они могут использоваться, перечислимые значения пути
используются на других языках.

%\paragraph{Pairs and lists}
\paragraph{Пары и списки}

\mainindex{pair}\mainindex{list}
%A pair is a data structure with two components.  The most common use
%of pairs is to represent (singly linked) lists, where the first
%component (the ``car'') represents the first element of the list, and
%the second component (the ``cdr'') the rest of the list.  Scheme also
%has a distinguished empty list, which is the last cdr in a chain of
%pairs that form a list.
Пара является структурой данных с двумя компонентами. Наиболее общее применение пар относится
к представлению (отдельно связанных) списков, где первый компонент ("car") представляет первый
элемент списка, а второй компонент ("cdr") - остальную часть списка. Scheme также имеет
отдельный пустой список, который является последним cdr в цепочке пар, формирующих
список.

%\paragraph{Vectors}
\paragraph{Векторы}

%\mainindex{vector}Vectors, like lists, are linear data structures
%representing finite sequences of arbitrary objects.
%Whereas the elements of a list are accessed
%sequentially through the chain of pairs representing it,
%the elements of a vector are addressed by integer indices.
%Thus, vectors are more appropriate than
%lists for random access to elements.
\mainindex{vector}Векторы, как и списки, являются линейными структурами данных, представляющими конечные
последовательности произвольных объектов. Поскольку к элементам списка получают
доступ последовательно через цепь пар, представляющих его, к элементам вектора обращаются
целыми индексами. Таким образом, векторы предпочтительнее списков для
произвольного доступа к элементам.

%\paragraph{Procedures}
\paragraph{Процедуры}

%\mainindex{procedure}Procedures are values in Scheme.
\mainindex{procedure}В Scheme процедуры являются значениями.

%\section{Expressions}
\section{Выражения}

%The most important elements of Scheme code are
%\mainindex{expression}\textit{expressions}.  Expressions can be
%\textit{evaluated}, producing a \textit{value}.  (Actually, any number
%of values---see section~\ref{multiplereturnvaluessection}.)  The most
%fundamental expressions are literal expressions:
Важнейшими элементами кода Scheme являются \mainindex{expression}\textit{выражения}. Выражения
могут быть \textit{вычислены}, порождая \textit{значение}. (Фактически, любое количество
значений---см. раздел~\ref{multiplereturnvaluessection}.) Самые фундаментальные выражения -
литеральные выражения:

\begin{scheme}
\bfseries{\schtrue{}} \ev \bfseries{\schtrue}
\bfseries{23} \ev \bfseries{23}%
\end{scheme}

%This notation means that the expression \schtrue{} evaluates to
%\schtrue{}, that is, the value for ``true'',  and that the expression
%{\cf 23} evaluates to a number object representing the number 23.
Эта форма записи означает, что выражение \schtrue {} вычисляется как
\schtrue {}, то есть, значение для ``true'', и что выражение {\cf 23} вычисляется к
численному объекту, представляющему число 23.

%Compound expressions are formed by placing parentheses around their
%subexpressions.  The first subexpression identifies an operation; the
%remaining subexpressions are operands to the operation:
Составные выражения формируются помещением круглых скобок вокруг своих подвыражений. Первое
подвыражение идентифицирует операцию; остальные подвыражения являются операндами
операции:
%
\begin{scheme}
\bfseries{(+ 23 42)} \ev \bfseries{65}
\bfseries{(+ 14 (* 23 42))} \ev \bfseries{980}%
\end{scheme}
%
%In the first of these examples, {\cf +} is the name of
%the built-in operation for addition, and {\cf 23} and {\cf 42} are the
%operands.  The expression {\cf (+ 23 42)} reads as ``the sum of 23 and
%42''.  Compound expressions can be nested---the second example reads
%as ``the sum of 14 and the product of 23 and 42''.
В начале этого примера {\bfseries\cf +} является именем встроенной операции сложения, а
{\bfseries\cf 23} и {\bfseries\cf 42} - операндами. Выражение {\bfseries\cf (+ 23 42)}
читается как ``сумма 23 и 42''. Составные выражения могут вкладываться---следующий пример
читается как ``сумма 14 и произведения 23 и 42''.

%As these examples indicate, compound expressions in Scheme are always
%written using the same prefix notation\mainindex{prefix notation}.  As
%a consequence, the parentheses are needed to indicate structure.
%Consequently, ``superfluous'' parentheses, which are often permissible in
%mathematical notation and also in many programming languages, are not
%allowed in Scheme.
Как показывают эти примеры, составные выражения в Scheme всегда записываются с помощью одной и
той же префиксной нотации\mainindex{prefix notation}. Следствием является необходимость круглых
скобок для указания структуры. Следовательно, ``лишние'' круглые скобки, которые часто
допускаются в математической нотации, а также во многих языках программирования, в Scheme не
позволяются.

%As in many other languages, whitespace (including line endings) is not
%significant when it separates subexpressions of an expression, and
%can be used to indicate structure.
Как и во многих других языках, пробельные символы (включая конец строки) являются не значащими,
когда они отделяют подвыражения выражений, и могут использоваться для указания структуры.

%\section{Variables and binding}
\section{Переменные и привязка}

%\mainindex{variable}\mainindex{binding}\mainindex{identifier}Scheme
%allows identifiers to stand for locations containing values.
%These identifiers are called variables.  In many cases, specifically
%when the location's value is never modified after its creation, it is
%useful to think of the variable as standing for the value directly.
\mainindex{variable}\mainindex{binding}\mainindex{identifier}Scheme позволяет привязывать
идентификаторы к содержащим значения ячейкам памяти. Такие идентификаторы называются
переменными. Во многих случаях, определенно когда значение ячейки памяти никогда не изменяется
после её создания, полезно думать о переменной как привязанной к значению непосредственно.

\begin{scheme}
\bfseries(let ((x 23)
\bfseries      (y 42))
\bfseries  (+ x y)) \ev \textbf{65}%
\end{scheme}

%In this case, the expression starting with {\cf let} is a binding
%construct.  The parenthesized structure following the {\cf let} lists
%variables alongside expressions: the variable {\cf x} alongside {\cf
%  23}, and the variable {\cf y} alongside {\cf 42}.  The {\cf let}
%expression binds {\cf x} to 23, and {\cf y} to 42.  These bindings are
%available in the \textit{body} of the {\cf let} expression, {\cf (+ x
%  y)}, and only there.
В данном случае выражение, начинающееся с {\cf\bfseries let} является конструкцией привязки. В
окружённой скобками структуре после {\cf\bfseries let} перечислены переменные совместно с выражениями:
переменная {\cf\bfseries x} совместно с {\cf\bfseries 23} и переменная {\cf\bfseries y}
совместно с {\cf\bfseries 42}. Выражение {\cf\bfseries let} связывает {\cf\bfseries x} с 23 и
{\cf\bfseries y} c 42. Эти привязки доступны в \textit{теле} выражения {\cf\bfseries let},
{\cf\bfseries (+ x y)}, и только там.

%\section{Definitions}
\section{Определения}

%\index{definition}The variables bound by a {\cf let} expression
%are \textit{local}, because their bindings are visible only in the
%{\cf let}'s body.  Scheme also allows creating top-level bindings for
%identifiers as follows:
\index{definition}Переменные, связанные выражением {\cf\bfseries let}, являются
\textit{локальными}, так как их связывания видимы только в теле {\cf\bfseries let}. Scheme также
позволяет создавать связывания верхнего уровня для идентификаторов следующим образом:

\begin{scheme}
\bfseries(define x 23)
\bfseries(define y 42)
\bfseries(+ x y) \ev \textbf{65}%
\end{scheme}

%(These are actually ``top-level'' in the body of a top-level program or library;
%see section~\ref{librariesintrosection} below.)
(Они фактически являются "верхним уровнем" в теле программы верхнего уровня или библиотеки; см. секцию
~\ref {librariesintrosection} ниже.)

%The first two parenthesized structures are \textit{definitions}; they
%create top-level bindings, binding {\cf x} to 23 and {\cf y} to 42.
%Definitions are not expressions, and cannot appear in all places
%where an expression can occur.  Moreover, a definition has no value.
Первые две заключенные в скобки структуры являются \textit {определениями}; они создают
привязки верхнего уровня, связывая {\cf x} с 23, а {\cf y} с 42. Определения не являются
выражениями и не могут находиться там, тде где может находиться выражение.
Кроме того, определение не имеет значения.

%Bindings follow the lexical structure of the program:  When several
%bindings with the same name exist, a variable refers to the binding
%that is closest to it, starting with its occurrence in the program
%and going from inside to outside, and referring to a top-level
%binding if no
%local binding can be found along the way:
Привязки подчиняются лексической структуре программы: при наличии нескольких одноименных
привязок переменная соотносится с ближайшей к ней на пути изнутри снаружу привязкой, начиная от
её появления в программе, и с привязкой верхнего уровня, если локальная привязка не может быть
найдена на этом пути:

\begin{scheme}
\bfseries(define x 23)
\bfseries(define y 42)
\bfseries(let ((y 43))
\end{scheme}

\newpage

\begin{scheme}
\bfseries  (+ x y)) \ev \textbf{66}

\bfseries(let ((y 43))
\bfseries  (let ((y 44))
\bfseries    (+ x y))) \ev \textbf{67}%
\end{scheme}

%\section{Forms}
\section{Формы}

%While definitions are not expressions, compound expressions and
%definitions exhibit similar syntactic structure:
Хотя определения не являются выражениями, составные выражения и определения
имеют схожую синтаксическую структуру:
%
\begin{scheme}
\bfseries(define x 23)
\bfseries(* x 2)%
\end{scheme}
%
%While the first line contains a definition, and the second an
%expression, this distinction depends on the bindings for {\cf define}
%and {\cf *}.  At the purely syntactical level, both are
%\textit{forms}\index{form}, and \textit{form} is the general name for
%a syntactic part of a Scheme program.  In particular, {\cf 23} is a
%\textit{subform}\index{subform} of the form {\cf (define x 23)}.
При этом первая линия содержит определение, а следующая - выражение, данное различие
основывается на связывании {\cf\bfseries define} и {\cf\bfseries *}. На чисто синтаксическом
уровне обе являются \textit{формами}\index{form}, а \textit{форма} является обобщённым названием
синтаксической части программы Scheme. В частности, {\cf\bfseries 23} является
\textit{подформой} \index{subform} формы {\cf\bfseries (define x 23)}.

%\section{Procedures}
\section{Процедуры}
\label{proceduressection}

%\index{procedure}Definitions can also be used to define
%procedures:
\index{procedure}Определения могут также использоваться для определения процедур.

\begin{scheme}
\bfseries(define (f x)
\bfseries  (+ x 42))

\bfseries(f 23) \ev \textbf{65}%
\end{scheme}

%A procedure is, slightly simplified, an abstraction of an
%expression over objects.  In the example, the first definition defines a procedure
%called {\cf f}.  (Note the parentheses around {\cf f x}, which
%indicate that this is a procedure definition.)  The expression {\cf (f
%  23)} is a \index{procedure call}procedure call, meaning,
%roughly, ``evaluate {\cf (+ x 42)} (the body of the procedure) with
%{\cf x} bound to 23''.
Процедура, несколько упрощённо, является абстракцией выражения посредством объектов. В первом определении
примера определена процедура, названная {\cf\bfseries f}. (Обратите внимание на круглые скобки
вокруг {\cf\bfseries f x}, обозначающие, что это - определение процедуры.) Выражение
{\cf\bfseries (f 23)} является \index{procedure call} вызовом процедуры, приблизительно означающим
"вычислить {\cf\bfseries (+ x 42)} (тело процедуры) с {\cf\bfseries x}, привязанным к \textbf{23}".

%As procedures are objects, they can be passed to other
%procedures:
Поскольку процедуры являются объектами, их можно передавать в другие процедуры:

%
\begin{scheme}
\bfseries(define (f x)
\bfseries  (+ x 42))

\bfseries(define (g p x)
\bfseries  (p x))

\bfseries(g f 23) \ev \textbf{65}%
\end{scheme}

%In this example, the body of {\cf g} is evaluated with {\cf p}
%bound to {\cf f} and {\cf x} bound to 23, which is equivalent
%to {\cf (f 23)}, which evaluates to 65.
В этом примере тело {\cf\bfseries g} вычисляется с {\cf\bfseries p}, привязанным к {\cf\bfseries
  f}, и {\cf\bfseries x}, привязанным к \textbf{23}, что эквивалентно {\cf\bfseries (f 23)} и
вычисляется как \textbf{65}.

%In fact, many predefined operations of Scheme are provided not by
%syntax, but by variables whose values are procedures.
%The {\cf +} operation, for example, which receives
%special syntactic treatment in many other languages, is just a regular
%identifier in Scheme, bound to a procedure that adds number objects.  The
%same holds for {\cf *} and many others:
Фактически многие предопределённые операции Scheme обеспечиваются не синтаксисом, а
переменными, значениями которых являются процедуры. Операция {\cf\bfseries +}, например, приобретающая
специальную синтаксическую трактовку во многих других языках, в Scheme является всего лишь
регулярным идентификатором, связанным с процедурой, складывающей числовые объекты.
То же самое касается и {\cf\bfseries *}, и многих других:

\begin{scheme}
\bfseries(define (h op x y)
\bfseries  (op x y))

\bfseries(h + 23 42) \ev \textbf{65}
\bfseries(h * 23 42) \ev \textbf{966}%
\end{scheme}

%Procedure definitions are not the only way to create procedures.  A
%{\cf lambda} expression creates a new procedure as an object, with no
%need to specify a name:
Определения процедур - не единственный способ создания процедур. {\cf\bfseries lambda}-выражение
создаёт новую процедуру в качестве объекта без необходимости указания имени:

\begin{scheme}
\bfseries((lambda(x)(+ x 42))23) \ev \textbf{65}%
\end{scheme}

%The entire expression in this example is a procedure call; {\cf
%  (lambda (x) (+ x 42))}, evaluates to a procedure that takes a single
%number object and adds 42 to it.
Всё выражение в этом примере является вызовом процедуры; {\cf\bfseries (lambda (x) (+ x 42))}
вычисляется как процедура, принимающая одиночный числовой объект и добавляющая к нему 42.

%\section{Procedure calls and syntactic keywords}
\section{Вызовы процедур и синтаксические ключевые слова}

%Whereas {\cf (+ 23 42)}, {\cf (f 23)}, and {\cf ((lambda (x) (+ x 42))
%  23)} are all examples of procedure calls, {\cf lambda} and {\cf
%  let} expressions are not.  This is because {\cf let}, even though
%it is an identifier, is not a variable, but is instead a \textit{syntactic
%  keyword}\index{syntactic keyword}.  A form that has a
%syntactic keyword as its first subexpression obeys special rules determined by
%the keyword.  The {\cf define} identifier in a definition is also a
%syntactic keyword.  Hence, definitions are also not procedure calls.
Хотя и {\cf\bfseries (+ 23 42)}, и {\cf\bfseries (f 23)}, и {\cf\bfseries ((lambda (x) (+ x 42))
  23)} являются примерами вызовов процедур, выражения {\cf\bfseries lambda} и {\cf\bfseries let} -
нет. Это потому что {\cf\bfseries let}, хоть и идентификатор, но
не переменная, а \textit{синтаксическое ключевое слово}\index{syntactic
  keyword}. Форма, содежащая синтаксическое ключевое слово в качестве своего первого подвыражения,
подчиняется специальным правилам, определяемым ключевым словом. Идентификатор {\cf\bfseries
  define} в определении также является синтаксическим ключевым словом. Следовательно,
определения также не являются вызовами процедур.

%The rules for the {\cf lambda} keyword specify that the first
%subform is a list of parameters, and the remaining subforms are the body of
%the procedure.  In {\cf let} expressions, the first subform is a list
%of binding specifications, and the remaining subforms constitute a body of
%expressions.
Правилами для ключевого слова {\cf\bfseries lambda} определено, что первая подформа является
списком параметров, а остальные подформы - телом процедуры. В выражении {\cf\bfseries let}
первая подформа является списком спецификаций привязки, а остальные подформы образовывают тело
выражений.

%Procedure calls can generally be distinguished from these
%\textit{special forms}\mainindex{special form} by
%looking for a syntactic keyword in the first position of an
%form: if the first position does not contain a syntactic keyword, the expression
%is a procedure call.
%(So-called \textit{identifier macros} allow creating other kinds of
%special forms, but are comparatively rare.)
%The set of syntactic keywords of Scheme is
%fairly small, which usually makes this task fairly simple.
%It is possible, however, to create new bindings for syntactic keywords; see
%section~\ref{macrosintrosection} below.
Обычно вызовы процедур можно отличить от таких \textit{специальных форм} \mainindex{special
  form}, с помощью поиска синтаксического ключевого слова в первом положении формы: если
в первое положении не содержится синтаксического ключевого слова, выражение является вызовом
процедуры. (Так называемый \textit {макрос идентификатора} позволяет создавать
другие виды специальных форм, но сравнительно редко.) Набор синтаксических ключевых слов Scheme
является довольно небольшим, что обычно делает эту задачу довольно простой. Возможно,
однако, создание новых привязок для синтаксических ключевых слов;
см. секцию~\ref{macrosintrosection} ниже.

%\section{Assignment}
\section{Присваивание}

%Scheme variables bound by definitions or {\cf let} or {\cf lambda}
%expressions are not actually bound directly to the objects specified in the
%respective bindings, but to locations containing these objects.  The
%contents of these locations can subsequently be modified destructively
%via \textit{assignment}\index{assignment}:
Переменные Scheme, связанные с определениями или с выражениями {\cf\bfseries let} или
{\cf\bfseries lambda}, привязываются фактически не непосредственно к объектам, определённым в
соответствующих привязках, а к адресам памяти, содержащим эти объекты. Содержимое по этим
адресам впоследствии может быть деструктивно изменено с помощью \textit{присваивания}
\index{assignment}:
%
\begin{scheme}
\bfseries(let ((x 23))
\bfseries  (set! x 42)
\bfseries  x) \ev \textbf{42}%
\end{scheme}

\newpage

%In this case, the body of the {\cf let} expression consists of two
%expressions which are evaluated sequentially, with the value of the
%final expression becoming the value of the entire {\cf let}
%expression.  The expression {\cf (set! x 42)} is an assignment, saying
%``replace the object in the location referenced by {\cf x} with 42''.
%Thus, the previous value of {\cf x}, 23, is replaced by 42.
В данном случае тело выражения {\cf\bfseries let} состоит из двух вычисляемых последовательно
выражений со значением финального выражения, принимающего значение всего выражения {\cf\bfseries
  let}. Выражение {\cf\bfseries (set! x 42),} является присваиванием, указывающим "заменить объект по
адресу, на который указывает {\cf\bfseries x}, на 42". Таким образом, предыдущее значение
{\cf\bfseries x} - 23 изменяется на 42.

%\section{Derived forms and macros}
\section{Производные формы и макросы}
\label{macrosintrosection}

%Many of the special forms specified in this report
%can be translated into more basic special forms.
%For example, a {\cf let} expression can be translated
%into a procedure call and a {\cf lambda} expression.  The following two
%expressions are equivalent:
Большинство специальных форм, определённых в данной работе, могут быть приведены к более
простым специальным формам. Например, выражение {\cf\bfseries let} может быть приведено к вызову
процедуры и выражению {\cf\bfseries lambda}. Следующие два выражения эквивалентны:
%
\begin{scheme}
\bfseries(let ((x 23)
\bfseries      (y 42))
\bfseries  (+ x y)) \ev \textbf{65}

\bfseries((lambda (x y) (+ x y)) 23 42) \lev \textbf{65}%
\end{scheme}

%Special forms like {\cf let} expressions are called \textit{derived
%  forms}\index{derived form} because their semantics can be
%derived from that of other kinds of forms by a syntactic
%transformation.  Some procedure definitions are also derived forms.  The
%following two definitions are equivalent:
Специальные формы, такие, как выражения {\cf\bfseries let}, называются \textit{производными
  формами}\index{derived form}, так как их семантика может быть получена из того или иного вида
форм синтаксическим преобразованием. Некоторые определения процедур также являются производными
формами. Следующие два определения эквивалентны:

\begin{scheme}
\bfseries(define (f x)
\bfseries  (+ x 42))

\bfseries(define f
\bfseries  (lambda (x)
\bfseries    (+ x 42)))%
\end{scheme}

%In Scheme, it is possible for a program to create its own derived
%forms by binding syntactic keywords to macros\index{macro}:
В программе Scheme имеется возможность создания своих собственных производных форм путём связывания
синтаксических ключевых слов с макросами \index{macro}:

\begin{scheme}
\bfseries(define-syntax def
\bfseries  (syntax-rules ()
\bfseries    ((def f (p ...) body)
\bfseries     (define (f p ...)
\bfseries       body))))

\bfseries(def f (x)
\bfseries  (+ x 42))%
\end{scheme}

%The {\cf define-syntax} construct specifies that a parenthesized
%structure matching the pattern {\cf (def f (p ...) body)}, where {\cf
%  f}, {\cf p}, and {\cf body} are pattern variables, is translated to
%{\cf (define (f p ...) body)}.  Thus, the {\cf def} form appearing in
%the example gets translated to:
Конструкция {\cf\bfseries define-syntax} определяет, что заключенная в скобки структура,
соответствующая шаблону {\cf\bfseries (def f (p ...) body)}, где {\cf\bfseries f}, {\cf\bfseries
  p} и {\cf\bfseries body} - переменные шаблона, приводится к {\cf\bfseries (define (f p ...)
  body)}. Таким образом, форма {\cf\bfseries def}, находящаяся в примере, приводится к:

\begin{scheme}
\bfseries(define (f x)
\bfseries  (+ x 42))%
\end{scheme}

%The ability to create new syntactic keywords makes Scheme extremely
%flexible and expressive, allowing many of the features
%built into other languages to be derived forms in Scheme.
Возможность создания новых синтаксических ключевых слов делает Scheme чрезвычайно гибким и
выразительным, что позволяет большинству особенностей, встроенных в другие языки, быть
производными формами в Scheme.

%\section{Syntactic data and datum values}
\section{Синтаксические данные и значения datum}\vspace{-4mm}

%A subset of the Scheme objects is called \textit{datum
%  values}\index{datum value}.
%These include booleans, number objects, characters, symbols,
%and strings as well as lists and vectors whose elements are data.  Each
%datum value may be represented in textual form as a
%\textit{syntactic datum}\index{syntactic datum}, which can be written out
%and read back in without loss of information.
%A datum value may be represented by several different syntactic data.
%Moreover, each datum value
%can be trivially translated to a literal expression in a program by
%prepending a {\cf\singlequote} to a corresponding syntactic datum:
Подмножество объектов Scheme называется \textit{значениями datum}\index{datum value}.
Оно включает булевы, численные объекты, знаки, символы и строки, равно как списки и векторы,
элементы которых являются данными. Каждое значение datum может быть представлено в текстовом
виде как \textit{синтаксический datum}\index{syntactic datum}, который может записываться и
считываться без потери информации. Значение datum может быть представлено различными
синтаксическими данными. Кроме того, каждое базисное значение может быть тривиально приведено к
литеральному выражению в программе путём добавления {\cf\singlequote} к соответствующему
синтаксическому datum:

\begin{scheme}
\bfseries'23 \ev \textbf{23}
\bfseries'\schtrue{} \ev \bfseries\schtrue{}
\bfseries'foo \ev \textbf{foo}
\bfseries'(1 2 3) \ev \textbf{(1 2 3)}
\bfseries'\#(1 2 3) \ev \textbf{\#(1 2 3)}%
\end{scheme}

%The {\cf\singlequote} shown in the previous examples
%is not needed for representations of number objects or booleans.
%The syntactic datum {\cf foo} represents a
%symbol with name ``foo'', and {\cf 'foo} is a literal expression with
%that symbol as its value.  {\cf (1 2 3)} is a syntactic datum that
%represents a list with elements 1, 2, and 3, and {\cf '(1 2 3)} is a literal
%expression with this list as its value.  Likewise, {\cf \#(1 2 3)}
%is a syntactic datum that represents a vector with elements 1, 2 and 3, and
%{\cf '\#(1 2 3)} is the corresponding literal.
{\cf\singlequote}, показанный в предыдущих примерах, не нужен для представлений численных
объектов или булевых переменных. Синтаксический базис {\cf\bfseries foo} представляет символ с
именем ``foo'', а {\cf\bfseries 'foo} - литеральное выражение с этим символом в качестве его
значения. {\cf\bfseries (1 2 3)} - синтаксический базис, представляющий список с
элементами 1, 2 и 3, а {\cf\bfseries '(1 2 3)} - литеральное выражение с этим списком в качестве
его значения. Аналогично, {\cf\bfseries \#(1 2 3)} - синтаксический базис, представляющий
вектор с элементами 1, 2 и 3, а {\cf\bfseries '\#(1 2 3)} - соответствующий литерал.


%The syntactic data are a superset of the Scheme forms.  Thus, data
%can be used to represent Scheme forms as data objects.  In
%particular, symbols can be used to represent identifiers.
Синтаксические данные являются расширенным набором форм Scheme. Таким образом, данные могут
использоваться, для представления форм Scheme в виде объектов данных. В частности, символы могут
использоваться для представления идентификаторов.\vspace{-1mm}

\begin{scheme}
\bfseries'(+ 23 42) \ev \bfseries(+ 23 42)
\bfseries'(define (f x) (+ x 42)) \lev \bfseries(define (f x) (+ x 42))%
\end{scheme}

%This facilitates writing programs that operate on Scheme source code,
%in particular interpreters and program transformers.
Это облегчает написание программ, работающих с исходным кодом Scheme, в частности,
интерпретаторов и программных преобразователей.\vspace{-4mm}

%\section{Continuations}
\section{Продолжения}\vspace{-4mm}

%Whenever a Scheme expression is evaluated there is a
%\textit{continuation}\index{continuation} wanting the result of the
%expression.  The continuation represents an entire (default) future
%for the computation.  For example, informally the continuation of {\cf 3}
%in the expression
Всякий раз, когда выражение Scheme оценено есть \textit{продолжение}\index{continuation}, ожидая
результат выражения. Продолжение представляет все будущее (по умолчанию) для
вычисления. Например, произвольно, продолжение {\cf\bfseries 3} в выражении
%
\begin{scheme}
\bfseries(+ 1 3)%
\end{scheme}
%
%adds 1 to it.  Normally these ubiquitous continuations are hidden
%behind the scenes and programmers do not think much about them.  On
%rare occasions, however, a programmer may need to deal with
%continuations explicitly.  The {\cf call-with-current-continuation}
%procedure (see section~\ref{call-with-current-continuation}) allows
%Scheme programmers to do that by creating a procedure that reinstates
%the current continuation.  The {\cf call-with-current-continuation}
%procedure accepts a procedure, calls it immediately with an argument
%that is an \textit{escape procedure}\index{escape procedure}.  This
%escape procedure can then be called with an argument that becomes the
%result of the call to {\cf call-with-current-continuation}.  That is,
%the escape procedure abandons its own continuation, and reinstates the
%continuation of the call to {\cf call-with-current-continuation}.
добавляет 1 к нему. Обычно эти встречающиеся повсюду продолжения скрыты на заднем плане, и
программисты особо не заботятся о них. В редких случаях, однако, программист, возможно, должен
иметь дело с продолжениями явно. Процедура {\cf\bfseries call-with-current-continuation}
(см. секцию ~\ref{call-with-current-continuation}) позволяет программистам Scheme делать это,
создавая процедуру, которая восстанавливает текущее продолжение. Процедура {\cf\bfseries
  call-with-current-continuation} принимает процедуру, вызывает её немедленно с аргументом,
который является \textit{аварийной процедурой} \index{escape procedure}. Эту аварийную процедуру
можно тогда вызвать с аргументом, который становится результатом вызова {\cf\bfseries
  call-with-current-continuation}. Таким образом, аварийная процедура оставляет ее собственное
продолжение, и восстанавливает продолжение вызова к {\cf\bfseries call-with-current-continuation}.

%In the following example, an escape procedure representing the
%continuation that adds 1 to its argument is bound to {\cf escape}, and
%then called with 3 as an argument.  The continuation of the call to
%{\cf escape} is abandoned, and instead the 3 is passed to the
%continuation that adds 1:
В следующем примере, аварийная процедура, представляющая продолжение, которое добавляет 1 к его
аргументу, привязана к {\cf\bfseries escape}, и затем вызывается с 3 как аргумент. Продолжение
вызова {\cf\bfseries escape} оставлено, и вместо этого эти 3 передают к продолжению, которое добавляет
1:
%
\begin{scheme}
\bfseries(+ 1 (call-with-current-continuation
\bfseries       (lambda (escape)
\bfseries         (+ 2 (escape 3))))) \lev \bfseries 4%
\end{scheme}
%
%An escape procedure has unlimited extent: It can be called after the
%continuation it captured has been invoked, and it can be called
%multiple times.  This makes {\cf call-with-current-continuation}
%significantly more powerful than typical non-local control constructs
%such as exceptions in other languages.
Аварийная процедура имеет неограниченный экстент: Она может быть вызвана после вызова
захватившего его продолжения, и это можно вызвать несколько раз. Это делает {\cf\bfseries
  call-with-current-continuation} значительно более мощным чем типичные нелокальные управляющие
конструкции, типа исключений в других языках.

%\section{Libraries}
\section{Библиотеки}
\label{librariesintrosection}

%Scheme code can be organized in components called
%\textit{libraries}\index{library}.  Each library contains
%definitions and expressions.  It can import definitions
%from other libraries and export definitions to other libraries.
Код Scheme может быть организован в компонентах, которые называются
\textit{библиотеками}\index{library}. Каждая библиотека содержит определения и выражения. Она
может импортировать определения из других библиотек и экспортировать определения в другие
библиотеки.

%The following library called {\cf (hello)} exports a definition called
%{\cf hello-world},  and imports the base library (see
%chapter~\ref{baselibrarychapter}) and the simple I/O library (see
%library section~\extref{lib:simpleiosection}{Simple I/O}).  The {\cf
%  hello-world} export is a procedure that displays {\cf Hello World}
%on a separate line:
Следующая библиотека называется {\cf\bfseries (hello)}, экспортирует определение, которое
называется {\cf\bfseries hello-world}, и импортирует основную библиотеку (см. главу
~\ref{baselibrarychapter}), и простая библиотека I/O (см. библиотечную
секцию~\extref{lib:simpleiosection} {Простой ввод/вывод}). Экспорт {\cf\bfseries hello-world} является
процедурой, которая показывает {\cf\bfseries Hello World} на отдельной линии:
%
\begin{scheme}
\bfseries(library (hello)
\bfseries  (export hello-world)
\bfseries  (import (rnrs base)
\bfseries          (rnrs io simple))
\bfseries  (define (hello-world)
\bfseries    (display "Hello World")
\bfseries    (newline)))%
\end{scheme}

%\section{Top-level programs}
\section{Программы верхнего уровня}

%A Scheme program is invoked via a \textit{top-level
%  program}\index{top-level program}.  Like a library, a top-level
%program contains imports, definitions and expressions, and specifies
%an entry point for execution.  Thus a top-level program defines, via
%the transitive closure of the libraries it imports, a Scheme program.
Программа Scheme активизируется посредством \textit{программы верхнего уровня}\index{top-level
  program}. Как и библиотека, программа верхнего уровня содержит импорт, определения и
выражения, и определяет точку входа для выполнения. Таким образом, программа верхнего уровня
определяет, через замкнутое выражение библиотек, которые это импортирует, программу Scheme.

%The following top-level program obtains the first argument from the command line
%via the {\cf command-line} procedure from the \rsixlibrary{programs}
%library (see library chapter~\extref{lib:programlibchapter}{Command-line
%  access and exit values}).  It then opens the file using {\cf
%  open-file-input-port} (see library section~\extref{lib:portsiosection}),
%yielding a \textit{port}, i.e.\ a connection to the file as a data
%source, and calls the {\cf get-bytes-all} procedure to obtain the
%contents of the file as binary data.  It then uses {\cf put-bytes} to
%output the contents of the file to standard output:
Следующая программа верхнего уровня получает первый аргумент из командной строки через процедуру
{\cf\bfseries command-line} из библиотеки {\bfseries\rsixlibrary{programs}} (см. библиотечную
главу~\extref{lib:programlibchapter}{Доступ к командной строке и выходные значения}). Это тогда
открывает файл с помощью {\cf\bfseries open-file-input-port} (см. секцию
~\extref{lib:portsiosection}), приводя к \textit{порт}, т.е.\ к связи с файлом как источником
данных, и вызывает процедуру {\cf\bfseries get-bytes-all} получения содержимого файла в виде
двоичных данных. Это тогда использует {\cf\bfseries put-bytes} для вывода содержимого файла в
стандартный вывод:
%
\begin{scheme}
\#!r6rs
\bfseries(import (rnrs base)
\bfseries        (rnrs io ports)
\bfseries        (rnrs programs))
\bfseries(put-bytes (standard-output-port)
\bfseries           (call-with-port
\bfseries               (open-file-input-port
\bfseries                 (cadr (command-line)))
\bfseries             get-bytes-all))%
\end{scheme}

%%% Local Variables:
%%% mode: latex
%%% TeX-master: "r6rs"
%%% End:
  \par
%\chapter{Requirement levels}
\chapter{Уровни требований}
\label{requirementchapter}

%The key words ``must'', ``must not'', ``should'',
%``should not'', ``recommended'', ``may'', and ``optional'' in this
%report are to be interpreted as described in RFC~2119~\cite{mustard}.
%Specifically:
Ключевые слова, ``must'', ``must not'' ``should'', ``should not'', ``recommended'',
``may'', и ``optional'' в данной работе должны интерпретироваться как описано в RFC~2119~\cite
{mustard}. А именно:

\begin{description}
%\item[must]\mainindex{must} This word means that a statement is an absolute
%  requirement of the specification.
\item[must]\mainindex{must} Это слово означает, что инструкция является абсолютным требованием
  спецификации.
%\item[must not]\mainindex{must not} This phrase means that a statement is an absolute
%  prohibition of the specification.
\item[must not]\mainindex{must not} Эта фраза означает, что инструкция является абсолютным запрещением
  спецификации.
%\item[should]\mainindex{should} This word, or the adjective ``recommended'', means that
%  valid reasons may exist in particular circumstances to ignore a
%  statement, but that the implications must be understood and weighed
%  before choosing a different course.
\item[should]\mainindex{should} Это слово, или прилагательное ``recommended'', означают, что
  в особых обстоятельствах могут существовать веские причины для игнорирования инструкции, но перед
  выбором иной линии поведения последствия должны быть осознаны и взвешены.
%\item[should not]\mainindex{should not} This phrase, or the phrase ``not recommended'', means
%  that valid reasons may exist in particular circumstances when the
%  behavior of a statement is acceptable, but that the implications
%  should be understood and weighed before choosing the course described
%  by the statement.
\item[should not]\mainindex{should not} Эта фраза, или фраза ``not recommended'', означают, что
  в особых обстоятельствах могут существовать веские причины для принятия функционирования
  инструкции, но перед выбором описываемой инструкцией линии поведения последствия
  должны быть осознаны и взвешены.
%\item[may]\mainindex{may} This word, or the adjective ``optional'', means that an item
%  is truly optional.
\item[may]\mainindex{may} Это слово, или прилагательное ``optional'' означают, что
  элемент является действительно дополнительным.
\end{description}

%In particular, this report occasionally uses ``should'' to designate
%circumstances that are outside the specification of this report, but
%cannot be practically detected by an implementation; see
%section~\ref{argumentcheckingsection}.  In such circumstances, a
%particular implementation may allow the programmer to ignore the
%recommendation of the report and even exhibit reasonable behavior.
%However, as the report does not specify the behavior,
%these programs may be unportable, that is, their execution might
%produce different results on different implementations.
В частности, ``should'' в данной работе иногда используется для обозначения обстоятельств,
лежащих вне спецификации данной работы, но не обнаруживаемых реализацией на практике;
см. секцию~\ref{argumentcheckingsection}. При таких обстоятельствах конкретная реализация может
позволить программисту игнорировать рекомендацию в данной работе и даже демонстрировать
адекватное функционирование. Однако, поскольку данная работа не специфицирует функционирование,
такие программы могут быть непортируемыми, то есть, их выполнение может привести к различным
результатам в разных реализациях.

\newpage

%Moreover, this report occasionally uses the phrase ``not required'' to note the
%absence of an absolute requirement.
Кроме того, в данной работе иногда используется фраза ``not required'' для обозначения
отсутствия абсолютного требования.

%%% Local Variables:
%%% mode: latex
%%% TeX-master: "r6rs"
%%% End:
 \par
%\chapter{Numbers}
\chapter{Числа}
\label{numbertypeschapter}
\mainindex{number}

%This chapter describes Scheme's model for numbers.  It is important to
%distinguish between the mathematical numbers, the Scheme objects that
%attempt to model them, the machine representations used to implement
%the numbers, and notations used to write numbers.  In this report, the
%term \textit{number} refers to a mathematical number, and the term
%\textit{number object} refers to a Scheme object representing a
%number.  This report uses the types \type{complex}, \type{real},
%\type{rational}, and \type{integer} to refer to both mathematical
%numbers and number objects.  The \type{fixnum} and \type{flonum} types
%refer to special subsets of the number objects, as determined by
%common machine representations, as explained below.
В этой главе описывается модель Scheme для чисел. Важно различать математические числа, объекты
Scheme, стремящиеся представить их, машинные представления, используемые для реализации чисел, и
используемые для записи чисел нотации. В данной работе термин \textit{число} относится к
математическому числу, а термин \textit{числовой объект} - к объекту Scheme, представляющему
число. В данной работе используются типы \type{complex}, \type{real}, \type{rational} и
\type{integer} для обращения и к математическим числам, и к числовым объектам. Типы
\type{fixnum} и \type{flonum} относятся к специальным подмножествам числовых объектов, как
определяется общими машинными представлениями и объясняется ниже.

%\section{Numerical tower}
\section{Числовая башня}
\label{numericaltypes}
\index{numerical types}

%Numbers may be arranged into a tower of subsets in which each level
%is a subset of the level above it:
Числа могут быть квалифицированы башней подмножеств, в которой каждый уровень является
подмножеством уровня выше него:
\begin{tabbing}
\ \ \ \ \ \ \ \ \ \=\tupe{number} \\
\> \tupe{complex} \\
\> \tupe{real} \\
\> \tupe{rational} \\
\> \tupe{integer}
\end{tabbing}

%For example, 5 is an integer.  Therefore 5 is also a rational,
%a real, and a complex.  The same is true of the number objects
%that model 5.
Например, 5 является целым числом. Поэтому 5 также является рациональным, действительным и
комплексным. То же верно и для числовых объектов, представляющих 5.


%Number objects are organized as a corresponding tower of subtypes
%defined by the predicates {\cf number?}, {\cf complex?}, {\cf real?},
%{\cf rational?}, and {\cf integer?}; see section~\ref{number?}.
%Integer number objects are also called \textit{integer
%  objects}\mainindex{integer object}.
Числовые объекты организованы как соответствующая башня подтипов, определённых предикатами
{\cf\bfseries number?}, {\cf\bfseries complex?}, {\cf\bfseries real?}, {\cf\bfseries racional?},
и {\cf\bfseries integer?}; см. секцию~\ref{number?}. Числовые объекты целого также
называются \textit{объектами целого}\mainindex{integer objects}.

%There is no simple relationship between the subset that contains a
%number and its representation inside a computer.  For example, the
%integer 5 may have several representations.  Scheme's numerical
%operations treat number objects as abstract data, as independent of
%their representation as possible.  Although an implementation of
%Scheme may use many different representations for numbers, this should
%not be apparent to a casual programmer writing simple programs.
Не существует очевидной взаимосвязи между подмножеством, содержащим число, и его представлением
в компьютере. Например, целое число 5 может иметь несколько представлений. Операции с числами в
Scheme интерпретируют числовые объекты как абстрактные данные, столь же независимые от их
представления, насколько возможно. Хотя реализация Scheme может использовать множество различных
представлений чисел, это не должно быть очевидно случайному программисту, пишущему простые
программы.

%\section{Exactness}
\section{Точность}
\label{exactly}

%\mainindex{exactness}It is useful to distinguish between number objects
%that are known to correspond to a number exactly, and those number
%objects whose computation involved rounding or other errors.  For
%example, index operations into data structures may need to know the index
%exactly, as may some operations on polynomial coefficients in a symbolic algebra
%system.  On the other hand, the results of measurements are inherently
%inexact, and irrational numbers may be approximated by rational and
%therefore inexact approximations.  In order to catch uses of numbers
%known only inexactly where exact numbers are required, Scheme
%explicitly distinguishes \defining{exact} from \defining{inexact} number objects.  This
%distinction is orthogonal to the dimension of type.
\mainindex{exactness}Целесообразно различать числовые объекты, о которых известно, что они точно
соответствуют числу, и числовые объекты, вычисление которых повлекло за собой округление или
другие ошибки. Например, операциям индексации структур данных может потребоваться знание точного
индекса, как и некоторым операциям с коэффициентами полинома в системе символьной алгебры. С
другой стороны, результаты измерений являются неточными по определению, и иррациональные
числа могут быть апроксимированы рациональной, и поэтому неточной, апроксимацией. Для обнаружения
использования чисел, известных только приблизительно, там, где требуются точные числа,
Scheme явно различает \defining{точные} числовые объекты от \defining{неточных}. Это
различие независимо от измерения типа.

%A
%number object is exact if it is the value of an exact numerical
%literal or was derived from exact number objects using only exact
%operations.  Exact number objects correspond to mathematical numbers
%in the obvious way.
Числовой объект является точным, если он является значением точного числового литерала или был
получен из точных числовых объектов с помощью только точных операций. Точные числовые объекты
соответствуют математическим числам явным образом.

%Conversely, a number object is inexact if it is the value of an
%inexact numerical literal, or was derived from inexact number objects,
%or was derived using inexact operations.  Thus inexactness is
%contagious.
В свою очередь, числовой объект является неточным, если он является значением неточного
числового литерала или был получен из неточных числовых объектов числа, или с помощью неточных
операций. Таким образом неточность передаётся непосредственно.

%Exact arithmetic is reliable in the following sense: If exact number
%objects are passed to any of the arithmetic procedures described in
%section~\ref{propagationsection}, and an exact number object is
%returned, then the result is mathematically correct.  This is
%generally not true of computations involving inexact number objects
%because approximate methods such as floating-point arithmetic may be
%used, but it is the duty of each implementation to make the result as
%close as practical to the mathematically ideal result.
Точная арифметика безопасна в следующем смысле: Если точные числовые объекты передаются любой из
арифметических процедур, описанных в секции~\ref{propagationsection}, и возвращается точный
числовой объект, результат математически корректен. Это в общем случае не верно для вычислений с
использованием неточных числовых объектов, так как могут использоваться методы аппроксимации,
типа арифметики с плавающей запятой, но это уже является обязанностью каждой реализации -
выдать результат как можно ближе практически к математически идеальному результату.

\section{Fixnums and flonums}

%A \defining{fixnum} is an exact integer object that lies
%within a certain implementation-dependent subrange of the
%exact integer objects. (Library section \extref{lib:fixnumssection}{Fixnums} describes a
%library for computing with fixnums.)
%Likewise, every implementation must
%designate a subset of its inexact real number objects as \defining{flonum}s, and
%to convert certain external representations into flonums.
%(Library section \extref{lib:flonumssection}{Flonums} describes a library for
%computing with flonums.)  Note that
%this does not imply that an implementation must use
%floating-point representations.
\defining{fixnum} является точным объектом целого, который лежит в пределах конкретного
зависимого от реализации поддиапазона точного объекта целого. (В секции
\extref{lib:fixnumssection}{Fixnums} описана библиотека для вычислений с fixnums.) Аналогично,
каждая реализация должно обозначать подмножество своих неточных вещественных числовых объектов
как \defining {flonum}s, и преобразовывать конкретные внешние представления в flonums. (В секции
\extref{lib:flonumssection}{Flonums} описана библиотека для вычислений с flonums.) Заметьте, что
не подразумевается, что реализация должна использовать представления с плавающей запятой.

%\section{Implementation requirements}
\section{Требования к реализациям}

\index{implementation restriction}\label{restrictions}

%Implementations of Scheme must support number objects for
%the entire tower of subtypes given in section~\ref{numericaltypes}.
%Moreover, implementations must support exact integer
%objects and exact rational number objects of practically unlimited
%size and precision, and to implement certain procedures (listed in
%\ref{propagationsection}) so they always return exact results when
%given exact arguments.  (``Practically unlimited'' means that the size
%and precision of these numbers should only be limited by the size of
%the available memory.)
Реализации Scheme должны поддерживать числовые объекты для всей башни подтипов, приведённых в
секции~\ref{numericaltypes}. Кроме того, реализации должны поддерживать точные объекты целого
точные рациональные числовые объекты практически неограниченного размера и точности,
и реализовывать конкретные процедуры (перечисленные в \ref{propagationsection}) с тем, чтобы
они всегда возвращали точные результаты при предоставлении точных аргументов. (``Практически
неограниченный'' означает, что размер и точность таких чисел должны
ограничиваться только размером доступной памяти.)

\newpage

%Implementations may support only a limited range of inexact number
%objects of any type, subject to the requirements of this section.  For
%example, an implementation may limit the range of the inexact real
%number objects (and therefore the range of inexact integer and
%rational number objects) to the dynamic range of the flonum format.
%Furthermore the gaps between the inexact integer objects and
%rationals are likely to be very large in such an implementation as the
%limits of this range are approached.
Реализации могут поддерживать только ограниченный диапазон неточных числовых объектов любого
типа, подчининяясь требованиям данной секции. Например, реализация может ограничить диапазон
неточных вещественных числовых объектов (и, следовательно, диапазон неточных целых и
рациональных числовых объектов) динамическим диапазоном формата flonum. Кроме того, зазоры
между неточными целыми объектами и rationals, вероятно, будут очень большими в такой
реализации при приближении к границам этого диапазона.

%An implementation may use floating point and other approximate
%representation strategies for \tupe{inexact} numbers.
%This report recommends, but does not require, that the IEEE
%floating-point standards be followed by implementations that use
%floating-point representations, and that implementations using
%other representations should match or exceed the precision achievable
%using these floating-point standards~\cite{IEEE}.
Реализация может использовать плавающую запятую и другие неточные способы представления
\tupe{неточных} чисел. В данной работе рекомендуется, но не требуется, чтобы реализации,
использующие представления с плавающей запятой, следовали стандартам плавающей запятой IEEE, а
реализации, использующие другие представления, должны иметь точность, соответствовующую или
превышающую точность, достигаемую этими стандартами с плавающей запятой~\cite{IEEE}.

%In particular, implementations that use floating-point representations
%must follow these rules: A floating-point result must be represented
%with at least as much precision as is used to express any of the
%inexact arguments to that operation.
%Potentially inexact operations such as {\cf sqrt}, when
%applied to exact arguments, should produce exact answers whenever possible
%(for example the square root of an exact 4 ought to be an exact 2).
%However, this is not required.
%If, on the other hand, an exact number object is operated upon so as to produce an
%inexact result (as by {\cf sqrt}), and if the result is represented in
%floating point, then the most precise floating-point format available
%must be used; but if the result is represented in some other way then
%the representation must have at least as much precision as the most
%precise floating-point format available.
В частности, реализации, использующие представления с плавающей запятой, должны соблюдать
следующие правила: результат с плавающей запятой должен быть представлен с точностью, по крайней
мере не менее используемой для выражения любого неточного аргумента в данной операции. При
применении потенциально неточных операций, типа {\cf\bfseries sqrt} к точным аргументам должны
выдаваться точные ответы всегда, когда возможно (например, квадратный корень точного 4 должен
быть точным 2). Однако это не является требованием. Если, с другой стороны, точным числовым
объектом управляют с целью получения неточного результата (как {\cf\bfseries sqrt}), и если
результат представляется с плавающей запятой, должен использоваться самый точный доступный
формат с плавающей запятой; но если результат представлен неким иным способом, тогда,
представление должно иметь точность, по крайней мере не менее самого точного доступного формата с
плавающей запятой.

%It is the programmer's responsibility to avoid using inexact number
%objects with magnitude or significand too large to be represented in
%the implementation.
Недопущение использования неточных числовых объектов со слишком большим для представления в
реализации значением или значащей частью является заботой программистов.\vspace{-5mm}

%\section{Infinities and NaNs}
\section{Бесконечность и NaN}\vspace{-2mm}

%Some Scheme implementations, specifically those that follow the IEEE
%floating-point standards, distinguish special number objects called
%\mainindex{infinity}\defining{positive infinity}, \defining{negative
%  infinity}, and \defining{NaN}.
Некоторое реализации Scheme, например, придерживающиеся стандартов IEEE с плавающей
точкой, различают специальные числовые объекты, называемые \mainindex
{infinity}\defining{положительная бесконечность}, \defining {отрицательная бесконечность} и
\defining{NaN}.

%Positive infinity is regarded as an inexact real (but not rational) number
%object that represents an indeterminate number greater than the
%numbers represented by all rational number objects.  Negative infinity
%is regarded as an inexact real (but not rational) number object that represents
%an indeterminate number less than the numbers represented by all
%rational numbers.
Положительная бесконечность рассматривается как неточный вещественный (но не рациональный)
числовой объект, представляющий неопределённое число, большее, чем числа, представляемые всеми
рациональными числовыми объектами. Отрицательная бесконечность рассматривается как неточный
вещественный (но не рациональный) числовой объект, представляющий неизвестное число, меньшее,
чем числа, представляемые всеми рациональными числами.

%A NaN is regarded as an inexact real (but not rational) number object so
%indeterminate that it might represent any real number, including
%positive or negative infinity, and might even be greater than positive
%infinity or less than negative infinity.
NaN рассматривается как неточный вещественный (но не рациональный) числовой объект, настолько
неопределённый, что он может представлять любое вещественное число, включая положительную или
отрицательную бесконечность, и даже может быть больше положительной бесконечности или
меньше отрицательной бесконечности.\vspace{-5mm}

%\section{Distinguished -0.0}
\section{Распознавание -0.0}\vspace{-2mm}

%\index{-0.0}
%Some Scheme implementations, specifically those that follow the IEEE
%floating-point standards, distinguish between number objects for $0.0$
%and $-0.0$, i.e., positive and negative inexact zero.  This report
%will sometimes specify the behavior of certain arithmetic operations
%on these number objects.  These specifications are marked with ``if
%$-0.0$ is distinguished'' or ``implementations that distinguish
%$-0.0$''.
\index{-0.0} Некоторые реализации Scheme, например, придерживающиеся стандартов IEEE с
плавающей точкой, различают числовые объекты $0.0$ и $-0.0$, то есть, положительный и
отрицательный неточный ноль. В данной работе в ряде случаев специфицируется функционирование
конкретных арифметических операций с такими числовыми объектами. Такие спецификации записываются
вместе с ``если $-0.0$ распознан'' или ``реализация, различающая $-0.0$''.\vspace{-4mm}

%%% Local Variables:
%%% mode: latex
%%% TeX-master: "r6rs"
%%% End:
 \par
% Lexical structure
\hyphenation{white-space}
%%\vfill\eject
%\chapter{Lexical syntax and datum syntax}
\chapter{Лексический и datum-синтаксис}
\label{readsyntaxchapter}

%The syntax of Scheme code is organized in three levels:
Синтаксис кода Scheme организован на трёх уровнях:\vspace{-3mm}
%
\begin{enumerate}
%\item the \textit{lexical syntax} that describes how a program text is split
%  into a sequence of lexemes,
\item \textit{Лексический синтаксис}, описывающий, как текст программы разбивается на
  последовательность лексем,
%\item the \textit{datum syntax}, formulated in terms of the lexical
%  syntax, that structures the lexeme sequence as a sequence of
%  \textit{syntactic data\mainindex{datum}\mainindex{syntactic
%      datum}}, where a syntactic datum is
%    a recursively structured entity,
\item \textit{Datum-синтаксис}, сформулированный в терминах лексического синтаксиса,
структурирующий последовательность лексем в виде последовательности \textit{синтаксических
  данных\mainindex{datum}\mainindex{syntactic datum}}, где
синтаксический datum является рекурсивно структурированным элементом,
%\item the \textit{program syntax} formulated in terms of the read
%  syntax, imposing further structure and assigning meaning to
%  syntactic data.
\item \textit{Программный синтаксис}, сформулированный в терминах синтаксиса считывания, задающий
  дальнейшую структуру и наполняющий смысловым содержанием синтаксические данные.
\end{enumerate}\vspace{-3mm}
%
%Syntactic data (also called \textit{external
%  representations\index{external representation}}) double
%as a notation for objects, and Scheme's \rsixlibrary{io ports} library
%(library section~\extref{lib:portsiosection}{Port I/O})
%provides the {\cf get-datum} and {\cf put-datum} procedures
%for reading and writing syntactic data, converting between their
%textual representation and the corresponding objects.
%Each syntactic datum represents a corresponding \defining{datum value}.
%A syntactic datum can be used in a program to obtain the corresponding
%datum value using {\cf quote} (see section~\ref{quote}).
Синтаксические данные (называемые также \textit{внешними представлениями\index{external
    representation}}) одновременно служат как формой записи объектов, так и библиотекой Scheme
{\bfseries\rsixlibrary{io ports}} (секция библиотек~\extref{lib:portsiosection}{Port I/O}),
предоставляющей процедуры {\cf\bfseries get-datum} и {\cf\bfseries put-datum} для чтения и
записи синтаксических данных, преобразовывающие их из текстового представления в
соответствующие объекты, и наоборот. Каждый синтаксический datum представляет соответствующее
\defining{datum-значение}. Синтаксический datum может использоваться в программе для
получения соответствующего datum-значения с помощью {\cf\bfseries quote}
(см. секцию~\ref{quote}).

%Scheme source code consists of syntactic data and (non-significant) comments.
%Syntactic data in Scheme source code are called
%\textit{forms}\mainindex{form}.
%(A form nested inside another form is
%called a \defining{subform}.)
%Consequently, Scheme's syntax has the property that any sequence of
%characters that is a form is also a syntactic datum representing
%some object.  This can lead to confusion, since it may not be obvious
%out of context whether a given sequence of characters is intended to
%be a representation of objects or the text of a program.
%It is also a source of power, since it
%facilitates writing programs such as interpreters or compilers that
%treat programs as objects (or vice versa).
Исходный текст Scheme состоит из синтаксических данных и (незначащих)
комментариев. Синтаксические данные в исходном тексте Scheme называются
\textit{формами}\mainindex{form}. (Форма, вложенная в другую форму, называется
\defining{подформой}.) Следовательно, синтаксис Scheme обладает свойством, что любая
последовательность символов, являющаяся формой, является также и синтаксическим datum,
представляющей некоторый объект. Это может привести к замешательству, так как из контекста может
быть не ясно, предназначена ли данная последовательность символов для представления объектов или
текста программы. Это - также источник мощи, так как это облегчает написание программ,
типа интерпретаторов или компиляторов, интерпретирующих программы в качестве объектов (или
наоборот).

%A datum value may have several different external representations.
%For example, both ``{\tt \#e28.000}'' and
%``{\tt\#x1c}'' are syntactic data representing the exact integer
%object 28, and the syntactic data ``{\tt(8 13)}'', ``{\tt( 08 13 )}'', ``{\tt(8 .\
%  (13 .\ ()))}''
%all represent a list containing the exact integer objects 8 and 13.
%Syntactic data that represent equal objects (in the sense of {\cf
%  equal?}; see section~\ref{equal?}) are always equivalent
%as forms of a program.
Datum-значение может иметь несколько различных внешних представлений. Например, и ``{\tt\bfseries
  \#e28.000}'', и ``{\tt\bfseries\#x1c}'' являются синтаксическими данными, представляющими
точный целый объект 28, а синтаксические данные ``{\tt\bfseries(8 13)}'', ``{\tt\bfseries( 08 13
  )}'', ``{\tt\bfseries(8 .\ (13 .\ ()))}'' представляют список, содержащий точные целые
объекты 8 и 13. Синтаксические данные, представляющие равные объекты (в смысле {\cf\bfseries
  equal?}; см. секцию~\ref{equal?}), всегда эквивалентны как формы программы.

%Because of the close correspondence between syntactic data and datum
%values, this report sometimes uses the term \defining{datum} for
%either a syntactic datum or a datum value when the exact meaning
%is apparent from the context.
Вследствие точного соответствия синтаксических данных и datum-значений в данной работе термин
\defining{datum} иногда используется и для синтаксического datum, и для datum-значения, когда
точный смысл очевиден из контекста.

%An implementation must not extend the lexical or datum syntax in
%any way, with one exception: it need not treat the syntax
%{\cf \sharpsign{}!\meta{identifier}}, for any \meta{identifier} (see
%section~\ref{identifiersection}) that is not {\cf r6rs}, as a syntax
%violation, and it may use specific {\cf \sharpsign{}!}-prefixed
%identifiers as flags indicating that subsequent input contains extensions
%to the standard lexical or datum syntax.
%The syntax {\cf \sharpsign{}!r6rs} may be used to signify that
%the input afterward is written with the lexical syntax and
%datum syntax described by
%this report.
%{\cf \sharpsign{}!r6rs} is otherwise treated as a comment; see section~\ref{whitespaceandcomments}.
Реализации запрещено расширять лексический или datum-синтаксис, за одним исключением: синтаксис
{\cf{\bfseries \sharpsign{}!}\meta {identifier}} при любом \meta{identifier} (см. секцию~\ref
{identifiersection}), отличном от {\cf\bfseries r6rs}, не должен считаться синтаксическим
нарушением, а специфические идентификаторы с префиксом {\cf\bfseries \sharpsign {}!} могут
использоваться в качестве флагов, указывающих на наличие в последующем вводе расширений
лексического стандарта или datum-синтаксиса. Синтаксис {\cf \sharpsign \bfseries{}!r6rs} может
применяться для указания, что последующий ввод записан с лексическим и datum-синтаксисом,
описанным в данной работе. В противном случае {\cf \sharpsign\bfseries {}!r6rs} считается
комментарием; см. секцию~\ref{whitespaceandcomments}.\vspace{-1mm}

%\section{Notation}
\section{Нотация}
\label{BNF}

%The formal syntax for Scheme is written in an extended BNF.
%Non-terminals are written using angle brackets.  Case is insignificant
%for non-terminal names.
Формальный синтаксис Scheme записан в расширенной BNF. Нетерминальные символы окружены
угловыми скобками. Для нетерминальных имён регистр не имеет значения.

%All spaces in the grammar are for legibility.
%\meta{Empty} stands for the empty string.
Все пробелы в грамматике применяются для удобочитаемости. \meta{Empty} обозначает пустую строку.

%The following extensions to BNF are used to make the description more
%concise:  \arbno{\meta{thing}} means zero or more occurrences of
%\meta{thing}, and \atleastone{\meta{thing}} means at least one
%\meta{thing}.
Для более лаконичного описания используются следующие расширения BNF: \arbno{\meta{thing}} означает
ноль или более вхождений \meta{thing}, а \atleastone{\meta{thing}} означает не менее
одного \meta{thing}.

%Some non-terminal names refer to the Unicode scalar values of the same
%name: \meta{character tabulation} (U+0009), \meta{linefeed} (U+000A),
%\meta{carriage return} (U+000D), \meta{line tabulation} (U+000B),
%\meta{form feed} (U+000C), \meta{carriage return} (U+000D),
%\meta{space} (U+0020), \meta{next line} (U+0085), \meta{line
%  separator} (U+2028), and \meta{paragraph separator} (U+2029).
Некоторые нетерминальные имена относятся к одноимённым скалярным значениям Unicode:
\meta{character tabulation} (U+0009), \meta{linefeed} (U+000A), \meta{carriage return} (U+000D),
\meta{line tabulation} (U+000B), \meta{form feed} (U+000C), \meta{carriage return} (U+000D),
\meta{space (U+0020)}, \meta{next line} (U+0085), \meta{line separator} (U+2028) и
\meta{paragraph separator} (U+2029).\vspace{-1mm}

%\section{Lexical syntax}
\section{Лексический синтаксис}
\label{lexicalsyntaxsection}

%The lexical syntax determines how a character sequence is split into a
%sequence of lexemes\index{lexeme}, omitting non-significant portions
%such as comments and whitespace.  The character sequence is assumed to
%be text according to the Unicode standard~\cite{Unicode}.  Some of
%the lexemes, such as identifiers, representations of number objects, strings etc., of the lexical
%syntax are syntactic data in the datum syntax, and thus represent objects.
%Besides the formal account of the syntax, this section also describes
%what datum values are represented by these syntactic data.
Лексический синтаксис определяет, как символьная последовательность разбивается на
последовательность лексем\index{lexeme}, пропуская незначащие части, типа комментариев и
пробелов. Полагается, что символьная последовательность является текстом согласно стандарту
Unicode~\cite{Unicode}. Некоторые лексемы, типа идентификаторов, представлений числовых объектов,
строки и т.д., лексического синтаксиса являются синтаксическими данными в datum-синтаксисе
и, таким образом, представляют объекты. Помимо формального описания синтаксиса, в этой
секции также описывается, какие datum-значения представляют эти синтаксические
данные.

%The lexical syntax, in the description of comments, contains
%a forward reference to \meta{datum}, which is described as part of the
%datum syntax.  Being comments, however, these \meta{datum}s do not play
%a significant role in the syntax.
Лексический синтаксис в описании комментариев содержит прямую ссылку на \meta{datum},
описанную как часть datum-синтаксиса. Будучи комментариями, однако,
такие \meta{datum} не играют существенную роль в синтаксисе.

%Case is significant except in representations of booleans, number objects, and
%in hexadecimal numbers specifying Unicode scalar values.  For example, {\cf \#x1A}
%and {\cf \#X1a} are equivalent.  The identifier {\cf Foo} is, however,
%distinct from the identifier {\cf FOO}.
Регистр является значащим, за исключением представлений booleans, числовых объектов и
шестнадцатеричных чисел, определяющих скалярные значения Unicode. Например, {\cf\bfseries \#x1A}
и {\cf\bfseries \#X1a} эквивалентны. А вот идентификатор {\cf\bfseries Foo} отличается от
идентификатора {\cf\bfseries FOO}.

%\subsection{Formal account}
\subsection{Формальное описание}
\label{lexicalgrammarsection}

%\meta{Interlexeme space} may occur on either side of any lexeme, but not
%within a lexeme.
\meta{Interlexeme space} может находиться с любой стороны лексемы, но не внутри лексемы\vspace{1mm}

%\hyper{Identifier}s, {\cf .}, \hyper{number}s, \hyper{character}s, and
%\hyper{boolean}s, must be terminated by a \meta{delimiter} or by the
%end of the input.
\hyper{Identifier}, {\cf .}, \hyper{number}, \hyper{character} и
\hyper{boolean} могут заканчиваться \meta{delimiter} или концом ввода.\vspace{1mm}

%The following two characters are reserved for future extensions to the
%language: {\tt \verb"{" \verb"}"}
Следующие два знака зарезервированы для будущих расширений языка:
{\tt \verb"{" \verb"}"}\vspace{1mm}

{%
\renewcommand{\baselinestretch}{1.1}
\selectfont
\begin{grammar}%
\meta{lexeme} \: \meta{identifier} \| \meta{boolean} \| \meta{number}\index{identifier}
\>  \| \meta{character} \| \meta{string}
\>  \| ( \| ) \| \openbracket{} \| \closedbracket{} \| \sharpsign( \| \sharpsign{}vu8( | \singlequote{} \| \backquote{} \| , \| ,@ \| {\bf.}
\>  \| \sharpsign\singlequote{} \| \sharpsign\backquote{} \| \sharpsign, \| \sharpsign,@
\meta{delimiter} \: ( \| ) \| \openbracket{} \| \closedbracket{} \| " \| ; \| \sharpsign{}
\>  \| \meta{whitespace}
\meta{whitespace} \: \meta{character tabulation}
\> \| \meta{linefeed} \| \meta{line tabulation} \| \meta{form feed}
\> \| \meta{carriage return} \| \meta{next line}
\> \| \meta{any character whose category is Zs, Zl, or Zp}
\meta{line ending} \: \meta{linefeed} \| \meta{carriage return}
\> \| \meta{carriage return} \meta{linefeed} \| \meta{next line}
\> \| \meta{carriage return} \meta{next line} \| \meta{line separator}
\meta{comment} \: ; \= $\langle$\rm all subsequent characters up to a
                    \>\ \rm \meta{line ending} or \meta{paragraph separator}$\rangle$\index{comment}
\qquad \= \| \meta{nested comment}
\> \| \#; \meta{interlexeme space} \meta{datum}
\> \| \#!r6rs
\meta{nested comment} \: \#| \= \meta{comment text}
\> \arbno{\meta{comment cont}} |\#
\meta{comment text} \: \= $\langle$\rm character sequence not containing
\>\ \rm {\tt \#|} or {\tt |\#}$\rangle$
\meta{comment cont} \: \meta{nested comment} \meta{comment text}
\meta{atmosphere} \: \meta{whitespace} \| \meta{comment}
\meta{interlexeme space} \: \arbno{\meta{atmosphere}}%
\end{grammar}

\label{extendedalphas}
\label{identifiersyntax}

% This is a kludge, but \multicolumn doesn't work in tabbing environments.
\setbox0\hbox{\cf\meta{variable} \goesto{} $\langle$}

\begin{grammar}%
\meta{identifier} \: \meta{initial} \arbno{\meta{subsequent}}
 \>  \| \meta{peculiar identifier}
\meta{initial} \: \meta{constituent} \| \meta{special initial}
 \> \| \meta{inline hex escape}
\meta{letter} \:  a \| b \| c \| ... \| z
\> \| A \| B \| C \| ... \| Z
\meta{constituent} \: \meta{letter}
 \> \| $\langle${\rm any character whose Unicode scalar value is greater than}
 \> \quad {\rm 127, and whose category is Lu, Ll, Lt, Lm, Lo, Mn,}
 \> \quad {\rm Nl, No, Pd, Pc, Po, Sc, Sm, Sk, So, or Co}$\rangle$
%\end{grammar}
%
%\newpage
%
%\begin{grammar}
\meta{special initial} \: ! \| \$ \| \% \| \verb"&" \| * \| / \| : \| < \| =
 \>  \| > \| ? \| \verb"^" \| \verb"_" \| \verb"~"
\meta{subsequent} \: \meta{initial} \| \meta{digit}
 \>  \| \meta{any character whose category is Nd, Mc, or Me}
 \>  \| \meta{special subsequent}
\meta{digit} \: 0 \| 1 \| 2 \| 3 \| 4 \| 5 \| 6 \| 7 \| 8 \| 9
\meta{hex digit} \: \meta{digit}
 \> \| a \| A \| b \| B \| c \| C \| d \| D \| e \| E \| f \| F
\meta{special subsequent} \: + \| - \| .\ \| @
\meta{inline hex escape} \: \backwhack{}x\meta{hex scalar value};
\meta{hex scalar value} \: \atleastone{\meta{hex digit}}
\meta{peculiar identifier} \: + \| - \| ... \| -> \arbno{\meta{subsequent}}
\meta{boolean} \: \schtrue{} \| \#T \| \schfalse{} \| \#F
\meta{character} \: \#\backwhack{}\meta{any character}
 \>  \| \#\backwhack{}\meta{character name}
 \>  \| \#\backwhack{}x\meta{hex scalar value}
\meta{character name} \: nul \| alarm \| backspace \| tab
\> \| linefeed \| newline \| vtab \| page \| return
\> \| esc \| space \| delete
\meta{string} \: " \arbno{\meta{string element}} "
\meta{string element} \: \meta{any character other than \doublequote{} or \backwhack}
 \> \| \backwhack{}a \| \backwhack{}b \| \backwhack{}t \| \backwhack{}n \| \backwhack{}v \| \backwhack{}f \| \backwhack{}r
 \>  \| \backwhack\doublequote{} \| \backwhack\backwhack
 \>  \| \backwhack\meta{intraline whitespace}\meta{line ending}
 \>  \hspace*{4em}\meta{intraline whitespace}
 \>  \| \meta{inline hex escape}
\meta{intraline whitespace} \: \meta{character tabulation}
\> \| \meta{any character whose category is Zs}%
\end{grammar}

}

A \meta{hex scalar value} represents a Unicode scalar value
between 0 and \sharpsign{}x10FFFF, excluding the range
$\left[\sharpsign{}x\textrm{D800}, \sharpsign{}x\textrm{DFFF}\right]$.

\label{numbersyntax}%
The rules for \meta{num $R$}, \meta{complex $R$}, \meta{real
$R$}, \meta{ureal $R$}, \meta{uinteger $R$}, and \meta{prefix $R$} below
should be replicated for \hbox{$R = 2, 8, 10,$}
and $16$.  There are no rules for \meta{decimal $2$}, \meta{decimal
$8$}, and \meta{decimal $16$}, which means that number representations containing
decimal points or exponents must be in decimal radix.

\begin{grammar}%
\meta{number} \: \meta{num $2$} \| \meta{num $8$}
   \>  \| \meta{num $10$} \| \meta{num $16$}
\meta{num $R$} \: \meta{prefix $R$} \meta{complex $R$}
\meta{complex $R$} \: %
         \meta{real $R$} %
      \| \meta{real $R$} @ \meta{real $R$}
   \> \| \meta{real $R$} + \meta{ureal $R$} i %
      \| \meta{real $R$} - \meta{ureal $R$} i
   \> \| \meta{real $R$} + \meta{naninf} i %
      \| \meta{real $R$} - \meta{naninf} i
   \> \| \meta{real $R$} + i %
      \| \meta{real $R$} - i
   \> \| + \meta{ureal $R$} i %
      \| - \meta{ureal $R$} i
   \> \| + \meta{naninf} i %
      \| - \meta{naninf} i
   \> \| + i %
      \| - i
\meta{real $R$} \: \meta{sign} \meta{ureal $R$}
  \> \| + \meta{naninf} \| - \meta{naninf}
\meta{naninf} \: nan.0 \| inf.0
\meta{ureal $R$} \: %
         \meta{uinteger $R$}
   \> \| \meta{uinteger $R$} / \meta{uinteger $R$}
   \> \| \meta{decimal $R$} \meta{mantissa width}
\meta{decimal $10$} \: %
         \meta{uinteger $10$} \meta{suffix}
   \> \| . \atleastone{\meta{digit $10$}} \meta{suffix}
   \> \| \atleastone{\meta{digit $10$}} . \arbno{\meta{digit $10$}} \meta{suffix}
   \> \| \atleastone{\meta{digit $10$}} . \meta{suffix}
\meta{uinteger $R$} \: \atleastone{\meta{digit $R$}}
\meta{prefix $R$} \: %
         \meta{radix $R$} \meta{exactness}
   \> \| \meta{exactness} \meta{radix $R$}
\end{grammar}

\begin{grammar}%
\meta{suffix} \: \meta{empty}
   \> \| \meta{exponent marker} \meta{sign} \atleastone{\meta{digit $10$}}
\meta{exponent marker} \: e \| E \| s \| S \| f \| F
   \> \| d \| D \| l \| L
\meta{mantissa width} \: \meta{empty}
   \> \| | \atleastone{\meta{digit 10}}
\meta{sign} \: \meta{empty}  \| + \|  -
\meta{exactness} \: \meta{empty}
   \> \| \#i\sharpindex{i} \| \#I \| \#e\sharpindex{e} \| \#E
\meta{radix 2} \: \#b\sharpindex{b} \| \#B
\meta{radix 8} \: \#o\sharpindex{o} \| \#O
\meta{radix 10} \: \meta{empty} \| \#d \| \#D
\meta{radix 16} \: \#x\sharpindex{x} \| \#X
\meta{digit 2} \: 0 \| 1
\meta{digit 8} \: 0 \| 1 \| 2 \| 3 \| 4 \| 5 \| 6 \| 7
\meta{digit 10} \: \meta{digit}
\meta{digit 16} \: \meta{hex digit}
\end{grammar}

%\subsection{Line endings}
\subsection{Окончание строки}
\label{lineendings}

%Line endings are significant in Scheme in single-line comments (see
%section~\ref{whitespaceandcomments}) and within string literals.  In
%Scheme source code, any of the line endings in \meta{line ending}
%marks the end of a line.  Moreover, the two-character line endings
%\meta{carriage return} \meta{linefeed} and \meta{carriage return}
%\meta{next line} each count as a single line ending.
В Scheme окончание строки является значащим в однострочных комментариях
(см. секцию~\ref{whitespaceandcomments}) и внутри строковых литералов. В исходном тексте Scheme
любое окончание строки в \meta{line ending} означает конец строки. Кроме того, двухсимвольное
окончание строки \meta{carriage return} \meta{linefeed} и \meta{carriage return} \meta{next
  line} считается одним окончанием строки.

%In a string literal, a \hyper{line ending} not preceded by a {\cf\backwhack}
%stands for a linefeed character, which is the standard line-ending
%character of Scheme.
В строковом литерале \hyper{line ending} без {\cf\backwhack} перед ним обозначает
символ linefeed, являющийся стандартным символом конца строки Scheme.

%\subsection{Whitespace and comments}
\subsection{Пробельные символы и комментарии}
\label{whitespaceandcomments}

%\defining{Whitespace} characters are spaces, linefeeds,
%carriage returns, character tabulations, form feeds, line tabulations,
%and any other character whose category is Zs, Zl, or Zp.
%Whitespace is used for improved readability and
%as necessary to separate lexemes from each other.  Whitespace may
%occur between any two lexemes,
%but not within a lexeme.  Whitespace may also occur inside a string,
%where it is significant.
\defining{Пробельными} символами являются пробелы, обратные переводы строк, переводы каретки, символы
табуляции, переводы страницы, линейные табуляции и любой другой символ, категория которого - Zs,
Zl, или Zp. Пробельные символы используются для улучшения удобочитаемости и при необходимости
отделения лексем друг от друга. Пробельные символы могут находиться между любыми двумя
лексемами, но не внутри лексемы. Пробельный символ может также находиться в строке, где он
является значащим.

%The lexical syntax includes several comment forms. In all cases,
%comments are invisible to Scheme, except that they act as delimiters,
%so, for example, a comment cannot appear in the middle of an
%identifier or representation of a number object.
Лексический синтаксис включает несколько форм комментариев. Во всех случаях, комментарии
невидимы, в Scheme, за исключением того, что они действуют как разделители, таким
образом, например, комментарий не может находиться в середине идентификатора или представления
числового объекта.

%\newpage

%A semicolon ({\tt;}) indicates the start of a line
%comment.\mainindex{comment}\mainschindex{;} The comment continues to
%the end of the line on which the semicolon appears.
Точка с запятой ({\tt;}), указывает начало строки комментария.\mainindex{comment}
  \mainschindex{;} Комментарий продолжается до конца строки, на которой находится точка с
запятой.\vspace{1mm}

%Another way to indicate a comment is to prefix a \hyper{datum}
%(cf.\ section~\ref{datumsyntax}) with {\tt \#;}\sharpindex{;}, possibly with
%\meta{interlexeme space} before the \hyper{datum}.  The comment consists of
%the comment prefix {\tt \#;} and the \hyper{datum} together.  This
%notation is useful for ``commenting out'' sections of code.
Другим способом указания комментария является добавления к \hyper{datum} (см.\ секцию~\ref
{datumsyntax}) префикса {\tt\bfseries \#;}\sharpindex{;}, возможно с \meta{interlexeme space} перед
\hyper{datum}. Комментарий состоит из приставки комментария {\tt\bfseries \#;} и \hyper{datum}.
Такая нотация полезна для ``комментирования'' секций кода.\vspace{1mm}

%Block comments may be indicated with properly nested {\tt
%  \#|}\index{#"|@\texttt{\sharpsign\verticalbar}}\index{"|#@\texttt{\verticalbar\sharpsign}}
%and {\tt |\#} pairs.
Блоковые комментарии могут быть обозначены соответствующим образом вложенными парами {\tt\bfseries \#|}
\index{#"|'@\texttt{\sharpsign\verticalbar}}\index{"|#@\texttt{\verticalbar\sharpsign}} и {\tt\bfseries |\#}.\vspace{1mm}

\begin{scheme}
\bfseries\#|
\bfseries   The FACT procedure computes the factorial
\bfseries   of a non-negative integer.
\bfseries|\#
\bfseries(define fact
\bfseries  (lambda (n)
\bfseries    ;; base case
\bfseries    (if (= n 0)
\bfseries        \#;(= n 1)
\bfseries        1       ; identity of *
\bfseries        (* n (fact (- n 1))))))%
\end{scheme}

%The lexeme {\cf \sharpsign{}!r6rs}, which signifies that the program text
%that follows is written with the lexical and datum syntax described in this
%report, is also otherwise treated as a comment.
Лексема {\cf\bfseries \sharpsign{}!r6rs}, которая означает, что текст следующей далее программы
написан с лексическим и data-синтаксисом, описанным в данной работе, также иным образом
интерпретируется как комментарий.

%\subsection{Identifiers}
\subsection{Идентификаторы}\vspace{2mm}
\label{identifiersection}

%Most identifiers\mainindex{identifier} allowed by other programming
%languages are also acceptable to Scheme.  In general,
%a sequence of letters, digits, and ``extended alphabetic
%characters'' is
%an identifier when it begins with a character that cannot begin a
%representation of a number object.
%In addition, \ide{+}, \ide{-}, and \ide{...} are identifiers, as is
%a sequence of letters, digits, and extended alphabetic
%characters that begins with the two-character sequence \ide{->}.
%Here are some examples of identifiers:
Большинство идентификаторов\mainindex{identifier}, допустимых в других языках программирования,
также допустимы и в Scheme. В общем случае, последовательность букв, цифр, и ``расширенных алфавитных
символов'' является идентификатором, если она начинается с символа, с которого не может начинаться
представление числового объекта. Кроме того, идентификаторами являются \ide{+}, \ide{-} и \ide{...},
равно как и последовательность букв, цифр и расширенных алфавитных символов, начинающаяся
с двухсимвольной последовательности \ide{->}. Некоторые примеры идентификаторов:\vspace{2mm}

\begin{scheme}
\bfseries lambda         q                soup
\bfseries list->vector   {+}                V17a
\bfseries <=             a34kTMNs         ->-
\bfseries the-word-recursion-has-many-meanings%
\end{scheme}\vspace{1mm}

%Extended alphabetic characters may be used within identifiers as if
%they were letters.  The following are extended alphabetic characters:
Расширенные алфавитные символы могут использоваться внутри идентификаторов, как если бы они
были буквами. Расширенными алфавитными символами являются:\vspace{2mm}

\begin{scheme}
\bfseries !\ \$ \% \verb"&" * + - . / :\ < = > ? @ \verb"^" \verb"_" \verb"~" %
\end{scheme}\vspace{1mm}

%Moreover, all characters whose Unicode scalar values are greater than 127 and
%whose Unicode category is Lu, Ll, Lt, Lm, Lo, Mn, Mc, Me, Nd, Nl, No, Pd,
%Pc, Po, Sc, Sm, Sk, So, or Co can be used within identifiers.
%In addition, any character can be used within an identifier
%when specified via an \meta{inline hex escape}.  For example, the
%identifier \verb|H\x65;llo| is the same as the identifier
%\verb|Hello|, and the identifier \verb|\x3BB;| is the same as the
%identifier $\lambda$.
Кроме того, в идентификаторах могут использоваться все символы, чьи скалярные значения Unicode
больше, чем 127, и чьей категорией Unicode является Lu, Ll, Lt, Lm, Lo, Mn, Mc, Me, Nd, Nl, No,
Pd, Pc, Po, Sc, Sm, Sk, So, или Co. Кроме того, любой символ может использоваться в
идентификаторе, если он специфицирован через \meta{inline hex escape}. Например,
идентификатор \verb|H\x65;llo| - точо такой же, как и идентификатор \verb|Hello|, а идентификатор
\verb|\x3BB;| - точно такой же, как и идентификатор $\lambda$.

%Any identifier may be used as a variable\index{variable} or as a
%syntactic keyword\index{syntactic keyword} (see
%sections~\ref{variablesection} and~\ref{macrosection}) in a Scheme
%program.
%Any identifier may also be used as a syntactic datum, in which case it
%represents a \textit{symbol}\index{symbol} (see section~\ref{symbolsection}).
В программе Scheme любой идентификатор может использоваться как переменная\index{variable} или
как синтаксическое ключевое слово\index{syntactic keyword} (см. секции~\ref{variablesection}
и~\ref{macrosection}) . Любой идентификатор может также использоваться как синтаксический datum,
в этом случае он представляет \textit{символ}\index{symbol} (см. секцию~\ref{symbolsection}).

\subsection{Булевы значения}

%The standard boolean objects for true and false have external representations
%\schtrue{} and \schfalse.\sharpindex{t}\sharpindex{f}
Стандартные булевые объекты для true и false имеют внешние представления {\bfseries\schtrue{}} и
{\bfseries\schfalse}.\sharpindex{t}\sharpindex{f}

\subsection{Символы}

%Characters are represented using the notation
%\sharpsign\backwhack\hyper{character}\index{#\@\texttt{\sharpsign\backwhack}} or
%\sharpsign\backwhack\hyper{character name} or
%\sharpsign\backwhack{}x\meta{hex scalar value}.
Символы представляются с помощью нотации
{\bfseries\sharpsign\backwhack}\hyper{character}\index{#\@\texttt{\sharpsign\backwhack}} или
{\bfseries\sharpsign\backwhack}\hyper{character name} или
{\bfseries\sharpsign\backwhack{}x}\meta{hex scalar value}.

%For example:
Например:

\texonly
\newcommand{\extab}{\>}
{%
\renewcommand{\baselinestretch}{1.05}
\selectfont
\begin{tabbing}
{\cf\#\backwhack{}x0000000000}\=\kill
\endtexonly
\htmlonly
\newcommand{\extab}{&}
\begin{tabular}{ll}
\endhtmlonly
{\cf\bfseries\#\backwhack{}a}          \extab \textrm{lower case letter a}\\
{\cf\bfseries\#\backwhack{}A}          \extab \textrm{upper case letter A}\\
{\cf\bfseries\#\backwhack{}(}          \extab \textrm{left parenthesis}\\
{\cf\bfseries\#\backwhack{}}           \extab \textrm{space character}\\
{\cf\bfseries\#\backwhack{}nul}        \extab \textrm{U+0000}\\
{\cf\bfseries\#\backwhack{}alarm}      \extab \textrm{U+0007}\\
{\cf\bfseries\#\backwhack{}backspace}  \extab \textrm{U+0008}\\
{\cf\bfseries\#\backwhack{}tab}        \extab \textrm{U+0009}\\
{\cf\bfseries\#\backwhack{}linefeed}   \extab \textrm{U+000A}\\
{\cf\bfseries\#\backwhack{}newline}   \extab \textrm{U+000A}\\
{\cf\bfseries\#\backwhack{}vtab}       \extab \textrm{U+000B}\\
{\cf\bfseries\#\backwhack{}page}       \extab \textrm{U+000C}\\
{\cf\bfseries\#\backwhack{}return}     \extab \textrm{U+000D}\\
{\cf\bfseries\#\backwhack{}esc}        \extab \textrm{U+001B}\\
{\cf\bfseries\#\backwhack{}space}      \extab \textrm{U+0020}\\
 \extab preferred way to write a space\\
{\cf\bfseries\#\backwhack{}delete}     \extab \textrm{U+007F}\\[1ex]
{\cf\bfseries\#\backwhack{}xFF}        \extab \textrm{U+00FF}\\
{\cf\bfseries\#\backwhack{}x03BB}      \extab \textrm{U+03BB}\\
{\cf\bfseries\#\backwhack{}x00006587}  \extab \textrm{U+6587}\\
{\cf\bfseries\#\backwhack{}\(\lambda\)} \extab \textrm{U+03BB}\\[1ex]
{\cf\bfseries\#\backwhack{}x0001z}     \extab \exception{\bfseries\&lexical}\\
{\cf\bfseries\#\backwhack{}\(\lambda\)x}         \extab \exception{\bfseries\&lexical}\\
{\cf\bfseries\#\backwhack{}alarmx}     \extab \exception{\bfseries\&lexical}\\
{\cf\bfseries\#\backwhack{}alarm x}    \extab \textrm{U+0007}\\
 \extab followed by {\cf{}x}\\
{\cf\bfseries\#\backwhack{}Alarm}      \extab \exception{\bfseries\&lexical}\\
{\cf\bfseries\#\backwhack{}alert}      \extab \exception{\bfseries\&lexical}\\
{\cf\bfseries\#\backwhack{}xA}         \extab \textrm{U+000A}\\
{\cf\bfseries\#\backwhack{}xFF}        \extab \textrm{U+00FF}\\
{\cf\bfseries\#\backwhack{}xff}        \extab \textrm{U+00FF}\\
{\cf\bfseries\#\backwhack{}x ff}       \extab \textrm{U+0078}\\
 \extab followed by another datum, {\bfseries\cf{}ff}\\%%%%%%%%%%%%%%%%%%%[34mm]%%%%%%%%%%%%%%%%%%%%%%%%%%%%%%%%%%%%
{\cf\bfseries\#\backwhack{}x(ff)}      \extab \textrm{U+0078}\\
 \extab followed by another datum,\\
 \extab a parenthesized {\bfseries\cf{}ff}\\
{\cf\bfseries\#\backwhack{}(x)}        \extab \exception{\bfseries\&lexical}\\
{\cf\bfseries\#\backwhack{}(x}         \extab \exception{\bfseries\&lexical}\\
{\cf\bfseries\#\backwhack{}((x)}       \extab \textrm{U+0028}\\
 \extab followed by another datum,\\
 \extab parenthesized {\bfseries\cf{}x}\\
{\cf\bfseries\#\backwhack{}x00110000}  \extab \exception{\bfseries\&lexical}\\
 \extab out of range\\
{\cf\bfseries\#\backwhack{}x000000001} \extab \textrm{U+0001}  \\
{\cf\bfseries\#\backwhack{}xD800}      \extab \exception{\bfseries\&lexical}\\
 \extab in excluded range
\htmlonly
\end{tabular}
\endhtmlonly
\texonly
\end{tabbing}

}
\endtexonly



%(The notation \exception{\&lexical} means that the line in question is
%a lexical syntax violation.)
(Нотация \exception{\bfseries\&lexical} означает, что рассматриваемая строка является
лексическим нарушением синтаксиса.)

%Case is significant in \sharpsign\backwhack\hyper{character}, and in
%\sharpsign\backwhack{\rm$\langle$character name$\rangle$}, % \hyper doesn't allow a linebreak
%but not in {\cf\sharpsign\backwhack{}x}\meta{hex scalar value}.
%A \meta{character} must be followed by a \meta{delimiter} or by the end of the input.
%This rule resolves various ambiguous cases involving named characters,
%requiring, for
%example, the sequence of characters ``{\tt\sharpsign\backwhack space}''
%to be interpreted as the space character rather than as
%the character ``{\tt\sharpsign\backwhack s}'' followed
%by the identifier ``{\tt pace}''.
Регистр является значащим в {\bfseries\sharpsign\backwhack}\hyper{character} и в
{\bfseries\sharpsign\backwhack} {\rm$\langle$character name$\rangle$}, но не в
{\bfseries\cf\sharpsign\backwhack{}x}\meta{hex scalar value}. За \meta{character} должен находиться
\meta{delimeter} или конец ввода. Это правило разрешает различные неоднозначные случаи с участием
названных символов, требуя, например, чтобы последовательность символов
``{\bfseries\tt\sharpsign\backwhack space}'' интерпретировалась как символ пробела, а не как символ
``{\bfseries\tt\sharpsign\backwhack s}''и следующим за ним идентификатором ``{\bfseries\tt pace}''.

\begin{note}
  %The {\cf\sharpsign\backwhack{}newline} notation is retained for
  %backward compatibility.  Its use is deprecated;
  %{\cf\sharpsign\backwhack{}linefeed} should be used instead.
  Нотация {\bfseries\cf\sharpsign\backwhack{}newline} сохранена с целью обратной совместимости. Её
  использование устарело; вместо неё должна использоваться
  {\bfseries\cf\sharpsign\backwhack{}linefeed}.
\end{note}

%\subsection{Strings}
\subsection{Строки}

%\vest String are represented by sequences of characters enclosed within doublequotes
%({\cf "}).  Within a string literal, various escape
%sequences\mainindex{escape sequence} represent characters other than
%themselves.  Escape sequences always start with a backslash (\backwhack{}):
\vest Строка представляется последовательностью символов, окружённых двойными кавычками
({\bfseries\cf "}). Внутри стрового литерала различные управляющие последовательности\mainindex
{escape sequence} представляют символы, кроме самих себя. Управляющие последовательности
всегда начинаются с обратного слеша ({\bfseries\backwhack{}}):


\begin{itemize}
\item{\bfseries\cf\backwhack{}a} : alarm, U+0007
\item{\bfseries\cf\backwhack{}b} : backspace, U+0008
\item{\bfseries\cf\backwhack{}t} : character tabulation, U+0009
\item{\bfseries\cf\backwhack{}n} : linefeed, U+000A
\item{\bfseries\cf\backwhack{}v} : line tabulation, U+000B
\item{\bfseries\cf\backwhack{}f} : formfeed, U+000C
\item{\bfseries\cf\backwhack{}r} : return, U+000D
\item{\bfseries\cf\backwhack{}}\verb|"| : doublequote, U+0022
\item{\bfseries\cf\backwhack{}\backwhack{}} : backslash, U+005C
\item{\bfseries\cf\backwhack{}}\hyper{intraline whitespace}\hyper{line ending}\\\hspace*{2em}\hyper{intraline whitespace} : nothing
\item{{\bfseries\cf\backwhack{}x}\meta{hex scalar value};} : specified character (note the
  terminating semi-colon).
\end{itemize}

%These escape sequences are case-sensitive, except that the alphabetic
%digits of a \meta{hex scalar value} can be uppercase or lowercase.
Эти управляющие последовательности регистрочувствительны, за исключением того, что алфавитные
цифры \meta{hex scalar value} могут быть заглавными или строчными.

%Any other character in a string after a backslash is a syntax violation. Except
%for a line ending, any
%character outside of an escape sequence and not a doublequote stands
%for itself in the string literal. For example the single-character
%string literal {\tt "$\lambda$"} (doublequote, a lower case lambda, doublequote)
%represents the same string as {\tt "\backwhack{}x03bb;"}.
%A line ending that does not follow a backslash stands for a linefeed character.
Любой другой символ в строке после обратного слеша является нарушением синтаксиса. За
исключением окончания строки, любой символ в строковом литерале вне управляющей
последовательностии, не являющийся двойными кавычками, обозначает самого себя. Например,
односимвольный строковый литерал {\tt "$\lambda$"} (двойные кавычки, строчная lambda, двойные
кавычки) представляет ту же самую строку, как и {\tt\bfseries "\backwhack{}x03bb;"}. Окончание строки,
находящееся не за обратным слешем, обозначает символ linefeed.

%Examples:
Примеры:

\texonly
\begin{tabbing}
{\cf "\backwhack{}x0000000000;"} \=\kill
\endtexonly
\htmlonly
\begin{tabular}{ll}
\endhtmlonly
{\bfseries\cf "abc"} \extab  \textrm{U+0061, U+0062, U+0063}\\
{\bfseries\cf "\backwhack{}x41;bc"} \extab  {\bfseries\cf "Abc"} ; \textrm{U+0041, U+0062, U+0063}\\
{\bfseries\cf "\backwhack{}x41; bc"} \extab {\bfseries\cf "A bc"}\\
 \extab U+0041, U+0020, U+0062, U+0063\\
{\bfseries\cf "\backwhack{}x41bc;"} \extab  \textrm{U+41BC}\\
{\bfseries\cf "\backwhack{}x41"} \extab \exception{\bfseries\&lexical}\\
{\bfseries\cf "\backwhack{}x;"} \extab \exception{\bfseries\&lexical}\\
{\bfseries\cf "\backwhack{}x41bx;"} \extab \exception{\bfseries\&lexical}\\
{\bfseries\cf "\backwhack{}x00000041;"} \extab  {\bfseries\cf "A"} ; \textrm{U+0041}\\
{\bfseries\cf "\backwhack{}x0010FFFF;"} \extab \textrm{U+10FFFF}\\
{\bfseries\cf "\backwhack{}x00110000;"} \extab  \exception{\bfseries\&lexical}\\
 \extab out of range\\
{\bfseries\cf "\backwhack{}x000000001;"} \extab \textrm{U+0001}\\
{\bfseries\cf "\backwhack{}xD800;"} \extab \exception{\bfseries\&lexical}\\
 \extab in excluded range\\
{\bfseries\cf "A}\\
{\bfseries\cf bc"} \extab \textrm{U+0041, U+000A, U+0062, U+0063}\\
 \extab if no space occurs after the {\bfseries\cf{}A}
\htmlonly
\end{tabular}
\endhtmlonly
\texonly
\end{tabbing}
\endtexonly

%\subsection{Numbers}
\subsection{Числа}
\label{numbernotations}

%The syntax of external representations for number objects is described
%formally by the \meta{number} rule in the formal grammar.
%Case is not significant in external representations of number objects.
Синтаксис внешних представлений числовых объектов формально описан правилом \meta{number}
формальной грамматики. Во внешних представлениях числовых объектов регистр является незначащим.

%A representation of a number object may be written in binary, octal, decimal, or
%hexadecimal by the use of a radix prefix.  The radix prefixes are {\cf
%\#b}\sharpindex{b} (binary), {\cf \#o}\sharpindex{o} (octal), {\cf
%\#d}\sharpindex{d} (decimal), and {\cf \#x}\sharpindex{x} (hexadecimal).  With
%no radix prefix, a representation of a number object is assumed to be expressed in decimal.
Представление числового объекта может быть записано в двоичном, восьмеричном, десятичном или
шестнадцатеричном виде при помощи приставки основания. Приставками основания являются
{\bfseries\cf \#b}\sharpindex{b} (двоичная), {\bfseries\cf \#o}\sharpindex{o} (восьмеричная),
{\bfseries\cf \#d}\sharpindex{d} (десятичная) и {\bfseries\cf \#x}\sharpindex{x}
(шестнадцатеричная). Без приставки основания представление числового объекта предполагается
выраженным в десятичном виде.

%A representation of a number object may be specified to be either exact or
%inexact by a prefix.  The prefixes are {\cf \#e}\sharpindex{e}
%for exact, and {\cf \#i}\sharpindex{i} for inexact.  An exactness
%prefix may appear before or after any radix prefix that is used.  If
%the representation of a number object has no exactness prefix, the
%constant is
%inexact if it contains a decimal point, an
%exponent, or
%a nonempty mantissa width;
%otherwise it is exact.
Представление числового объекта может указываться точным или неточным с помощью
приставки. Приставками являются {\bfseries\cf \#e}\sharpindex{e} для точного и {\bfseries\cf
  \#i}\sharpindex{i} для неточного. Приставка точности может находиться до или после любой
используемой приставки основания. Если представление числового объекта не содержит приставки
точности, константа является неточной, если она содержит десятичную точку, экспоненту или
непустую ширину мантиссы; в противном случае она является точной.

%In systems with inexact number objects
%of varying precisions, it may be useful to specify
%the precision of a constant.  For this purpose, representations of
%number objects
%may be written with an exponent marker that indicates the
%desired precision of the inexact
%representation.  The letters {\cf s}, {\cf f},
%{\cf d}, and {\cf l} specify the use of \var{short}, \var{single},
%\var{double}, and \var{long} precision, respectively.  (When fewer
%than four internal
%inexact
%representations exist, the four size
%specifications are mapped onto those available.  For example, an
%implementation with two internal representations may map short and
%single together and long and double together.)  In addition, the
%exponent marker {\cf e} specifies the default precision for the
%implementation.  The default precision has at least as much precision
%as \var{double}, but
%implementations may wish to allow this default to be set by the user.
В системах с неточными числовыми объектами переменной точности может быть полезно определение
точности константы. С этой целью представления числовых объектов могут записываться с
экспоненциальным маркером, указывающим желаемую точность неточного представления. Буквы
{\bfseries\cf s}, {\bfseries\cf f}, {\bfseries\cf d} и {\bfseries\cf l} указывают на использование
точности \var{short}, \var{single}, \var {double} и \var{long} соответственно. (Если существует
менее четырёх внутренних неточных представлений, четыре спецификации размера преобразовываются в
доступные. Например, реализация с двумя внутренними представлениями может преобразовывать вместе
short и single, а также long и double) Кроме того, маркер точности {\bfseries\cf e} указывает
точность по умолчанию для реализации. Точность по умолчанию имеет значение по крайней мере не
менее точности \var{double}, но реализации могут позволять пользователю устанавливать данное
умолчание.

\begin{scheme}
\bfseries 3.1415926535898F0
       {\rm{}Round to single, perhaps} {\bfseries 3.141593}
\bfseries 0.6L0
       {\rm{}Extend to long, perhaps} {\bfseries .600000000000000}%
\end{scheme}

%A representation of a number object with nonempty mantissa width,
%{\cf \var{x}|\var{p}}, represents the best binary
%floating-point approximation of \var{x} using a \var{p}-bit significand.
%For example, {\cf 1.1|53} is a
%representation of the best approximation of 1.1 in IEEE double
%precision.
%If \var{x} is an external representation of an inexact real number object
%that contains no vertical bar, then its numerical value should be computed
%as though it had a mantissa width of 53 or more.
Представление числового объекта с непустой шириной мантиссы {\cf \var{x}|\var{p}} представляет
наилучшее двоичное приближение с плавающей точкой \var{x} с помощью \var{p}-битной значащей
части. Например, {\cf\bfseries 1.1|53} является представлением наилучшего приближения 1.1 в IEEE
двойной точности. Если \var{x} является внешним представлением неточного действительного
числового объекта, не содержащим вертикальной черты, его численное значение должно
вычисляться так, как если бы оно имело ширину мантиссы 53 или более.

%Implementations that use binary floating-point representations
%of real number objects should represent {\cf \var{x}|\var{p}}
%using a \var{p}-bit significand if practical, or by a greater
%precision if a \var{p}-bit significand is not practical, or
%by the largest available precision if \var{p} or more bits
%of significand are not practical within the implementation.
Реализации, использующие двоичные представления с плавающей точкой действительных числовых
объектов, должны представлять {\cf \var{x}|\var{p}} с помощью \var{p}-битной значащей части,
если это удобно, или с большей точностью, если \var{p}-битная значащая часть не
удобна, или с наибольшей доступной точностью, если \var{p} или более бит
значащей части не удобны в реализации.

\begin{note}
%The precision of a significand should not be confused with the
%number of bits used to represent the significand.  In the IEEE
%floating-point standards, for example, the significand's most
%significant bit is implicit in single and double precision but
%is explicit in extended precision.  Whether that bit is implicit
%or explicit does not affect the mathematical precision.
%In implementations that use binary floating point, the default
%precision can be calculated by calling the following procedure:
Точность значащей части не следует путать с количеством бит, используемых для представления
значащей части. В стандартах IEEE с плавающей точкой, например, старший значащий бит значащей
части является неявным при одинарной и двойной точности, и явным при расширенной точности. На
математическую точность не влияет, является ли этот бит явным или неявным. В реализациях,
использующих двоичную плавающую точку, точность по умолчанию может быть вычислена путём вызова
следующей процедуры:

\begin{scheme}
\bfseries (define (precision)
\bfseries   (do ((n 0 (+ n 1))
\bfseries        (x 1.0 (/ x 2.0)))
\bfseries     ((= 1.0 (+ 1.0 x)) n)))
\end{scheme}
\end{note}

\begin{note}
%When the underlying floating-point representation is IEEE double
%precision, the {\cf |\var{p}} suffix should not always be omitted:
%Denormalized floating-point numbers have diminished precision,
%and therefore their external representations should
%carry a {\cf |\var{p}} suffix with the actual width of the
%significand.
Если основным представлением с плавающей точкой является двойная точность IEEE, {\cf |\var{p}},
суффикс не должен всегда быть пропущеным: Денормализованные числа с плавающей точкой имеют пониженную
точность, и поэтому их внешние представления должны содержать суффикс {\cf |\var{p}} с
фактической шириной значащей части.
\end{note}

%The literals {\cf +inf.0} and {\cf -inf.0} represent positive and
%negative infinity, respectively.  The {\cf +nan.0}
%literal represents the NaN that is the result of {\cf (/ 0.0 0.0)},
%and may represent other NaNs as well.
Литералы {\cf\bfseries +inf.0} и {\cf\bfseries -inf.0} представляют положительную и
отрицательную бесконечность соответственно. Литерал {\cf\bfseries +nan.0} представляет NaN,
являющийся результатом {\cf\bfseries(/ 0.0 0.0)}, и может также представлять другие NaN.

%If \var{x} is an external representation of an inexact real number
%object and
%contains no vertical bar and no exponent marker
%other than {\cf e}, the inexact real number object it represents is a flonum
%(see library section~\extref{lib:flonumssection}{Flonums}).
%Some or all of the other external representations of
%inexact real number objects may also represent flonums, but that is not required by
%this report.
Если \var{x} является внешним представлением неточного действительного числового объекта и не
содержит вертикальной черты и экспоненциального маркера кроме {\cf\bfseries e}, неточным
действительным числовым объектом, представляющим его, является flonum (см. секцию
библиотек~\extref{lib:flonumssection}{Flonums}). Некоторые или все другие внешние представления
неточных действительных числовых объектов могут также представлять flonums, но это не является
требованием данной работы.

%\section{Datum syntax}
\section{Datum-синтаксис}
\label{datumsyntaxsection}

%The datum syntax describes the syntax of
%syntactic data\mainindex{syntactic datum} in terms of a sequence of
%\meta{lexeme}s, as defined in the lexical syntax.
Datum-синтаксис описывает синтаксис синтаксических данных\mainindex{syntactic datum}
в терминах последовательности \meta{лексем} как определено в лексическом
синтаксисе.

%Syntactic data include the lexeme data described in the
%previous section as well as the following constructs for forming
%compound data:
Синтаксические данные включают данные лексем, описанные в предыдущих секциях, а также
следующие конструкции для формирования составных данных:
%
\begin{itemize}
%\item pairs and lists, enclosed by \verb|( )| or \verb|[ ]| (see
%  section~\ref{pairlistsyntax})
\item пары и списки, заключённые в \verb|( )| или \verb|[ ]| (см.
  секцию~\ref{pairlistsyntax})
%\item vectors (see section~\ref{vectorsyntax})
\item векторы (см. секцию~\ref{vectorsyntax})
%\item bytevectors (see section~\ref{bytevectorsyntax})
\item байтовые векторы (см. секцию~\ref{bytevectorsyntax})
\end{itemize}

%\subsection{Formal account}
\subsection{Формальное описание}
\label{datumsyntax}

%The following grammar describes the syntax of syntactic data in terms
%of various kinds of lexemes defined in the grammar in
%section~\ref{lexicalsyntaxsection}:
Следующая грамматика описывает синтаксис синтаксических данных в терминах лексем различных видов,
определённых в грамматике в секции ~\ref{lexicalsyntaxsection}:

{%
\renewcommand{\baselinestretch}{1.05}
\selectfont
\begin{grammar}%
\meta{datum} \: \meta{lexeme datum}
\>  \| \meta{compound datum}
\meta{lexeme datum} \: \meta{boolean} \| \meta{number}
\>  \| \meta{character} \| \meta{string} \|  \meta{symbol}
\meta{symbol} \: \meta{identifier}
\meta{compound datum} \: \meta{list} \| \meta{vector} \| \meta{bytevector}
\meta{list} \: (\arbno{\meta{datum}}) \| [\arbno{\meta{datum}}]
\>    \| (\atleastone{\meta{datum}} .\ \meta{datum}) \| [\atleastone{\meta{datum}} .\ \meta{datum}]
\>    \| \meta{abbreviation}
\meta{abbreviation} \: \meta{abbrev prefix} \meta{datum}
\meta{abbrev prefix} \: ' \| ` \| , \| ,@
\>    \| \#' | \#` | \#, | \#,@
\meta{vector} \: \#(\arbno{\meta{datum}})
\meta{bytevector} \: \#vu8(\arbno{\meta{u8}})
\meta{u8} \: $\langle${\rm any \meta{number} representing an exact}
 \>\>\quad\quad {\rm integer in $\{0, \ldots, 255\}$}$\rangle$%
\end{grammar}

}

%\subsection{Pairs and lists}
\subsection{Пары и списки}
\label{pairlistsyntax}

%List and pair data, representing pairs and lists of values
%(see section~\ref{listsection}) are represented using parentheses or brackets.
%Matching pairs of brackets that occur in the rules of \meta{list} are
%equivalent to matching pairs of parentheses.
Данные списков и пар, представляющие значения пар и списков (см. секцию~\ref{listsection}),
представляются с помощью круглых или квадратных скобок. Соответствие пар квадратных скобок,
находящихся в правилах \meta{list}, эквивалентно соответствию пар круглых
скобок.

%\newpage

%The most general notation for Scheme pairs as syntactic data is
%the ``dotted'' notation \hbox{\cf (\hyperi{datum} .\ \hyperii{datum})} where
%\hyperi{datum} is the representation of the value of the car field and
%\hyperii{datum} is the representation of the value of the
%cdr field.  For example {\cf (4 .\ 5)} is a pair whose car is 4 and whose
%cdr is 5.
Наиболее общей нотацией для пар Scheme как синтаксических данных является ``точечная'' нотация
\hbox{\cf (\hyperi{datum} .\ \hyperii{datum})}, где \hyperi{datum} является представлением
значения car, а \hyperii{datum} - значения поля cdr. Например, {\cf\bfseries (4 .\ 5)} является
парой, car которой - 4, а cdr - 5.

%A more streamlined notation can be used for lists: the elements of the
%list are simply enclosed in parentheses and separated by spaces.  The
%empty list\index{empty list} is represented by {\tt()} .  For example,
Для списков может использоваться более чёткая нотация: элементы списка просто заключаются в
круглые скобки и разделяются пробелами. Пустой список\index{empty list} представляется, как
{\tt()}. Например,

\begin{scheme}
\bfseries(a b c d e)%
\end{scheme}

%and
и

\begin{scheme}
\bfseries (a . (b . (c . (d . (e . ())))))%
\end{scheme}

%are equivalent notations for a list of symbols.
являются эквивалентными формами записи списка символов

%The general rule is that, if a dot is followed by an open parenthesis,
%the dot, open parenthesis, and matching closing parenthesis
%can be omitted in the external representation.
Общее правило гласит, что если за точкой следует открывающаяся круглая скобка, точка,
открывающаяся круглая скобка и соответствующая ей закрывающаяся круглая скобка
во внешнем представлении могут быть пропущены.

%The sequence of characters ``{\cf (4 .\ 5)}'' is the external representation of a
%pair, not an expression that evaluates to a pair.
%Similarly, the sequence of characters ``{\tt(+ 2 6)}'' is {\em not} an
%external representation of the integer 8, even though it {\em is} an
%expression (in the language of the \rsixlibrary{base} library)
%evaluating to the integer 8; rather, it is a
%syntactic datum representing a three-element list, the elements of which
%are the symbol {\tt +} and the integers 2 and 6.
Последовательность символов ``{\cf\bfseries (4 .\ 5)}'' является внешним представлением пары, а не
выражением, которое вычисляется как пара. Аналогично, последовательность символов ``{\tt\bfseries (+ 2 6)}''
{\em не} является внешним представлением целого числа 8, даже при том, что она {\em является} выраженим
(на языке библиотеки {\bfseries\rsixlibrary{base}}), вычисляемым как целое число 8; наоборот, это
синтаксический datum, представляющий трёхэлементный список, элементами которого являются
символ {\tt\bfseries +} и целые числа 2 и 6.

%\subsection{Vectors}
\subsection{Векторы}
\label{vectorsyntax}

%Vector data, representing vectors of objects (see
%section~\ref{vectorsection}), are represented using the notation
%{\tt\#(\hyper{datum} \dotsfoo)}.  For example, a vector of length 3
%containing the number object for zero in element 0, the list {\cf(2 2 2 2)} in
%element 1, and the string {\cf "Anna"} in element 2 can be represented as
%follows:
Векторные данные, представляюшие вектора объектов (см. секцию~\ref{vectorsection}),
представляются с помощью нотации {\tt{\bfseries\#(}\hyper{datum}
  \dotsfoo{\bfseries)}}. Например, вектор с длиной 3, содержащий числовой объект для нуля в
элементе 0, список {\cf\bfseries (2 2 2 2)} в элементе 1 и строку {\cf\bfseries "Анна"} в
элементе 2, может быть представлен следующим образом:

\begin{scheme}
\bfseries \#(0 (2 2 2 2) "Anna")%
\end{scheme}

%This is the external representation of a vector, not an
%expression that evaluates to a vector.
Это внешнее представление вектора, а не выражение, вычисляемое как вектор.

%\subsection{Bytevectors}
\subsection{Байтовые векторы}
\label{bytevectorsyntax}

%Bytevector data, representing bytevectors (see
%library chapter~\extref{lib:bytevectorschapter}{Bytevectors}), are represented using the notation
%{\tt\#vu8(\hyper{u8} \dotsfoo)}, where the \hyper{u8}s represent the octets of
%the bytevector.  For example, a bytevector of length 3 containing the
%octets 2, 24, and 123 can be represented as follows:
Байт-векторные данные, представляющие байтовые вектора (см. библиотечную
главу~\extref{lib:bytevectorschapter}{Bytevectors}), представляются с помощью нотации
{\tt{\bfseries\#vu8(}\hyper{u8} \dotsfoo{\bfseries )}}, где \hyper{u8} представляет октет
байтового вектора. Например, байтовый вектор с длиной 3, содержащий октеты 2, 24 и 123, может быть
представлен следующим образом:

\begin{scheme}
\bfseries \#vu8(2 24 123)%
\end{scheme}

%This is the external representation of a bytevector, and also an
%expression that evaluates to a bytevector.
Это внешнее представление байтового вектора, а не выражение, вычисляемое как байтовый вектор.

%\subsection{Abbreviations}\unsection
\subsection{Сокращения}\unsection
\label{abbreviationsection}

{%
\renewcommand{\baselinestretch}{1.05}
\selectfont
\begin{entry}{%
\pproto{\singlequote\hyper{datum}}{}
\pproto{\backquote\hyper{datum}}{}
\pproto{,\hyper{datum}}{}
\pproto{,\atsign\hyper{datum}}{}
\pproto{\#'\hyper{datum}}{}
\pproto{\#\backquote\hyper{datum}}{}
\pproto{\#,\hyper{datum}}{}
\pproto{\#,@\hyper{datum}}{}
}

%Each of these is an abbreviation:
Каждый из них является сокращением:
\\\quad\schindex{'}\singlequote\hyper{datum}
for {\cf (quote \hyper{datum})},
\\\quad\schindex{`}\backquote\hyper{datum}
for {\cf (quasiquote \hyper{datum})},
\\\quad\schindex{,}{\cf,}\hyper{datum}
for {\cf (unquote \hyper{datum})},
\\\quad\index{,@\texttt{,\atsign}}{\cf,}\atsign\hyper{datum}
for {\cf (unquote-splicing \hyper{datum})},
\\\quad\sharpindex{'}{\cf\#'}\hyper{datum}
for {\cf (syntax \hyper{datum})},
\\\quad\sharpindex{`}{\cf\#`}\hyper{datum}
for {\cf (quasisyntax \hyper{datum})},
\\\quad\sharpindex{,}{\cf\#,}\hyper{datum}
for {\cf (unsyntax \hyper{datum})}, and
\\\quad\index{#,@\texttt{\#,\atsign}}{\cf\#,@}\hyper{datum}
for {\cf (unsyntax-splicing \hyper{datum})}.
\end{entry}

}

%%% Local Variables:
%%% mode: latex
%%% TeX-master: "r6rs"
%%% End:
     \par
%\vfill\eject
%\chapter{Semantic concepts}
\chapter{Семантические концепции}
\label{basicchapter}

%\section{Programs and libraries}
\section{Программы и библиотеки}

%A Scheme program consists of a \textit{top-level program\index{top-level program}}
%together with a set of \textit{libraries\index{library}}, each
%of which defines a part of the program connected to the others through
%explicitly specified exports and imports.  A library consists of a set
%of export and import specifications and a body, which consists of
%definitions, and expressions.
%A top-level program is similar to a library, but
%has no export specifications.
%Chapters~\ref{librarychapter} and \ref{programchapter}
%describe the syntax and semantics of libraries and top-level programs,
%respectively.
%Chapter~\ref{baselibrarychapter} describes a base
%library that defines many of the constructs traditionally associated with
%Scheme.
%A separate report~\cite{R6RS-libraries}
%describes the various \textit{standard libraries}\index{standard
%  library} provided by a Scheme system.
Программа Scheme состоит из \textit{программы верхнего уровня\index{top-level program}} совместно с
набором \textit{библиотек\index{library}}, каждая из которых определяет часть программы, связанную
с другими частями посредством явно указываемых экспорта и импорта. Библиотека состоит из ряда
спецификаций экспорта и импорта, а также тела, состоящего из определений и выражений. Программа
верхнего уровня похожа на библиотеку, но не имеет спецификаций
экспорта. В главах~\ref{librarychapter} и \ref{programchapter} описаны синтаксис и семантика
библиотек и программ верхнего уровня соответственно. В главе~\ref {baselibrarychapter} описана
основная библиотека, в которой определено большинство конструкций, традиционно ассоциированных со
Scheme. В отдельной работе~\cite{R6RS-libraries} описаны различные \textit{стандартные
библиотеки}, \index{standard library} предоставляемые системой Scheme.

%The division between the base library and the other standard libraries is
%based on use, not on construction.  In particular, some facilities
%that are typically implemented as ``primitives'' by a compiler or the
%run-time system rather than in terms of other standard procedures
% or syntactic forms are not part of the base library, but are defined in
%separate libraries.  Examples include the fixnums and flonums libraries,
%the exceptions and conditions libraries, and the libraries for
%records.
Деление на основную библиотеку и прочие стандартные библиотеки является прикладным, а не
конструктивным. В частности, некоторые средства, обычно реализуемые как ``примитивы''
компилятором или системой во время выполнения, а не в терминах других стандартных процедур или
синтаксических форм, не являются частью основной библиотеки, а определены в отдельных
библиотеках. Примеры включают библиотеки fixnums и flonums, библиотеки исключений и условий, а
также библиотеки для записей.

%\section{Variables, keywords, and regions}
\section{Переменные, ключевые слова и регионы}
\label{specialformsection}
\label{variablesection}

%Within the body of a library or top-level program,
%an identifier\index{identifier} may name a kind of syntax, or it may name
%a location where a value can be stored.  An identifier that names a kind
%of syntax is called a {\em keyword}\mainindex{keyword}, or {\em syntactic keyword}\mainindex{syntactic keyword},
%and is said to be {\em bound} to that kind of syntax (or, in the case of a
%syntactic abstraction, a {\em transformer} that translates the syntax into more
%primitive forms; see section~\ref{macrosection}).  An identifier that names a
%location is called a {\em variable}\mainindex{variable} and is said to be
%{\em bound} to that location.
%At each point within a top-level program or a library, a specific, fixed set
%of identifiers is bound.  The set of these identifiers, the set of \textit{visible
%bindings}\mainindex{binding}, is
%known as the {\em environment} in effect at that point.
В теле библиотеки или программы верхнего уровня идентификатор\index{identifier} может именовать
или вид синтаксиса, или ячейку памяти, где может храниться
значение. Идентификатор, именующий вид синтаксиса, называется {\em ключевым
  словом}\mainindex{keyword}, или {\em синтаксическим ключевым словом}\mainindex{syntactic
  keyword}, и считается {\em привязанным} к этому виду синтаксиса (или, в случае
синтаксической абстракции, {\em преобразователем}, транслирующим синтаксис в более
примитивные формы; см. секцию~\ref{macrosection}). Идентификатор, именующий ячейку памяти,
называется {\em переменной}\mainindex{variable} и считается {\em привязанным} к этой
ячейке памяти. К каждой точке программы верхнего уровня или библиотеке привязан конкретный,
постоянный набор идентификаторов. Набор таких идентификаторов, набор
\textit{видимого связывания}\mainindex{binding}, называется {\em окружением}, действующим в
данной точке.

%Certain forms are used to create syntactic abstractions
%and to bind keywords to transformers for those new syntactic abstractions, while other
%forms create new locations and bind variables to those
%locations.  Collectively, these forms are called {\em binding
%  constructs}.\mainindex{binding construct}
%Some binding constructs take the form of
%\textit{definitions}\index{definition}, while others are
%expressions.
%With the exception of exported library bindings, a binding created
%by a definition is visible only within the body in which the
%definition appears, e.g., the body of a library, top-level program,
%or {\cf lambda} expression.
%Exported library bindings are also visible within the bodies of
%the libraries and top-level programs that import them (see
%chapter~\ref{librarychapter}).
Одни формы используются для создания синтаксических абстракций и связывания ключевых
слов с преобразователями для этих новых синтаксических абстракций, в то время как другие формы
создают новые ячейки памяти и связывают переменные с этими ячейками. Все эти
формы обобщённо называются {\em конструкциями привязки}.\mainindex{binding construct}, Некоторые
конструкции привязки принимают форму \textit{определений}\index{definition}, в то время как
другие являются выражениями. За исключением экспортируемых библиотечных привязок, привязка,
созданная определением, видима только внутри тела, в котором находится определение,
например, в теле библиотеки, в программе верхнего уровня или в выражении {\cf\bfseries
  lambda}. Экспортируемые библиотечные привязки также видимы внутри тел библиотек и программ
верхнего уровня, импортирующих их (см. главу~\ref{librarychapter}).

%Expressions that bind variables include the {\cf lambda},
%{\cf let}, {\cf let*}, {\cf letrec}, {\cf letrec*}, {\cf let-values},
%and {\cf let*-values} forms from the base library (see
%sections~\ref{lambda}, \ref{letrec}).
%Of these, {\cf lambda} is the most fundamental.
%Variable definitions appearing within the body of
%such an expression, or within the bodies of a library or top-level
%program, are treated as a set of
%{\cf letrec*} bindings.
%In addition, for library bodies,
%the variables exported from the library can be referenced by
%importing libraries and top-level programs.
Выражения, связывающие переменные, включают формы {\cf\bfseries lambda}, {\cf\bfseries let},
{\cf\bfseries let*}, {\cf\bfseries letrec}, {\cf\bfseries letrec*}, {\cf\bfseries let-values}, и
{\cf\bfseries let*-values} из основной библиотеки (см. секции~\ref{lambda}, \ref {letrec}). Из
них {\cf\bfseries lambda} является самой фундаментальной. Определения переменных, находящиеся внутри
тела такого выражения, или внутри тела библиотеки или программы верхнего уровня,
интерпретируются как набор привязок {\cf\bfseries letrec*}. Кроме того, для тел библиотеки, к переменным,
экспортируемым из библиотеки, можно обратиться, импортируя программы верхнего уровня и
библиотеки.

%Expressions that bind keywords include the {\cf
%  let-syntax} and {\cf letrec-syntax} forms (see
%section~\ref{bindsyntax}).  A {\cf define} form (see section~\ref{define}) is a
%definition that creates a variable binding (see
%section~\ref{defines}), and a {\cf define-syntax} form is
%a definition that creates a keyword binding (see
%section~\ref{syntaxdefinitionsection}).
Выражения, привязывающие ключевые слова, включают формы {\cf\bfseries let-syntax} и
{\cf\bfseries letrec-syntax} (см. секцию~\ref{bindsyntax}). Форма {\cf\bfseries define}
(см. секцию~\ref{define}) является определением, создающим привязку переменной
(см. секцию~\ref{define}), а форма {\cf\bfseries define-syntax} - определением, создающим
привязку ключевого слова (см. секцию~\ref{syntaxdefinitionsection}).

%\vest Scheme is a statically scoped language with
%block structure.  To each place in a top-level program or library body where an identifier is bound
%there corresponds a \defining{region} of code within which
%the binding is visible.  The region is determined by the particular
%binding construct that establishes the binding; if the binding is
%established by a {\cf lambda} expression, for example, then its region
%is the entire {\cf lambda} expression.  Every mention of an identifier
%refers to the binding of the identifier that establishes the
%innermost of the regions containing the use.  If a use of an
%identifier appears in a place where none of the surrounding expressions
%contains a binding for the identifier, the use may refer to a
%binding established by a definition or import at the top of the
%enclosing library or top-level program
%(see chapter~\ref{librarychapter}).
%If there is no binding for the identifier,
%it is said to be \defining{unbound}.\mainindex{bound}
\vest Scheme -- язык со статическими областями видимости и блочной структурой. Каждой позиции в
программе верхнего уровня или тела библиотеки, где привязан идентификатор, соответствует
\defining{регион} кода, внутри которого видима привязка. Регион определяется конкретной
конструкцией привязки, устанавливающей привязку; если привязка установлена, например, выражением
{\cf\bfseries lambda}, её регионом является всё выражение {\cf\bfseries lambda}. Каждое
упоминание идентификатора обращается к привязке идентификатора, установленной самым внутренним
из регионов, содержащих её применение. Если применение идентификатора находится в позиции, где
ни одно из близлежащих выражений не содержит привязки идентификатора, применение может
обратиться к привязке, установленной определением или импортом из верхнего уровня окружающей
библиотеки, или программой верхнего уровня (см. главу~\ref {librarychapter}). Если для
идентификатора не существует привязки, он называется \defining{несвязанным}.\mainindex{bound}

%\section{Exceptional situations}
\section{Исключительные ситуации}
\label{exceptionalsituationsection}

%\mainindex{exceptional situation}A variety of exceptional situations
%are distinguished in this report, among them violations of syntax,
%violations of a procedure's specification, violations of
%implementation restrictions, and exceptional situations in the
%environment.  When an exceptional situation is detected by the
%implementation, an \textit{exception is raised}\mainindex{raise},
%which means that a special procedure called the \textit{current
%  exception handler} is called.  A program can also raise an
%exception, and override the current exception handler; see
%library section~\extref{lib:exceptionssection}{Exceptions}.
\mainindex{exceptional situation}В данной работе рассматривается разнообразие исключительных
ситуаций, среди которых нарушения синтаксиса, нарушения спецификаций процедур, нарушения
ограничений реализации и исключительные ситуации в окружении. При обнаружении реализацией
исключительной ситуации \textit{возбуждается исключение}\mainindex{raise}, что
означает вызов специальной процедуры, называемой \textit{текущим обработчиком исключений}.
Программа может также возбудить исключение и переопределить текущий обработчик
исключений; см. библиотечную секцию~\extref{lib:exceptionssection}{Exceptions}.

%When an exception is raised, an object is provided that
%describes the nature of the exceptional situation.  The report uses
%the condition system described in library section~\extref{lib:conditionssection}{Conditions} to
%describe exceptional situations, classifying them by condition types.
При возбуждении исключения порождается объект, описывающий характер исключительной ситуации. В
работе используется система состояний, описанная в библиотечной
секции~\extref{lib:conditionssection}{Conditions}, для описания исключительных ситуаций
классификацией их типами состояний.

%Some exceptional situations allow continuing the program if the
%exception handler takes appropriate action.  The corresponding
%exceptions are called \textit{continuable}\index{continuable exception}.
%For most of the exceptional situations described in this report,
%portable programs cannot rely upon the exception being continuable
%at the place where the situation was detected.
%For those exceptions, the exception handler that is invoked by the
%exception should not return.
%In some cases, however, continuing is permissible, and the
%handler may return.  See library section~\extref{lib:exceptionssection}{Exceptions}.
Некоторые исключительные ситуации допускают продолжение программы, если обработчик исключений
предпримет соответствующее действие. Соответствующие исключения называются
\textit{продолжаемыми}\index{continuable exception}. В большинстве исключительных ситуаций,
описанных в данной работе, переносимые программы не могут зависеть от продолжаемых в месте
обнаружения ситуации исключений. Для таких исключений обработчик исключений, вызванный
исключением, не должен возвращаться. В некоторых случаях, однако, продолжение допустимо, и
обработчик может возвращаться. См. библиотечную
секцию~\extref{lib:exceptionssection}{Exceptions}.

%Implementations must raise an exception
%when they are unable to continue correct execution
%of a correct program due to some \defining{implementation restriction}.  For
%example, an implementation that does not support infinities
%must raise an exception with condition type
%{\cf\&implementation-restriction} when it evaluates an expression
%whose result would be an infinity.
Реализации должны возбудить исключение в случае невозможности продолжения корректного выполнения
корректной программы из-за некоторого \defining{ограничения реализации}. Например, реализация,
не поддерживающая бесконечность, должна возбудить исключение с типом состояния
{\cf\bfseries\&implementation-restriction}, при вычислениии выражения, результатом которого
может быть бесконечность.

%Some possible implementation restrictions such as the lack of
%representations for NaNs and infinities (see
%section~\ref{infinitiesnanssection}) are anticipated by this report,
%and implementations typically must raise an exception of the
%appropriate condition type if they encounter such a situation.
Некоторые возможные ограничения реализации, типа нехватки представлений для NaN и
бесконечностей (см. секцию~\ref{infinitiesnanssection}) предугадываются в соответствии с данной
работой, и реализации, как правило, должны возбуждать исключение соответствующего типа
состояния, если они сталкиваются с такой ситуацией.

%This report uses the phrase ``an exception is raised'' synonymously
%with ``an exception must be raised''.
%This report uses the phrase ``an exception with condition type \var{t}''
%to indicate that the object provided with the
%exception is a condition object of the specified type.
%The phrase ``a continuable exception is raised'' indicates an
%exceptional situation that permits the exception handler to return.
В данной работе применение фразы ``исключение возбуждено'' синонимично с "исключение должно быть
возбуждено". В данной работе используется фраза ``исключение с типом состояния \var{t}'', для указания,
что объект, порождаемый с исключением, является объектом состояния указанного типа. Фраза
``продолжаемое исключение возбуждено'', указывает исключительную ситуацию, разрешающую возврат обработчика
исключений.

%\section{Argument checking}
\section{Проверка аргументов}
\label{argumentcheckingsection}

\mainindex{argument checking}
%Many procedures specified in this report or as part of a
%standard library restrict the arguments they accept.
%Typically, a procedure accepts only specific numbers and types of arguments.
%Many syntactic forms similarly restrict the values to which one or
%more of their subforms can evaluate.
%These restrictions imply responsibilities\mainindex{responsibility} for
%both the programmer and the implementation.
%Specifically, the programmer is responsible for ensuring
%that the values indeed adhere to the restrictions described
%in the specification.  The implementation must check
%that the restrictions in the specification are indeed met, to the
%extent that it is reasonable, possible, and necessary to allow the
%specified operation to complete successfully.  The implementation's
%responsibilities are specified in more detail in
%chapter~\ref{entryformatchapter} and throughout the report.
Многие процедуры, определённые в данной работе или в качестве части стандартной библиотеки,
накладывают ограничения на принимаемые ими аргументы. Обычно процедура принимает только
конкретное количество и конкретные типы аргументов. Аналогично, многие синтаксические формы
накладывают ограничения на значения результатов вычислений одной или более своих подформ. Эти
ограничения подразумевают обязанности\mainindex{responsibility} и программиста, и реализации. А
именно, программист обязан гарантировать, что значения действительно удовлетворяют описанным в
спецификации ограничениям. Реализация должна проверить, что ограничения спецификации
действительно выполнены в том смысле, что предоставление возможности успешного завершения
указанной операции является разумным, возможным и необходимым. Обязанности реализации определены
более подробно в главе~\ref{entryformatchapter} и повсюду в данной работе.

%Note that it is not always possible for an implementation to completely check
%the restrictions set forth in a specification.  For example, if an
%operation is specified to accept a procedure with specific properties,
%checking of these properties is undecidable in general.  Similarly,
%some operations accept both lists and procedures that are
%called by these operations.  Since lists can be mutated by the procedures
%through the \rsixlibrary{mutable-pairs} library (see library
%chapter~\extref{lib:pairmutationchapter}{Mutable pairs}), an argument that is a list
%when the operation starts may become a non-list during the execution of the operation.
%Also, the procedure might escape to a different continuation,
%preventing the operation from performing more checks.
%Requiring the operation to check that the argument is a list after
%each call to such a procedure would be impractical.  Furthermore, some
%operations that accept lists only need to traverse these lists
%partially to perform their function; requiring the implementation to
%traverse the remainder of the list to verify that all specified
%restrictions have been met might
%violate reasonable performance assumptions.  For these reasons, the
%programmer's obligations may exceed the checking obligations of the
%implementation.
Необходимо отметить, что у реализации не всегда имеется возможность полной проверки
сформулированных в спецификации ограничений. Например, если операция определена для приёма
процедуры с конкретными свойствами, проверка этих свойств неразрешима в принципе. Аналогично,
некоторые операции принимают как списки, так и процедуры, которые вызывают эти операции. Так как
списки могут быть видоизменены процедурами с помощью библиотеки \textbf{\rsixlibrary{mutable-pairs}}
(см. библиотечную главу~\extref{lib:pairmutationchapter}{Изменяемые пары}), аргумент, являвшийся
списком при запуске операции, может перестать быть списком во время выполнения операции. К тому
же, процедура может перейти к другому продолжению, что не позволит операции произвести больше
проверок. Требовать от операции проверки, что аргумент является списком после каждого вызова
такой процедуры было бы непрактично. Кроме того, некоторым операциям, принимающим только
списки, для выполнения своей функции необходимо частичное прохождение по данным спискам; требование
реализации проходить остаток списка для проверки выполнения всех указанных ограничений
может нарушить разумные условия производительности. По этим причинам обязанности программистов
могут превышать обязанности проверки реализацией.

%When an implementation detects a violation of a restriction for an
%argument, it must raise an exception with condition type
%{\cf\&assertion} in a way consistent with the safety of execution as
%described in section~\ref{safetysection}.
При обнаружении реализацией нарушения ограничения аргумента она должна возбудить исключение
с типом состояния {\cf\&assertion} совместимым с безопасностью выполнения образом, как описано в
секции ~\ref{safetysection}.\vspace{3mm}

%\section{Syntax violations}
\section{Нарушения синтаксиса}\vspace{3mm}

%The subforms of a special form usually need to obey certain syntactic
%restrictions.  As forms may be subject to macro expansion, which may
%not terminate, the question of whether they obey the specified
%restrictions is undecidable in general.
Подформы специальной формы обычно должны подчиняться некоторым синтаксическим
ограничениям. Поскольку формы могут быть представлены макро-разворачиванием, возможно, не
завершённым, вопрос их подчинения указанным ограничениям неразрешим в принципе.

%When macro expansion terminates, however, implementations must detect
%violations of the syntax.  A \defining{syntax violation} is an error
%with respect to the syntax of library bodies, top-level bodies,
%or the ``\exprtype'' entries in the
%specification of the base library or the standard libraries.
%Moreover, attempting to assign to an immutable variable (i.e., the
%variables exported by a library; see
%section~\ref{importsareimmutablesection}) is also
%considered a syntax violation.
При окончании макро-разворачивания, однако, реализации должны обнаруживать нарушения
синтаксиса. \defining{Нарушение синтаксиса} - ошибка относительно синтаксиса библиотечных тел,
тел верхнего уровня, или ``\exprtype'' записей в спецификации базовой библиотеки или
стандартных библиотек. Кроме того, попытка присваивания неизменяемой переменной (то есть,
переменным, экспортируемым библиотекой; см. секцию~\ref{importsareimmutablesection}) также
считается нарушением синтаксиса.

\newpage

%If a top-level or library form in a program is not syntactically
%correct, then the implementation must raise an exception with
%condition type {\cf\&syntax}, and execution of that top-level program
%or library must not be allowed to begin.
Если форма верхнего уровня или библиотеки в программе не является синтаксически корректной,
реализация должна возбудить исключение с типом состояния {\bfseries\cf\&syntax}, и не должна
позволить начать выполнение такой программы или библиотеки верхнего уровня.\vspace{-2mm}

%\section{Safety}
\section{Безопасность}\vspace{-1mm}
\label{safetysection}

%The standard libraries whose exports are described by this document
%are said to be \defining{safe libraries}.  Libraries and top-level
%programs that import only from safe libraries are also said to be safe.
Стандартные библиотеки, экспорт которых описан в данном документе, называются
\defining{безопасными библиотеками}. Библиотеки и программы верхнего уровня,
импортирующие только из безопасных библиотек, также называются безопасными.

%As defined by this document, the Scheme programming language
%is safe in the following sense:
%The execution of a safe top-level program
%cannot go so badly wrong as to crash or to continue to
%execute while behaving in ways that are
%inconsistent with the semantics described in this document,
%unless an exception is raised.
Как определено данным документом, язык программирования Scheme безопасен в
следующем смысле: выполнение безопасной программы верхнего уровня не может происходить
настолько неверно, чтобы привести к аварийному завершению или продолжению выполнения,
функционирование которого противоречит семантике, описанной в данном документе, если исключение
не возбуждено.

%Violations of an implementation restriction must raise an
%exception with condition type {\cf\&implementation-\hp{}restriction},
%as must all
%violations and errors that would otherwise threaten system
%integrity in ways that might result in execution that is
%inconsistent with the semantics described in this document.
Нарушения ограничений реализации должны возбуждать исключение с типом состояния
{\bfseries\cf\&implementation-\hp{}restriction}, как должны все нарушения и ошибки, которые в
противном случае угрожали бы целостности системы таким образом, что это могло бы привести к
выполнению, противоречащему семантике, описанной в данном документе.

%The above safety properties are guaranteed only for top-level programs
%and libraries that are said to be safe.  In particular,
%implementations may provide access to unsafe libraries in ways that
%cannot guarantee safety.
Вышеупомянутые свойства безопасности гарантируются только для программ верхнего уровня и
библиотек, которые называются безопасными. В частности, реализации могут обеспечивать
доступ к опасным библиотекам способами, которые не могут гарантировать безопасность.\vspace{-2mm}

%\section{Boolean values}
\section{Булевы значения}\vspace{-1mm}
\label{booleanvaluessection}

%Although there is a separate boolean type, any Scheme value can be
%used as a boolean value for the purpose of a conditional test.  In a
%conditional test, all values count as true in such a test except for
%\schfalse{}.  This report uses the word ``true'' to refer to any
%Scheme value except \schfalse{}, and the word ``false'' to refer to
%\schfalse{}. \mainindex{true} \mainindex{false}
Хотя существует отдельный булевый тип, любое значение Scheme может использоваться в качестве
булевого для проверки условия. В проверке условия все значения считаются истинными в таком тесте,
за исключением {\bfseries\schfalse{}}. В данной работе используется слово ``true'' для обращения
к любому значению Scheme, кроме {\bfseries\schfalse{}}, и слово ``false'' для обращения к
{\bfseries\schfalse{}}. \mainindex{true}\mainindex{false}\vspace{-2mm}

%\section{Multiple return values}
\section{Несколько возвращаемых значений}\vspace{-1mm}
\label{multiplereturnvaluessection}

%A Scheme expression can evaluate to an arbitrary finite number of
%values.  These values are passed to the expression's continuation.
Выражение Scheme может вычисляться, как произвольное конечное количество значений. Эти значения
передаются продолжению выражения.

%Not all continuations accept any number of values. For example, a continuation that
%accepts the argument to a procedure call is guaranteed to accept
%exactly one value.  The effect of passing some other number of values
%to such a continuation is unspecified.  The {\cf call-with-values}
%procedure
%described in section~\ref{controlsection} makes it possible to create
%continuations that accept specified numbers of return values.
%If the number of
%return values passed to a continuation created by a call to
%{\cf call-with-values} is not accepted by its consumer
%that was passed in that call, then an exception is raised.
%A more complete description of the number of values accepted by
%different continuations and the consequences of passing an unexpected
%number of values is given in the description of the {\cf values}
%procedure in section~\ref{values}.
Не все продолжения принимают несколько значений. Например, продолжение, принимающее аргумент
вызова процедуры, гарантированно примет ровно одно значение. Результат передачи любого другого
количества значений такому продолжению не определён. Процедура {\bfseries\cf call-with-values},
описанная в секции~\ref{controlsection}, позволяет создать продолжения, принимающие указанное
количество возвращаемых значений. Если количество возвращаемых значений, переданных продолжению,
созданному вызовом {\bfseries\cf call-with-values}, не принято его потребителем, переданным в
этом вызове, возбуждается исключение. Более полное описание количества значений, принимаемых
различными продолжениями и последствия передачи неожидаемого количества значений приведено в
описании процедуры {\bfseries\cf values} в секции~\ref{values}.

%A number of forms in the base library have sequences of expressions
%as subforms that are evaluated sequentially, with the return values of
%all but the last expression being discarded.  The continuations
%discarding these values accept any number of values.
Во множестве форм основной библиотеки содержатся последовательности выражений в качестве подформ,
вычисляемых последовательно, со всеми возвращаемыми ими значениями, кроме последнего, не учитывемого
выражения. Продолжения, не учитывающие эти значения, принимают любое количество значений.\vspace{-4mm}

%\section{Unspecified behavior}
\section{Неопределённое поведение}\vspace{-2mm}

%\vest If an expression is said to ``return unspecified values'',
%then the expression must evaluate without raising an exception, but
%the values returned depend on the implementation; this report
%explicitly does not say how many or what values should be returned.
%Programmers should not rely on a specific number of return values or
%the specific values themselves.
%\mainindex{unspecified behavior}\mainindex{unspecified values}
\vest Если сообщается, что выражение ``возвращает неопределённые значения'', выражение должно
вычисляться без возбуждения исключения, но возвращаемые значения зависят от реализации; в данной
работе явно не оговаривается, сколько или какие значения должны возвращаться. Программисты не
должны полагаться на конкретное количество возвращаемых значений или непосредственно конкретных
значений. \mainindex{unspecified behavior}\mainindex{unspecified values}\vspace{-4mm}

%\section{Storage model}
\section{Модель памяти}\vspace{-2mm}
\label{storagemodel}

%Variables and objects such as pairs, vectors, bytevectors, strings,
%hashtables, and records implicitly
%refer to locations\mainindex{location} or sequences of locations.  A string, for
%example, contains as many locations as there are characters in the string.
%(These locations need not correspond to a full machine word.) A new value may be
%stored into one of these locations using the {\tt string-set!} procedure, but
%the string contains the same locations as before.
Переменные и объекты типа пар, векторов, байтовых векторов, строк, хэш-таблиц и записей, неявно
адресуются к областям памяти\mainindex{location} или последовательностям областей памяти. Строка,
например, содержит столько областей памяти, сколько символов в строке. (Эти области памяти не
обязаны соответствовать полному машинному слову.) Новое значение может быть сохранено в одной из
этих областей памяти с помощью процедуры {\bfseries\tt string-set!}, но строка содержит те же
области памяти, как и прежде.

%An object fetched from a location, by a variable reference or by
%a procedure such as {\cf car}, {\cf vector-ref}, or {\cf string-ref}, is
%equivalent in the sense of \ide{eqv?} % and \ide{eq?} ??
%(section~\ref{equivalencesection})
%to the object last stored in the location before the fetch.
Объект, считанный из области памяти обращением к переменной или процедурой типа {\bfseries\cf
  car}, {\bfseries\cf vector-ref} или {\bfseries\cf string-ref}, эквивалентен в смысле
\textbf{\ide{eqv?}} (секция~\ref{equivalencesection}) последнему сохранённому в области памяти перед
считыванием объекту.

%Every location is marked to show whether it is in use.
%No variable or object ever refers to a location that is not in use.
%Whenever this report speaks of storage being allocated for a variable
%or object, what is meant is that an appropriate number of locations are
%chosen from the set of locations that are not in use, and the chosen
%locations are marked to indicate that they are now in use before the variable
%or object is made to refer to them.
Каждая область памяти помечается признаком её использования. Ни переменная, ни объект, никогда
не адресуются к неиспользуемой области памяти. При каждом упоминании в данной работе памяти,
выделяемой для переменной или объекта, имеется в виду, что соответствующее количество областей
памяти выбрано из множества неиспользуемых областей памяти, и выбранные области памяти
помечаются признаком их использования перед обращением к ним переменной или объекта.

%It is desirable for constants\index{constant} (i.e. the values of
%literal expressions) to reside in read-only memory.  To express this,
%it is convenient to imagine that every object that refers to locations
%is associated with a flag telling whether that object is
%mutable\index{mutable} or immutable\index{immutable}.  Literal
%constants, the strings returned by \ide{symbol->string}, records with
%no mutable fields, and other values explicitly designated as immutable
%are immutable objects, while all objects created by the other
%procedures listed in this report are mutable.  An attempt to store a
%new value into a location referred to by an immutable object
%should raise an exception with condition type {\cf\&assertion}.
Хранить константы\index{constant} (то есть значения литеральных выражений) целесообразно в
памяти только для чтения. Чтобы выразить это, удобно представить, что каждый объект,
адресующийся к области памяти, связан с флагом, указывающим, является ли объект
изменяемым\index{mutable} или неизменяемым\index{immutable}. Литеральные константы, строки,
возвращаемые {\bfseries\ide{symbol->string}}, записи без изменяемых полей и другие значения,
явно определённые неизменяемыми являются неизменяемыми объектами, в то время как все объекты,
созданные другими процедурами, перечисленными в данной работе - изменяемыми. Попытка сохранить
новое значение в области памяти, адресуемое неизменяемым объектом, должна возбудить исключение с
типом состояния {\bfseries\cf\&assertion}.

%\section{Proper tail recursion}
\section{Чистая хвостовая рекурсия}
\label{proper tail recursion}

%Implementations of Scheme must be
%{\em properly tail-recursive}\mainindex{proper tail recursion}.
%Procedure calls that occur in certain syntactic
%contexts called \textit{tail contexts}\index{tail context}
%are \textit{tail calls}\mainindex{tail call}.  A Scheme implementation is
%properly tail-recursive if it supports an unbounded number of active
%tail calls.  A call is {\em active} if the called procedure may still
%return.  Note that this includes regular returns as well as returns
%through continuations captured earlier by
%{\cf call-with-current-continuation} that are later invoked.
%In the absence of captured continuations, calls could
%return at most once and the active calls would be those that had not
%yet returned.
%A formal definition of proper tail recursion can be found
%in Clinger's paper~\cite{propertailrecursion}.  The rules for identifying tail calls
%in constructs from the \rsixlibrary{base} library are described in
%section~\ref{basetailcontextsection}.
Реализации Scheme должны быть {\em чистыми хвост-рекурсивными}\mainindex{proper tail
  recursion}. Вызовы процедур, находящихся в определённых синтаксических контекстах, называемых
\textit{хвостовыми контекстами}\index{tail context}, являются \textit{хвостовыми
  вызовами}\mainindex{tail call}. Реализация Scheme обладает свойством чистой хвостовой
рекурсии, если она поддерживает неограниченное количество активных хвостовых вызовов. Вызов
является {\em активным}, если вызываемая процедура может все еще возвращаться. Отметьте, что это
включает регулярные возвращения так же как возвращения через продолжения, захваченные ранее
{\bfseries\cf call-with-current-continuation}, вызываемые позже. В отсутствии захваченных
продолжений, вызовы могли возвратиться максимум один раз, и активными будут вызовы, которые еще
не возвратились. Формальное определение чистой хвостовой рекурсии может быть найдено в газете
Клайнджера~\cite{propertailrecursion}. Правила идентификации хвостовых вызовов в конструкциях из
библиотеки \textbf{\rsixlibrary{base}} описаны в секции ~\ref{basetailcontextsection}.

%\section{Dynamic extent and the dynamic environment}
\section{Динамический экстент и динамическое окружение}
\label{dynamicenvironmentsection}

%For a procedure call, the time between when it is initiated and when
%it returns is called its \defining{dynamic extent}.  In Scheme, {\cf
%  call-with-current-continuation}
%(section~\ref{call-with-current-continuation}) allows reentering a
%dynamic extent after its procedure call has returned.  Thus, the
%dynamic extent of a call may not be a single, connected time period.
Для вызова процедуры, интервал между её началом и возвращением,
называют её \defining {динамическим экстентом}. В Scheme, {\cf\bfseries call-with-current-continuation}
(секция~\ref{call-with-current-continuation}) позволяет повторно войти динамическую экстент после
того, как ее вызов процедуры возвратился. Таким образом, динамический экстент вызова может не быть
единственным, связанным периодом времени.

Some operations described in the report acquire information in
addition to their explicit arguments from the \defining{dynamic
  environment}.  For example, {\cf call-\hp{}with-\hp{}current-\hp{}continuation}
accesses an implicit context established
by {\cf dynamic-wind} (section~\ref{dynamic-wind}), and the {\cf
  raise} procedure (library
section~\extref{lib:exceptionssection}{Exceptions}) accesses the
current exception handler.  The operations that modify the dynamic
environment do so dynamically, for the dynamic extent of a call to a
procedure like {\cf dynamic-wind} or {\cf with-exception-handler}.
When such a call returns, the previous dynamic environment is
restored.  The dynamic environment can be thought of as part of the
dynamic extent of a call.  Consequently, it is captured by {\cf
  call-with-current-continuation}, and restored by invoking the escape
procedure it creates.

%%% Local Variables:
%%% mode: latex
%%% TeX-master: "r6rs"
%%% End:
   \par
%\vfill\eject
%\chapter{Entry format}
\chapter{Формат записи}
\label{entryformatchapter}

%The chapters that describe bindings in the base library and the standard
%libraries are organized
%into entries.  Each entry describes one language feature or a group of
%related features, where a feature is either a syntactic construct or a
%built-in procedure.  An entry begins with one or more header lines of the form
Главы, описывающие привязки в основной и стандартных библиотеках,
упорядочены записями. Каждая запись описывает одну языковую характеристику или группу
связанных характеристик, где характеристика является или синтаксической конструкцией, или
встроенной процедурой. Запись начинается с одной или более строки заголовка вида

\newpage

\noindent\pproto{\var{template}}{\var{category}}\unpenalty

%The \var{category} defines the kind of binding described by the entry,
%typically either ``\exprtype'' or ``procedure''.
%An entry may specify various restrictions on subforms or arguments.
%For background on this, see section~\ref{argumentcheckingsection}.
\var{category} определяет вид привязки, описываемой записью, обычно или ``\exprtype'', или
``procedure''. В записи могут указываться различные ограничения подформ или аргументов. Вводную
информацию о них см. в секции~\ref{argumentcheckingsection}.\vspace{2mm}

%\section{Syntax entries}
\section{Запись синтаксиса}\vspace{2mm}

%If \var{category} is ``\exprtype'', the entry describes a
%special syntactic construct, and the template gives the syntax of the
%forms of the construct.
%The template is written in a notation similar to a right-hand
%side of the BNF rules in chapter~\ref{readsyntaxchapter}, and describes
%the set of forms equivalent to the forms matching the
%template as syntactic data.  Some ``\exprtype'' entries carry a
%suffix ({\cf expand}), specifying that the syntactic keyword of the
%construct is exported with level
%$1$.  Otherwise, the syntactic keyword is exported with level $0$; see
%section~\ref{phasessection}.
Если \var{category} -- ``\exprtype'', запись описывает специальную синтаксическую
конструкцию, а шаблон демонстрирует синтаксис форм конструкции. Шаблон записывается в
нотации, аналогичной правой часть правил BNF в главе~\ref{readsyntaxchapter}, и описывает
набор форм, эквивалентных формам, соответствующим шаблону в качестве синтаксических
данных. Некоторые записи ``\exprtype'' содержат суффикс ({\bfseries\cf expand}), что указывает
на экспорт синтаксического ключевого слова конструкции с уровнем $1$. В противном случае
синтаксическое ключевое слово экспортируется с уровнем $0$; см. секцию~\ref{phasessection}.\vspace{2mm}

%Components of the form described by a template are designated
%by syntactic variables, which are written using angle brackets, for
%example, \hyper{expression}, \hyper{variable}.  Case is insignificant
%in syntactic variables.  Syntactic variables
%stand for other forms, or
%sequences of them.  A syntactic variable may refer to a non-terminal
%in the grammar for syntactic data (see section~\ref{datumsyntax}),
%in which case only forms matching
%that non-terminal are permissible in that position.
%For example, \hyper{identifier} stands for a form which must be an
%identifier.
%Also,
%\hyper{expression} stands for any form which is a
%syntactically valid expression.  Other non-terminals that are used in
%templates are defined as part of the specification.
Компоненты описываемой шаблоном формы устанавливаются синтаксическими переменными, заключёнными
в угловые скобки, например, \hyper{expression}, \hyper{variable}. В синтаксических переменных
регистр является незначащим. Синтаксические переменные представляют другие формы или их
последовательности. Синтаксическая переменная может обращаться к нетерминальному символу в
грамматике синтаксических данных (см. секцию~\ref{datumsyntax}) только в случае, если формы,
соответствующие нетерминальному символу, допустимы в этой позиции. Например, \hyper{identifier}
представляет форму, которая должна быть идентификатором. Кроме того, \hyper{expression}
представляет любую форму, являющуюся синтаксически верным выражением. Другие используемые
в шаблонах нетерминальные символы определены как часть спецификации.\vspace{2mm}

%The notation
Нотация
\begin{tabbing}
\qquad \hyperi{thing} $\ldots$
\end{tabbing}
%indicates zero or more occurrences of a \hyper{thing}, and
означает ноль или более вхождений \hyper{thing}, а
\begin{tabbing}
\qquad \hyperi{thing} \hyperii{thing} $\ldots$
\end{tabbing}
%indicates one or more occurrences of a \hyper{thing}.
означает одно или более вхождение \hyper{thing}\vspace{2mm}

%It is the programmer's responsibility to ensure that each component of
%a form has the shape specified by a template.  Descriptions of syntax
%may express other restrictions on the components of a form.
%Typically, such a restriction is formulated as a phrase of the form
%``\hyper{x} must be\mainindex{must be} a \ldots''.  Again, these
%specify the programmer's responsibility.  It is the implementation's
%responsibility to check that these restrictions are satisfied, as long
%as the macro transformers involved in expanding the form terminate.
%If the implementation detects that a component does not meet the
%restriction, an exception with condition type {\cf\&syntax} is raised.
Обеспечение каждого компонента формы шейпом, определённым в шаблоне, является компетенцией
программиста. В описании синтаксиса могут формулироваться другие ограничения на компоненты формы.
Обычно такое ограничение формулируется фразой вида ``\hyper{x} должен быть\mainindex{must be}
\ldots''. Опять же, это подразумевает компетенцию программиста. Обязанностью реализации является
проверка удовлетворения этих ограничений при условии завершения работы участвующих в
разворачивании формы макротрансформеров. При обнаружении реализацией невыполнения компонентом
ограничения генерируется исключение с типом состояния {\bfseries\cf\&syntax}.\vspace{2mm}

%\newpage

%\section{Procedure entries}
\section{Запись процедур}

%If \var{category} is ``procedure'', then the entry describes a procedure, and
%the header line gives a template for a call to the procedure.  Parameter
%names in the template are \var{italicized}.  Thus the header line
Если \var{category} -- ``procedure'', запись описывает процедуру, а строка заголовка
демонстрирует шаблон вызова процедуры. Имена параметров в шаблоне выделены \var{курсивом}.
Таким образом, строка заголовка\vspace{1mm}

%\noindent\pproto{(vector-ref \var{vector} \var{k})}{procedure}\unpenalty
\noindent\pproto{\textbf{(vector-ref} \var{vector} \var{k}\textbf{)}}{procedure}\unpenalty

%indicates that the built-in procedure {\tt vector-ref} takes
%two arguments, a vector \var{vector} and an exact non-negative integer
%object \var{k} (see below).  The header lines
указывает, что встроенная процедура {\bfseries\tt vector-ref} принимает два аргумента, вектор
\var{vector} и точный неотрицательный целый объект \var{k} (см. ниже). Строки заголовка

\noindent%
%\pproto{(make-vector \var{k})}{procedure}
%\pproto{(make-vector \var{k} \var{fill})}{procedure}\unpenalty
\pproto{\textbf{(make-vector} \var{k}\textbf{)}}{procedure}
\pproto{\textbf{(make-vector} \var{k} \var{fill}\textbf{)}}{procedure}\unpenalty

%indicate that the {\tt make-vector} procedure takes
%either one or two arguments.  The parameter names are
%case-insensitive: \var{Vector} is the same as \var{vector}.
указывает, что процедура {\bfseries\tt make-vector} принимает один или два
аргумента. Имена параметров нечувствительны к регистру: \var{Vector} тождественен
\var{vector}.

%As with syntax templates, an ellipsis \dotsfoo{} at the end of a header
%line, as in
Как и в случае синтаксических шаблонов, многоточие \dotsfoo{} в конце строки заголовка, как например

\noindent\pproto{\textbf{(=} \vari{z} \varii{z} \variii{z} \dotsfoo\textbf{)}}{procedure}\unpenalty

%indicates that the procedure takes arbitrarily many arguments of the
%same type as specified for the last parameter name.  In this case,
%{\cf =} accepts two or more arguments that must all be complex
%number objects.
указывает, что процедура принимает произвольное количество аргументов того же типа, который
указан для последнего имени параметра. В данном случае {\bfseries\cf =} принимает два или более
аргумента, которые должны быть комплексными числовыми объектами.\vspace{1mm}

\label{typeconventions}
%A procedure that detects an argument that it is not specified to
%handle must raise an exception with condition type
%{\cf\&assertion}.  Also, the argument specifications are exhaustive: if the
%number of arguments provided in a procedure call does not match
%any number of arguments accepted by the procedure, an exception with
%condition type {\cf\&assertion} must be raised.
Процедура, обнаружившая не определённый для обработки аргумент, должна возбудить
исключение с типом состояния {\bfseries\cf\&assertion}. Кроме того, спецификации аргументов
являются полными: если количество аргументов, предоставляемых в вызове процедуры, не
соответствует любому принимаемому процедурой количеству аргументов, должно быть возбуждено
исключение с типом состояния {\bfseries\cf\&assertion}.\vspace{1mm}

%For succinctness, the report follows the convention
%that if a parameter name is also the name of a type, then the corresponding argument must be of the named type.
%For example, the header line for {\tt vector-ref} given above dictates that the
%first argument to {\tt vector-ref} must be a vector.  The following naming
%conventions imply type restrictions:
Для краткости в работе выполняется соглашение, что если имя параметра является также и
именем типа, соответствующий аргумент должен иметь именованный тип. Например, строка
заголовка для {\bfseries\tt vector-ref}, приведённая выше, предписывает, что первый аргумент
{\bfseries\tt vector-ref} должен быть вектором. Следующие соглашения именования
подразумевают ограничения типа:\vspace{1mm}

%\texonly\begin{center}\endtexonly
%  \begin{tabular}{ll}
%    \var{obj}&any object\\
%    \var{z}&complex number object\\
%    \var{x}&real number object\\
%    \var{y}&real number object\\
%    \var{q}&rational number object\\
%    \var{n}&integer object\\
%    \var{k}&exact non-negative integer object\\
%    \var{bool}&boolean (\schfalse{} or \schtrue{})\\
%    \var{octet}&exact integer object in $\{0, \ldots, 255\}$\\
%    \var{byte}&exact integer object in $\{-128, \ldots, 127\}$\\
%    \var{char}&character (see section~\ref{charactersection})\\
%    \var{pair}&pair (see section~\ref{listsection})\\
%    \var{vector}&vector (see section~\ref{vectorsection})\\
%    \var{string}&string (see section~\ref{stringsection})\\
%    \var{condition}&condition (see library section~\extref{lib:conditionssection}{Conditions})\\
%    \var{bytevector}&bytevector (see library chapter~\extref{lib:bytevectorschapter}{Bytevectors})\\
%    \var{proc}&procedure (see section~\ref{proceduressection})
%  \end{tabular}
%\texonly\end{center}\endtexonly
\texonly\begin{center}\endtexonly
  \begin{tabular}{ll}
    \var{obj}&любой объект\\
    \var{z}&комплексный числовой объект\\
    \var{x}&действительный числовой объект\\
    \var{y}&действительный числовой объект\\
    \var{q}&рациональный числовой объект\\
    \var{n}&целый объект\\
    \var{k}&точный неотрицательный целый объект\\
    \var{bool}&булево значение ({\bfseries\schfalse{}} или \bfseries\schtrue{})\\
    \var{octet}&точный целый объект диапазона $\{0, \ldots, 255\}$\\
    \var{byte}&точный целый объект диапазона $\{-128, \ldots, 127\}$\\
    \var{char}&символ (см. секцию~\ref{charactersection})\\
    \var{pair}&пара (см. секцию~\ref{listsection})\\
    \var{vector}&вектор (см. секцию~\ref{vectorsection})\\
    \var{string}&строка (см. секцию~\ref{stringsection})\\
    \var{condition}&условие (см. библиотечную секцию~\extref{lib:conditionssection}{Conditions})\\
    \var{bytevector}&байт-вектор (см. библиотецный раздел~\extref{lib:bytevectorschapter}{Bytevectors})\\
    \var{proc}&процедура (см. секцию~\ref{proceduressection})
  \end{tabular}
\texonly\end{center}\endtexonly

%Other type restrictions are expressed through parameter-naming
%conventions that are described in specific chapters.  For example,
%library chapter~\extref{lib:numberchapter}{Arithmetic} uses a number of special
%parameter variables for the various subsets of the numbers.
Другие ограничения типа выражаются посредством соглашений именований параметров, описанных в
конкретных главах. Например, в библиотечной главе~\extref{lib:numberchapter}{Arithmetic}
используется множество специальных переменных параметра для различных подмножеств чисел.\vspace{1.5mm}

%With the listed type restrictions, it is the programmer's responsibility to
%ensure that the corresponding argument is of the specified type.
%It is the implementation's responsibility to check for
%that type.
При перечисленных ограничениях типа обязанностью программиста является гарантия, что
соответствующий аргумент имеет указанный тип. Проверка этого типа является обязанностью
реализации.\vspace{1.6mm}

%A parameter called \var{list} means that it is the
%programmer's responsibility to pass an argument that is a list (see
%section~\ref{listsection}).  It is the implementation's responsibility
%to check that the argument is appropriately structured for the
%operation to perform its function, to the extent that this is possible
%and reasonable.  The implementation must at least check that the
%argument is either an empty list or a pair.
Параметр, называемый \var{list}, означает, что обязанностью программиста является передача
аргумента, который является списком (см. секцию~\ref{listsection}). Обязанностью реализации
является проверка, что аргумент должным образом структурирован для операции, чтобы выполнить её
функцию, до степени, что это возможно и разумно. Реализация должна по крайней мере проверить,
что аргумент является или пустым списком или парой.\vspace{1.6mm}

%Descriptions of procedures may express other restrictions on the
%arguments of a procedure.  Typically, such a restriction is formulated
%as a phrase of the form ``\var{x} must be a \ldots'' (or otherwise
%using the word ``must'').
Описания процедур могут выражать другие ограничения аргументов процедур. Обычно такое
ограничение формулируется фразой вида ``\var{x} должна быть \ldots'' (или иначе
с помощью слова ``должна'').\vspace{1.6mm}

%\section{Implementation responsibilities}
\section{Обязанности реализации}\vspace{1.6mm}

%In addition to the restrictions implied by naming conventions, an
%entry may list additional explicit restrictions.
%These explicit restrictions usually describe both the
%programmer's responsibilities, who must ensure that the subforms of a
%form are appropriate, or that an appropriate
%argument is passed, and the implementation's responsibilities, which
%must check that subform adheres to the specified restrictions (if
%macro expansion terminates), or if the argument is appropriate.  A description
%may explicitly list the implementation's responsibilities for some
%arguments or subforms in a paragraph labeled ``\textit{Implementation
%  responsibilities}''.  In this case, the responsibilities specified
%for these subforms or arguments in the rest of the description are only for the
%programmer.  A paragraph describing implementation responsibility does not
%affect the implementation's responsibilities for checking subforms or arguments not
%mentioned in the paragraph.
Кроме ограничений, подразумеваемых соглашениями именования, в записи могут перечисляться
дополнительные явные ограничения. Эти явные ограничения обычно описывают и обязанности
программиста, который должен гарантировать, что подформы формы являются соответствующими, или
что передаётся соответствующий аргумент, и обязанности реализаций, которые должны проверить, что
подформа соблюдает указанные ограничения (если разворачивание макросов завершено), или если
аргумент является соответствующим. В описании могут явно перечисляться обязанности реализации
для некоторых аргументов или подформ в параграфе, помеченном ``\textit {Обязанности
  реализации}''. В этом случае обязанности, указанные для этих подформ или аргументов в
остальной части описания, касаются только программиста. Параграф, описывающий обязанность
реализации, не затрагивает обязанности реализации по проверке подформ или аргументов, не
упомянутых в параграфе.\vspace{1.6mm}

%\section{Other kinds of entries}
\section{Другие виды записей}\vspace{1.6mm}

%If \var{category} is something other than ``syntax'' and
%``procedure'', then the entry describes a non-procedural value, and
%the \var{category} describes the type of that value.  The header line
Если \var{category} -- нечто иное, чем ``syntax'' и ``procedure'', запись описывает
непроцедурное значение, и \var{category} описывает тип этого значения. Строка
заголовка\vspace{1.6mm}

\noindent\rvproto{\bfseries\&who}{condition type}\\
%indicates that {\cf\&who} is a condition type.  The header line
указывает, что {\bfseries\cf\&who} является типом условий. Строка заголовка

\noindent\rvproto{\bfseries unquote}{auxiliary syntax}\\
%indicates that {\cf unquote} is a syntax binding that may occur
%only as part of specific surrounding expressions.  Any use as an
%independent syntactic construct or identifier is a syntax violation.
%As with ``\exprtype'' entries, some ``auxiliary syntax'' entries  carry a
%suffix ({\cf expand}), specifying that the syntactic keyword of the
%construct is exported with level $1$.
%\section{Equivalent entries}
указывает, что {\bfseries\cf unquote} является синтаксической привязкой, которая может
фигурировать только как часть конкретных окружающих выражений. Любое применение
в качестве независимой синтаксической конструкции или идентификатора является нарушением
синтаксиса. Как и в случае с записями ``\exprtype'', некоторые записи ``вспомогательного синтаксиса''
содержат суффикс ({\bfseries\cf expand}), что указывает на экспорт синтаксического ключевого слова
конструкции с уровнем $1$.\vspace{1mm}

\section {Эквивалентные записи}\vspace{1mm}

%The description of an entry occasionally states that it is \textit{the
%  same} as another entry.  This means that both entries are
%equivalent.  Specifically, it means that if both entries have the same
%name and are thus exported from different libraries, the entries from
%both libraries can be imported under the same name without conflict.
В описании записи иногда утверждается, что она \textit{аналогична} другой записи. Это
означает, что обе записи эквивалентны. Реально это означает, что, если обе записи имеют
одинаковые имена и при этом экспортируются из различных библиотек, записи
могут импортироваться из обеих библиотек под одинаковыми именами без конфликта.\vspace{1mm}

%\section{Evaluation examples}
\section{Примеры вычислений}\vspace{1mm}

%The symbol ``\evalsto'' used in program examples can be read
%``evaluates to''.  For example,
Символ ``\evalsto'' применяется в примерах программ и может быть прочитан ``вычисляется
как''. Например,\vspace{1mm}

\begin{scheme}
\bfseries(* 5 8)      \ev\bfseries  40%
\end{scheme}\vspace{1mm}

%means that the expression {\tt(* 5 8)} evaluates to the object {\tt 40}.
%Or, more precisely:  the expression given by the sequence of characters
%``{\tt(* 5 8)}'' evaluates, in an environment that imports the relevant library, to an object
%that may be represented externally by the sequence of characters ``{\tt
%40}''.  See section~\ref{datumsyntaxsection} for a discussion of external
%representations of objects.
означает, что выражение {\bfseries\tt (* 5 8)} вычисляется как объект {\bfseries\tt 40}. Или,
точнее: заданное последовательностью символов выражение ``{\bfseries\tt (* 5 8)}'' в окружении,
импортирующем релевантную библиотеку, вычисляется как объект, который может быть представлен
внешне последовательностью символов ``{\bfseries\tt 40}''. См. секцию ~\ref{datumsyntaxsection},
где рассматриваются внешние представления объектов.\vspace{1mm}

%The ``\evalsto'' symbol is also used when the evaluation of an
%expression causes a violation.  For example,
Символ ``\evalsto'' также применяется, если вычисление выражения приводит к нарушению. Например,\vspace{1mm}

\begin{scheme}
\bfseries(integer->char \sharpsign{}xD800) \xev \exception{\bfseries\&assertion}%
\end{scheme}\vspace{1mm}
%
%means that the evaluation of the expression {\cf (integer->char
%  \sharpsign{}xD800)} must raise an exception with condition type
%{\cf\&assertion}.
означает, что вычисление выражения {\bfseries\cf (integer->char \sharpsign{}xD800)}
должно возбудить исключение с типом состояния {\bfseries\cf\&assertion}.\vspace{1mm}

%Moreover, the ``\evalsto'' symbol is also used to explicitly say that
%the value of an expression in unspecified.  For example:
Кроме того, символ ``\evalsto'' также используется для явного указания, что значение выражения
не определено. Например:\vspace{1mm}
%
\begin{scheme}
\bfseries(eqv? "" "")             \ev  \unspecified%
\end{scheme}\vspace{1mm}

%Mostly, examples merely illustrate the behavior specified in the
%entry.  In some cases, however, they disambiguate otherwise ambiguous
%specifications and are thus normative.  Note that, in some cases,
%specifically in the case of inexact number objects, the return value is only
%specified conditionally or approximately.  For example:
В основном примеры просто демонстрируют определённую записью функциональность. Однако в
некоторых случаях они устраняют неоднозначность двусмысленных в ином случае спецификаций и,
таким образом, нормативны. Необходимо отметить, что в некоторых случаях, как правило в случае
неточных числовых объектов, возвращаемое значение указывается только условно или
приблизительно. Например:\vspace{1mm}
%
\begin{scheme}
\bfseries(atan -inf.0)                  \lev \textbf{-1.5707963267948965} ; \textrm{approximately}%
\end{scheme}

%\newpage

%\section{Naming conventions}
\section{Соглашения по именованию}

%By convention, the names of procedures that store values into previously
%allocated locations (see section~\ref{storagemodel}) usually end in
%``\ide{!}''.
В соответствии с соглашением имена процедур, сохраняющих значения в отведённых ранее
областях памяти (см. секцию ~\ref{storagemodel}), обычно завершаются ``\ide{\bfseries !}''.

%By convention, ``\ide{->}'' appears within the names of procedures that
%take an object of one type and return an analogous object of another type.
%For example, {\cf list->vector} takes a list and returns a vector whose
%elements are the same as those of the list.
В соответствии с соглашением ``\ide{\bfseries ->}'' присутствует внутри имён процедур,
принимающих объект одного типа и возвращающих аналогичный объект другого типа. Например,
{\cf\bfseries list->vector} принимает список и возвращает вектор, элементы которого совпадают с
таковыми из списка.

%By convention, the names of predicates---procedures that always return
%a boolean value---end in ``\ide{?}'' when the name contains any
%letters; otherwise, the predicate's name does not end with a question
%mark.
В соответствии с соглашением имена предикатов --- процедур, всегда возвращающих
булево значение --- завершаются ``\ide{\bfseries ?}'', когда имя содержит произвольные буквы; в
противном случае имя предиката не завершается вопросительным знаком.

%By convention, the components of compound names are separated by ``\ide{-}''
%In particular, prefixes that are actual words or can be pronounced as
%though they were actual words are followed by a hyphen, except when
%the first character following the hyphen would be something other than
%a letter, in which case the hyphen is omitted.  Short,
%unpronounceable prefixes (``\ide{fx}'' and ``\ide{fl}'') are not
%followed by a hyphen.
В соответствии с соглашением компоненты составных имён разделяются ``\ide{\bfseries -}''. В
частности, после приставок, являющихся фактическими словами или которые могут быть произнесены,
как если бы они были фактическими словами, ставится дефис, кроме тех случаев, когда первый
символ после дефиса не является буквой, в этом случае дефис не ставится. После коротких,
труднопроизносимых приставок (``\ide{\bfseries fx}'' и ``\ide{\bfseries fl}'') дефис не
ставится.

%By convention, the names of condition types start with
В соответствии с соглашением имена типов состояния начинаются с
``{\bfseries\cf\&}''\index{&@\texttt{\&}}.

%%% Local Variables:
%%% mode: latex
%%% TeX-master: "r6rs"
%%% End:
 \par
%\chapter{Libraries}
\chapter{Библиотеки}
\label{librarychapter}
\mainindex{library}
%Libraries are parts of a program that can be distributed
%independently.
%The library system supports macro definitions within libraries,
%macro exports, and distinguishes the phases in which definitions
%and imports are needed.  This chapter defines the notation for
%libraries and a semantics for library expansion and execution.
Библиотеки являются частями программы, которые могут поставляться независимо. Система библиотек
поддерживает макроопределения внутри библиотек, макроэкспорт, а также различает фазы, в которых
необходимы определения и импорт. В данной главе описана нотация библиотек и семантика расширения
и реализации библиотек.

%\section{Library form}
\section{Библиотечная форма}
\label{librarysyntaxsection}

%A library definition must have the following form:\mainschindex{library}\mainschindex{import}\mainschindex{export}
Определение библиотеки имеет следующую форму:\mainschindex{library}\mainschindex{import}\mainschindex{export}

\begin{scheme}
(\textbf{library} \hyper{library~name}
  (\textbf{export} \hyper{export~spec} \ldots)
  (\textbf{import} \hyper{import~spec} \ldots)
  \hyper{library~body})%
\end{scheme}

%A library declaration contains the following elements:
Объявление библиотеки содержит следующие элементы:

\begin{itemize}
%\item The \hyper{library~name} specifies the name of the library
%  (possibly with version).
\item \hyper{library~name} указывает имя библиотеки (возможно с версией).
%\item The {\cf export} subform specifies a list of exports, which name
%  a subset of the bindings defined within or imported into the
%  library.
\item Подформа {\cf\bfseries export} указывает список экспорта, именующий подмножество
  привязок, определённых внутри или импортированных в библиотеку.
%\item The {\cf import} subform specifies the imported bindings as a
%  list of import dependencies, where each dependency specifies:
\item Подформа {\cf\bfseries import} указывает импортированные привязки в виде списка
  зависимостей импорта, где каждая зависимость указывает:
\begin{itemize}
%\item the imported library's name, and, optionally, constraints on its
%  version,
\item имя импортируемой библиотеки и, при необходимости, ограничения её версии,
%\item the relevant levels, e.g., expand or run time (see
%  section~\ref{phasessection}, and
\item релевантные уровни, например, время разворачивания или выполнения (см
  секцию~\ref{phasessection}, и
%\item the subset of the library's exports to make available within the
%      importing library, and the local names to use within the importing
%      library for each of the library's exports.
\item подмножество библиотечных экспортов, чтобы сделать доступный внутри библиотеки
  импортирования, и локальных имён, для использования внутри библиотеки
  импортирования для каждого из экспортов библиотеки.
\end{itemize}
%\item The \hyper{library body} is the library body, consisting of a
%  sequence of definitions followed by a sequence of expressions.  The
%  definitions may be both for local (unexported) and exported
%  bindings, and the expressions are initialization expressions to be evaluated
%  for their effects.
\item \hyper{library body} - тело библиотеки, состоящее из последовательности определений
  и следующей за ней последовательности выражений. Определения могут быть как для локальных
  (неэкспортируемых), так и для экспортируемых связываний, а выражения являются
  инициализирующими выражениями, которые будут вычисляться для их эффектов.
\end{itemize}

%An identifier can be imported with the same local name from two or
%more libraries or for two levels from the same library only if the
%binding exported by each library is the same (i.e., the binding is
%defined in one library, and it arrives through the imports only by
%exporting and re-exporting).  Otherwise, no identifier can be imported
%multiple times, defined multiple times, or both defined and imported.
%No identifiers are visible within a library except for those
%explicitly imported into the library or defined within the library.
Идентификатор может импортироваться с тем же локальным именем из двух или более библиотек, или
для двух уровней из той же библиотеки только в случае, если привязки, экспортируемые каждой
библиотекой, тождественны (то есть, связывание определено в одной библиотеке, и оно поступает
только через импорт, экспортируя и реэкспортируя). В противном случае идентификатор не может
быть импортирован несколько раз, определён несколько раз, или определён и
импортирован. Идентификаторы не видимы внутри библиотеки, за исключением явно импортированных в
библиотеку или определённых внутри библиотеки.

%A \hyper{library name} uniquely identifies a library within an
%implementation, and is globally visible in the {\cf import} clauses
%(see below) of all other libraries within an implementation.
%A \hyper{library name} has the following form:
\hyper{library name} однозначно идентифицирует библиотеку внутри реализации и является глобально
видимым в разделах {\cf\bfseries import} (см. ниже) всех других библиотек внутри
реализации. \hyper{library name} имеет следующую форму:

\begin{scheme}
(\hyperi{identifier} \hyperii{identifier} \ldots \hyper{version})%
\end{scheme}

%where \hyper{version} is empty or has the following form:
где \hyper{version} является пустым или принимает следующую форму:
%
\begin{scheme}
(\hyper{sub-version} \ldots)%
\end{scheme}

%Each \hyper{sub-version} must represent an exact nonnegative integer object.
%An empty \hyper{version} is equivalent to {\cf ()}.
Каждый \hyper{sub-version} должен представлять точный неотрицательный целый объект.
Пустой \hyper{version} эквивалентен {\cf ()}.

%An \hyper{export~spec} names a set of imported and locally defined bindings to
%be exported, possibly with different
%external names.  An \hyper{export~spec} must have one of the
%following forms:
\hyper{export~spec} именует совокупность импортированных и локально определённых привязок
для экспортирования, возможно с другими внешними именами. \hyper{export~spec} должен иметь
одну из следующих форм:

\begin{scheme}
\hyper{identifier}
(\textbf{rename} (\hyperi{identifier} \hyperii{identifier}) \ldots)%
\end{scheme}

%In an \hyper{export~spec}, an \hyper{identifier} names a single binding defined
%within or imported into the library, where the external name for the export is
%the same as the name of the binding within the library.
%A {\cf rename} spec exports the binding named by
%\hyperi{identifier} in each {\cf (\hyperi{identifier}
%  \hyperii{identifier})} pairing, using \hyperii{identifier} as the
%external name.
В \hyper{export~spec}, \hyper{identifier} именует одиночную привязку, определённую внутри или
импортированную в библиотеку, причём внешнее имя экспорта совпадает с именем привязки внутри
библиотеки. Спецификация {\cf\bfseries rename} экспортирует привязку с именем
\hyperi{identifier} в каждой паре {\cf (\hyperi{identifier} \hyperii{identifier})},
используя \hyperii{identifier} в качестве внешнего имени.

%Each \hyper{import~spec} specifies a set of bindings to be imported into
%the library, the levels at which they are to be available, and the local
%names by which they are to be known.  An \hyper{import spec} must
%be one of the following:
Каждый \hyper{import~spec} указывает множество привязок, импортируемых в
библиотеку, уровни, на которых они должны быть доступны и локальные имена, которыми
они должны быть названы. \hyper{import spec} должен быть одним из следующего:

\newpage

\begin{scheme}
\hyper{import set}
(\textbf{for} \hyper{import~set} \hyper{import~level} \ldots)%
\end{scheme}

%An \hyper{import level}  is one of the following:
\hyper{import level} является одним из следующего:
\begin{scheme}
\textbf{run}
\textbf{expand}
\textbf{(meta} \hyper{level}\textbf{)}%
\end{scheme}

%where \hyper{level} represents an exact integer object.
Где \hyper{level} представляет точный целый объект.

%As an \hyper{import level}, {\cf run} is an abbreviation for {\cf
%  (meta 0)}, and {\cf expand} is an abbreviation for {\cf (meta 1)}.
%Levels and phases are discussed in section~\ref{phasessection}.
Как \hyper{import level}, {\cf \bfseries run} является сокращением для {\cf\bfseries (meta 0)},
а {\cf\bfseries expand} - сокращением для {\cf\bfseries (meta 1)}. Уровни и фазы описаны в
секции~\ref{phasessection}.

%An \hyper{import~set} names a set of bindings from another library and
%possibly specifies local names for the imported bindings.  It must be
%one of the following:
\hyper{import~set} именует множество привязок из другой библиотеки и, возможно, определяет
локальные имена для импортированных привязок. Он должен быть одним из следующего:

\begin{scheme}
\hyper{library~reference}
\textbf{(library} \hyper{library~reference}\textbf{)}
\textbf{(only} \hyper{import~set} \hyper{identifier} \ldots\textbf{)}
\textbf{(except} \hyper{import~set} \hyper{identifier} \ldots\textbf{)}
\textbf{(prefix} \hyper{import~set} \hyper{identifier}\textbf{)}
\textbf{(rename} \hyper{import~set} \textbf{(}\hyperi{identifier} \hyperii{identifier}\textbf{)} \ldots\textbf{)}%
\end{scheme}

%A \hyper{library~reference} identifies a library by its
%name and optionally by its version.  It has one of the following forms:
\hyper{library~reference} идентифицирует библиотеку её именем и,
произвольно, её версией. Он имеет одну из следующих форм:

\begin{scheme}
\textbf{(}\hyperi{identifier} \hyperii{identifier} \ldots\textbf{)}
\textbf{(}\hyperi{identifier} \hyperii{identifier} \ldots \hyper{version~reference}\textbf{)}%
\end{scheme}

%A \hyper{library~reference} whose first \hyper{identifier} is
%{\cf for}, {\cf library}, {\cf only}, {\cf except}, {\cf prefix}, or {\cf rename} is
%permitted only within a {\cf library} \hyper{import~set}.
%The \hyper{import~set} {\cf (library \hyper{library~reference})} is
%otherwise equivalent to \hyper{library~reference}.
\hyper{library~reference}, чьим первым \hyper{identifier} является {\cf\bfseries for}, {\cf\bfseries
library}, {\cf\bfseries only}, {\cf\bfseries except}, {\cf\bfseries prefix} или {\cf\bfseries
rename}, разрешается только внутри {\cf\bfseries library}
\hyper{import~set}. \hyper{import~set} {\cf\textbf{(library}
  \hyper{library~reference}\textbf{)}} в противном случае
эквивалентен \hyper{library~reference}.

%A \hyper{library~reference} with no \hyper{version~reference}
%(first form above) is equivalent to a \hyper{library~reference} with a
%\hyper{version~reference} of {\cf ()}.
\hyper{library~reference} без \hyper{version~reference} (первая форма выше) эквивалентен
\hyper{library~reference} с \hyper{version~reference} {\cf\bfseries ()}.

%A \hyper{version~reference} specifies a set of \hyper{version}s that
%it matches.  The \hyper{library~reference} identifies all libraries of
%the same name and whose version is matched by the
%\hyper{version~reference}.  A \hyper{version~reference} has
%the following form:
\hyper{version~reference} указывает множество \hyper{version}, которым он
соответствует. \hyper{library~reference} идентифицирует все библиотеки того же самого
имени и чья версия соответствует \hyper{version~reference}. \hyper{version~reference}
имеет следующую форму:
%
\begin{scheme}
\textbf{(}\hyperi{sub-version reference} \ldots \hypern{sub-version reference}\textbf{)}
\textbf{(and} \hyper{version reference} \ldots\textbf{)}
\textbf{(or} \hyper{version reference} \ldots\textbf{)}
\textbf{(not} \hyper{version reference}\textbf{)}%
\end{scheme}
%
%A \hyper{version reference} of the first form matches a \hyper{version}
%with at least $n$ elements, whose \hyper{sub-version reference}s match
%the corresponding \hyper{sub-version}s.  An {\cf and} \hyper{version
%  reference} matches a version if all \hyper{version references}
%following the {\cf and} match it.  Correspondingly, an {\cf
%  or} \hyper{version reference} matches a version if one of
%\hyper{version references} following the {\cf or} matches it,
%and a {\cf not} \hyper{version reference} matches a version if the
%\hyper{version reference} following it does not match it.
\hyper{version reference} первой формы соответствует \hyper{version} с по крайней мере $n$
элементы, \hyper{sub-version reference} которых соответствует
соответств. \hyper{sub-version}. {\cf\bfseries and} \hyper{version reference} соответствует
версии, если все \hyper{version references} после {\cf\bfseries and} соответствуют
этому. Соответственно, {\cf\bfseries or} \hyper{version reference} соответствует версии, если
один из \hyper{version references} после {\cf\bfseries or} соответствуют этому, и {\cf\bfseries
  not} \hyper{version reference} соответствует версии, если \hyper{version references} после
этого не соответствует этому.

%A \hyper{sub-version reference} has one of the following forms:
\hyper{sub-version reference} имеет одну из следующих форм:

\begin{scheme}
\hyper{sub-version}
\textbf{(>=} \hyper{sub-version}\textbf{)}
\textbf{(<=} \hyper{sub-version}\textbf{)}
\textbf{(and} \hyper{sub-version~reference} \ldots\textbf{)}
\textbf{(or} \hyper{sub-version~reference} \ldots\textbf{)}
\textbf{(not} \hyper{sub-version~reference}\textbf{)}%
\end{scheme}

%A \hyper{sub-version reference} of the first form matches a
%\hyper{sub-version} if it is equal to it.  A {\cf >=}
%\hyper{sub-version reference} of the first form matches a sub-version
%if it is greater or equal to the \hyper{sub-version} following it;
%analogously for {\cf <=}.  An {\cf and} \hyper{sub-version reference}
%matches a sub-version if all of the subsequent \hyper{sub-version
%  reference}s match it.  Correspondingly, an {\cf or}
%\hyper{sub-version reference} matches a sub-version if one of the
%subsequent \hyper{sub-version reference}s matches it, and a {\cf not}
%\hyper{sub-version reference} matches a sub-version if the subsequent
%\hyper{sub-version reference} does not match it.
\hyper{sub-version reference} первой формы соответствует \hyper{sub-version}, если они равны.
{\cf\bfseries >=} \hyper{sub-version reference} первой формы соответствует sub-version, если оно
больше или равно \hyper{sub-version} после него; аналогично для {\cf\bfseries
  <=}. {\cf\bfseries and} \hyper{sub-version reference} соответствует sub-version, если все
последующие \hyper{sub-version reference} соответствуют ему. Соответственно, {\cf\bfseries or}
\hyper{sub-version reference} соответствует sub-version, если один из последующих \hyper{sub-version
reference} соответствует ему, и {\cf\bfseries not} \hyper{sub-version reference} соответствует
sub-version, если последующий \hyper{sub-version reference} не соответствует ему.

%Examples:
Примеры:

\texonly\begin{center}\endtexonly
  \begin{tabular}{lll}
    version reference & version & match?
    \\
    {\cf ()} & {\cf (1)} & yes\\
    {\cf (1)} & {\cf (1)} & yes\\
    {\cf (1)} & {\cf (2)} & no\\
    {\cf (2 3)} & {\cf (2)} & no\\
    {\cf (2 3)} & {\cf (2 3)} & yes\\
    {\cf (2 3)} & {\cf (2 3 5)} & yes\\
    {\cf (or (1 (>= 1)) (2))} & {\cf (2)} & yes\\
    {\cf (or (1 (>= 1)) (2))} & {\cf (1 1)} & yes\\
    {\cf (or (1 (>= 1)) (2))} & {\cf (1 0)} & no\\
    {\cf ((or 1 2 3))} & {\cf (1)} & yes\\
    {\cf ((or 1 2 3))} & {\cf (2)} & yes\\
    {\cf ((or 1 2 3))} & {\cf (3)} & yes\\
    {\cf ((or 1 2 3))} & {\cf (4)} & no
  \end{tabular}
\texonly\end{center}\endtexonly

%When more than one library is identified by a library reference, the
%choice of libraries is determined in some implementation-dependent manner.
Если при обращении к библиотеке идентифицировано более одной библиотеки, выбор библиотеки
определяется неким зависимым от реализации методом.

%To avoid problems such as incompatible types and replicated state,
%implementations should prohibit the two libraries whose library names
%consist of the same sequence of identifiers but whose versions do not
%match to co-exist in the same program.
Во избежании проблем, таких, как несовместимость типов и дублирование состояний, реализация
должна запретить сосуществование в одной программе двух библиотек, библиотечные имена которых
состоят из одинаковой последовательности идентификаторов, но версии которых являются
несоответствующими.

%By default, all of an imported library's exported bindings are made
%visible within an importing library using the names given to the bindings
%by the imported library.
%The precise set of bindings to be imported and the names of those
%bindings can be adjusted with the {\cf only}, {\cf except},
%{\cf prefix}, and {\cf rename} forms as described below.
По умолчанию все экспортируемые привязки из импортируемой библиотеки предполагаются видимыми
внутри импортированной библиотеки с именами, присвоенными привязкам в импортируемой
библиотеке. Точный набор импортитуемых привязок и имён этих привязок может быть установлен с
помощью форм {\cf\bfseries only}, {\cf\bfseries except}, {\cf\bfseries prefix} и {\cf\bfseries
rename} как описано ниже.

\begin{itemize}
%\item An {\cf only} form produces a subset of the bindings from another
%\hyper{import~set}, including only the listed
%\hyper{identifier}s.
%The included \hyper{identifier}s must be in
%the original \hyper{import~set}.
\item Форма {\cf\bfseries only} порождает подмножество привязок из иного \hyper{import~set},
  содержащее только перечисленные \hyper{identifier}. Включенные \hyper{identifier} должны
  существовать в исходном \hyper{import~set}.
\newpage
%\item An {\cf except} form produces a subset of the bindings from another
%\hyper{import~set}, including all but the listed
%\hyper{identifier}s.
%All of the excluded \hyper{identifier}s must be in
%the original \hyper{import~set}.
\item Форма {\cf\bfseries except} порождает подмножество привязок из другого \hyper{import~set},
  содержащее все \hyper{identifier}, кроме перечисленных. Включенные \hyper{identifier} должны
  существовать в исходном \hyper{import~set}.
%\item A {\cf prefix} form adds the \hyper{identifier} prefix to each
%name from another \hyper{import~set}.
\item Форма {\cf\bfseries prefix} форма добавляет приставку \hyper{identifier} к каждому
  имени из другого \hyper{import~set}.
%\item A {\cf rename} form, {\cf (rename (\hyperi{identifier} \hyperii{identifier}) \ldots)},
%removes the bindings for {\cf \hyperi{identifier} \ldots} to form an
%intermediate \hyper{import~set}, then adds the bindings back for the
%corresponding {\cf \hyperii{identifier} \ldots} to form the final
%\hyper{import~set}.
%Each \hyperi{identifier} must be in the original \hyper{import~set},
%each \hyperii{identifier} must not be in the intermediate \hyper{import~set},
%and the \hyperii{identifier}s must be distinct.
\item Форма {\cf\bfseries rename}, {\cf \textbf{(rename (}\hyperi{identifier}
  \hyperii{identifier}\textbf{)} \ldots\textbf{)}}, удаляет привязки для {\cf
  \hyperi{identifier} \ldots} для формирования промежуточного \hyper{import~set}, затем
  добавляет привязки назад для соответствующего {\cf \hyperii{identifier} \ldots} для
  формирования конечного \hyper{import~set}. Каждый \hyperi{identifier} должен быть в исходном
  \hyper{import~set}, каждый \hyperii{identifier} не должен быть в промежуточном
  \hyper{import~set}, а все \hyperii{identifier} должны быть разными.
\end{itemize}
%It is a syntax violation if a constraint given above is not met.
Невыполнение приведённого выше ограничения является синтаксическим нарушением.\vspace{1mm}

\label{librarybodysection}
%The \hyper{library~body} of a {\cf library} form consists of forms
%that are classified as
%\textit{definitions}\mainindex{definition} or
%\textit{expressions}\mainindex{expression}.  Which forms belong to
%which class depends on the imported libraries and the result of
%expansion---see chapter~\ref{expansionchapter}.  Generally, forms that
%are not
%definitions (see section~\ref{defines} for definitions available
%through the base library) are expressions.
\hyper{library~body} формы {\cf\bfseries library} состоит из форм, классифицируемых как
\textit{определения}\mainindex{definition} или
\textit{выражения}\mainindex{expression}. Принадлежность конкретной формы конкретному классу
зависит от импортированных библиотек и результата
расширения---см. главу~\ref{expansionchapter}. В общем случае формы, не являющиеся
определениями (о доступных посредством базовой
библиотеки определениях см. секцию ~\ref {определяют}), являются выражениями.\vspace{1mm}

%A \hyper{library~body} is like a \hyper{body} (see section~\ref{bodiessection}) except that
%a \hyper{library~body}s need not include any expressions.  It must
%have the following form:
\hyper{library~body} похоже на \hyper{body} (см. секцию ~\ref{bodiessection}) за исключением
того, что в \hyper{library~body} не обязаны содержаться выражения. Оно должно иметь следующую
форму:\vspace{1mm}

\begin{scheme}
\hyper{definition} \ldots \hyper{expression} \ldots%
\end{scheme}\vspace{1mm}

%When {\cf begin}, {\cf let-syntax}, or {\cf letrec-syntax} forms
%occur in a top-level body prior to the first
%expression, they are spliced into the body; see section~\ref{begin}.
%Some or all of the body, including portions wrapped in {\cf begin},
%{\cf let-syntax}, or {\cf letrec-syntax}
%forms, may be specified by a syntactic abstraction
%(see section~\ref{macrosection}).
Если формы {\cf\bfseries begin}, {\cf\bfseries let-syntax} или {\cf\bfseries letrec-syntax}
находятся в теле верхнего уровня до первого выражения, они встраиваются в тело; см.
секцию ~\ref{begin}. Всё тело или его часть, включая части, обёрнутые в формы {\cf\bfseries begin},
{\cf\bfseries let-syntax} или {\cf\bfseries letrec-syntax}, может быть определено
синтаксической абстракцией (см. секцию~\ref{macrosection}).\vspace{1mm}

%The transformer expressions and bindings are evaluated and created
%from left to right, as described in chapter~\ref{expansionchapter}.
%The expressions of variable definitions are evaluated
%from left to right, as if in an implicit {\cf letrec*},
%and the body expressions are also evaluated from left to right
%after the expressions of the variable definitions.
%A fresh location is created for each exported variable and initialized
%to the value of its local counterpart.
%The effect of returning twice to the continuation of the last body
%expression is unspecified.
Преобразующие выражения и привязки вычисляются и создаются слева направо, как описано в главе
~\ref{expansionchapter}. Выражения определений переменных вычисляются слева направо, как будто в
неявном {\cf\bfseries letrec*}, и выражения тела также вычисляются слева направо после выражений
определений переменных. Для каждой экспортируемой переменной создаётся новая ячейка и
инициализируется значением её локальной копии. Эффект возвращения дважды к продолжению
последнего выражения тела не определён.\vspace{1mm}

\begin{note}
%The names {\cf library}, {\cf export}, {\cf import},
%{\cf for}, {\cf run}, {\cf expand}, {\cf meta},
%{\cf import}, {\cf export}, {\cf only}, {\cf except}, {\cf
%  prefix}, {\cf rename}, {\cf and}, {\cf or}, {\cf not}, {\cf >=}, and {\cf <=}
%appearing in the library syntax are part of the
%syntax and are not reserved, i.e., the same names can be used for other
%purposes within the library or even exported from or imported
%into a library with different meanings, without affecting their
%use in the {\cf library} form.
Имена {\cf\bfseries library}, {\cf\bfseries export}, {\cf\bfseries import}, {\cf\bfseries for},
{\cf\bfseries run}, {\cf\bfseries expand}, {\cf\bfseries meta}, {\cf\bfseries import},
{\cf\bfseries export}, {\cf\bfseries only}, {\cf\bfseries except}, {\cf\bfseries prefix},
{\cf\bfseries rename}, {\cf\bfseries and}, {\cf\bfseries or}, {\cf\bfseries not}, {\cf\bfseries
  >=} и {\cf\bfseries <=}, входящие в библиотечный синтаксис, являются частью синтаксиса и не
зарезервированы, то есть, те же имена могут использоваться в других целях внутри
библиотеки или даже экспортироваться из библиотеки или импортироваться в библиотеку с другим
смысловым содержанием, не затрагивая их использование в форме {\cf\bfseries library}.
\end{note}

%Bindings defined with a library are not visible in code
%outside of the library, unless the bindings are explicitly exported from the
%library.
%An exported macro may, however, \emph{implicitly export} an otherwise
%unexported identifier defined within or imported into the library.
%That is, it may insert a reference to that identifier into the output code
%it produces.
Определённые с библиотекой привязки невидимы в коде вне библиотеки, если они явно не
экспортированы из библиотеки. Однако экспортируемый макрос может \emph{неявно экспортировать}
в иных случаях неэкспортируемый идентификатор, определённый в библиотеке или
импортированный в неё. Иными словами, он может вставить обращение к этому идентификатору
в генерируемый им выходной код.\vspace{-1.2mm}

\label{importsareimmutablesection}
%All explicitly exported variables are immutable in both the
%exporting and importing libraries.
%It is thus a syntax violation if an
%explicitly exported variable appears on the left-hand side of a {\cf set!}
%expression, either in the exporting or importing libraries.
Все явно экспортированные переменные являются неизменяемыми и в экспортируемых, и в импортируемых
библиотеках. Таким образом, наличие явно экспортированной переменной в левой части выражения
{\cf\bfseries set!} является нарушением синтаксиса как в экспортируемых, так и в импортируемых
библиотеках.\vspace{-1.2mm}

%All implicitly exported variables are also immutable in both the
%exporting and importing libraries.
%It is thus a syntax violation if a
%variable appears on the left-hand side of a {\cf set!}
%expression in any code produced by an exported macro outside of the
%library in which the variable is defined.
%It is also a syntax violation if a
%reference to an assigned variable appears in any code produced by
%an exported macro outside of the library in which the variable is defined,
%where an assigned variable is one that appears on the left-hand
%side of a {\cf set!} expression in the exporting library.
Все неявно экспортированные переменные также являются неизменяемыми и в экспортируемых, и в
импортируемых библиотеках. Таким образом, наличие переменной в левой части выражения
{\cf\bfseries set!} в любом коде, сгенерированном экспортируемым макросом вне библиотеки, в
которой определена переменная, является нарушением синтаксиса. Также является нарушением
синтаксиса наличие обращения к присваиваемой переменной в любом коде, сгенерированном
экспортируемым макросом вне библиотеки, в которой определена переменная, где присваиваемой
переменной является находящаяся в левой части выражения {\cf\bfseries set!} переменная в
экспортируемой библиотеке.\vspace{-1.2mm}

%All other variables defined within a library are mutable.
Все остальные определённые в библиотеке переменные являются изменяемыми.\vspace{-6mm}

%\section{Import and export levels}
\section{Уровни экспорта и импорта}\vspace{-3.8mm}
\label{phasessection}

%Expanding a library may require run-time information from another
%library.  For example, if a macro transformer calls a
%procedure from library $A$, then the library $A$ must be
%instantiated before expanding any use of the macro in library $B$.  Library $A$ may
%not be
%needed when library $B$ is eventually run as part of a program, or it
%may be needed for run time of library $B$, too.  The library
%mechanism distinguishes these times by phases, which are explained in
%this section.
При разворачивании библиотеки может потребоваться информация этапа выполнения из другой
библиотеки. Например, при вызове макропреобразователем процедуры из библиотеки $A$, библиотека
$A$ должна быть инстанцирована перед разворачиванием любого применения макроса в библиотеке
$B$. Библиотека $A$ в итоге может и не потребоваться при выполнении библиотеки $B$ как части
программы, или же она может потребоваться библиотеке $B$ также и для этапа выполнения.
Механизм библиотек характеризует эти этапы фазами, описанными в данной секции.\vspace{-1.2mm}

%Every library can be characterized by expand-time information (minimally,
%its imported libraries, a list of the exported keywords, a list of the
%exported variables, and code to evaluate the transformer expressions) and
%run-time information (minimally, code to evaluate the variable definition
%right-hand-side expressions, and code to evaluate the body expressions).
%The expand-time information must be available to expand references to
%any exported binding, and the run-time information must be available to
%evaluate references to any exported variable binding.
Каждая библиотека может характеризоваться информацией этапа разворачивания (как минимум, её
импортированные библиотеки, список экспортируемых ключевых слов, список экспортируемых
переменных и код вычисления преобразовательных выражений) и информацией этапа выполнения
(как минимум, код вычисления выражений правой части определений переменных и код
вычисления выражений тела). Информация этапа разворачивания должна быть доступна для раскрытия
обращений к любым экспортируемым привязкам, а информация этапа выполнения -
для вычисления обращений к любым экспортированным привязкам переменных.\vspace{-1.2mm}

\mainindex{phase}
%
%A \emph{phase} is a time at which the expressions within a library are
%evaluated.
%Within a library body, top-level expressions and
%the right-hand sides of {\cf define} forms are evaluated at run time,
%i.e., phase $0$, and the right-hand
%sides of {\cf define-syntax} forms are evaluated at expand time, i.e.,
%phase $1$.
%When {\cf define-syntax},
%{\cf let-syntax}, or {\cf letrec-syntax}
%forms appear within code evaluated at phase $n$, the right-hand sides
%are evaluated at phase $n+1$.
\emph{Фаза} -- это этап, в течение которого вычисляются выражения в библиотеке. Внутри тела
библиотеки выражения верхнего уровня и правые части форм {\cf\bfseries define} вычисляются на
этапе выполнения, то есть, в фазе $0$, а правые части форм {\cf\bfseries define-syntax} -- на этапе
разворачивания, то есть, в фазе $1$. Если формы {\cf\bfseries define-syntax}, {\cf\bfseries
  let-syntax} или {\cf\bfseries letrec-syntax} находятся внутри кода, вычисляемого в фазе $n$,
правые части вычисляются в фазе $n+1$.

%These phases are relative to the phase in which the library itself is
%used.
%An \defining{instance} of a library corresponds to an evaluation of its
%variable definitions and expressions in a particular phase relative to another
%library---a process called \defining{instantiation}.
%For example, if a top-level expression in a library $B$ refers to
%a variable export from another library $A$, then it refers to the export from an
%instance of $A$ at phase $0$ (relative to the phase of $B$).
%But if a phase $1$ expression within $B$ refers to the same binding from
%$A$, then it refers to the export from an instance of $A$ at phase $1$
%(relative to the phase of $B$).
Эти фазы связаны с фазой, в которой используется сама библиотека. \defining{Экземпляр}
библиотеки соответствует вычислению её определений переменных и выражений в конкретной фазе
относительно другой библиотеки---процессу, называемому \defining{инстанцированием}. Например,
если выражение верхнего уровня библиотеки $B$ обращается к переменной, экспортируемой из
другой библиотеки $A$, то оно обращается к экспорту из экземпляра $A$ в фазе $0$ (относительно
фазы $B$). Но если выражение фазы $1$ внутри $B$ обращается к тому же связыванию из
$A$, то оно обращается к экспорту из экземпляра $A$ в фазе $1$ (относительно фазы $B$).

%A \defining{visit} of a library corresponds to the evaluation of its syntax
%definitions in a particular phase relative to another
%library---a process called \defining{visiting}.
%For example, if a top-level expression in a library $B$ refers to
%a macro export from another library $A$, then it refers to the export from a
%visit of $A$ at phase $0$ (relative to the phase of $B$), which corresponds
%to the evaluation of the macro's transformer expression at phase $1$.
\defining{Посещение} библиотеки соответствует вычислению её определений синтаксиса
в конкретной фазе относительно другой библиотеки---процессу, называемому \defining
{посещением}. Например, если выражение верхнего уровня библиотеки $B$ обращается к
макро-экспорту из другой библиотеки $A$, то оно обращается к экспорту из посещения
$A$ в фазе $0$ (относительно фазы $B$), которая соответствует вычислению выражения
макропреобразователя в фазе $1$.

\mainindex{level}\mainindex{import level}
%
%A \emph{level} is a lexical property of an identifier that determines
%in which phases it can be referenced. The level for each identifier
%bound by a definition within a library is $0$; that is, the identifier
%can be referenced only at phase $0$ within the library.
%The level for each imported binding is determined by the enclosing {\cf
%  for} form of the {\cf import} in the importing library, in
%addition to the levels of the identifier in the exporting
%library. Import and export levels are combined by pairwise addition of
%all level combinations.  For example, references to an imported
%identifier exported for levels $p_a$ and $p_b$ and imported for levels
%$q_a$, $q_b$, and $q_c$ are valid at levels $p_a+q_a$, $p_a+q_b$,
%$p_a+q_c$, $p_b+q_a$, $p_b+q_b$, and $p_b+q_c$. An \hyper{import~set}
%without an enclosing {\cf for} is equivalent to {\cf (for
%  \hyper{import~set} run)}, which is the same as {\cf (for
%  \hyper{import~set} (meta 0))}.
\emph{Уровень} -- это лексическое свойство идентификатора, определяющее, в каких фазах допускаются
обращения к нему. Уровнем каждого идентификатора, привязанного определением внутри
библиотеки, является $0$; то есть, к идентификатору допускается обращение только в фазе $0$ внутри
библиотеки. Уровень каждой импортированной привязки определяется не только ограничивающей формой
{\cf\bfseries for} формы {\cf\bfseries import} в импортируемой библиотеке, но и уровнями
идентификаторов в экспортируемой библиотеке. Уровни импорта и экспорта объединяются попарным
суммированием всех комбинаций уровней. Например, обращения к импортируемому идентификатору,
экспортируемому для уровней $p_a$ и $p_b$ и импортированному для уровней $q_a$, $q_b$, и $q_c$,
применимы на уровнях $p_a+q_a$, $p_a+q_b$, $p_a+q_c$, $p_b+q_a$, $p_b+q_b$ и
$p_b+q_c$. \hyper{import~set} без ограничивающего {\cf\bfseries for} эквивалентен {\cf \textbf{(for}
  \hyper {import~set} \textbf{run)}}, что тождественно {\cf \textbf{(for}
  \hyper{import~set} \bfseries(meta 0))}.

%The export level of an exported binding is $0$ for all bindings
%that are defined within the exporting library. The export levels of a
%reexported binding, i.e., an export imported from another library, are the
%same as the effective import levels of that binding within the reexporting
%library.
Уровнем экспорта экспортируемой привязки является $0$ для всех привязок, определённых внутри
экспортируемой библиотеки. Уровни зкспорта реэкспортированной привязки, то есть, экспорта,
импортированного из другой библиотеки, являются такими же, как и эффективные уровни импорта этой
привязки внутри реэкспортируемой библиотеки.

%For the libraries defined in the library report, the export level is
%$0$ for nearly all bindings. The exceptions are {\cf syntax-rules},
%{\cf identifier-syntax}, {\cf ...}, and {\cf \_} from the
%\rsixlibrary{base} library, which are exported with level $1$, {\cf
%  set!} from the \rsixlibrary{base} library, which is exported with
%levels $0$ and $1$, and all bindings from the composite
%\thersixlibrary{} library (see library
%chapter~\extref{lib:complibchapter}{Composite library}), which are
%exported with levels $0$ and $1$.
Для библиотек, определённых в библиотечном документе, экспортным уровнем практически всех
привязок является $0$. Исключениями являются {\cf\bfseries syntax-rules}, {\cf\bfseries
  identifier-syntax}, {\cf\bfseries ...} и {\cf\bfseries \_} из библиотеки
\textbf{\rsixlibrary{base}}, экспортируемые с уровнем $1$, {\cf\bfseries set!} из библиотеки
\textbf{\rsixlibrary{base}}, экспортируемая с уровнями $0$ и $1$, и все привязки из композитной
библиотеки \textbf{\thersixlibrary{}} (см. библиотечную
главу~\extref{lib:complibchapter}{Composite library}), экспортируемые с уровнями $0$ и $1$.

%Macro expansion within a library can introduce a reference to an
%identifier that is not explicitly imported into the library. In that
%case, the phase of the reference must match the identifier's level as
%shifted by the difference between the phase of the source library
%(i.e., the library that supplied the identifier's lexical context) and
%the library that encloses the reference. For example, suppose that
%expanding a library invokes a macro transformer, and the evaluation of
%the macro transformer refers to an identifier that is exported from
%another library (so the phase-$1$ instance of the library is used);
%suppose further that the value of the binding is a syntax object
%representing an identifier with only a level-$n$ binding; then, the
%identifier must be used only at phase $n+1$ in the
%library being expanded. This combination of levels and phases is why
%negative levels on identifiers can be useful, even though libraries
%exist only at non-negative phases.
Раскрытие макроса в библиотеке может привести к обращению к идентификатору, явно не
импортированному в библиотеку. В этом случае фаза обращения должна соответствовать уровню
идентификатора, сдвинутому на разность фаз исходной библиотеки (то есть, библиотеки,
предоставляющей лексический контекст идентификатора) и библиотеки, содержащей
обращение. Например, предположим, что при разворачивании библиотеки активизируется
макропреобразователь, и при вычислении макропреобразователя происходит обращение к
идентификатору, экспортируемому из другой библиотеки (таким образом, используется экземпляр
библиотеки фазы $1$); предположим также, что значением привязки является синтаксический объект,
представляющий идентификатор с привязкой только уровня $n$; тогда, в разворачиваемой библиотеке
идентификатор должен использоваться только в фазе $n+1$. Поэтому при такой комбинации уровней и
фаз могут применяться отрицательные уровни идентификаторов несмотря на то, что библиотеки
существуют только в неотрицательных фазах.

%If any of a library's definitions are referenced at phase $0$ in the
%expanded form of a program, then an instance of the referenced library
%is created for phase $0$ before the program's definitions and
%expressions are evaluated. This rule applies transitively: if the
%expanded form of one library references at phase $0$ an identifier
%from another library, then before the referencing library is
%instantiated at phase $n$, the referenced library must be instantiated
%at phase $n$. When an identifier is referenced at any phase $n$
%greater than $0$, in contrast, then the defining library is
%instantiated at phase $n$ at some unspecified time before the
%reference is evaluated. Similarly, when a macro keyword is referenced at
%phase $n$ during the expansion of a library, then the
%defining library is visited at phase $n$ at some unspecified time
%before the reference is evaluated.
При обращении из развёрнутой формы программы к какому-либо библиотечному определению в фазе $0$,
перед вычислением определений и выражений программы создаётся экземпляр адресуемой библиотеки
для фазы $0$. Это правило применяется транзитивно: при обращении из развёрнутой формы некоторой
библиотеки в фазе $0$ к идентификатору из другой библиотеки, перед инстанцированием библиотеки
обращения в фазе $n$, должна инстанцироваться адресуемая библиотека в фазе $n$. При обращении к
идентификатору в некоторой фазе $n$, большей чем $0$, напротив, библиотека определения
инстанцируется в фазе $n$ в произвольное неспецифицированное время до вычисления
обращения. Аналогично, при обращении к макро-ключевому слову в фазе $n$ в течение разворачивания
библиотеки, библиотека определения посещается в фазе $n$ в произвольное неспецифицированное
время до вычисления обращения.

%An implementation may distinguish instances/visits of a library for
%different phases or to use an instance/visit at any phase as an instance/visit at
%any other phase. An implementation may further
%expand each {\cf library} form with distinct
%visits of libraries in any phase and/or instances of
%libraries in phases above $0$. An implementation may
%create instances/visits of more libraries at more phases than required to
%satisfy references. When an identifier appears as an expression in a
%phase that is inconsistent with the identifier's level, then an
%implementation may raise an exception either at expand time or run
%time, or it may allow the reference. Thus, a library whose meaning depends on whether the
%instances of a library are distinguished or shared across phases or
%{\cf library} expansions may be unportable.
Реализация может разделять экземпляры/посещения библиотек для различных фаз или
использовать экземпляр/посещение в любой фазе как экземпляр/посещение в любой другой
фазе. Реализация может также раскрывать каждую форму {\cf\bfseries library} с неодинаковыми
посещениями библиотек в любой фазе и/или экземпляры библиотек в фазах выше $0$. Реализация
может создавать экземпляры/посещения большего количества библиотек в большем количестве фаз, чем
необходимо для удовлетворения обращений. Когда идентификатор находится как выражение в фазе,
противоречащей уровню идентификатора, реализация может поднять исключение или на этапе разворачивания,
или на этапе выполнения, или она может позволить обращение. Таким образом, библиотека,
содержание которой зависит от того, различаются ли экземпляры библиотек, или же разделяется при
разворачивания фаз или {\cf\bfseries library}, может быть непортируемой.\vspace{-4mm}

%\section{Examples}
\section{Примеры}

Examples for various \hyper{import~spec}s and \hyper{export~spec}s:

\begin{scheme}
(library (stack)
  (export make push! pop! empty!)
  (import (rnrs))

  (define (make) (list '()))
  (define (push! s v) (set-car! s (cons v (car s))))
  (define (pop! s) (let ([v (caar s)])
                     (set-car! s (cdar s))
                     v))
  (define (empty! s) (set-car! s '())))
\end{scheme}

\begin{scheme}
(library (balloons)
  (export make push pop)
  (import (rnrs))

  (define (make w h) (cons w h))
  (define (push b amt)
    (cons (- (car b) amt) (+ (cdr b) amt)))
  (define (pop b) (display "Boom! ")
                  (display (* (car b) (cdr b)))
                  (newline)))

(library (party)
  ;; Total exports:
  ;; make, push, push!, make-party, pop!
  (export (rename (balloon:make make)
                  (balloon:push push))
          push!
          make-party
          (rename (party-pop! pop!)))
  (import (rnrs)
          (only (stack) make push! pop!) ; not empty!
          (prefix (balloons) balloon:))

  ;; Creates a party as a stack of balloons,
  ;; starting with two balloons
  (define (make-party)
    (let ([s (make)]) ; from stack
      (push! s (balloon:make 10 10))
      (push! s (balloon:make 12 9))
      s))
  (define (party-pop! p)
    (balloon:pop (pop! p))))


(library (main)
  (export)
  (import (rnrs) (party))

  (define p (make-party))
  (pop! p)        ; displays "Boom! 108"
  (push! p (push (make 5 5) 1))
  (pop! p))       ; displays "Boom! 24"%
\end{scheme}

Examples for macros and phases:

\begin{schemenoindent}
(library (my-helpers id-stuff)
  (export find-dup)
  (import (rnrs))

  (define (find-dup l)
    (and (pair? l)
         (let loop ((rest (cdr l)))
           (cond
            [(null? rest) (find-dup (cdr l))]
            [(bound-identifier=? (car l) (car rest))
             (car rest)]
            [else (loop (cdr rest))])))))

(library (my-helpers values-stuff)
  (export mvlet)
  (import (rnrs) (for (my-helpers id-stuff) expand))

  (define-syntax mvlet
    (lambda (stx)
      (syntax-case stx ()
        [(\_ [(id ...) expr] body0 body ...)
         (not (find-dup (syntax (id ...))))
         (syntax
           (call-with-values
               (lambda () expr)
             (lambda (id ...) body0 body ...)))]))))

(library (let-div)
  (export let-div)
  (import (rnrs)
          (my-helpers values-stuff)
          (rnrs r5rs))

  (define (quotient+remainder n d)
    (let ([q (quotient n d)])
      (values q (- n (* q d)))))
  (define-syntax let-div
    (syntax-rules ()
     [(\_ n d (q r) body0 body ...)
      (mvlet [(q r) (quotient+remainder n d)]
        body0 body ...)])))%
\end{schemenoindent}


%%% Local Variables:
%%% mode: latex
%%% TeX-master: "r6rs"
%%% End:
 \par
%\chapter{Top-level programs}
\chapter{Программы верхнего уровня}
\label{programchapter}

%A \defining{top-level program} specifies an entry point for defining and running
%a Scheme program.  A top-level program specifies a set of libraries to import and
%code to run.  Through the imported libraries, whether directly or through the
%transitive closure of importing, a top-level program defines a complete Scheme
%program.
\defining{Программа верхнего уровня} задаёт точку входа для определения и выполнения программы
Scheme. В программе верхнего уровня указывается набор библиотек для импорта и код для
выполнения. (Через импортированные библиотеки, или непосредственно или через переходное закрытие
импортирования ?), программа верхнего уровня определяет полную программу Scheme.

%\section{Top-level program syntax}
\section{Синтаксис программ верхнего уровня}
\label{programsyntaxsection}

%A top-level program is a delimited piece of text, typically a file, that
%has the following form:
Программа верхнего уровня является ограниченной частью текста, обычно файлом, имеющим следующую
форму:
%
\begin{scheme}
\hyper{import form} \hyper{top-level body}%
\end{scheme}
%
%An \hyper{import form} has the following form:
\hyper{import form} имеет следующую форму:
%
\begin{scheme}
(\textbf{import} \hyper{import spec} \dotsfoo)%
\end{scheme}
%
%A \hyper{top-level body} has the following form:
\hyper{top-level body} имеет следующую форму:
\begin{scheme}
\hyper{top-level body form} \dotsfoo%
\end{scheme}
%
%A \hyper{top-level body form} is either a \hyper{definition} or an
%\hyper{expression}.
\hyper{top-level body form} является или \hyper{definition} или
\hyper{expression}.

%The \hyper{import form} is identical to the import clause in
%libraries (see section~\ref{librarysyntaxsection}),
%and specifies a set of libraries to import.  A \hyper{top-level
% body} is like a \hyper{library body} (see
%section~\ref{librarybodysection}), except that
%definitions and expressions may occur in any order.  Thus, the syntax
%specified by \hyper{top-level body form} refers to the result of macro
%expansion.
\hyper{import form} идентичен разделу import в библиотеках
(см. секцию~\ref{librarysyntaxsection}) и задаёт ряд библиотек импорта. \hyper{top-level body}
похож на \hyper{library body} (см. секцию~\ref{librarybodysection}), за исключением того, что
определения и выражения могут находиться в произвольном порядке. Таким образом, синтаксис,
задаваемый \hyper{top-level body form} относится к результату разворачивания макросов.

%When uses of {\cf begin}, {\cf let-syntax}, or {\cf letrec-syntax}
%from the \rsixlibrary{base} library
%occur in a top-level body prior to the first
%expression, they are spliced into the body; see section~\ref{begin}.
%Some or all of the body, including portions wrapped in {\cf begin},
%{\cf let-syntax}, or {\cf letrec-syntax}
%forms, may be specified by a syntactic abstraction
%(see section~\ref{macrosection}).
При использовании {\bfseries\cf begin}, {\bfseries\cf let-syntax} или {\bfseries\cf
  letrec-syntax} из библиотеки \rsixlibrary{base} происходит{встречается} в теле верхнего уровня
до первого выражения, они соединены в тело; см. секцию~\ref{begin}. Некоторые или все тело,
включая части, обернутые в формы {\bfseries\cf begin}, {\bfseries\cf let-syntax} или
{\bfseries\cf letrec-syntax}, может быть определен синтаксической абстракцией
(см. секцию~\ref{macrosection}).

%\section{Top-level program semantics}
\section{Семантика программы верхнего уровня}

A top-level program is executed by treating the program similarly to a library, and
evaluating its definitions and expressions.
The semantics of a top-level body may be roughly explained by
a simple translation into a library body:
Each \hyper{expression} that appears before a
definition in
the top-level body is converted into a dummy definition
%
\begin{scheme}
(define \hyper{variable} (begin \hyper{expression} \hyper{unspecified}))%
\end{scheme}
%
where \hyper{variable} is a fresh identifier and \hyper{unspecified}
is a side-effect-free expression returning an unspecified value.
(It is generally impossible to determine which forms are
definitions and expressions without concurrently expanding the body, so
the actual translation is somewhat more complicated; see
chapter~\ref{expansionchapter}.)

On platforms that support it, a top-level program may access its command line
by calling the {\cf command-line} procedure (see library
section~\extref{lib:command-line}{Command-line access and exit values}).

%%% Local Variables:
%%% mode: latex
%%% TeX-master: "r6rs"
%%% End:
 \par
%\chapter{Primitive syntax}
\chapter{Синтаксис примитивов}\vspace{2mm}

%After the {\cf import} form within a {\cf library} form or a top-level
%program, the forms
%that constitute the body of the library or the top-level program
%depend on the libraries that are
%imported. In particular, imported syntactic keywords determine
%the available syntactic abstractions and whether each form is a
%definition or expression. A few form types are
%always available independent of imported libraries, however,
%including constant literals, variable references, procedure calls,
% and macro uses.
После формы {\bfseries\cf import} внутри формы {\bfseries\cf library} или программы верхнего уровня,
формы, составляющие тело библиотеки или программы верхнего уровня, зависят от
импортированных библиотек.
В частности, импортированные синтаксические ключевые слова определяют
доступные синтаксические абстракции и является ли каждая форма определением или
выражением. Однако некоторые типы форм всегда доступны независимо от импортированных библиотек,
включая константные литералы, обращения к переменным, вызовы
процедур и применения макросов.

%\section{Primitive expression types}
\section{Типы примитивных выражений}
\label{primitiveexpressionsection}

%The entries in this section all describe expressions, which may occur
%in the place of \hyper{expression} syntactic variables.  See
%also section~\ref{expressionsection}.
Все записи в этой секции описывают выражения, которые могут находиться на месте синтаксических
переменных \hyper{expression}. См. также секцию~\ref{expressionsection}.

%\subsection*{Constant literals}\unsection
\subsection*{Константные литералы}\unsection

\begin{entry}{%
\pproto{\hyper{number}}{\exprtype}
\pproto{\hyper{boolean}}{\exprtype}
\pproto{\hyper{character}}{\exprtype}
\pproto{\hyper{string}}{\exprtype}
\pproto{\hyper{bytevector}}{\exprtype}}\mainindex{literal}\vspace{1mm}

%An expression consisting of a representation of a number object, a
%boolean, a character, a string, or a bytevector, evaluates ``to
%itself''.
Выражение, состоящее из представления числового объекта, булевого значения, символа, строки или
байт-вектора, вычисляется ``как есть''.\vspace{1mm}

\begin{scheme}
\bfseries 145932     \ev  \bfseries 145932
\bfseries \schtrue   \ev  \bfseries \schtrue
\bfseries "abc"      \ev  \bfseries "abc"
\bfseries \#vu8(2 24 123) \ev \bfseries\#vu8(2 24 123)%
\end{scheme}\vspace{1mm}

%As noted in section~\ref{storagemodel}, the value of a literal
%expression is immutable.
Как отмечено в секции ~\ref{storagemodel}, значение литерального выражения является неизменяемым.
\end{entry}

%\subsection*{Variable references}\unsection
\subsection*{Обращения к переменным}\unsection
\begin{entry}{%
\pproto{\hyper{variable}}{\exprtype}}\vspace{1mm}

%An expression consisting of a variable\index{variable}
%(section~\ref{variablesection}) is a variable reference if it is not a
%macro use (see below).  The value of
%the variable reference is the value stored in the location to which the
%variable is bound.  It is a syntax violation to reference
%an unbound\index{unbound} variable.
Выражение, состоящее из переменной\index{variable} (секция~\ref{variablesection}), является
обращением к переменной в случае, если оно не является применением макроса (см. ниже). Значением
обращения к переменной является значение,
хранящееся в области памяти, с которой связана переменная.
Обращение к непривязанной \index{unbound} переменной является нарушением синтаксиса.

%The following example examples assumes the base library
%has been imported:
В следующих примерах предполагается, что основная библиотека импортирована:\vspace{1mm}
%
\begin{scheme}
\bfseries(define x 28)
\bfseries x   \ev  \bfseries 28%
\end{scheme}\vspace{1mm}
\end{entry}

%\subsection*{Procedure calls}\unsection
\subsection*{Вызовы процедур}\unsection

\begin{entry}{%
\pproto{\textbf{(}\hyper{operator} \hyperi{operand} \dotsfoo\textbf{)}}{\exprtype}}\vspace{1mm}

%A procedure call consists of expressions for the procedure to be
%called and the arguments to be passed to it, with enclosing
%parentheses.  A form in an expression context is a procedure call if
%\hyper{operator} is not an identifier bound as a syntactic keyword
%(see section~\ref{macrosection} below).
Вызов процедуры состоит из заключённых в круглые скобки выражений вызова процедуры и передаваемых ей
аргументов. Форма в контексте выражения является вызовом процедуры, если \hyper{operator} не
является идентификатором, привязанным к синтаксическому ключевому слову
(см. секцию~\ref{macrosection} ниже).

%When a procedure call is evaluated, the operator and operand
%expressions are evaluated (in an unspecified order) and the resulting
%procedure is passed the resulting
%arguments.\mainindex{call}\mainindex{procedure call}
При вычислении вызова процедуры вычисляются (в произвольном порядке) выражения оператора и операндов,
и полученной процедуре передаются полученные аргументы.\mainindex{call}\mainindex{procedure
  call}

%The following examples assume the \rsixlibrary{base} library
%has been imported:
В следующих примерах предполагается, что библиотека \textbf{\rsixlibrary{base}} импортирована.\vspace{1mm}
%
\begin{scheme}%
\bfseries (+ 3 4)                          \ev\bfseries 7
\bfseries ((if \schfalse + *) 3 4)         \ev\bfseries 12%
\end{scheme}\vspace{1mm}
%
%If the value of \hyper{operator} is not a procedure, an exception with
%condition type {\cf\&assertion} is raised.  Also, if \hyper{operator}
%does not accept as many arguments as there are \hyper{operand}s, an
%exception with condition type {\cf\&assertion} is raised.

Если значение \hyper{operator} не является процедурой, возбуждается исключение с типом состояния
{\cf\bfseries\&assertion}. Если количество \hyper{operand} превышает количество принимаемых
\hyper{operand} аргументом, также возбуждается исключение с типом состояния {\cf\bfseries\&assertion}.

\begin{note} %In contrast to other dialects of Lisp, the order of
%evaluation is unspecified, and the operator expression and the operand
%expressions are always evaluated with the same evaluation rules.
В отличие от других диалектов Lisp, порядок вычисления не определён, и выражение оператора, и
выражения операнда всегда вычисляются с теми же правилами вычисления.

%Although the order of evaluation is otherwise unspecified, the effect of
%any concurrent evaluation of the operator and operand expressions is
%constrained to be consistent with some sequential order of evaluation.
%The order of evaluation may be chosen differently for each procedure call.
Хотя порядок вычисления иначе не определён, эффект любого параллельного вычисления выражений
оператора и операнда ограничен совместимостью с некоторым последовательным порядком
вычисления. Порядок вычисления может быть выбран другим при каждом вызове процедуры.
\end{note}

%\newpage

\begin{note} %In many dialects of Lisp, the form {\tt
%()} is a legitimate expression.  In Scheme, expressions written as
%list/pair forms must have at
%least one subexpression, so {\tt ()} is not a syntactically valid
%expression.
Во многих диалектах Lisp форма {\tt\textbf{()}} является допустимым выражением. В Scheme
выражения, записанные как формы списков/пар, должны иметь по крайней мере одно подвыражение, таким
образом, {\tt\textbf{()}} не является синтаксически допустимым выражением.
\end{note}

\end{entry}

%\section{Macros}
\section{Макросы}\vspace{1mm}
\label{macrosection}

%Libraries and top-level programs can define and use new kinds of derived expressions and
%definitions called {\em syntactic abstractions} or
%{\em macros}.\mainindex{syntactic abstraction}\mainindex{macro}
%A syntactic abstraction is created by binding a keyword to a
%{\em macro transformer} or, simply, {\em transformer}.
%\index{macro transformer}\index{transformer}
%The transformer determines
%how a use of the macro (called a \defining{macro use})
%is transcribed into a more primitive form.
В библиотеках и программах верхнего уровня могут определяться и использоваться новые виды
производных выражений и определений, называемых {\em синтаксическими абстракциями} или {\em
  макросами}.\mainindex{syntactic abstraction} \mainindex {macro} Синтаксическая абстракция
создаётся путём привязки ключевого слова к {\em макротрансформеру}, или просто {\em
  трансформеру}.\index{macro transformer}\index{macro transformer} Трансформер
определяет, как применение макроса (называемое \defining{макроприменением}) расшифровывается в
более примитивную форму.\vspace{1mm}

%Most macro uses have the form:
Большинство макросов имеют форму:\vspace{1mm}
\begin{scheme}
\textbf{(}\hyper{keyword} \hyper{datum} \dotsfoo\textbf{)}%
\end{scheme}\vspace{1mm}%
%where \hyper{keyword} is an identifier that uniquely determines the
%kind of form.  This identifier is called the {\em syntactic
%keyword}\index{syntactic keyword}, or simply {\em
%keyword}\index{keyword}, of the macro\index{macro keyword}.
%The number of \hyper{datum}s and the syntax
%of each depends on the syntactic abstraction.
где \hyper{keyword} -- идентификатор, уникально определяющий вид формы. Этот
идентификатор называется {\em синтаксическим ключевым словом}\index{syntactic keyword},
или просто {\em ключевым словом}\index{keyword}, макроса\index{macro keyword}. Количество
\hyper{datum} и синтаксис каждого из них зависит от синтаксической абстракции.\vspace{1mm}

%Macro uses can also take the form of improper lists, singleton
%identifiers, or {\cf set!} forms, where the second subform of the
%{\cf set!} is the keyword (see section~\ref{identifier-syntax})
%library section~\extref{lib:make-variable-transformer}{{\cf make-variable-transformer}}):
Макроприменения могут также принимать форму нестрогих списков, еденичных идентификаторов
или форм {\cf\bfseries set!}, где вторая подформа {\cf\bfseries set!} является
ключевым словом (см. секцию~\ref{identifier-syntax}), библиотечную
секцию~\extref{lib:make-variable-transformer} {{\cf\bfseries make-variable-transformer}}):\vspace{1mm}
\begin{scheme}
\textbf{(}\hyper{keyword} \hyper{datum} \dotsfoo . \hyper{datum}\textbf{)}
\hyper{keyword}
\textbf{(}set! \hyper{keyword} \hyper{datum}\textbf{)}%
\end{scheme}\vspace{1mm}

%The {\cf define-syntax}, {\cf let-syntax} and {\cf letrec-syntax}
%forms, described in sections~\ref{define-syntax} and \ref{let-syntax},
%create bindings for keywords, associate them with macro transformers,
%and control the scope within which they are visible.
Формы {\cf\bfseries define-syntax}, {\cf\bfseries let-syntax} и {\cf\bfseries letrec-syntax},
описанные в секциях~\ref{define-syntax} и \ref{letrec-syntax}, создают привязки для ключевых
слов, связывают их с макротрансформерами, и задают область, внутри которых они видимы.\vspace{1mm}

%The {\cf syntax-rules} and {\cf identifier-syntax} forms, described in
%section~\ref{syntaxrulessection}, create transformers via a pattern
%language.  Moreover, the {\cf syntax-case} form, described in library
%chapter~\extref{lib:syntaxcasechapter}{{\cf syntax-case}},
%allows creating transformers via arbitrary Scheme code.
Формы {\cf\bfseries syntax-rules} и {\cf\bfseries identifier-syntax}, описанные в секции
~\ref{syntaxrulessection}, создают трансформеры посредством языка шаблонов. Кроме того, форма
{\cf\bfseries syntax-case}, описанная в библиотечной главе~\extref{lib:syntaxcasechapter}
{{\cf\bfseries syntax-case}}, позволяет создавать трансформеры посредством произвольного кода
Scheme.\vspace{1mm}

%Keywords occupy the same name space as variables.
%That is, within the same
%scope, an identifier can be bound as a variable or keyword, or neither, but
%not both, and local bindings of either kind may shadow other bindings of
%either kind.
Ключевые слова занимают то же пространство имён, что и переменные. Таким образом, внутри той же
области видимости идентификатор может быть привязан как переменная или как ключевое слово, или
не привязан вообще, но не привязан одновременно, а локальные привязки любого вида могут
маскировать другие привязки любого вида.\vspace{1mm}

%Macros defined using {\cf syntax-rules} and {\cf identifier-\hp{}syntax}
%are ``hygienic'' and ``referentially transparent'' and thus preserve
%Scheme's lexical scoping~\cite{Kohlbecker86,
%  hygienic,Bawden88,macrosthatwork,syntacticabstraction}:
%\mainindex{hygienic} \mainindex{referentially transparent}
Макрос, определённый с помощью {\cf\bfseries syntax-rules} и {\cf\bfseries
  identifier-\hp{}syntax}, является ``гигиеничным'' и ``референциально прозрачным'' и, таким
образом, предохраняет лексическую сферу действия
Scheme~\cite{Kohlbecker86,hygienic,Bawden88,macrosthatwork,syntacticabstraction}:\mainindex{hygienic}
\mainindex{referentially transparent}

\begin{itemize}
%\item If a macro transformer inserts a binding for an identifier
%(variable or keyword) not appearing in the macro use, the identifier is in effect renamed
%throughout its scope to avoid conflicts with other identifiers.
\item Если макротрансформер вносит привязку для идентификатора (переменной или ключевого
слова), отсутствующего в макроприменении, идентификатор используется переименованным
по всей своей области видимости для предотвращения конфликтов с другими идентификаторами.

%\item If a macro transformer inserts a free reference to an
%identifier, the reference refers to the binding that was visible
%where the transformer was specified, regardless of any local
%bindings that may surround the use of the macro.
\item Если макротрансформер вносит свободное обращение к идентификатору, обращение относится к
  привязке, которая была видима в месте определения трансформера, независимо от всех
  локальных привязкок, которые могут окружать применение макроса.
\end{itemize}

%Macros defined using the {\cf syntax-case} facility are also
%hygienic unless {\cf datum\coerce{}syntax}
%(see library section~\extref{lib:conversionssection}{Syntax-object and datum conversions}) is
%used.
Макрос, определённый с помощью средства {\cf\bfseries syntax-case},
также является гигиеничным, кроме случаев, когда используется {\cf\bfseries
  datum\coerce{}syntax} (см. библиотечную секцию~\extref{lib:conversionssection}{Syntax-object
  and datum conversions}).

%%% Local Variables:
%%% mode: latex
%%% TeX-master: "r6rs"
%%% End:
 \par
\chapter{Expansion process}
\label{expansionchapter}

Macro uses (see section~\ref{macrosection}) are expanded into \textit{core
forms}\mainindex{core form} at the start of evaluation (before compilation or interpretation)
by a syntax \emph{expander}.
The set of core forms is implementation-dependent, as is the
representation of these forms in the expander's output.
If the expander encounters a syntactic abstraction, it invokes
the associated transformer to expand the syntactic abstraction, then
repeats the expansion process for the form returned by the transformer.
If the expander encounters a core form, it recursively
processes its subforms that are in expression or definition context,
if any, and reconstructs the form from the
expanded subforms.
Information about identifier bindings is maintained during expansion
to enforce lexical scoping for variables and keywords.

To handle definitions, the expander processes the initial
forms in a \hyper{body} (see section~\ref{bodiessection}) or
\hyper{library body} (see section~\ref{librarybodysection})
from left to
right.  How the expander processes each form encountered 
depends upon the kind of form.

\begin{description}
\item[macro use]
The expander invokes the associated transformer to transform the macro
use, then recursively performs whichever of these actions are appropriate
for the resulting form.

\item[{\cf define-syntax} form]
The expander expands and evaluates the right-hand-side expression and binds the
keyword to the resulting transformer.

\item[{\cf define} form]
The expander records the fact that the defined identifier is a variable but defers
expansion of the right-hand-side expression until after all of the
definitions have been processed.

\item[{\cf begin} form]
The expander splices the subforms into the list of body forms it is
processing.  (See section~\ref{begin}.)

\item[{\cf let-syntax} or {\cf letrec-syntax} form]
The expander splic\-es the inner body forms into the list of (outer) body forms it is
processing, arranging for the keywords bound by the {\cf let-syntax}
and {\cf letrec-syntax} to be visible only in the inner body forms.

\item[expression, i.e., nondefinition]
The expander completes the expansion of the deferred right-hand-side expressions
and the current and remaining expressions in the body, and then
creates the equivalent of a {\cf letrec*} form from the defined variables,
expanded right-hand-side expressions, and expanded body expressions.
\end{description}

For the right-hand side of the definition of a variable, expansion is
deferred until after all of the definitions have been seen.  Consequently,
each keyword and variable reference within the right-hand side
resolves to the local binding, if any.

A definition in the sequence of forms must not define any identifier whose
binding is used to determine the meaning of the undeferred portions of the
definition or any definition that precedes it in the sequence of forms.
For example, the bodies of the following expressions violate this
restriction.

\begin{scheme}
(let ()
  (define define 17)
  (list define))

(let-syntax ([def0 (syntax-rules ()
                     [(\_ x) (define x 0)])])
  (let ([z 3])
    (def0 z)
    (define def0 list)
    (list z)))

(let ()
  (define-syntax foo
    (lambda (e)
      (+ 1 2)))
  (define + 2)
  (foo))%
\end{scheme}

The following do not violate the restriction.

\begin{scheme}
(let ([x 5])
  (define lambda list)
  (lambda x x))         \ev  (5 5)

(let-syntax ([def0 (syntax-rules ()
                     [(\_ x) (define x 0)])])
  (let ([z 3])
    (define def0 list)
    (def0 z)
    (list z)))          \ev  (3)

(let ()
  (define-syntax foo
    (lambda (e)
      (let ([+ -]) (+ 1 2))))
  (define + 2)
  (foo))                \ev  -1%
\end{scheme}

The implementation should treat a violation of the restriction as a
syntax violation.

% Andre's proposed implementation:
% To detect this violation, the expander can record each
% identifier whose denotation is determined during expansion
% of the body, together with the denotation.
% Before an identifier is bound, its current denotation is compared
% against denotations already used for the same (in the sense of
% bound-identifier=?) identifier in the scope of the intended binding,
% to determine if its current denotation has already been used
% during the expansion of the body.

Note that this algorithm does not directly reprocess any form.
It requires a single left-to-right pass over the definitions followed by a
single pass (in any order) over the body expressions and deferred
right-hand sides.

Example:

\begin{scheme}
(lambda (x)
  (define-syntax defun
    (syntax-rules ()
      [(\_ x a e) (define x (lambda a e))]))
  (defun even? (n) (or (= n 0) (odd? (- n 1))))
  (define-syntax odd?
    (syntax-rules () [(\_ n) (not (even? n))]))
  (odd? (if (odd? x) (* x x) x)))%
\end{scheme}

In the example, the definition of {\cf defun} is encountered first, and the keyword
{\cf defun} is associated with the transformer resulting from
the expansion and evaluation of the corresponding right-hand side.
A use of {\cf defun} is encountered next and expands into a
{\cf define} form.
Expansion of the right-hand side of this define form is deferred.
The definition of {\cf odd?} is next and results in the association
of the keyword {\cf odd?} with the transformer resulting from
expanding and evaluating the corresponding right-hand side.
A use of {\cf odd?} appears next and is expanded; the resulting
call to {\cf not} is recognized as an expression
because {\cf not} is bound as a variable.
At this point, the expander completes the expansion of the current
expression (the call to {\cf not}) and the deferred right-hand side of the
{\cf even?} definition;
the uses of {\cf odd?} appearing in these expressions are expanded
using the transformer associated with the keyword {\cf odd?}.
The final output is the equivalent of

\begin{scheme}
(lambda (x)
  (letrec* ([even?
              (lambda (n)
                (or (= n 0)
                    (not (even? (- n 1)))))])
    (not (even? (if (not (even? x)) (* x x) x)))))%
\end{scheme}

although the structure of the output is implementation-dependent.

Because definitions and expressions can be interleaved in a
\hyper{top-level body} (see chapter~\ref{programchapter}),
the expander's processing of a \hyper{top-level body} is somewhat
more complicated.
It behaves as described above for a \hyper{body} or
\hyper{library body} with the following exceptions:
When the expander finds a nondefinition,
it defers its expansion and continues scanning for definitions.
Once it reaches the end of the set of forms, it processes the
deferred right-hand-side and body expressions, then
generates the equivalent of a {\cf letrec*} form from the defined variables,
expanded right-hand-side expressions, and expanded body expressions.
For each body expression \hyper{expression} that appears before a variable definition
in the body, a dummy binding is created at the corresponding place within
the set of {\cf letrec*} bindings, with a fresh temporary variable on the
left-hand side and the equivalent of {\cf (begin \hyper{expression}
  \hyper{unspecified})},
where \hyper{unspecified} is a side-effect-free expression returning
an unspecified value,
on the right-hand side, so that
left-to-right evaluation order is preserved.
The {\cf begin} wrapper allows \hyper{expression} to evaluate to an
arbitrary number of values.

%%% Local Variables: 
%%% mode: latex
%%% TeX-master: "r6rs"
%%% End: 
 \par
%\vfill\eject
\chapter{Base library}
\label{baselibrarychapter}

This chapter describes Scheme's \defrsixlibrary{base} library, which exports many of
the procedure and syntax bindings that are traditionally associated
with Scheme.

Section~\ref{basetailcontextsection} defines the rules that identify
tail calls and tail contexts in constructs from the \rsixlibrary{base}
library.

\section{Base types}
\label{disjointness}

No object satisfies more than one of the following predicates:

\begin{scheme}
boolean?          pair?
symbol?           number?
char?             string?
vector?           procedure?
null?%
\end{scheme}

These predicates define the base types {\em boolean}, {\em pair}, {\em
symbol}, {\em number}, {\em char} (or {\em character}), {\em string}, {\em
vector}, and {\em procedure}.  Moreover, the empty list is a special
object of its own type.
\mainindex{type}\schindex{boolean?}\schindex{pair?}\schindex{symbol?}
\schindex{number?}\schindex{char?}\schindex{string?}\schindex{vector?}
\schindex{procedure?}\index{empty list}\schindex{null?}

Note that, although there is a separate boolean type, any Scheme value
can be used as a boolean value for the purpose of a conditional test;
see section~\ref{booleanvaluessection}.

\section{Definitions}
\label{defines}

Definitions\mainindex{definition} may appear within a
\meta{top-level body} (section~\ref{programsyntaxsection}),
at the top of a \meta{library body} (section~\ref{librarysyntaxsection}),
or at the top of a \meta{body} (section~\ref{bodiessection}).

A \hyper{definition} may be a variable definition
(section~\ref{variabledefinitionsection}) or
keyword definition
(section~\ref{variabledefinitionsection}).
Macro uses that expand into definitions or groups of
definitions (packaged in a {\cf begin}, {\cf let-syntax}, or
{\cf letrec-syntax} form; see section~\ref{begin}) may also appear
wherever other definitions may appear.

\subsection{Variable definitions}
\label{variabledefinitionsection}

The {\cf define} form described in this section is a
\hyper{definition}\mainindex{definition} used to create variable bindings
and may appear anywhere other definitions may appear.

\begin{entry}{%
\proto{define}{ \hyper{variable} \hyper{expression}}{\exprtype}
\rproto{define}{ \hyper{variable}}{\exprtype}
\pproto{(define (\hyper{variable} \hyper{formals}) \hyper{body})}{\exprtype}
\pproto{(define (\hyper{variable} .\ \hyper{formal}) \hyper{body})}{\exprtype}}


The first from of {\cf define} binds \hyper{variable} to a new
location before assigning the value of \hyper{expression} to it.
\begin{scheme}
(define add3
  (lambda (x) (+ x 3)))
(add3 3)                            \ev  6
(define first car)
(first '(1 2))                      \ev  1%
\end{scheme}
%
The continuation of \hyper{expression} should not be invoked more than
once.

\implresp Implementations should detect that the
continuation of \hyper{expression} is invoked more than once.
If the implementation detects this, it must raise an
exception with condition type {\cf\&assertion}.

The second form of {\cf define} is equivalent to
\begin{scheme}
(define \hyper{variable} \hyper{unspecified})%
\end{scheme}
where \hyper{unspecified} is a side-effect-free expression returning
an unspecified value.

In the third form of {\cf define}, \hyper{formals} must be either a
sequence of zero or more variables, or a sequence of one or more
variables followed by a dot {\cf .} and another variable (as
in a lambda expression, see section~\ref{lambda}).  This form is equivalent to
\begin{scheme}
(define \hyper{variable}
  (lambda (\hyper{formals}) \hyper{body}))\rm.%
\end{scheme}

In the fourth form of {\cf define}, 
\hyper{formal} must be a single
variable.  This form is equivalent to
\begin{scheme}
(define \hyper{variable}
  (lambda \hyper{formal} \hyper{body}))\rm.%
\end{scheme}
\end{entry}

\subsection{Syntax definitions}
\label{syntaxdefinitionsection}

The {\cf define-syntax} form described in this section is a
\hyper{definition}\mainindex{definition} used to create keyword bindings
and may appear anywhere other definitions may appear.

\begin{entry}{%
\proto{define-syntax}{ \hyper{keyword} \hyper{expression}}{\exprtype}}

Binds \hyper{keyword} to the value of
\hyper{expression}, which must evaluate, at macro-expansion
time, to a transformer.  Macro transformers can be created using the
{\cf syntax-rules} and {\cf identifier-syntax} forms described in
section~\ref{syntaxrulessection}.  See library
section~\extref{lib:transformerssection}{Transformers} for a more
complete description of transformers.

Keyword bindings established by {\cf define-syntax} are visible
throughout the body in which they appear, except where shadowed by
other bindings, and nowhere else, just like variable bindings established
by {\cf define}.
All bindings established by a set of definitions, whether
keyword or variable definitions, are visible within the definitions
themselves.

\implresp The implementation should detect if the value of
\hyper{expression} cannot possibly be a transformer.

Example:

\begin{scheme}
(let ()
  (define even?
    (lambda (x)
      (or (= x 0) (odd? (- x 1)))))
  (define-syntax odd?
    (syntax-rules ()
      ((odd?  x) (not (even? x)))))
  (even? 10))                       \ev \schtrue{}%
\end{scheme}

An implication of the left-to-right processing order
(section~\ref{expansionchapter}) is that one definition can
affect whether a subsequent form is also a definition.  

Example:

\begin{scheme}
(let ()
  (define-syntax bind-to-zero
    (syntax-rules ()
      ((bind-to-zero id) (define id 0))))
  (bind-to-zero x)
  x) \ev 0%
\end{scheme}

The behavior is unaffected by any binding for
{\cf bind-to-zero} that might appear outside of the {\cf let}
expression.
\end{entry}

\section{Bodies}
\label{bodiessection}

\index{body}The \hyper{body} of a \ide{lambda}, \ide{let}, \ide{let*},
\ide{let-values}, \ide{let*-values}, \ide{letrec}, or \ide{letrec*}
expression, or that of a definition with a body
consists of zero or more definitions followed by one or more
expressions.

{\cf \hyper{definition} \ldots{} \hyperi{expression} \hyperii{expression} \ldots}

Each identifier defined by a
definition is local to the \hyper{body}.  That is, the identifier is
bound, and the region of the binding is the
entire \hyper{body} (see section~\ref{variablesection}).

Example:
%
\begin{scheme}
(let ((x 5))
  (define foo (lambda (y) (bar x y)))
  (define bar (lambda (a b) (+ (* a b) a)))
  (foo (+ x 3)))                \ev  45%
\end{scheme}
%
When {\cf begin}, {\cf let-syntax}, or {\cf letrec-syntax} forms
occur in a body prior to the first
expression, they are spliced into the body; see section~\ref{begin}.
Some or all of the body, including portions wrapped in {\cf begin},
{\cf let-syntax}, or {\cf letrec-syntax}
forms, may be specified by a macro use
(see section~\ref{macrosection}).

An expanded \hyper{body} (see chapter~\ref{expansionchapter})
containing variable definitions can
always be converted into an equivalent {\cf letrec*}
expression.  For example, the {\cf let} expression in the above
example is equivalent to

\begin{scheme}
(let ((x 5))
  (letrec* ((foo (lambda (y) (bar x y)))
            (bar (lambda (a b) (+ (* a b) a))))
    (foo (+ x 3))))%
\end{scheme}

\section{Expressions}
\label{expressionsection}

The entries in this section describe the expressions of the \rsixlibrary{base}
library, which may occur in the position of the \hyper{expression}
syntactic variable in addition to the primitive
expression types as described in
section~\ref{primitiveexpressionsection}.

\subsection{Quotation}\unsection
\label{quotesection}

\begin{entry}{%
\proto{quote}{ \hyper{datum}}{\exprtype}}

\syntax \hyper{Datum} should be a syntactic datum.

\semantics
{\cf (quote \hyper{datum})} evaluates to the datum value
represented by \hyper{datum}
(see
section~\ref{datumsyntaxsection}).  This notation is used to include
constants.

\begin{scheme}%
(quote a)                     \ev  a
(quote \sharpsign(a b c))     \ev  \#(a b c)
(quote (+ 1 2))               \ev  (+ 1 2)%
\end{scheme}

As noted in section~\ref{abbreviationsection}, {\cf (quote \hyper{datum})}
may be abbreviated as \singlequote\hyper{datum}:

\begin{scheme}
'"abc"               \ev  "abc"
'145932              \ev  145932
'a                   \ev  a
'\#(a b c)           \ev  \#(a b c)
'()                  \ev  ()
'(+ 1 2)             \ev  (+ 1 2)
'(quote a)           \ev  (quote a)
''a                  \ev  (quote a)%
\end{scheme}

As noted in section~\ref{storagemodel}, constants are immutable.

\begin{note}
  Different constants that are the value of a {\cf quote} expression may
  share the same locations.
\end{note}
\end{entry}

\subsection{Procedures}\unsection
\label{lamba}

\begin{entry}{%
\proto{lambda}{ \hyper{formals} \hyper{body}}{\exprtype}}

\syntax
\hyper{Formals} must be a formal parameter list as described below,
and \hyper{body} must be as described in section~\ref{bodiessection}.

\semantics
\vest A \lambdaexp{} evaluates to a procedure.  The environment in
effect when the \lambdaexp{} is evaluated is remembered as part of the
procedure.  When the procedure is later called with some 
arguments, the environment in which the \lambdaexp{} was evaluated is
extended by binding the variables in the parameter list to
fresh locations, and the resulting argument values are stored
in those locations.  Then, the expressions in the body of the \lambdaexp{}
(which may contain definitions and thus represent a {\cf
  letrec*} form, see section~\ref{bodiessection}) are evaluated
sequentially in the extended environment.
The results of the last expression in the body are returned as
the results of the procedure call.

\begin{scheme}
(lambda (x) (+ x x))      \ev  {\em{}a procedure}
((lambda (x) (+ x x)) 4)  \ev  8

((lambda (x)
   (define (p y)
     (+ y 1))
   (+ (p x) x))
 5) \ev 11

(define reverse-subtract
  (lambda (x y) (- y x)))
(reverse-subtract 7 10)         \ev  3

(define add4
  (let ((x 4))
    (lambda (y) (+ x y))))
(add4 6)                        \ev  10%
\end{scheme}

\hyper{Formals} must have one of the following forms:

\begin{itemize}
\item {\tt(\hyperi{variable} \dotsfoo)}:
The procedure takes a fixed number of arguments; when the procedure is
called, the arguments are stored in the bindings of the
corresponding variables.

\item \hyper{variable}:
The procedure takes any number of arguments; when the procedure is
called, the sequence of arguments is converted into a newly
allocated list, and the list is stored in the binding of the
\hyper{variable}.

\item {\tt(\hyperi{variable} \dotsfoo{} \hyper{variable$_{n}$}\ {\bf.}\
\hyper{variable$_{n+1}$})}:
If a period {\cf .} precedes the last variable, then
the procedure takes $n$ or more arguments, where $n$ is the
number of parameters before the period (there must
be at least one).
The value stored in the binding of the last variable is a
newly allocated
list of the arguments left over after all the other 
arguments have been matched up against the other parameters.
\end{itemize}

\begin{scheme}
((lambda x x) 3 4 5 6)          \ev  (3 4 5 6)
((lambda (x y . z) z)
 3 4 5 6)                       \ev  (5 6)%
\end{scheme}

Any \hyper{variable} must not appear more than once in
\hyper{formals}.
\end{entry}


\subsection{Conditionals}\unsection

\begin{entry}{%
\proto{if}{ \hyper{test} \hyper{consequent} \hyper{alternate}}{\exprtype}
\rproto{if}{ \hyper{test} \hyper{consequent}}{\exprtype}}  %\/ if hyper = italic

\syntax
\hyper{Test}, \hyper{consequent}, and \hyper{alternate} must be 
expressions.

\semantics
An {\cf if} expression is evaluated as follows: first,
\hyper{test} is evaluated.  If it yields a true value\index{true} (see
section~\ref{booleanvaluessection}), then \hyper{consequent} is evaluated and
its values are returned.  Otherwise \hyper{alternate} is evaluated and its
values are returned.  If \hyper{test} yields \schfalse{} and no
\hyper{alternate} is specified, then the result of the expression \isunspecified.

\begin{scheme}
(if (> 3 2) 'yes 'no)           \ev  yes
(if (> 2 3) 'yes 'no)           \ev  no
(if (> 3 2)
    (- 3 2)
    (+ 3 2))                    \ev  1
(if \#f \#f)                    \ev \theunspecified%
\end{scheme}

The \hyper{consequent} and \hyper{alternate} expressions are in
tail context if the {\cf if} expression itself is; see
section~\ref{basetailcontextsection}.
\end{entry}


\subsection{Assignments}\unsection
\label{assignment}

\begin{entry}{%
\proto{set!}{ \hyper{variable} \hyper{expression}}{\exprtype}}

\hyper{Expression} is evaluated, and the resulting value is stored in
the location to which \hyper{variable} is bound.  \hyper{Variable} must
be bound either in some region\index{region} enclosing the {\cf set!}\ expression
or at the top level.  The result of the {\cf set!}
expression \isunspecified.

\begin{scheme}
(let ((x 2))
  (+ x 1)
  (set! x 4)
  (+ x 1)) \ev  5%
\end{scheme}

It is a syntax violation if \hyper{variable} refers to an
immutable binding.

\begin{note}
  The identifier {\cf set!} is exported with level $1$ as well.  See
  section~\ref{identifier-syntax}.
\end{note}
\end{entry}

\subsection{Derived conditionals}\unsection

\begin{entry}{%
\proto{cond}{ \hyperi{cond clause} \hyperii{cond clause} \dotsfoo}{\exprtype}
\litproto{=>}
\litproto{else}}

\syntax
Each \hyper{cond clause} must be of the form
\begin{scheme}
(\hyper{test} \hyperi{expression} \dotsfoo)%
\end{scheme}
where \hyper{test} is an expression.  Alternatively, a \hyper{cond clause} may be
of the form
\begin{scheme}
(\hyper{test} => \hyper{expression})%
\end{scheme}
The last \hyper{cond clause} may be
an ``{\cf else} clause'', which has the form
\begin{scheme}
(else \hyperi{expression} \hyperii{expression} \dotsfoo)\rm.%
\end{scheme}

\semantics
A {\cf cond} expression is evaluated by evaluating the \hyper{test}
expressions of successive \hyper{cond clause}s in order until one of them
evaluates to a true value\index{true} (see
section~\ref{booleanvaluessection}).  When a \hyper{test} evaluates to a true
value, then the remaining \hyper{expression}s in its \hyper{cond clause} are
evaluated in order, and the results of the last \hyper{expression} in the
\hyper{cond clause} are returned as the results of the entire {\cf cond}
expression.  If the selected \hyper{cond clause} contains only the
\hyper{test} and no \hyper{expression}s, then the value of the
\hyper{test} is returned as the result.  If the selected \hyper{cond clause} uses the
\ide{=>} alternate form, then the \hyper{expression} is evaluated.
Its value must be a procedure.  This procedure should accept one argument; it is
called on the value of the \hyper{test} and the values returned by this
procedure are returned by the {\cf cond} expression.
If all \hyper{test}s evaluate
to \schfalse, and there is no {\cf else} clause, then 
the conditional expression returns \unspecifiedreturn; if there is an {\cf else}
clause, then its \hyper{expression}s are evaluated, and the values of
the last one are returned.

\begin{scheme}
(cond ((> 3 2) 'greater)
      ((< 3 2) 'less))         \ev  greater%

(cond ((> 3 3) 'greater)
      ((< 3 3) 'less)
      (else 'equal))            \ev  equal%

(cond ('(1 2 3) => cadr)
      (else \schfalse{}))         \ev  2%
\end{scheme}

For a \hyper{cond clause} of one of the following forms
%
\begin{scheme}
(\hyper{test} \hyperi{expression} \dotsfoo)
(else \hyperi{expression} \hyperii{expression} \dotsfoo)%
\end{scheme}
%
the last \hyper{expression} is in tail context if the {\cf cond} form
itself is.  For a \hyper{cond clause} of the form
\begin{scheme}
(\hyper{test} => \hyper{expression})%
\end{scheme}
the (implied) call to the procedure that results from the evaluation
of \hyper{expression} is in a tail context if the {\cf cond} form
itself is. See section~\ref{basetailcontextsection}.

A sample definition of {\cf cond} in terms of simpler forms is in
appendix~\ref{derivedformsappendix}.
\end{entry}


\begin{entry}{%
\proto{case}{ \hyper{key} \hyperi{case clause} \hyperii{case clause} \dotsfoo}{\exprtype}}

\syntax
\hyper{Key} must be an expression.  Each \hyper{case clause} must have one of
the following forms:
\begin{scheme}
((\hyperi{datum} \dotsfoo) \hyperi{expression} \hyperii{expression} \dotsfoo)
(else \hyperi{expression} \hyperii{expression} \dotsfoo)%
\end{scheme}
\schindex{else}
The second form, which specifies an ``{\cf else} clause'',
may only appear as the last \hyper{case clause}.
Each \hyper{datum} is an external representation of some object.
The data represented by the \hyper{datum}s need not be distinct.

\semantics
A {\cf case} expression is evaluated as follows.  \hyper{Key} is
evaluated and its result is compared using {\cf eqv?} (see
section~\ref{eqv?}) against the data
represented by the \hyper{datum}s of each \hyper{case clause} in turn, proceeding
in order from left to right through the set of clauses.  If the
result of evaluating \hyper{key} is equivalent to a datum of a \hyper{case clause}, the
corresponding \hyper{expression}s are evaluated from left
to right and the results of the last expression in the \hyper{case clause} are
returned as the results of the {\cf case} expression.  Otherwise, the
comparison process continues.  If the result of
evaluating \hyper{key} is different from every datum in each set, then if
there is an {\cf else} clause its expressions are evaluated and the
results of the last are the results of the {\cf case} expression;
otherwise the {\cf case} expression returns \unspecifiedreturn.

\begin{scheme}
(case (* 2 3)
  ((2 3 5 7) 'prime)
  ((1 4 6 8 9) 'composite))     \ev  composite
(case (car '(c d))
  ((a) 'a)
  ((b) 'b))                     \ev  \theunspecified
(case (car '(c d))
  ((a e i o u) 'vowel)
  ((w y) 'semivowel)
  (else 'consonant))            \ev  consonant%
\end{scheme}

The last \hyper{expression} of a \hyper{case clause} is in tail
context if the {\cf case} expression itself is; see
section~\ref{basetailcontextsection}.

% A sample definition of {\cf case} in terms of simpler forms is in
% appendix~\ref{derivedformsappendix}.
\end{entry}


\begin{entry}{%
\proto{and}{ \hyperi{test} \dotsfoo}{\exprtype}}

\syntax The \hyper{test}s must be expressions.

\semantics If there are no \hyper{test}s, \schtrue{} is returned.
Otherwise, the \hyper{test} expressions are evaluated from left to
right until a \hyper{test} returns \schfalse{} or the last
\hyper{test} is reached.  In the former case, the {\cf and} expression
returns \schfalse{} without evaluating the remaining expressions.
In the latter case, the last expression is evaluated and its values
are returned.

\begin{scheme}
(and (= 2 2) (> 2 1))           \ev  \schtrue
(and (= 2 2) (< 2 1))           \ev  \schfalse
(and 1 2 'c '(f g))             \ev  (f g)
(and)                           \ev  \schtrue%
\end{scheme}

The {\cf and} keyword could be defined in terms of {\cf if} using {\cf
  syntax-rules} (see section~\ref{syntaxrulessection}) as follows:

\begin{scheme}
(define-syntax \ide{and}
  (syntax-rules ()
    ((and) \sharpfoo{t})
    ((and test) test)
    ((and test1 test2 ...)
     (if test1 (and test2 ...) \sharpfoo{f}))))%
\end{scheme}

The last \hyper{test} expression is in tail context if the {\cf and}
expression itself is; see section~\ref{basetailcontextsection}.
\end{entry}


\begin{entry}{%
\proto{or}{ \hyperi{test} \dotsfoo}{\exprtype}}

\syntax The \hyper{test}s must be expressions.

\semantics If there are no \hyper{test}s, \schfalse{} is returned.
Otherwise, the \hyper{test} expressions are evaluated from left to
right until a \hyper{test} returns a true value \var{val}
(see section~\ref{booleanvaluessection}) or the last
\hyper{test} is reached.  In the former case, the {\cf or} expression
returns \var{val} without evaluating the remaining expressions.
In the latter case, the last expression is evaluated and its values
are returned.

\begin{scheme}
(or (= 2 2) (> 2 1))            \ev  \schtrue
(or (= 2 2) (< 2 1))            \ev  \schtrue
(or \schfalse \schfalse \schfalse) \ev  \schfalse
(or '(b c) (/ 3 0))             \ev  (b c)%
\end{scheme}

The {\cf or} keyword could be defined in terms of {\cf if} using {\cf
  syntax-rules} (see section~\ref{syntaxrulessection}) as follows:

\begin{scheme}
(define-syntax \ide{or}
  (syntax-rules ()
    ((or) \sharpfoo{f})
    ((or test) test)
    ((or test1 test2 ...)
     (let ((x test1))
       (if x x (or test2 ...))))))%
\end{scheme}

The last \hyper{test} expression is in tail context if the {\cf or}
expression itself is; see section~\ref{basetailcontextsection}.
\end{entry}


\subsection{Binding constructs}

The binding constructs described in this section
create local bindings for variables that are visible only in a
delimited region.  The syntax of the 
constructs
{\cf let}, {\cf let*}, {\cf letrec}, and {\cf letrec*}
 is identical, but they differ in the regions\index{region}
(see section~\ref{variablesection}) they establish
for their variable bindings and in the order in which the values for
the bindings are computed.  In a {\cf let} expression, the initial
values are computed before any of the variables become bound; in a
{\cf let*} expression, the bindings and evaluations are performed
sequentially.  In a {\cf letrec} or {\cf letrec*}
expression, all the bindings are in
effect while their initial values are being computed, thus allowing
mutually recursive definitions.  In a {\cf letrec} expression, the
initial values are computed before being assigned to the variables;
in a {\cf letrec*}, the evaluations and assignments are performed
sequentially.

In addition, the binding constructs {\cf let-values} and {\cf
  let*-values} generalize {\cf let} and {\cf let*} to allow multiple
variables to be bound to the results of expressions that evaluate to
multiple values.
They are analogous to {\cf let} and {\cf let*} in the
way they establish regions: in a {\cf let-values} expression, the
initial values are computed before any of the variables become bound;
in a {\cf let*-values} expression, the bindings are performed
sequentially. 

Sample definitions of all the binding forms of this section in terms
of simpler forms are in appendix~\ref{derivedformsappendix}.

\begin{entry}{%
\proto{let}{ \hyper{bindings} \hyper{body}}{\exprtype}}

\syntax
\hyper{Bindings} must have the form
\begin{scheme}
((\hyperi{variable} \hyperi{init}) \dotsfoo)\rm,%
\end{scheme}
where each \hyper{init} is an expression, and \hyper{body} 
is as described in section~\ref{bodiessection}.  
Any variable must not appear more than once in the \hyper{variable}s.

\semantics
The \hyper{init}s are evaluated in the current environment (in some
unspecified order), the \hyper{variable}s are bound to fresh locations
holding the results, the \hyper{body} is evaluated in the extended
environment, and the values of the last expression of \hyper{body}
are returned.  Each binding of a \hyper{variable} has \hyper{body} as its
region.\index{region}

\begin{scheme}
(let ((x 2) (y 3))
  (* x y))                      \ev  6

(let ((x 2) (y 3))
  (let ((x 7)
        (z (+ x y)))
    (* z x)))                   \ev  35%
\end{scheme}

See also named {\cf let}, section \ref{namedlet}.

\end{entry}


\begin{entry}{%
\proto{let*}{ \hyper{bindings} \hyper{body}}{\exprtype}}\nobreak

\nobreak
\syntax
\hyper{Bindings} must have the form
\begin{scheme}
((\hyperi{variable} \hyperi{init}) \dotsfoo)\rm,%
\end{scheme}
where each \hyper{init} is an expression, and \hyper{body} 
is as described in section~\ref{bodiessection}.

\semantics
The {\cf let*} form is similar to {\cf let}, but the \hyper{init}s are
evaluated and bindings created sequentially from left to right, with
the region\index{region} of each binding including the bindings to
its right as well as \hyper{body}.  Thus the second \hyper{init} is evaluated
in an environment in which the first binding is visible and initialized,
and so on.

\begin{scheme}
(let ((x 2) (y 3))
  (let* ((x 7)
         (z (+ x y)))
    (* z x)))             \ev  70%
\end{scheme}

\begin{note}
  While the variables bound by a {\cf let} expression must be distinct,
  the variables bound by a {\cf let*} expression need not be distinct.
\end{note}
\end{entry}

\begin{entry}{%
\proto{letrec}{ \hyper{bindings} \hyper{body}}{\exprtype}}

\syntax
\hyper{Bindings} must have the form
\begin{scheme}
((\hyperi{variable} \hyperi{init}) \dotsfoo)\rm,%
\end{scheme}
where each \hyper{init} is an expression, and \hyper{body} 
is as described in section~\ref{bodiessection}.  Any
variable must not appear more than once in the
\hyper{variable}s.

\semantics
The \hyper{variable}s are bound to fresh locations, the \hyper{init}s
are evaluated in the resulting environment (in
some unspecified order), each \hyper{variable} is assigned to the result
of the corresponding \hyper{init}, the \hyper{body} is evaluated in the
resulting environment, and the values of the last expression in
\hyper{body} are returned.  Each binding of a \hyper{variable} has the
entire {\cf letrec} expression as its region\index{region}, making it possible to
define mutually recursive procedures.

\begin{scheme}
%(letrec ((x 2) (y 3))
%  (letrec ((foo (lambda (z) (+ x y z))) (x 7))
%    (foo 4)))                   \ev  14
%
(letrec ((even?
          (lambda (n)
            (if (zero? n)
                \schtrue
                (odd? (- n 1)))))
         (odd?
          (lambda (n)
            (if (zero? n)
                \schfalse
                (even? (- n 1))))))
  (even? 88))   
                \ev  \schtrue%
\end{scheme}

It should be possible
to evaluate each \hyper{init} without assigning or referring to the
value of any \hyper{variable}.  In the most
common uses of {\cf letrec}, all the \hyper{init}s are \lambdaexp{}s
and the restriction is satisfied automatically.
Another restriction is that the continuation of each \hyper{init} should not be invoked
more than once.

\implresp Implementations must detect references to a \hyper{variable} during the
evaluation of the \hyper{init} expressions (using one particular
evaluation order and order of evaluating the \hyper{init} expressions).
If an implementation detects such a violation of the
restriction, it must raise an exception with condition type
{\cf\&assertion}.
Implementations may or may not detect that the continuation of each
\hyper{init} is invoked more than once.  However, if the
implementation detects this, it must raise an exception with condition
type {\cf\&assertion}.
\end{entry}

\begin{entry}{%
\proto{letrec*}{ \hyper{bindings} \hyper{body}}{\exprtype}}

\syntax
\hyper{Bindings} must have the form
\begin{scheme}
((\hyperi{variable} \hyperi{init}) \dotsfoo)\rm,%
\end{scheme}
where each \hyper{init} is an expression, and \hyper{body} 
is as described in section~\ref{bodiessection}. 
Any variable must not appear more than once in the
\hyper{variable}s.

\semantics
The \hyper{variable}s are bound to fresh locations,
each \hyper{variable} is assigned in left-to-right order to the
result of evaluating the corresponding \hyper{init}, the \hyper{body} is
evaluated in the resulting environment, and the values of the last
expression in \hyper{body} are returned. 
Despite the left-to-right evaluation and assignment order, each binding of
a \hyper{variable} has the entire {\cf letrec*} expression as its
region\index{region}, making it possible to define mutually recursive
procedures.

\begin{scheme}
(letrec* ((p
           (lambda (x)
             (+ 1 (q (- x 1)))))
          (q
           (lambda (y)
             (if (zero? y)
                 0
                 (+ 1 (p (- y 1))))))
          (x (p 5))
          (y x))
  y)
                \ev  5%
\end{scheme}

It must be possible
to evaluate each \hyper{init} without assigning or referring to the value
of the corresponding \hyper{variable} or the \hyper{variable} of any of
the bindings that follow it in \hyper{bindings}.
Another restriction is that the continuation of each \hyper{init} should not be invoked
more than once.

\implresp Implementations must, during the evaluation of an
\hyper{init} expression, detect references to the
value of the corresponding \hyper{variable} or the \hyper{variable} of
any of the bindings that follow it in \hyper{bindings}.
If an implementation detects such a
violation of the restriction, it must raise an exception with
condition type {\cf\&assertion}.  Implementations may or may not
detect that the continuation of each \hyper{init} is invoked more than
once.  However, if the implementation detects this, it must raise an
exception with condition type {\cf\&assertion}.
\end{entry}

\begin{entry}{%
\proto{let-values}{ \hyper{mv-bindings} \hyper{body}}{\exprtype}}

\syntax
\hyper{Mv-bindings} must have the form
\begin{scheme}
((\hyperi{formals} \hyperi{init}) \dotsfoo)\rm,%
\end{scheme}
where each \hyper{init} is an expression, and \hyper{body} 
is as described in section~\ref{bodiessection}.  
Any variable must not appear more
than once in the set of \hyper{formals}.

\semantics The \hyper{init}s are evaluated in the current environment
(in some unspecified order), and the variables occurring in the
\hyper{formals} are bound to fresh locations containing the values
returned by the \hyper{init}s, where the \hyper{formals} are matched
to the return values in the same way that the \hyper{formals} in a
\lambdaexp{} are matched to the arguments in a procedure call.
Then, the \hyper{body} is evaluated in the extended environment, and the
values of the last expression of \hyper{body} are returned.
Each binding of a variable has \hyper{body} as its
region.\index{region}
If the \hyper{formals} do not match, an exception with condition type
{\cf\&assertion} is raised.

\begin{scheme}
(let-values (((a b) (values 1 2))
             ((c d) (values 3 4)))
  (list a b c d)) \ev (1 2 3 4)

(let-values (((a b . c) (values 1 2 3 4)))
  (list a b c))            \ev (1 2 (3 4))

(let ((a 'a) (b 'b) (x 'x) (y 'y))
  (let-values (((a b) (values x y))
               ((x y) (values a b)))
    (list a b x y)))       \ev (x y a b)%
\end{scheme}
\end{entry}

\begin{entry}{%
\proto{let*-values}{ \hyper{mv-bindings} \hyper{body}}{\exprtype}}

\syntax
\hyper{Mv-bindings} must have the form
\begin{scheme}
((\hyperi{formals} \hyperi{init}) \dotsfoo)\rm,%
\end{scheme}
where each \hyper{init} is an expression, and \hyper{body} 
is as described in section~\ref{bodiessection}.
In each \hyper{formals}, any variable must not appear more than once.

\semantics
The {\cf let*-values} form is similar to {\cf let-values}, but the \hyper{init}s are
evaluated and bindings created sequentially from left to right, with
the region\index{region} of the bindings of each \hyper{formals} including
the bindings to its right as well as \hyper{body}. 
Thus the second \hyper{init} is evaluated in an environment in which the
bindings of the first \hyper{formals} is visible and initialized, and so
on.

\begin{scheme}
(let ((a 'a) (b 'b) (x 'x) (y 'y))
  (let*-values (((a b) (values x y))
                ((x y) (values a b)))
    (list a b x y)))  \ev (x y x y)%
\end{scheme}

\begin{note}
  While all of the variables bound by a {\cf let-values} expression
  must be distinct, the variables bound by different \hyper{formals} of a
  {\cf let*-values} expression need not be distinct.
\end{note}

\end{entry}

\subsection{Sequencing}\unsection

\begin{entry}{%
\proto{begin}{ \hyper{form} \dotsfoo}{\exprtype}
\rproto{begin}{ \hyper{expression} \hyper{expression} \dotsfoo}{\exprtype}}

The \hyper{begin} keyword has two different roles, depending on its
context:
\begin{itemize}
\item It may appear as a form in a \hyper{body} (see
  section~\ref{bodiessection}), \hyper{library body} (see
  section~\ref{librarybodysection}), or \hyper{top-level body} (see
  chapter~\ref{programchapter}), or directly nested in a {\cf begin}
  form that appears in a body.  In this case, the {\cf begin} form
  must have the shape specified in the first header line.  This use of
  {\cf begin} acts as a \defining{splicing} form---the forms inside
  the \hyper{body} are spliced into the surrounding body, as if the
  {\cf begin} wrapper were not actually present.
  
  A {\cf begin} form in a \hyper{body} or \hyper{library body} must
  be non-empty if it appears after the first \hyper{expression}
  within the body.
\item It may appear as an ordinary expression and must have the shape
  specified in the second header line.  In this case, the
  \hyper{expression}s are evaluated sequentially from left to right,
  and the values of the last \hyper{expression} are returned.
  This expression type is used to sequence side effects such as
  assignments or input
  and output.
\end{itemize}

\begin{scheme}
(define x 0)

(begin (set! x 5)
       (+ x 1))                  \ev  6

(begin (display "4 plus 1 equals ")
       (display (+ 4 1)))      \ev  \unspecified
 \>{\em and prints}  4 plus 1 equals 5%
\end{scheme}
\end{entry}

\section{Equivalence predicates}
\label{equivalencesection}

A \defining{predicate} is a procedure that always returns a boolean
value (\schtrue{} or \schfalse).  An \defining{equivalence predicate} is
the computational analogue of a mathematical equivalence relation (it is
symmetric, reflexive, and transitive).  Of the equivalence predicates
described in this section, {\cf eq?}\ is the finest or most
discriminating, and {\cf equal?}\ is the coarsest.  The {\cf eqv?} predicate is
slightly less discriminating than {\cf eq?}.  \todo{Pitman doesn't like
this paragraph.  Lift the discussion from the Maclisp manual.  Explain
why there's more than one predicate.}


\begin{entry}{%
\proto{eqv?}{ \vari{obj} \varii{obj}}{procedure}}

The {\cf eqv?} procedure defines a useful equivalence relation on objects.
Briefly, it returns \schtrue{} if \vari{obj} and \varii{obj} should
normally be regarded as the same object and \schfalse{} otherwise.  This relation is left slightly
open to interpretation, but the following partial specification of
{\cf eqv?} must hold for all implementations.

The {\cf eqv?} procedure returns \schtrue{} if one of the following holds:

\begin{itemize}
\item \vari{Obj} and \varii{obj} are both booleans and are the same
  according to the {\cf boolean=?} procedure (section~\ref{boolean=?}).

\item \vari{Obj} and \varii{obj} are both symbols and are the same
  according to the {\cf symbol=?} procedure (section~\ref{symbol=?}).

\item \vari{Obj} and \varii{obj} are both exact\index{exact} number objects
  and are numerically equal (see {\cf =}, 
  section~\ref{genericarithmeticsection}).

\item \vari{Obj} and \varii{obj} are both inexact\index{inexact}
  number objects, are numerically
  equal (see {\cf =}, section~\ref{genericarithmeticsection}), and
  yield the same results (in the sense of {\cf eqv?}) when passed
  as arguments to any other procedure that can be defined
  as a finite composition of Scheme's standard arithmetic
  procedures.

\item \vari{Obj} and \varii{obj} are both characters and are the same
character according to the {\cf char=?} procedure
(section~\ref{charactersection}).

\item Both \vari{obj} and \varii{obj} are the empty list.

\item \vari{Obj} and \varii{obj} are objects such as pairs, vectors, bytevectors
  (library chapter~\extref{lib:bytevectorschapter}{Bytevectors}),
  strings, hashtables, records (library
  chapter~\extref{lib:recordschapter}{Records}), ports (library
  section~\extref{lib:portsiosection}{Port I/O}), or hashtables
  (library chapter~\extref{lib:hashtablechapter}{Hash tables}) that
  refer to the same locations in the store (section~\ref{storagemodel}).

\item \vari{Obj} and \varii{obj} are record-type descriptors that are
  specified to be {\cf eqv?} in library
  section~\extref{lib:recordsproceduralsection}{Procedural layer}.
\end{itemize}

The {\cf eqv?} procedure returns \schfalse{} if one of the following holds:

\begin{itemize}
\item \vari{Obj} and \varii{obj} are of different types
(section~\ref{disjointness}).

\item \vari{Obj} and \varii{obj} are booleans for which the {\cf
    boolean=?} procedure returns \schfalse.

\item \vari{Obj} and \varii{obj} are symbols for which the {\cf
    symbol=?} procedure returns \schfalse.

\item One of \vari{obj} and \varii{obj} is an exact number object but the other is
        an inexact number object.

\item \vari{Obj} and \varii{obj} are rational number objects for which the {\cf =} procedure
  returns \schfalse{}.

\item \vari{Obj} and \varii{obj} yield different results (in the sense of
  {\cf eqv?}) when passed as arguments to any other procedure
  that can be defined as a finite composition of Scheme's
  standard arithmetic procedures.

\item \vari{Obj} and \varii{obj} are characters for which the {\cf char=?}
  procedure returns \schfalse{}.

\item One of \vari{obj} and \varii{obj} is the empty list, but the other is not.

\item \vari{Obj} and \varii{obj} are objects such as pairs, vectors,
  bytevectors (library
  chapter~\extref{lib:bytevectorschapter}{Bytevectors}), strings,
  records (library
  chapter~\extref{lib:recordschapter}{Records}), ports (library
  section~\extref{lib:portsiosection}{Port I/O}), or hashtables
  (library chapter~\extref{lib:hashtablechapter}{Hashtables}) that
  refer to distinct locations.

\item \vari{Obj} and \varii{obj} are pairs, vectors, strings, or
  records, or hashtables, where the applying the same accessor (i.e.\
  {\cf car}, {\cf cdr}, {\cf vector-ref}, {\cf string-ref}, or record
  accessors) to both yields results for which {\cf eqv?} returns
  \schfalse.

\item \vari{Obj} and \varii{obj} are procedures that would behave differently
(return different values or have different side effects) for some arguments.

\end{itemize}

\begin{note}
  The {\cf eqv?} procedure returning \schtrue{} when \vari{obj} and
  \varii{obj} are number objects does not imply that {\cf =} would also
  return \schtrue{} when called with \vari{obj} and \varii{obj} as
  arguments.
\end{note}


\begin{scheme}
(eqv? 'a 'a)                     \ev  \schtrue
(eqv? 'a 'b)                     \ev  \schfalse
(eqv? 2 2)                       \ev  \schtrue
(eqv? '() '())                   \ev  \schtrue
(eqv? 100000000 100000000)       \ev  \schtrue
(eqv? (cons 1 2) (cons 1 2))     \ev  \schfalse
(eqv? (lambda () 1)
      (lambda () 2))             \ev  \schfalse
(eqv? \#f 'nil)                  \ev  \schfalse%
\end{scheme}

The following examples illustrate cases in which the above rules do
not fully specify the behavior of {\cf eqv?}.  All that can be said
about such cases is that the value returned by {\cf eqv?} must be a
boolean.

\begin{scheme}
(let ((p (lambda (x) x)))
  (eqv? p p))                    \ev  \unspecified
(eqv? "" "")             \ev  \unspecified
(eqv? '\#() '\#())         \ev  \unspecified
(eqv? (lambda (x) x)
      (lambda (x) x))    \ev  \unspecified
(eqv? (lambda (x) x)
      (lambda (y) y))    \ev  \unspecified
(eqv? +nan.0 +nan.0)             \ev \unspecified%
\end{scheme}

The next set of examples shows the use of {\cf eqv?}\ with procedures
that have local state.  Calls to {\cf gen-counter} must return a
distinct procedure every time, since each procedure has its own
internal counter.  Calls to {\cf gen-loser} return procedures that
behave equivalently when called.  However, {\cf eqv?} may
not detect this equivalence.

\begin{scheme}
(define gen-counter
  (lambda ()
    (let ((n 0))
      (lambda () (set! n (+ n 1)) n))))
(let ((g (gen-counter)))
  (eqv? g g))           \ev  \unspecified
(eqv? (gen-counter) (gen-counter))
                        \ev  \schfalse
(define gen-loser
  (lambda ()
    (let ((n 0))
      (lambda () (set! n (+ n 1)) 27))))
(let ((g (gen-loser)))
  (eqv? g g))           \ev  \unspecified
(eqv? (gen-loser) (gen-loser))
                        \ev  \unspecified

(letrec ((f (lambda () (if (eqv? f g) 'both 'f)))
         (g (lambda () (if (eqv? f g) 'both 'g))))
  (eqv? f g)) \ev  \unspecified

(letrec ((f (lambda () (if (eqv? f g) 'f 'both)))
         (g (lambda () (if (eqv? f g) 'g 'both))))
  (eqv? f g)) \ev  \schfalse%
\end{scheme}

Implementations may
share structure between constants where appropriate.
Furthermore, a constant may be copied at any time by the implementation so
as to exist simultaneously in different sets of locations, as noted in
section~\ref{quote}.
Thus the value of {\cf eqv?} on constants is sometimes
implementation-dependent.

\begin{scheme}
(eqv? '(a) '(a))                 \ev  \unspecified
(eqv? "a" "a")                   \ev  \unspecified
(eqv? '(b) (cdr '(a b)))         \ev  \unspecified
(let ((x '(a)))
  (eqv? x x))                    \ev  \schtrue%
\end{scheme}
\end{entry}


\begin{entry}{%
\proto{eq?}{ \vari{obj} \varii{obj}}{procedure}}

The {\cf eq?} predicate is similar to {\cf eqv?}\ except that in some cases it is
capable of discerning distinctions finer than those detectable by
{\cf eqv?}.

The {\cf eq?}\ and {\cf eqv?} predicates are guaranteed to have the
same behavior on symbols, booleans, the empty list, pairs, procedures,
non-empty strings, bytevectors, and vectors, and records.  The
behavior of {\cf eq?} on number objects and characters is
implementation-dependent, but it always returns either \schtrue{} or
\schfalse{}, and returns \schtrue{} only when {\cf eqv?}\ would also
return \schtrue.  The {\cf eq?} predicate may also behave differently
from {\cf eqv?} on empty vectors, empty bytevectors, and empty strings.

\begin{scheme}
(eq? 'a 'a)                     \ev  \schtrue
(eq? '(a) '(a))                 \ev  \unspecified
(eq? (list 'a) (list 'a))       \ev  \schfalse
(eq? "a" "a")                   \ev  \unspecified
(eq? "" "")                     \ev  \unspecified
(eq? '() '())                   \ev  \schtrue
(eq? 2 2)                       \ev  \unspecified
(eq? \#\backwhack{}A \#\backwhack{}A) \ev  \unspecified
(eq? car car)                   \ev  \schtrue
(let ((n (+ 2 3)))
  (eq? n n))      \ev  \unspecified
(let ((x '(a)))
  (eq? x x))      \ev  \schtrue
(let ((x '\#()))
  (eq? x x))      \ev  \unspecified
(let ((p (lambda (x) x)))
  (eq? p p))      \ev  \unspecified%
\end{scheme}

\todo{Needs to be explained better above.  How can this be made to be
not confusing?  A table maybe?}

\end{entry}

\begin{entry}{%
\proto{equal?}{ \vari{obj} \varii{obj}}{procedure}}

The {\cf equal?}  predicate returns \schtrue{} if and only if the
(possibly infinite) unfoldings of its arguments into regular trees are
equal as ordered trees.

The {\cf equal?} predicate treats pairs and vectors
as nodes with outgoing edges, uses {\cf
  string=?} to compare strings, uses {\cf
  bytevector=?} to compare bytevectors (see library chapter~\extref{lib:bytevectorschapter}{Bytevectors}),
  and uses {\cf eqv?} to compare other nodes.

\begin{scheme}
(equal? 'a 'a)                  \ev  \schtrue
(equal? '(a) '(a))              \ev  \schtrue
(equal? '(a (b) c)
        '(a (b) c))             \ev  \schtrue
(equal? "abc" "abc")            \ev  \schtrue
(equal? 2 2)                    \ev  \schtrue
(equal? (make-vector 5 'a)
        (make-vector 5 'a))     \ev  \schtrue
(equal? '\#vu8(1 2 3 4 5)
        (u8-list->bytevector
         '(1 2 3 4 5))          \ev  \schtrue
(equal? (lambda (x) x)
        (lambda (y) y))  \ev  \unspecified

(let* ((x (list 'a))
       (y (list 'a))
       (z (list x y)))
  (list (equal? z (list y x))
        (equal? z (list x x))))             \lev  (\schtrue{} \schtrue{})%
\end{scheme}

\begin{note}
  The {\cf equal?} procedure must always terminate, even if its
  arguments contain cycles.
\end{note}

\end{entry}

\section{Procedure predicate}

\begin{entry}{%
\proto{procedure?}{ obj}{procedure}}

Returns \schtrue{} if \var{obj} is a procedure, otherwise returns \schfalse.

\begin{scheme}
(procedure? car)            \ev  \schtrue
(procedure? 'car)           \ev  \schfalse
(procedure? (lambda (x) (* x x)))   
                            \ev  \schtrue
(procedure? '(lambda (x) (* x x)))  
                            \ev  \schfalse%
\end{scheme}

\end{entry}

\section{Arithmetic}
\label{genericarithmeticsection}

The procedures described here implement arithmetic that is
generic over
the numerical tower described in chapter~\ref{numbertypeschapter}.
The generic procedures described in this section
accept both exact and inexact number objects as arguments,
performing coercions and selecting the appropriate operations
as determined by the numeric subtypes of their arguments.

Library chapter~\extref{lib:numberchapter}{Arithmetic} describes
libraries that define other numerical procedures.

\subsection{Propagation of exactness and inexactness}
\label{propagationsection}

The procedures listed below must return the mathematically correct exact result
provided all their arguments are exact:

\begin{scheme}
+            -            *
max          min          abs
numerator    denominator  gcd
lcm          floor        ceiling
truncate     round        rationalize
real-part    imag-part    make-rectangular%
\end{scheme}

The procedures listed below must return the correct exact result
provided all their arguments are exact, and no divisors are zero:

\begin{scheme}
/
div          mod           div-and-mod
div0         mod0          div0-and-mod0%
\end{scheme}

Moreover, the procedure {\cf expt} must return the correct exact
result provided its first argument is an exact real number object and
its second argument is an exact integer object.

The general rule is that the generic operations return the correct
exact result when all of their arguments are exact and the result is
mathematically well-defined, but return an inexact result when any
argument is inexact.  Exceptions to this rule include
{\cf sqrt}, {\cf exp}, {\cf log},
{\cf sin}, {\cf cos}, {\cf tan},
{\cf asin}, {\cf acos}, {\cf atan},
{\cf expt}, {\cf make-polar}, {\cf magnitude}, and {\cf angle}, which
may (but are not required to) return inexact results even when
given exact arguments, as indicated in the specification of these
procedures.

One general exception to the rule above is that an implementation may
return an exact result despite inexact arguments if that exact result
would be the correct result for all possible substitutions of exact
arguments for the inexact ones.  An example is {\cf (* 1.0 0)} which
may return either {\cf 0} (exact) or {\cf 0.0} (inexact).

\subsection{Representability of infinities and NaNs}
\label{infinitiesnanssection}

The specification of the numerical operations is written as though
infinities and NaNs are representable, and specifies many operations
with respect to these number objects in ways that are consistent with the
IEEE-754 standard for binary floating-point arithmetic.  
An implementation of Scheme may or may not represent infinities and
NaNs; however,
an implementation must raise a continuable exception with
condition type {\cf\&no-infinities} or {\cf\&no-nans} (respectively;
see library section~\extref{lib:flonumssection}{Flonums})
whenever it is unable to represent an infinity or NaN as specified. 
In this case, the continuation of the exception
handler is the continuation that otherwise would have received
the infinity or NaN value.  This requirement also applies to
conversions between number objects and external representations, including
the reading of program source code.

\subsection{Semantics of common operations}

Some operations are the semantic basis for several arithmetic
procedures.  The behavior of these operations is described in this
section for later reference.

\subsubsection{Integer division}
\label{integerdivision}

Scheme's operations for performing integer
division rely on mathematical operations $\mathrm{div}$,
$\mathrm{mod}$, $\mathrm{div}_0$, and
$\mathrm{mod}_0$, that are defined as follows:

$\mathrm{div}$, $\mathrm{mod}$, $\mathrm{div}_0$, and $\mathrm{mod}_0$
each accept two real numbers $x_1$ and $x_2$ as operands, where
$x_2$ must be nonzero.

$\mathrm{div}$ returns an integer, and $\mathrm{mod}$ returns a real.
Their results are specified by
%
\begin{eqnarray*}
x_1~\mathrm{div}~x_2 &=& n_d\\
x_1~\mathrm{mod}~x_2 &=& x_m
\end{eqnarray*}
%
where
%
\begin{displaymath}
\begin{array}{c}
x_1 = n_d \cdot x_2 + x_m\\
0 \leq x_m < |x_2|
\end{array}
\end{displaymath}
%
Examples:
\begin{eqnarray*}
123~\mathrm{div}~10    &=&  12\\
123~\mathrm{mod}~10    &=&  3\\
123~\mathrm{div}~\textrm{$-10$}   &=&  -12\\
123~\mathrm{mod}~\textrm{$-10$}   &=&  3\\
-123~\mathrm{div}~10    &=&  -13\\
-123~\mathrm{mod}~10    &=&  7\\
-123~\mathrm{div}~\textrm{$-10$}   &=&  13\\
-123~\mathrm{mod}~\textrm{$-10$}   &=&  7
\end{eqnarray*}
%
$\mathrm{div}_0$ and $\mathrm{mod}_0$ are like $\mathrm{div}$ and
$\mathrm{mod}$, except the result of $\mathrm{mod}_0$ lies within a
half-open interval centered on zero.  The results are specified by
%
\begin{eqnarray*}
x_1~\mathrm{div}_0~x_2 &=& n_d\\
x_1~\mathrm{mod}_0~x_2 &=& x_m
\end{eqnarray*}
%
where:
%
\begin{displaymath}
\begin{array}{c}
x_1 = n_d \cdot x_2 + x_m\\
-|\frac{x_2}{2}| \leq x_m < |\frac{x_2}{2}|
\end{array}
\end{displaymath}
%
Examples:
%
\begin{eqnarray*}
123~\mathrm{div}_0~10    &=&  12\\
123~\mathrm{mod}_0~10    &=&  3\\
123~\mathrm{div}_0~\textrm{$-10$}   &=&  -12\\
123~\mathrm{mod}_0~\textrm{$-10$}   &=&  3\\
-123~\mathrm{div}_0~10    &=&  -12\\
-123~\mathrm{mod}_0~10    &=&  -3\\
-123~\mathrm{div}_0~\textrm{$-10$}   &=&  12\\
-123~\mathrm{mod}_0~\textrm{$-10$}   &=&  -3
\end{eqnarray*}

\subsubsection{Transcendental functions}
\label{transcendentalfunctions}

In general, the transcendental functions $\log$, $\sin^{-1}$
(arcsine), $\cos^{-1}$ (arccosine), and $\tan^{-1}$ are multiply
defined.  The value of $\log z$ is defined to be the one whose
imaginary part lies in the range from $-\pi$ (inclusive if $-0.0$ is
distinguished, exclusive otherwise) to $\pi$ (inclusive).  $\log 0$ is
undefined.

The value of $\log z$ for non-real $z$ is defined in terms of log on real numbers as 

\begin{displaymath}
\log z = \log |z| + (\mathrm{angle}~z)i
\end{displaymath}
%
where $\mathrm{angle}~z$ is the angle of $z = a\cdot e^{ib}$ specified
as:
$$\mathrm{angle}~z = b+2\pi n$$
with $-\pi \leq \mathrm{angle}~z\leq \pi$ and $\mathrm{angle}~z =
b+2\pi n$ for some integer $n$.

With the one-argument version of $\log$ defined this way, the values
of the two-argument-version of $\log$, $\sin^{-1} z$, $\cos^{-1} z$,
$\tan^{-1} z$, and the two-argument version of $\tan^{-1}$ are
according to the following formul\ae:
\begin{eqnarray*}
\log z~b &=& \frac{\log z}{\log b}\\
\sin^{-1} z &=& -i \log (i z + \sqrt{1 - z^2})\\
\cos^{-1} z &=& \pi / 2 - \sin^{-1} z\\
\tan^{-1} z &=& (\log (1 + i z) - \log (1 - i z)) / (2 i)\\
\tan^{-1} x~y &=& \mathrm{angle}(x+ yi)
\end{eqnarray*}

The range of $\tan^{-1} x~y$ is as in the following table. The
asterisk (*) indicates that the entry applies to implementations that
distinguish minus zero.

\begin{center}
\begin{tabular}{clll}
& $y$ condition & $x$ condition & range of result $r$\\\hline
& $y = 0.0$ & $x > 0.0$ & $0.0$\\
$\ast$ & $y = +0.0$  & $x > 0.0$ & $+0.0$\\     
$\ast$ & $y = -0.0$ & $x > 0.0$ & $-0.0$\\
& $y > 0.0$ & $x > 0.0$ & $0.0 < r < \frac{\pi}{2}$\\
& $y > 0.0$ & $x = 0.0$ & $\frac{\pi}{2}$\\
& $y > 0.0$ & $x < 0.0$ & $\frac{\pi}{2} < r < \pi$\\
& $y = 0.0$ & $x < 0$ & $\pi$\\
$\ast$ & $y = +0.0$ & $x < 0.0$ & $\pi$\\
$\ast$ & $y = -0.0$ & $x < 0.0$ & $-\pi$\\      
&$y < 0.0$ & $x < 0.0$ & $-\pi< r< -\frac{\pi}{2}$\\
&$y < 0.0$ & $x = 0.0$ & $-\frac{\pi}{2}$\\
&$y < 0.0$ & $x > 0.0$ & $-\frac{\pi}{2} < r< 0.0$\\    
&$y = 0.0$ & $x = 0.0$ & undefined\\
$\ast$& $y = +0.0$ & $x = +0.0$ & $+0.0$\\
$\ast$& $y = -0.0$ & $x = +0.0$& $-0.0$\\
$\ast$& $y = +0.0$ & $x = -0.0$ & $\pi$\\
$\ast$& $y = -0.0$ & $x = -0.0$ & $-\pi$\\
$\ast$& $y = +0.0$ & $x = 0$ & $\frac{\pi}{2}$\\
$\ast$& $y = -0.0$ & $x = 0$    & $-\frac{\pi}{2}$
\end{tabular}
\end{center}

\subsection{Numerical operations}

\subsubsection{Numerical type predicates}

\begin{entry}{%
\proto{number?}{ obj}{procedure}
\proto{complex?}{ obj}{procedure}
\proto{real?}{ obj}{procedure}
\proto{rational?}{ obj}{procedure}
\proto{integer?}{ obj}{procedure}}

These numerical type predicates can be applied to any kind of
argument.  They return \schtrue{} if the object is a number object
of the named type, and \schfalse{} otherwise.
In general, if a type predicate is true of a number object then all higher
type predicates are also true of that number object.  Consequently, if a type
predicate is false of a number object, then all lower type predicates are
also false of that number object.

If \var{z} is a complex number object, then {\cf (real? \var{z})} is true if
and only if {\cf (zero? (imag-part \var{z}))} and {\cf (exact?
  (imag-part \var{z}))} are both true.

If \var{x} is a real number object, then {\cf (rational? \var{x})} is true if
and only if there exist exact integer objects \vari{k} and \varii{k} such that
{\cf (= \var{x} (/ \vari{k} \varii{k}))} and {\cf (= (numerator
  \var{x}) \vari{k})} and {\cf (= (denominator \var{x}) \varii{k})} are
all true.  Thus infinities and NaNs are not rational number objects.

If \var{q} is a rational number objects, then {\cf (integer?
\var{q})} is true if and only if {\cf (= (denominator
\var{q}) 1)} is true.  If \var{q} is not a rational number object,
then {\cf (integer? \var{q})} is \schfalse.

\begin{scheme}
(complex? 3+4i)                        \ev  \schtrue{}
(complex? 3)                           \ev  \schtrue{}
(real? 3)                              \ev  \schtrue{}
(real? -2.5+0.0i)                      \ev  \schfalse{}
(real? -2.5+0i)                        \ev  \schtrue{}
(real? -2.5)                           \ev  \schtrue{}
(real? \sharpsign{}e1e10)                         \ev  \schtrue{}
(rational? 6/10)                       \ev  \schtrue{}
(rational? 6/3)                        \ev  \schtrue{}
(rational? 2)                          \ev  \schtrue{}
(integer? 3+0i)                        \ev  \schtrue{}
(integer? 3.0)                         \ev  \schtrue{}
(integer? 8/4)                         \ev  \schtrue{}

(number? +nan.0)                       \ev  \schtrue{}
(complex? +nan.0)                      \ev  \schtrue{}
(real? +nan.0)                         \ev  \schtrue{}
(rational? +nan.0)                     \ev  \schfalse{}
(complex? +inf.0)                      \ev  \schtrue{}
(real? -inf.0)                         \ev  \schtrue{}
(rational? -inf.0)                     \ev  \schfalse{}
(integer? -inf.0)                      \ev  \schfalse{}%
\end{scheme}

\begin{note}
Except for {\cf number?}, the behavior of these type predicates
on inexact number objects is
unreliable, because any inaccuracy may
affect the result.
\end{note}
\end{entry}

\begin{entry}{%
\proto{real-valued?}{ obj}{procedure}
\proto{rational-valued?}{ obj}{procedure}
\proto{integer-valued?}{ obj}{procedure}}

These numerical type predicates can be applied to any kind of
argument.  The {\cf real-valued?} procedure
returns \schtrue{} if the object is a number object and is equal in the
sense of {\cf =} to some real number object, or if the object is a NaN, or a
complex number object whose real part is a NaN and whose imaginary
part is zero
in the sense of {\cf zero?}.  The {\cf rational-valued?} and {\cf
  integer-valued?} procedures return \schtrue{} if the object is a
number object and is equal in the sense of {\cf =} to some object of the
named type, and otherwise they return \schfalse{}.

\begin{scheme}
(real-valued? +nan.0)                  \ev  \schtrue{}
(real-valued? +nan.0+0i)                  \ev  \schtrue{}
(real-valued? -inf.0)                  \ev  \schtrue{}
(real-valued? 3)                       \ev  \schtrue{}
(real-valued? -2.5+0.0i)               \ev  \schtrue{}
(real-valued? -2.5+0i)                 \ev  \schtrue{}
(real-valued? -2.5)                    \ev  \schtrue{}
(real-valued? \sharpsign{}e1e10)                  \ev  \schtrue{}

(rational-valued? +nan.0)              \ev  \schfalse{}
(rational-valued? -inf.0)              \ev  \schfalse{}
(rational-valued? 6/10)                \ev  \schtrue{}
(rational-valued? 6/10+0.0i)           \ev  \schtrue{}
(rational-valued? 6/10+0i)             \ev  \schtrue{}
(rational-valued? 6/3)                 \ev  \schtrue{}

(integer-valued? 3+0i)                 \ev  \schtrue{}
(integer-valued? 3+0.0i)               \ev  \schtrue{}
(integer-valued? 3.0)                  \ev  \schtrue{}
(integer-valued? 3.0+0.0i)             \ev  \schtrue{}
(integer-valued? 8/4)                  \ev  \schtrue{}%
\end{scheme}

\begin{note}
  These procedures test whether a given number object can be coerced
  to the specified type without loss of numerical accuracy.
  Specifically, the behavior of these predicates differs from the
  behavior of {\cf real?}, {\cf rational?}, and {\cf integer?} on
  complex number objects whose imaginary part is inexact zero.
\end{note}

\begin{note}
The behavior of these type predicates on inexact number objects is
unreliable, because any inaccuracy may
affect the result.
\end{note}
\end{entry}

\begin{entry}{%
\proto{exact?}{ z}{procedure}
\proto{inexact?}{ z}{procedure}}

These numerical predicates provide tests for the exactness of a
quantity.  For any number object, precisely one of these predicates is
true.

\begin{scheme}
(exact? 5)                   \ev  \schtrue{}
(inexact? +inf.0)            \ev  \schtrue{}%
\end{scheme}
\end{entry}

\subsubsection{Generic conversions}

\begin{entry}{%
\proto{inexact}{ z}{procedure}
\proto{exact}{ z}{procedure}}

The {\cf inexact} procedure returns an inexact representation of \var{z}.  If
inexact number objects of the appropriate type have bounded precision, then
the value returned is an inexact number object that is nearest to the
argument.  If an exact argument has no reasonably close inexact
equivalent, an exception with condition type
{\cf\&implementation-violation} may be
raised.

\begin{note}
  For a real number object whose magnitude is finite but so large that it has
  no reasonable finite approximation as an inexact number, a
  reasonably close inexact equivalent may be {\cf +inf.0} or {\cf
    -inf.0}.  Similarly, the inexact representation of a complex
  number object whose components are finite may have infinite components.
\end{note}

The {\cf exact} procedure returns an exact representation of \var{z}.  The value
returned is the exact number object that is numerically closest to the
argument; in most cases, the result of this procedure should be
numerically equal to its argument.  If an inexact argument has no
reasonably close exact equivalent, an exception with condition type
{\cf\&implementation-violation} may be
raised.

These procedures implement the natural one-to-one correspondence
between exact and inexact integer objects throughout an
implementation-dependent range.

The {\cf inexact} and {\cf exact} procedures are idempotent.
\end{entry}

\subsubsection{Arithmetic operations}

\begin{entry}{%
\proto{=}{ \vari{z} \varii{z} \variii{z} \dotsfoo}{procedure}
\proto{<}{ \vari{x} \varii{x} \variii{x} \dotsfoo}{procedure}
\proto{>}{ \vari{x} \varii{x} \variii{x} \dotsfoo}{procedure}
\proto{<=}{ \vari{x} \varii{x} \variii{x} \dotsfoo}{procedure}
\proto{>=}{ \vari{x} \varii{x} \variii{x} \dotsfoo}{procedure}}

These procedures return \schtrue{} if their arguments are
(respectively): equal, monotonically increasing, monotonically
decreasing, monotonically nondecreasing, or monotonically
nonincreasing, and \schfalse{} otherwise.

\begin{scheme}
(= +inf.0 +inf.0)           \ev  \schtrue{}
(= -inf.0 +inf.0)           \ev  \schfalse{}
(= -inf.0 -inf.0)           \ev  \schtrue{}%
\end{scheme}

For any real number object \var{x} that is neither infinite nor NaN:

\begin{scheme}
(< -inf.0 \var{x} +inf.0))        \ev  \schtrue{}
(> +inf.0 \var{x} -inf.0))        \ev  \schtrue{}%
\end{scheme}

For any number object \var{z}:
%
\begin{scheme}
(= +nan.0 \var{z})               \ev  \schfalse{}%
\end{scheme}
%
For any real number object \var{x}:
%
\begin{scheme}
(< +nan.0 \var{x})               \ev  \schfalse{}
(> +nan.0 \var{x})               \ev  \schfalse{}%
\end{scheme}

These predicates must be transitive.

\begin{note}
The traditional implementations of these predicates in Lisp-like
languages are not transitive.
\end{note}

\begin{note}
While it is possible to compare inexact number objects using these
predicates, the results may be unreliable because a small inaccuracy
may affect the result; this is especially true of {\cf =} and {\cf zero?} (below).

When in doubt, consult a numerical analyst.
\end{note}
\end{entry}

\begin{entry}{%
\proto{zero?}{ z}{procedure}
\proto{positive?}{ x}{procedure}
\proto{negative?}{ x}{procedure}
\proto{odd?}{ n}{procedure}
\proto{even?}{ n}{procedure}
\proto{finite?}{ x}{procedure}
\proto{infinite?}{ x}{procedure}
\proto{nan?}{ x}{procedure}}

These numerical predicates test a number object for a particular property,
returning \schtrue{} or \schfalse{}.  The {\cf zero?}
procedure
tests if the number object is {\cf =} to zero, {\cf positive?} tests whether it is
greater than zero, {\cf negative?} tests whether it is less than zero, {\cf
  odd?} tests whether it is odd, {\cf even?} tests whether it is even, {\cf
  finite?} tests whether it is not an infinity and not a NaN, {\cf
  infinite?} tests whether it is an infinity, {\cf nan?} tests whether it is a
NaN.

\begin{scheme}
(zero? +0.0)                  \ev  \schtrue{}
(zero? -0.0)                  \ev  \schtrue{}
(zero? +nan.0)                \ev  \schfalse{}
(positive? +inf.0)            \ev  \schtrue{}
(negative? -inf.0)            \ev  \schtrue{}
(positive? +nan.0)            \ev  \schfalse{}
(negative? +nan.0)            \ev  \schfalse{}
(finite? +inf.0)              \ev  \schfalse{}
(finite? 5)                   \ev  \schtrue{}
(finite? 5.0)                 \ev  \schtrue{}
(infinite? 5.0)               \ev  \schfalse{}
(infinite? +inf.0)            \ev  \schtrue{}%
\end{scheme}

\begin{note}
  As with the predicates above, the results may be unreliable because
  a small inaccuracy may affect the result.
\end{note}
\end{entry}

\begin{entry}{%
\proto{max}{ \vari{x} \varii{x} \dotsfoo}{procedure}
\proto{min}{ \vari{x} \varii{x} \dotsfoo}{procedure}}

These procedures return the maximum or minimum of their arguments.

\begin{scheme}
(max 3 4)                              \ev  4
(max 3.9 4)                            \ev  4.0%
\end{scheme}

For any real number object \var{x}:

\begin{scheme}
(max +inf.0 \var{x})                         \ev  +inf.0
(min -inf.0 \var{x})                         \ev  -inf.0%
\end{scheme}

\begin{note}
If any argument is inexact, then the result is also inexact (unless
the procedure can prove that the inaccuracy is not large enough to affect the
result, which is possible only in unusual implementations).  If {\cf min} or
{\cf max} is used to compare number objects of mixed exactness, and the numerical
value of the result cannot be represented as an inexact number object without loss of
accuracy, then the procedure may raise an exception with condition
type {\cf\&implementation-restriction}.
\end{note}

\end{entry}

\begin{entry}{%
\proto{+}{ \vari{z} \dotsfoo}{procedure}
\proto{*}{ \vari{z} \dotsfoo}{procedure}}

These procedures return the sum or product of their arguments.

\begin{scheme}
(+ 3 4)                                \ev  7
(+ 3)                                  \ev  3
(+)                                    \ev  0
(+ +inf.0 +inf.0)                      \ev  +inf.0
(+ +inf.0 -inf.0)                      \ev  +nan.0

(* 4)                                  \ev  4
(*)                                    \ev  1
(* 5 +inf.0)                           \ev  +inf.0
(* -5 +inf.0)                          \ev  -inf.0
(* +inf.0 +inf.0)                      \ev  +inf.0
(* +inf.0 -inf.0)                      \ev  -inf.0
(* 0 +inf.0)                           \ev  0 \textnormal{\textit{or}} +nan.0
(* 0 +nan.0)                           \ev  0 \textnormal{\textit{or}} +nan.0
(* 1.0 0)                              \ev  0 \textnormal{\textit{or}} 0.0%
\end{scheme}

For any real number object \var{x} that is neither infinite nor NaN:

\begin{scheme}
(+ +inf.0 \var{x})                           \ev  +inf.0
(+ -inf.0 \var{x})                           \ev  -inf.0%
\end{scheme}

For any real number object \var{x}:

\begin{scheme}
(+ +nan.0 \var{x})                           \ev  +nan.0%
\end{scheme}

For any real number object \var{x} that is not an exact 0:

\begin{scheme}
(* +nan.0 \var{x})                           \ev  +nan.0%
\end{scheme}

If any of these procedures are applied to mixed non-rational real and
non-real complex arguments, they either raise an exception with
condition type {\cf\&implementation-restriction} or return an
unspecified number object.

Implementations that distinguish $-0.0$ should adopt behavior
consistent with the following examples:

\begin{scheme}
(+ 0.0 -0.0)  \ev 0.0
(+ -0.0 0.0)  \ev 0.0
(+ 0.0 0.0)   \ev 0.0
(+ -0.0 -0.0) \ev -0.0%
\end{scheme}
\end{entry}

\begin{entry}{%
\proto{-}{ z}{procedure}
\rproto{-}{ \vari{z} \varii{z} \dotsfoo}{procedure}}

With two or more arguments, this procedures returns the difference of
its arguments, associating to the left.  With one argument, however,
it returns the additive inverse of its argument.

\begin{scheme}
(- 3 4)                                \ev  -1
(- 3 4 5)                              \ev  -6
(- 3)                                  \ev  -3
(- +inf.0 +inf.0)                      \ev  +nan.0%
\end{scheme}

If this procedure is applied to mixed non-rational real and
non-real complex arguments, it either raises an exception with
condition type {\cf\&implementation-restriction} or returns an
unspecified number object.

Implementations that distinguish $-0.0$ should adopt behavior
consistent with the following examples:

\begin{scheme}
(- 0.0)       \ev -0.0
(- -0.0)      \ev 0.0
(- 0.0 -0.0)  \ev 0.0
(- -0.0 0.0)  \ev -0.0
(- 0.0 0.0)   \ev 0.0
(- -0.0 -0.0) \ev 0.0%
\end{scheme}
\end{entry}

\begin{entry}{%
\proto{/}{ z}{procedure}
\rproto{/}{ \vari{z} \varii{z} \dotsfoo}{procedure}}

\domain{If all of the arguments are exact, then the divisors must all
  be nonzero.}
With two or more arguments, this procedure returns the 
quotient of its arguments, associating to the left.  With one
argument, however, it returns the multiplicative inverse
of its argument.

\begin{scheme}
(/ 3 4 5)                              \ev  3/20
(/ 3)                                  \ev  1/3
(/ 0.0)                                \ev  +inf.0
(/ 1.0 0)                              \ev  +inf.0
(/ -1 0.0)                             \ev  -inf.0
(/ +inf.0)                             \ev  0.0
(/ 0 0)                                \xev \exception{\&assertion}
(/ 3 0)                                \xev \exception{\&assertion}
(/ 0 3.5)                              \ev  0.0
(/ 0 0.0)                              \ev  +nan.0
(/ 0.0 0)                              \ev  +nan.0
(/ 0.0 0.0)                            \ev  +nan.0%
\end{scheme}

If this procedure is applied to mixed non-rational real and
non-real complex arguments, it either raises an exception with
condition type {\cf\&implementation-restriction} or returns an
unspecified number object.
\end{entry}

\begin{entry}{%
\proto{abs}{ x}{procedure}}

Returns the absolute value of its argument.

\begin{scheme}
(abs -7)                               \ev  7
(abs -inf.0)                           \ev  +inf.0%
\end{scheme}

\end{entry}

\begin{entry}{%
\proto{div-and-mod}{ \vari{x} \varii{x}}{procedure}
\proto{div}{ \vari{x} \varii{x}}{procedure}
\proto{mod}{ \vari{x} \varii{x}}{procedure}
\proto{div0-and-mod0}{ \vari{x} \varii{x}}{procedure}
\proto{div0}{ \vari{x} \varii{x}}{procedure}
\proto{mod0}{ \vari{x} \varii{x}}{procedure}}

These procedures implement number-theoretic integer division and
return the results of the corresponding mathematical operations
specified in section~\ref{integerdivision}.  In each case, \vari{x}
must be neither infinite nor a NaN, and \varii{x} must be nonzero;
otherwise, an exception with condition type {\cf\&assertion} is raised.

\begin{scheme}
(div \vari{x} \varii{x})         \ev \(\vari{x}~\mathrm{div}~\varii{x}\)
(mod \vari{x} \varii{x})         \ev \(\vari{x}~\mathrm{mod}~\varii{x}\)
(div-and-mod \vari{x} \varii{x})     \ev \(\vari{x}~\mathrm{div}~\varii{x}, \vari{x}~\mathrm{mod}~\varii{x}\)\\\>\>\>; \textrm{two return values}
(div0 \vari{x} \varii{x})        \ev \(\vari{x}~\mathrm{div}_0~\varii{x}\)
(mod0 \vari{x} \varii{x})        \ev \(\vari{x}~\mathrm{mod}_0~\varii{x}\)
(div0-and-mod0 \vari{x} \varii{x})   \lev \(\vari{x}~\mathrm{div}_0~\varii{x}, \vari{x}~\mathrm{mod}_0~\varii{x}\)\\\>\>; \textrm{two return values}%
\end{scheme}

\begin{entry}{%
\proto{gcd}{ \vari{n} \dotsfoo}{procedure}
\proto{lcm}{ \vari{n} \dotsfoo}{procedure}}

These procedures return the greatest common divisor or least common
multiple of their arguments.  The result is always non-negative.

\begin{scheme}
(gcd 32 -36)                           \ev  4
(gcd)                                  \ev  0
(lcm 32 -36)                           \ev  288
(lcm 32.0 -36)                         \ev  288.0
(lcm)                                  \ev  1%
\end{scheme}
\end{entry}

\begin{entry}{%
\proto{numerator}{ q}{procedure}
\proto{denominator}{ q}{procedure}}

These procedures return the numerator or denominator of their
argument; the result is computed as if the argument was represented as
a fraction in lowest terms.  The denominator is always positive.  The
denominator of $0$ is defined to be $1$.

\begin{scheme}
(numerator (/ 6 4))                    \ev  3
(denominator (/ 6 4))                  \ev  2
(denominator
  (inexact (/ 6 4)))                   \ev  2.0%
\end{scheme}
\end{entry}

\begin{entry}{%
\proto{floor}{ x}{procedure}
\proto{ceiling}{ x}{procedure}
\proto{truncate}{ x}{procedure}
\proto{round}{ x}{procedure}}

These procedures return inexact integer objects for inexact arguments that are
not infinities or NaNs, and exact integer objects for exact rational
arguments.  For such arguments, {\cf floor} returns the largest
integer object not larger than \var{x}.  The {\cf ceiling} procedure returns the smallest
integer object not smaller than \var{x}.  The {\cf truncate} procedure returns the integer
object closest to \var{x} whose absolute value is not larger than the
absolute value of \var{x}.  The {\cf round} procedure returns the
closest integer object to
\var{x}, rounding to even when \var{x} represents a number halfway between two
integers.

\begin{note}
If the argument to one of these procedures is inexact, then the result
is also inexact.  If an exact value is needed, the
result should be passed to the {\cf exact} procedure.
\end{note}

Although infinities and NaNs are not integer objects, these procedures return
an infinity when given an infinity as an argument, and a NaN when
given a NaN.

\begin{scheme}
(floor -4.3)                           \ev  -5.0
(ceiling -4.3)                         \ev  -4.0
(truncate -4.3)                        \ev  -4.0
(round -4.3)                           \ev  -4.0

(floor 3.5)                            \ev  3.0
(ceiling 3.5)                          \ev  4.0
(truncate 3.5)                         \ev  3.0
(round 3.5)                            \ev  4.0

(round 7/2)                            \ev  4
(round 7)                              \ev  7

(floor +inf.0)                         \ev  +inf.0
(ceiling -inf.0)                       \ev  -inf.0
(round +nan.0)                         \ev  +nan.0%
\end{scheme}

\end{entry}

\begin{entry}{%
\proto{rationalize}{ \vari{x} \varii{x}}{procedure}}

The {\cf rationalize} procedure returns the a number object
representing the {\em simplest} rational
number differing from \vari{x} by no more than \varii{x}.    A rational number $r_1$ is
{\em simpler} \mainindex{simplest rational} than another rational number
$r_2$ if $r_1 = p_1/q_1$ and $r_2 = p_2/q_2$ (in lowest terms) and $|p_1|
\leq |p_2|$ and $|q_1| \leq |q_2|$.  Thus $3/5$ is simpler than $4/7$.
Although not all rationals are comparable in this ordering (consider $2/7$
and $3/5$) any interval contains a rational number that is simpler than
every other rational number in that interval (the simpler $2/5$ lies
between $2/7$ and $3/5$).  Note that $0 = 0/1$ is the simplest rational of
all.
%
\begin{scheme}
(rationalize (exact .3) 1/10)          \lev 1/3
(rationalize .3 1/10)                  \lev \sharpsign{}i1/3  ; \textrm{approximately}

(rationalize +inf.0 3)                 \ev  +inf.0
(rationalize +inf.0 +inf.0)            \ev  +nan.0
(rationalize 3 +inf.0)                 \ev  0.0%
\end{scheme}
%
The first two examples hold only in implementations whose inexact real
number objects have sufficient precision.

\end{entry}

\begin{entry}{%
\proto{exp}{ z}{procedure}
\proto{log}{ z}{procedure}
\rproto{log}{ \vari{z} \varii{z}}{procedure}
\proto{sin}{ z}{procedure}
\proto{cos}{ z}{procedure}
\proto{tan}{ z}{procedure}
\proto{asin}{ z}{procedure}
\proto{acos}{ z}{procedure}
\proto{atan}{ z}{procedure}
\rproto{atan}{ \vari{x} \varii{x}}{procedure}}

These procedures compute the usual transcendental functions.  The {\cf
  exp} procedure computes the base-$e$ exponential of \var{z}. 
The {\cf log} procedure with a single argument computes the natural logarithm of
\var{z} (not the base-ten logarithm); {\cf (log \vari{z}
  \varii{z})} computes the base-\varii{z} logarithm of \vari{z}.
The {\cf asin}, {\cf acos}, and {\cf atan} procedures compute arcsine,
arccosine, and arctangent, respectively.  The two-argument variant of
{\cf atan} computes {\cf (angle (make-rectangular \varii{x}
\vari{x}))}.

See section~\ref{transcendentalfunctions} for the underlying
mathematical operations. These procedures may return inexact results
even when given exact arguments.

\begin{scheme}
(exp +inf.0)                   \ev +inf.0
(exp -inf.0)                   \ev 0.0
(log +inf.0)                   \ev +inf.0
(log 0.0)                      \ev -inf.0
(log 0)                        \xev \exception{\&assertion}
(log -inf.0)                   \lev +inf.0+3.141592653589793i\\\> ; \textrm{approximately}
(atan -inf.0)                  \lev -1.5707963267948965 ; \textrm{approximately}
(atan +inf.0)                  \lev 1.5707963267948965 ; \textrm{approximately}
(log -1.0+0.0i)                \lev 0.0+3.141592653589793i ; \textrm{approximately}
(log -1.0-0.0i)                \lev 0.0-3.141592653589793i ; \textrm{approximately}\\\>; \textrm{if -0.0 is distinguished}%
\end{scheme}
\end{entry}

\begin{entry}{%
\proto{sqrt}{ z}{procedure}}

Returns the principal square root of \var{z}.  For rational \var{z},
the result has either positive real part, or zero real part and
non-negative imaginary part.  With $\log$ defined as in
section~\ref{transcendentalfunctions}, the value of {\cf (sqrt
  \var{z})} could be expressed as $e^{\frac{\log z}{2}}$.

The {\cf sqrt} procedure may return an inexact result even when given an exact
argument.

\begin{scheme}
(sqrt -5)                   \lev  0.0+2.23606797749979i ; \textrm{approximately}
(sqrt +inf.0)               \ev  +inf.0
(sqrt -inf.0)               \ev  +inf.0i%
\end{scheme}
\end{entry}

\begin{entry}{%
\proto{exact-integer-sqrt}{ k}{procedure}}

The {\cf exact-integer-sqrt} procedure returns two non-negative exact
integer objects $s$ and $r$ where $\var{k} = s^2 +
r$ and $\var{k} < (s+1)^2$.

\begin{scheme}
(exact-integer-sqrt 4) \ev 2 0\\\>\>\>; \textrm{two return values}
(exact-integer-sqrt 5) \ev 2 1\\\>\>\>; \textrm{two return values}
\end{scheme}
\end{entry}

\begin{entry}{%
\proto{expt}{ \vari{z} \varii{z}}{procedure}}

Returns \vari{z} raised to the power \varii{z}.  For nonzero \vari{z},
this is $e^{z_2 \log z_1}$.
$0.0^{z}$ is $1.0$ if $\var{z} = 0.0$, and $0.0$ if {\cf
  (real-part \var{z})} is positive.  For other cases in which
the first argument is zero, either an exception is raised with
condition type {\cf\&implementation-restriction}, or an unspecified
number object is returned.

For an exact real number object \vari{z} and an exact
integer object \varii{z}, {\cf (expt \vari{z}
\varii{z})} must return an exact result.  For all other
values of \vari{z} and \varii{z}, {\cf (expt \vari{z}
\varii{z})} may return an inexact result, even when both
\vari{z} and \varii{z} are exact.

\begin{scheme}
(expt 5 3)                  \ev  125
(expt 5 -3)                 \ev  1/125
(expt 5 0)                  \ev  1
(expt 0 5)                  \ev  0
(expt 0 5+.0000312i)        \ev  0
(expt 0 -5)                 \ev  \unspecified
(expt 0 -5+.0000312i)       \ev  \unspecified
(expt 0 0)                  \ev  1
(expt 0.0 0.0)              \ev  1.0%
\end{scheme}
\end{entry}

\begin{entry}{%
\proto{make-rectangular}{ \vari{x} \varii{x}}{procedure}
\proto{make-polar}{ \variii{x} \variv{x}}{procedure}
\proto{real-part}{ z}{procedure}
\proto{imag-part}{ z}{procedure}
\proto{magnitude}{ z}{procedure}
\proto{angle}{ z}{procedure}}

Suppose $a_1$, $a_2$, $a_3$, and $a_4$ are real
numbers, and $c$ is a complex number such that the
following holds:
%
\begin{displaymath}
c = a_1 + a_2 i = a_3 e^{i a_4}
\end{displaymath}

Then, if \vari{x}, \varii{x}, \variii{x}, and \variv{x} are number
objects representing $a_1$, $a_2$, $a_3$, and $a_4$, respectively,
{\cf (make-rectangular \vari{x} \varii{x})} returns $c$, and {\cf
  (make-polar \variii{x} \variv{x})} returns $c$.
%
\begin{scheme}
(make-rectangular 1.1 2.2) \lev 1.1+2.2i ; \textrm{approximately}
(make-polar 1.1 2.2) \lev 1.1@2.2 ; \textrm{approximately}
\end{scheme}
%
Conversely, if $-\pi \leq a_4 \leq \pi$, and if $z$ is a number object
representing $c$, then {\cf (real-part \var{z})} returns $a_1$ {\cf
  (imag-part \var{z})} returns $a_2$, {\cf (magnitude \var{z})}
returns $a_3$, and {\cf (angle \var{z})} returns $a_4$.

\begin{scheme}
(real-part 1.1+2.2i)              \ev 1.1 ; \textrm{approximately}
(imag-part 1.1+2.2i)              \ev 2.2i ; \textrm{approximately}
(magnitude 1.1@2.2)              \ev 1.1 ; \textrm{approximately}
(angle 1.1@2.2)                  \ev 2.2 ; \textrm{approximately}

(angle -1.0)         \lev 3.141592653589793 ; \textrm{approximately}
(angle -1.0+0.0i)    \lev 3.141592653589793 ; \textrm{approximately}
(angle -1.0-0.0i)    \lev -3.141592653589793 ; \textrm{approximately}\\\>; \textrm{if -0.0 is distinguished}
(angle +inf.0)       \ev 0.0
(angle -inf.0)       \lev 3.141592653589793 ; \textrm{approximately}%
\end{scheme}

Moreover, suppose \vari{x}, \varii{x} are such that either \vari{x}
or \varii{x} is an infinity, then
%
\begin{scheme}
(make-rectangular \vari{x} \varii{x}) \ev \var{z}
(magnitude \var{z})              \ev +inf.0%
\end{scheme}
\end{entry}

The {\cf make-polar}, {\cf magnitude}, and
{\cf angle} procedures may return inexact results even when given exact
arguments.

\begin{scheme}
(angle -1)                    \lev 3.141592653589793 ; \textrm{approximately}
\end{scheme}
\end{entry}

\subsubsection{Numerical Input and Output}

\begin{entry}{%
\proto{number->string}{ z}{procedure}
\rproto{number->string}{ z radix}{procedure}
\rproto{number->string}{ z radix precision}{procedure}}

\var{Radix} must be an exact integer object, either 2, 8, 10, or 16.  If
omitted, \var{radix} defaults to 10.  If a \var{precision} is
specified, then \var{z} must be an inexact complex number object,
\var{precision} must be an exact positive integer object, and \var{radix}
must be 10.  The {\cf number->string} procedure takes a number object and a
radix and returns as a string an external representation of the given
number object in the given radix such that
%
\begin{scheme}
(let ((number \var{z}) (radix \var{radix}))
  (eqv? (string->number
          (number->string number radix)
          radix)
        number))%
\end{scheme}
%
is true.  If no possible result makes this expression
true, an exception with condition type
{\cf\&implementation-\hp{}restriction} is raised.

\begin{note}
The error case can occur only when \var{z} is not a complex number object
or is a complex number object with a non-rational real or imaginary part.
\end{note}

If a \var{precision} is specified, then the representations of the
inexact real components of the result, unless they are infinite or
NaN, specify an explicit \meta{mantissa width} \var{p}, and \var{p} is the
least $\var{p} \geq \var{precision}$ for which the above expression is
true.

If \var{z} is inexact, the radix is 10, and the above expression and
condition can be satisfied by a result that contains a decimal point,
then the result contains a decimal point and is expressed using the
minimum number of digits (exclusive of exponent, trailing zeroes, and
mantissa width) needed to make the above expression and condition
true~\cite{howtoprint,howtoread}; otherwise the format of the result
is unspecified.

The result returned by {\cf number->string} never contains an explicit
radix prefix.
\end{entry}

\begin{entry}{%
\proto{string->number}{ string}{procedure}
\rproto{string->number}{ string radix}{procedure}}

Returns a number object with maximally precise representation expressed by the
given \var{string}.  \var{Radix} must be an exact integer object, either 2, 8, 10,
or 16.  If supplied, \var{radix} is a default radix that may be overridden
by an explicit radix prefix in \var{string} (e.g., {\tt "\#o177"}).  If \var{radix}
is not supplied, then the default radix is 10.  If \var{string} is not
a syntactically valid notation for a number object or a notation for a
rational number object with a zero denominator, then {\cf string->number}
returns \schfalse{}.
%
\begin{scheme}
(string->number "100")                 \ev  100
(string->number "100" 16)              \ev  256
(string->number "1e2")                 \ev  100.0
(string->number "0/0")                 \ev  \schfalse
(string->number "+inf.0")              \ev  +inf.0
(string->number "-inf.0")              \ev  -inf.0
(string->number "+nan.0")              \ev  +nan.0%
\end{scheme}

\begin{note}
  The {\cf string->number} procedure always returns a number object or
  \schfalse{}; it never raises an exception.
\end{note}
\end{entry}


\section{Booleans}
\label{booleansection}

The standard boolean objects for true and false have external representations
\schtrue{} and \schfalse.\sharpindex{t}\sharpindex{f} However, of all
objects, only \schfalse{} counts as false in
conditional expressions.  See section~\ref{booleanvaluessection}.

\begin{note}
Programmers accustomed to other dialects of Lisp should be aware that
Scheme distinguishes both \schfalse{} and the empty list \index{empty list}
from each other and from the symbol \ide{nil}.
\end{note}

\begin{entry}{%
\proto{not}{ obj}{procedure}}

Returns \schtrue{} if \var{obj} is \schfalse, and returns
\schfalse{} otherwise.

\begin{scheme}
(not \schtrue)   \ev  \schfalse
(not 3)          \ev  \schfalse
(not (list 3))   \ev  \schfalse
(not \schfalse)  \ev  \schtrue
(not '())        \ev  \schfalse
(not (list))     \ev  \schfalse
(not 'nil)       \ev  \schfalse%
\end{scheme}

\end{entry}


\begin{entry}{%
\proto{boolean?}{ obj}{procedure}}

Returns \schtrue{} if \var{obj} is either \schtrue{} or
\schfalse{} and returns \schfalse{} otherwise.

\begin{scheme}
(boolean? \schfalse)  \ev  \schtrue
(boolean? 0)          \ev  \schfalse
(boolean? '())        \ev  \schfalse%
\end{scheme}

\begin{entry}{%
\proto{boolean=?}{ \vari{bool} \varii{bool} \variii{bool}
  \dotsfoo}{procedure}}

Returns \schtrue{} if the booleans are the same.
\end{entry}

\end{entry}

 
\section{Pairs and lists}
\label{listsection}

A \defining{pair} is a
compound structure with two fields called the car and cdr fields (for
historical reasons).  Pairs are created by the procedure {\cf cons}.
The car and cdr fields are accessed by the procedures {\cf car} and
{\cf cdr}.

Pairs are used primarily to represent lists.  A list can
be defined recursively as either the empty list\index{empty list} or a pair whose
cdr is a list.  More precisely, the set of lists is defined as the smallest
set \var{X} such that

\begin{itemize}
\item The empty list is in \var{X}.
\item If \var{list} is in \var{X}, then any pair whose cdr field contains
      \var{list} is also in \var{X}.
\end{itemize}

The objects in the car fields of successive pairs of a list are the
elements of the list.  For example, a two-element list is a pair whose car
is the first element and whose cdr is a pair whose car is the second element
and whose cdr is the empty list.  The length of a list is the number of
elements, which is the same as the number of pairs.

The empty list\mainindex{empty list} is a special object of its own type.
It is not a pair.  It has no elements and its length is zero.

\begin{note}
The above definitions imply that all lists have finite length and are
terminated by the empty list.
\end{note}

A chain of pairs not ending in the empty list is called an
\defining{improper list}.  Note that an improper list is not a list.
The list and dotted notations can be combined to represent
improper lists:

\begin{scheme}
(a b c . d)%
\end{scheme}

is equivalent to

\begin{scheme}
(a . (b . (c . d)))%
\end{scheme}

Whether a given pair is a list depends upon what is stored in the cdr
field.

\begin{entry}{%
\proto{pair?}{ obj}{procedure}}

Returns \schtrue{} if \var{obj} is a pair, and otherwise
returns \schfalse.

\begin{scheme}
(pair? '(a . b))        \ev  \schtrue
(pair? '(a b c))        \ev  \schtrue
(pair? '())             \ev  \schfalse
(pair? '\#(a b))         \ev  \schfalse%
\end{scheme}
\end{entry}


\begin{entry}{%
\proto{cons}{ \vari{obj} \varii{obj}}{procedure}}

Returns a newly allocated pair whose car is \vari{obj} and whose cdr is
\varii{obj}.  The pair is guaranteed to be different (in the sense of
{\cf eqv?}) from every existing object.

\begin{scheme}
(cons 'a '())           \ev  (a)
(cons '(a) '(b c d))    \ev  ((a) b c d)
(cons "a" '(b c))       \ev  ("a" b c)
(cons 'a 3)             \ev  (a . 3)
(cons '(a b) 'c)        \ev  ((a b) . c)%
\end{scheme}
\end{entry}


\begin{entry}{%
\proto{car}{ pair}{procedure}}

Returns the contents of the car field of \var{pair}.

\begin{scheme}
(car '(a b c))          \ev  a
(car '((a) b c d))      \ev  (a)
(car '(1 . 2))          \ev  1
(car '())               \xev \exception{\&assertion}%
\end{scheme}
 
\end{entry}


\begin{entry}{%
\proto{cdr}{ pair}{procedure}}

Returns the contents of the cdr field of \var{pair}.

\begin{scheme}
(cdr '((a) b c d))      \ev  (b c d)
(cdr '(1 . 2))          \ev  2
(cdr '())               \xev \exception{\&assertion}%
\end{scheme}
 
\end{entry}



\setbox0\hbox{\tt(cadr \var{pair})}
\setbox1\hbox{procedure}


\begin{entry}{%
\proto{caar}{ pair}{procedure}
\proto{cadr}{ pair}{procedure}
\texonly
\pproto{\hbox to 1\wd0 {\hfil$\vdots$\hfil}}{\hbox to 1\wd1 {\hfil$\vdots$\hfil}}
\endtexonly
\htmlonly $\vdots$ \endhtmlonly
\proto{cdddar}{ pair}{procedure}
\proto{cddddr}{ pair}{procedure}}

These procedures are compositions of {\cf car} and {\cf cdr}, where
for example {\cf caddr} could be defined by

\begin{scheme}
(define caddr (lambda (x) (car (cdr (cdr x))))){\rm.}%
\end{scheme}

Arbitrary compositions, up to four deep, are provided.  There are
twenty-eight of these procedures in all.

\end{entry}


\begin{entry}{%
\proto{null?}{ obj}{procedure}}

Returns \schtrue{} if \var{obj} is the empty list\index{empty list},
\schfalse otherwise.

\end{entry}

\begin{entry}{%
\proto{list?}{ obj}{procedure}}

Returns \schtrue{} if \var{obj} is a list, \schfalse{} otherwise.
By definition, all lists are chains of pairs that have finite length and are terminated by
the empty list.

\begin{scheme}
(list? '(a b c))     \ev  \schtrue
(list? '())          \ev  \schtrue
(list? '(a . b))     \ev  \schfalse%
\end{scheme}
\end{entry}


\begin{entry}{%
\proto{list}{ \var{obj} \dotsfoo}{procedure}}

Returns a newly allocated list of its arguments.

\begin{scheme}
(list 'a (+ 3 4) 'c)            \ev  (a 7 c)
(list)                          \ev  ()%
\end{scheme}
\end{entry}


\begin{entry}{%
\proto{length}{ list}{procedure}}

Returns the length of \var{list}.

\begin{scheme}
(length '(a b c))               \ev  3
(length '(a (b) (c d e)))       \ev  3
(length '())                    \ev  0%
\end{scheme}
\end{entry}


\begin{entry}{%
\proto{append}{ list \dotsfoo{} obj}{procedure}}

Returns a possibly improper list consisting of the elements of the first \var{list}
followed by the elements of the other \var{list}s, with \var{obj} as
the cdr of the final pair.
An improper list results if \var{obj} is not a
list.

\begin{scheme}
(append '(x) '(y))              \ev  (x y)
(append '(a) '(b c d))          \ev  (a b c d)
(append '(a (b)) '((c)))        \ev  (a (b) (c))
(append '(a b) '(c . d))        \ev  (a b c . d)
(append '() 'a)                 \ev  a%
\end{scheme}

If {\cf append} constructs a nonempty chain of pairs, it is always
newly allocated.  If no pairs are allocated, \var{obj} is returned.
\end{entry}


\begin{entry}{%
\proto{reverse}{ list}{procedure}}

Returns a newly allocated list consisting of the elements of \var{list}
in reverse order.

\begin{scheme}
(reverse '(a b c))              \ev  (c b a)
(reverse '(a (b c) d (e (f))))  \lev  ((e (f)) d (b c) a)%
\end{scheme}
\end{entry}


\begin{entry}{%
\proto{list-tail}{ list k}{procedure}}

\domain{\var{List} should be a list of size at least \var{k}.}
The {\cf list-tail} procedure returns the subchain of pairs of \var{list}
obtained by omitting the first \var{k} elements.

\begin{scheme}
(list-tail '(a b c d) 2)                 \ev  (c d)%
\end{scheme}

\implresp The implementation must check that \var{list} is a chain of
pairs whose length is at least \var{k}.  It should not check that it is a chain
of pairs beyond this length.
\end{entry}


\begin{entry}{%
\proto{list-ref}{ list k}{procedure}}

\domain{\var{List} must be a list whose length is at least $\var{k}+1$.}
The {\cf list-tail} procedure returns the \var{k}th element of \var{list}.

\begin{scheme}
(list-ref '(a b c d) 2)                 \ev c%
\end{scheme}

\implresp The implementation must check that \var{list} is a chain of
pairs whose length is at least $\var{k}+1$.  It should not check that it is a list
of pairs beyond this length.
\end{entry}


\begin{entry}{%
\proto{map}{ proc \vari{list} \varii{list} \dotsfoo}{procedure}}

\domain{The \var{list}s should all have the same length.  \var{Proc}
  should accept as many arguments as there are
  \var{list}s and return a single value.  \var{Proc} should not mutate
  any of the \var{list}s.}

The {\cf map} procedure applies \var{proc} element-wise to the elements of the
\var{list}s and returns a list of the results, in order.
\var{Proc} is always called in the same dynamic environment 
as {\cf map} itself.
The order in which \var{proc} is applied to the elements of the
\var{list}s is unspecified.
If multiple returns occur from {\cf map}, the 
values returned by earlier returns are not mutated.

\begin{scheme}
(map cadr '((a b) (d e) (g h)))   \lev  (b e h)

(map (lambda (n) (expt n n))
     '(1 2 3 4 5))                \lev  (1 4 27 256 3125)

(map + '(1 2 3) '(4 5 6))         \ev  (5 7 9)

(let ((count 0))
  (map (lambda (ignored)
         (set! count (+ count 1))
         count)
       '(a b)))                 \ev  (1 2) \var{or} (2 1)%
\end{scheme}

\implresp The implementation should check that the \var{list}s all
have the same length.  The implementation must check the restrictions
on \var{proc} to the extent performed by applying it as described.  An
implementation may check whether \var{proc} is an appropriate argument
before applying it.
\end{entry}


\begin{entry}{%
\proto{for-each}{ proc \vari{list} \varii{list} \dotsfoo}{procedure}}

\domain{The \var{list}s should all have the same length.  \var{Proc}
  should accept as many arguments as there are
  \var{list}s.  \var{Proc} should not mutate
  any of the \var{list}s.}

The {\cf for-each} procedure applies \var{proc}
element-wise to the elements of the
\var{list}s for its side effects,  in order from the first elements to the
last.
\var{Proc} is always called in the same dynamic environment 
as {\cf for-each} itself.
The return values of {\cf for-each} \areunspecified.

\begin{scheme}
(let ((v (make-vector 5)))
  (for-each (lambda (i)
              (vector-set! v i (* i i)))
            '(0 1 2 3 4))
  v)                                \ev  \#(0 1 4 9 16)

(for-each (lambda (x) x) '(1 2 3 4)) \lev \theunspecified

(for-each even? '()) \ev \theunspecified%
\end{scheme}

\implresp The implementation should check that the \var{list}s all
have the same length.  The implementation must check the restrictions
on \var{proc} to the extent performed by applying it as described.
An implementation may check whether \var{proc} is an appropriate argument
before applying it.

\begin{note}
Implementations of {\cf for-each} may or may not tail-call
\var{proc} on the last elements.
\end{note}

\end{entry}


\section{Symbols}
\label{symbolsection}

Symbols are objects whose usefulness rests on the fact that two
symbols are identical (in the sense of {\cf eq?}, {\cf eqv?} and {\cf equal?}) if and only if their
names are spelled the same way. 
A symbol literal is formed using {\cf quote}.

\begin{entry}{%
\proto{symbol?}{ obj}{procedure}}

Returns \schtrue{} if \var{obj} is a symbol, otherwise returns \schfalse.

\begin{scheme}
(symbol? 'foo)          \ev  \schtrue
(symbol? (car '(a b)))  \ev  \schtrue
(symbol? "bar")         \ev  \schfalse
(symbol? 'nil)          \ev  \schtrue
(symbol? '())           \ev  \schfalse
(symbol? \schfalse)     \ev  \schfalse%
\end{scheme}
\end{entry}


\begin{entry}{%
\proto{symbol->string}{ symbol}{procedure}}

Returns the name of \var{symbol} as an immutable string.  

\begin{scheme}
(symbol->string 'flying-fish)     
                                  \ev  "flying-fish"
(symbol->string 'Martin)          \ev  "Martin"
(symbol->string
   (string->symbol "Malvina"))     
                                  \ev  "Malvina"%
\end{scheme}
\end{entry}

\begin{entry}{%
\proto{symbol=?}{ \vari{symbol} \varii{symbol} \variii{symbol}
  \dotsfoo}{procedure}}

Returns \schtrue{} if the symbols are the same, i.e., if their names
are spelled the same.
\end{entry}

\begin{entry}{%
\proto{string->symbol}{ string}{procedure}}

Returns the symbol whose name is \var{string}. 

\begin{scheme}
(eq? 'mISSISSIppi 'mississippi)  \lev  \schfalse
(string->symbol "mISSISSIppi")  \lev%
  {\rm{}the symbol with name} "mISSISSIppi"
(eq? 'bitBlt (string->symbol "bitBlt"))     \lev  \schtrue
(eq? 'JollyWog
     (string->symbol
       (symbol->string 'JollyWog)))  \lev  \schtrue
(string=? "K. Harper, M.D."
          (symbol->string
            (string->symbol "K. Harper, M.D.")))  \lev  \schtrue%
\end{scheme}

\end{entry}


\section{Characters}
\label{charactersection}

\mainindex{Unicode}
\mainindex{scalar value}

\defining{Characters} are objects that represent Unicode scalar
values~\cite{Unicode}.

\begin{note}
  Unicode defines a standard mapping between sequences of
  \textit{Unicode scalar values}\mainindex{Unicode scalar
    value}\mainindex{scalar value} (integers in the range 0 to
  \#x10FFFF, excluding the range \#xD800 to \#xDFFF) in the latest
  version of the standard and human-readable ``characters''. More
  precisely, Unicode distinguishes between glyphs, which are printed
  for humans to read, and characters, which are abstract entities that
  map to glyphs (sometimes in a way that's sensitive to surrounding
  characters).  Furthermore, different sequences of scalar values
  sometimes correspond to the same character.  The relationships among
  scalar, characters, and glyphs are subtle and complex.

  Despite this complexity, most things that a literate human would
  call a ``character'' can be represented by a single Unicode scalar
  value (although several sequences of Unicode scalar values may
  represent that same character). For example, Roman letters, Cyrillic
  letters, Hebrew consonants, and most Chinese characters fall into
  this category.
  
  Unicode scalar values exclude the range \#xD800 to \#xDFFF, which
  are part of the range of Unicode \textit{code points}\mainindex{code
    point}.  However, the Unicode code points in this range,
  the so-called \textit{surrogates}\mainindex{surrogate}, are an
  artifact of the UTF-16 encoding, and can only appear in specific
  Unicode encodings, and even then only in pairs that encode scalar
  values.  Consequently, all characters represent code points, but the
  surrogate code points do not have representations as characters.
\end{note}

\begin{entry}{%
\proto{char?}{ obj}{procedure}}

Returns \schtrue{} if \var{obj} is a character, otherwise returns \schfalse.

\end{entry}

\begin{entry}{%
\proto{char->integer}{ char}{procedure}
\proto{integer->char}{ \vr{sv}}{procedure}}

\domain{\var{Sv} must be a Unicode scalar value, i.e., a non-negative exact
  integer object in $\left[0, \#x\textrm{D7FF}\right] \cup
  \left[\#x\textrm{E000}, \#x\textrm{10FFFF}\right]$.}

Given a character, {\cf char\coerce{}integer} returns its Unicode scalar value
as an exact integer object.  
For a Unicode scalar value \var{sv}, {\cf integer\coerce{}char}
returns its associated character.

\begin{scheme}
(integer->char 32) \ev \sharpsign\backwhack{}space
(char->integer (integer->char 5000))
\ev 5000
(integer->char \sharpsign{}\backwhack{}xD800) \xev \exception{\&assertion}%
\end{scheme}
\end{entry}


\begin{entry}{%
\proto{char=?}{ \vari{char} \varii{char} \variii{char} \dotsfoo}{procedure}
\proto{char<?}{ \vari{char} \varii{char} \variii{char} \dotsfoo}{procedure}
\proto{char>?}{ \vari{char} \varii{char} \variii{char} \dotsfoo}{procedure}
\proto{char<=?}{ \vari{char} \varii{char} \variii{char} \dotsfoo}{procedure}
\proto{char>=?}{ \vari{char} \varii{char} \variii{char} \dotsfoo}{procedure}}

\label{characterequality}
These procedures impose a total ordering on the set of characters
according to their Unicode scalar values.

\begin{scheme}
(char<? \sharpsign\backwhack{}z \sharpsign\backwhack{}\ss) \ev \schtrue
(char<? \sharpsign\backwhack{}z \sharpsign\backwhack{}Z) \ev \schfalse%
\end{scheme}

\end{entry}

\section{Strings}
\label{stringsection}

Strings are sequences of characters.  

\vest The {\em length} of a string is the number of characters that it
contains.  This number is fixed when the
string is created.  The \defining{valid indices} of a string are the
integers less than the length of the string.  The first
character of a string has index 0, the second has index 1, and so on.

\begin{entry}{%
\proto{string?}{ obj}{procedure}}

Returns \schtrue{} if \var{obj} is a string, otherwise returns \schfalse.
\end{entry}


\begin{entry}{%
\proto{make-string}{ k}{procedure}
\rproto{make-string}{ k char}{procedure}}

Returns a newly allocated string of
length \var{k}.  If \var{char} is given, then all elements of the string
are initialized to \var{char}, otherwise the contents of the
\var{string} are unspecified.

\end{entry}

\begin{entry}{%
\proto{string}{ char \dotsfoo}{procedure}}

Returns a newly allocated string composed of the arguments.

\end{entry}

\begin{entry}{%
\proto{string-length}{ string}{procedure}}

Returns the number of characters in the given \var{string} as an exact
integer object.
\end{entry}


\begin{entry}{%
\proto{string-ref}{ string k}{procedure}}

\domain{\var{K} must be a valid index of \var{string}.}
The {\cf string-ref} procedure returns character \vr{k} of \var{string} using zero-origin indexing.

\begin{note}
  Implementors should make {\cf string-ref} run in constant
  time.
\end{note}
\end{entry}

\begin{entry}{%
\proto{string=?}{ \vari{string} \varii{string} \variii{string} \dotsfoo}{procedure}}

Returns \schtrue{} if the strings are the same length and contain the same
characters in the same positions.  Otherwise, the {\cf string=?}
procedure returns \schfalse.

\begin{scheme}
(string=? "Stra\ss{}e" "Strasse") \lev \schfalse%
\end{scheme}
\end{entry}

\begin{entry}{%
\proto{string<?}{ \vari{string} \varii{string} \variii{string} \dotsfoo}{procedure}
\proto{string>?}{ \vari{string} \varii{string} \variii{string} \dotsfoo}{procedure}
\proto{string<=?}{ \vari{string} \varii{string} \variii{string} \dotsfoo}{procedure}
\proto{string>=?}{ \vari{string} \varii{string} \variii{string} \dotsfoo}{procedure}}

These procedures are the lexicographic extensions to strings of the
corresponding orderings on characters.  For example, {\cf string<?}\ is
the lexicographic ordering on strings induced by the ordering
{\cf char<?}\ on characters.  If two strings differ in length but
are the same up to the length of the shorter string, the shorter string
is considered to be lexicographically less than the longer string.

\begin{scheme}
(string<? "z" "\ss") \ev \schtrue
(string<? "z" "zz") \ev \schtrue
(string<? "z" "Z") \ev \schfalse%
\end{scheme}
\end{entry}


\begin{entry}{%
\proto{substring}{ string start end}{procedure}}

\domain{\var{String} must be a string, and \var{start} and \var{end}
must be exact integer objects satisfying
$$0 \leq \var{start} \leq \var{end} \leq \hbox{\tt(string-length \var{string})\rm.}$$}
The {\cf substring} procedure returns a newly allocated string formed from the characters of
\var{string} beginning with index \var{start} (inclusive) and ending with index
\var{end} (exclusive).
\end{entry}


\begin{entry}{%
\proto{string-append}{ \var{string} \dotsfoo}{procedure}}

Returns a newly allocated string whose characters form the concatenation of the
given strings.
\end{entry}


\begin{entry}{%
\proto{string->list}{ string}{procedure}
\proto{list->string}{ list}{procedure}}

\domain{\var{List} must be a list of characters.}
The {\cf string\coerce{}list} procedure returns a newly allocated list of the
characters that make up the given string.  The {\cf
  list\coerce{}string} procedure
returns a newly allocated string formed from the characters in 
\var{list}. The {\cf string\coerce{}list}
and {\cf list\coerce{}string} procedures are
inverses so far as {\cf equal?}\ is concerned.  
\end{entry}

\begin{entry}{%
\proto{string-for-each}{ proc \vari{string} \varii{string} \dotsfoo}{procedure}}

\domain{The \var{string}s must all have the same length.  \var{Proc}
  should accept as many arguments as there are {\it string}s.}
The {\cf string-for-each} procedure applies \var{proc}
element-wise to the characters of the
\var{string}s for its side effects,  in order from the first characters to the
last.
\var{Proc} is always called in the same dynamic environment 
as {\cf string-for-each} itself.
The return values of {\cf string-for-each} \areunspecified.

Analogous to {\cf for-each}.

\implresp The implementation must check the restrictions
on \var{proc} to the extent performed by applying it as described.
An
implementation may check whether \var{proc} is an appropriate argument
before applying it.
\end{entry}

\begin{entry}{%
\proto{string-copy}{ string}{procedure}}

Returns a newly allocated copy of the given \var{string}.

\end{entry}

\section{Vectors}
\label{vectorsection}

Vectors are heterogeneous structures whose elements are indexed
by integers.  A vector typically occupies less space than a list
of the same length, and the average time needed to access a randomly
chosen element is typically less for the vector than for the list.

\vest The {\em length} of a vector is the number of elements that it
contains.  This number is a non-negative integer that is fixed when the
vector is created.
The {\em valid indices}\index{valid indices} of a
vector are the exact non-negative integer objects less than the length of the
vector.  The first element in a vector is indexed by zero, and the last
element is indexed by one less than the length of the vector.

Like list constants, vector constants must be quoted:

\begin{scheme}
'\#(0 (2 2 2 2) "Anna")  \lev  \#(0 (2 2 2 2) "Anna")%
\end{scheme}

\begin{entry}{%
\proto{vector?}{ obj}{procedure}}
 
Returns \schtrue{} if \var{obj} is a vector.  Otherwise the procedure
returns \schfalse.
\end{entry}


\begin{entry}{%
\proto{make-vector}{ k}{procedure}
\rproto{make-vector}{ k fill}{procedure}}

Returns a newly allocated vector of \var{k} elements.  If a second
argument is given, then each element is initialized to \var{fill}.
Otherwise the initial contents of each element is unspecified.

\end{entry}


\begin{entry}{%
\proto{vector}{ obj \dotsfoo}{procedure}}

Returns a newly allocated vector whose elements contain the given
arguments.  Analogous to {\cf list}.

\begin{scheme}
(vector 'a 'b 'c)               \ev  \#(a b c)%
\end{scheme}
\end{entry}


\begin{entry}{%
\proto{vector-length}{ vector}{procedure}}

Returns the number of elements in \var{vector} as an exact integer object.
\end{entry}


\begin{entry}{%
\proto{vector-ref}{ vector k}{procedure}}

\domain{\var{K} must be a valid index of \var{vector}.}
The {\cf vector-ref} procedure returns the contents of element \vr{k} of
\var{vector}.

\begin{scheme}
(vector-ref '\#(1 1 2 3 5 8 13 21) 5)  \lev  8%
\end{scheme}
\end{entry}


\begin{entry}{%
\proto{vector-set!}{ vector k obj}{procedure}}

\domain{\var{K} must be a valid index of \var{vector}.}
The {\cf vector-set!} procedure stores \var{obj} in element \vr{k} of
\var{vector}, and returns \unspecifiedreturn.

Passing an immutable vector to {\cf vector-set!} should cause an exception
with condition type {\cf\&assertion} to be raised.

\begin{scheme}
(let ((vec (vector 0 '(2 2 2 2) "Anna")))
  (vector-set! vec 1 '("Sue" "Sue"))
  vec)      \lev  \#(0 ("Sue" "Sue") "Anna")

(vector-set! '\#(0 1 2) 1 "doe")  \lev  \unspecified
             ; \textrm{constant vector}
             ; \textrm{should raise} \exception{\&assertion}%
\end{scheme}

\end{entry}


\begin{entry}{%
\proto{vector->list}{ vector}{procedure}
\proto{list->vector}{ list}{procedure}}

The {\cf vector->list} procedure returns a newly allocated list of the objects contained
in the elements of \var{vector}.  The {\cf list->vector} procedure returns a newly
created vector initialized to the elements of the list \var{list}.

\begin{scheme}
(vector->list '\#(dah dah didah))  \lev  (dah dah didah)
(list->vector '(dididit dah))   \lev  \#(dididit dah)%
\end{scheme}
\end{entry}


\begin{entry}{%
\proto{vector-fill!}{ vector fill}{procedure}}

Stores \var{fill} in every element of \var{vector}
and returns \unspecifiedreturn.
\end{entry}

\begin{entry}{%
\proto{vector-map}{ proc \vari{vector} \varii{vector} \dotsfoo}{procedure}}

\domain{The \var{vector}s must all have the same length.  \var{Proc}
  should accept as many arguments as there are {\it vector}s and return a
  single value.}

The {\cf vector-map} procedure applies \var{proc} element-wise to the elements of the
\var{vector}s and returns a vector of the results, in order.
\var{Proc} is always called in the same dynamic environment 
as {\cf vector-map} itself.
The order in which \var{proc} is applied to the elements of the
\var{vector}s is unspecified.
If multiple returns occur from {\cf vector-map}, the return
values returned by earlier returns are not mutated.


Analogous to {\cf map}.

\implresp The implementation must check the restrictions
on \var{proc} to the extent performed by applying it as described.
An
implementation may check whether \var{proc} is an appropriate argument
before applying it.
\end{entry}


\begin{entry}{%
\proto{vector-for-each}{ proc \vari{vector} \varii{vector} \dotsfoo}{procedure}}

\domain{The \var{vector}s must all have the same length.  \var{Proc}
  should accept as many arguments as there are {\it vector}s.}
The {\cf vector-for-each} procedure applies \var{proc}
element-wise to the elements of the
\var{vector}s for its side effects,  in order from the first elements to the
last.
\var{Proc} is always called in the same dynamic environment 
as {\cf vector-for-each} itself.
The return values of {\cf vector-for-each} \areunspecified.

Analogous to {\cf for-each}.

\implresp The implementation must check the restrictions
on \var{proc} to the extent performed by applying it as described.
An
implementation may check whether \var{proc} is an appropriate argument
before applying it.
\end{entry}

\section{Errors and violations}
\label{errorviolation}

\begin{entry}{%
\proto{error}{ who message \vari{irritant} \dotsfoo}{procedure}
\proto{assertion-violation}{ who message \vari{irritant} \dotsfoo}{procedure}}

\domain{\var{Who} must be a string or a symbol or \schfalse{}.
  \var{Message} must be a string.
  The \var{irritant}s are arbitrary objects.}

These procedures raise an exception.  The {\cf error}
procedure should be called when an error has occurred, typically caused by
something that has gone wrong in the interaction of the program with the
external world or the user.  The {\cf assertion-violation} procedure
should be called when an invalid call to a procedure was made, either passing an
invalid number of arguments, or passing an argument that it is not
specified to handle.

The \var{who} argument should describe the procedure or operation that
detected the exception.  The \var{message} argument should describe
the exceptional situation.  The \var{irritant}s should be the arguments
to the operation that detected the operation.

The condition object provided with the exception (see
library chapter~\extref{lib:exceptionsconditionschapter}{Exceptions
  and conditions}) has the following condition types:
%
\begin{itemize}
\item If \var{who} is not \schfalse, the condition has condition type
  {\cf \&who}, with \var{who} as the value of its field.  In
  that case, \var{who} should be the name of the procedure or entity that
  detected the exception.  If it is \schfalse, the condition does not
  have condition type {\cf \&who}.
\item The condition has condition type {\cf \&message}, with
  \var{message} as the value of its field.
\item The condition has condition type {\cf \&irritants}, and its
  field has as its value a list of the \var{irritant}s.
\end{itemize}
%
Moreover, the condition created by {\cf error} has condition type 
{\cf \&error}, and the condition created by {\cf assertion-\hp{}violation} has
condition type {\cf \&assertion}.

\begin{scheme}
(define (fac n)
  (if (not (integer-valued? n))
      (assertion-violation
       'fac "non-integral argument" n))
  (if (negative? n)
      (assertion-violation
       'fac "negative argument" n))
  (letrec
    ((loop (lambda (n r)
             (if (zero? n)
                 r
                 (loop (- n 1) (* r n))))))
      (loop n 1)))

(fac 5) \ev 120
(fac 4.5) \xev \exception{\&assertion}
(fac -3) \xev \exception{\&assertion}%
\end{scheme}
\end{entry}

\begin{entry}{%
\proto{assert}{ \hyper{expression}}{\exprtype}}

An {\cf assert} form is evaluated by evaluating \hyper{expression}.
If \hyper{expression} returns a true value, that value is returned
from the {\cf assert} expression.  If \hyper{expression} returns
\schfalse, an exception with condition types {\cf \&assertion} and
{\cf \&message} is raised.  The message provided in the condition
object is implementation-dependent.

\begin{note}
  Implementations should exploit the fact that
  {\cf assert} is syntax to provide as much information as possible
  about the location of the assertion failure.
\end{note}
\end{entry}

\section{Control features}
\label{controlsection}
\label{valuessection}
 
This chapter describes various primitive procedures which control the
flow of program execution in special ways.

\begin{entry}{%
\proto{apply}{ proc \vari{arg} $\ldots$ rest-args}{procedure}}

\domain{\var{Rest-args} must be a list.
 \var{Proc} should accept $n$ arguments, where $n$ is
  number of \var{arg}s plus the length of \var{rest-args}.}
The {\cf apply} procedure calls \var{proc} with the elements of the list
{\cf(append (list \vari{arg} \dotsfoo) \var{rest-args})} as the actual
arguments.

If a call to {\cf apply} occurs in a tail context, the call
to \var{proc} is also in a tail context.

\begin{scheme}
(apply + (list 3 4))              \ev  7

(define compose
  (lambda (f g)
    (lambda args
      (f (apply g args)))))

((compose sqrt *) 12 75)              \ev  30%
\end{scheme}
\end{entry}


\begin{entry}{%
\proto{call-with-current-continuation}{ proc}{procedure}
\proto{call/cc}{ proc}{procedure}}

\label{continuations} \domain{\var{Proc} should accept one
argument.} The procedure {\cf call-with-current-continuation} 
(which is the same as the procedure {\cf call/cc}) packages
the current continuation as an ``escape
procedure''\mainindex{escape procedure} and passes it as an argument to
\var{proc}.  The escape procedure is a Scheme procedure that, if it is
later called, will abandon whatever continuation is in effect at that later
time and will instead reinstate the continuation that was in effect
when the escape procedure was created.  Calling the escape procedure
may cause the invocation of \var{before} and \var{after} procedures installed using
\ide{dynamic-wind}.

The escape procedure accepts the same number of arguments as the
continuation of the original call to {\cf call-\hp{}with-\hp{}current-\hp{}continuation}.

The escape procedure that is passed to \var{proc} has
unlimited extent just like any other procedure in Scheme.  It may be stored
in variables or data structures and may be called as many times as desired.

If a call to {\cf call-with-current-continuation} occurs in a tail
context, the call to \var{proc} is also in a tail context.

The following examples show only some ways in which
{\cf call-with-current-continuation} is used.  If all real uses were as
simple as these examples, there would be no need for a procedure with
the power of {\cf call-\hp{}with-\hp{}current-\hp{}continuation}.

\begin{scheme}
(call-with-current-continuation
  (lambda (exit)
    (for-each (lambda (x)
                (if (negative? x)
                    (exit x)))
              '(54 0 37 -3 245 19))
    \schtrue))                        \ev  -3

(define list-length
  (lambda (obj)
    (call-with-current-continuation
      (lambda (return)
        (letrec ((r
                  (lambda (obj)
                    (cond ((null? obj) 0)
                          ((pair? obj)
                           (+ (r (cdr obj)) 1))
                          (else (return \schfalse))))))
          (r obj))))))

(list-length '(1 2 3 4))            \ev  4

(list-length '(a b . c))            \ev  \schfalse%

(call-with-current-continuation procedure?)
                            \ev  \schtrue%
\end{scheme}

\begin{note}
  Calling an escape procedure reenters the dynamic extent of the call
  to {\cf call-with-current-continuation}, and thus restores its
  dynamic environment; see section~\ref{dynamicenvironmentsection}.
\end{note}

\end{entry}

\begin{entry}{%
\proto{values}{ obj $\ldots$}{procedure}}

Delivers all of its arguments to its continuation.
The {\cf values} procedure might be defined as follows:
\begin{scheme}
(define (values . things)
  (call-with-current-continuation 
    (lambda (cont) (apply cont things))))%
\end{scheme}

The continuations of all non-final expressions within a sequence of
expressions, such as in {\cf lambda}, {\cf begin}, {\cf let}, {\cf
  let*}, {\cf letrec}, {\cf letrec*}, {\cf let-values}, {\cf
  let*-values}, {\cf case}, and {\cf cond} forms, usually take an
arbitrary number of values.

Except for these and the continuations created by {\cf
  call-\hp{}with-\hp{}values}, {\cf let-values}, and {\cf let*-values},
continuations implicitly accepting a single value, such as the
continuations of \hyper{operator} and \hyper{operand}s of procedure
calls or the \hyper{test} expressions in conditionals, take exactly
one value.  The effect of passing an inappropriate number of values to
such a continuation is undefined.
\end{entry}

\begin{entry}{%
\proto{call-with-values}{ producer consumer}{procedure}}

\domain{\var{Producer} must be a procedure and should accept zero
  arguments.  \var{Consumer} must be a procedure and should accept as many
  values as \var{producer} returns.}
The {\cf call-\hp{}with-\hp{}values} procedure calls \var{producer} with no arguments and
a continuation that, when passed some values, calls the
\var{consumer} procedure with those values as arguments.
The continuation for the call to \var{consumer} is the
continuation of the call to {\tt call-with-values}.

\begin{scheme}
(call-with-values (lambda () (values 4 5))
                  (lambda (a b) b))
                                                   \ev  5

(call-with-values * -)                             \ev  -1%
\end{scheme}

If a call to {\cf call-with-values} occurs in a tail context, the call
to \var{consumer} is also in a tail context.

\implresp After \var{producer} returns, the implementation must check
that \var{consumer} accepts as many values as \var{consumer} has
returned.
\end{entry}

\begin{entry}{%
\proto{dynamic-wind}{ before thunk after}{procedure}}

\domain{\var{Before}, \var{thunk}, and \var{after} must be procedures,
  and each should accept zero arguments.  These procedures may return
  any number of values.}  The {\cf dynamic-wind} procedure calls
\var{thunk} without arguments, returning the results of this call.
Moreover, {\cf dynamic-wind} calls \var{before} without arguments
whenever the dynamic extent of the call to \var{thunk} is entered, and
\var{after} without arguments whenever the dynamic extent of the call
to \var{thunk} is exited.  Thus, in the absence of calls to escape
procedures created by {\cf call-with-current-continuation}, {\cf
  dynamic-wind} calls \var{before}, \var{thunk}, and \var{after}, in
that order.

While the calls to \var{before} and \var{after} are not considered to be
within the dynamic extent of the call to \var{thunk}, calls to the \var{before}
and \var{after} procedures of any other calls to {\cf dynamic-wind} that occur
within the dynamic extent of the call to \var{thunk} are considered to be
within the dynamic extent of the call to \var{thunk}.

More precisely, an escape procedure transfers control out of the
dynamic extent of a set of zero or more active {\cf dynamic-wind}
calls $x\ \dots$ and transfer control into the dynamic extent
of a set of zero or more active {\cf dynamic-wind} calls
$y\ \dots$.  
It leaves the dynamic extent of the most recent $x$ and calls without
arguments the corresponding \var{after} procedure.
If the \var{after} procedure returns, the escape procedure proceeds to
the next most recent $x$, and so on.
Once each $x$ has been handled in this manner,
the escape procedure calls without arguments the \var{before} procedure
corresponding to the least recent $y$.
If the \var{before} procedure returns, the escape procedure reenters the
dynamic extent of the least recent $y$ and proceeds with the next least
recent $y$, and so on.
Once each $y$ has been handled in this manner, control is transferred to
the continuation packaged in the escape procedure.

\implresp The implementation must check the restrictions on
\var{thunk} and \var{after} only if they are actually called.

\begin{scheme}
(let ((path '())
      (c \#f))
  (let ((add (lambda (s)
               (set! path (cons s path)))))
    (dynamic-wind
      (lambda () (add 'connect))
      (lambda ()
        (add (call-with-current-continuation
               (lambda (c0)
                 (set! c c0)
                 'talk1))))
      (lambda () (add 'disconnect)))
    (if (< (length path) 4)
        (c 'talk2)
        (reverse path))))
    \lev (connect talk1 disconnect
               connect talk2 disconnect)

(let ((n 0))
  (call-with-current-continuation
    (lambda (k)
      (dynamic-wind
        (lambda ()
          (set! n (+ n 1))
          (k))
        (lambda ()
          (set! n (+ n 2)))
        (lambda ()
          (set! n (+ n 4))))))
  n) \ev 1

(let ((n 0))
  (call-with-current-continuation
    (lambda (k)
      (dynamic-wind
        values
        (lambda ()
          (dynamic-wind
            values
            (lambda ()
              (set! n (+ n 1))
              (k))
            (lambda ()
              (set! n (+ n 2))
              (k))))
        (lambda ()
          (set! n (+ n 4))))))
  n) \ev 7%
\end{scheme}

\begin{note}
  Entering a dynamic extent restores its dynamic environment; see
  section~\ref{dynamicenvironmentsection}.
\end{note}
\end{entry}

\section{Iteration}\unsection

\begin{entry}{%
\rproto{let}{ \hyper{variable} \hyper{bindings} \hyper{body}}{\exprtype}}

\label{namedlet}
``Named {\cf let}'' is a variant on the syntax of \ide{let} that provides
a general looping construct and may also be used to express
recursion.
It has the same syntax and semantics as ordinary {\cf let}
except that \hyper{variable} is bound within \hyper{body} to a procedure
whose parameters are the bound variables and whose body is
\hyper{body}.  Thus the execution of \hyper{body} may be repeated by
invoking the procedure named by \hyper{variable}.

%                                              |  <-- right margin
\begin{scheme}
(let loop ((numbers '(3 -2 1 6 -5))
           (nonneg '())
           (neg '()))
  (cond ((null? numbers) (list nonneg neg))
        ((>= (car numbers) 0)
         (loop (cdr numbers)
               (cons (car numbers) nonneg)
               neg))
        ((< (car numbers) 0)
         (loop (cdr numbers)
               nonneg
               (cons (car numbers) neg))))) %
  \lev  ((6 1 3) (-5 -2))%
\end{scheme}

\end{entry}

\section{Quasiquotation}\unsection
\label{quasiquotesection}

\begin{entry}{%
\proto{quasiquote}{ \hyper{qq template}}{\exprtype}
\litproto{unquote}
\litproto{unquote-splicing}}

``Backquote'' or ``quasiquote''\index{backquote} expressions are useful
for constructing a list or vector structure when some but not all of the
desired structure is known in advance.  

\syntax \hyper{Qq template} should be as specified by the grammar at
the end of this entry.

\semantics If no
{\cf unquote} or {\cf unquote-splicing} forms
appear within the \hyper{qq template}, the result of
evaluating
{\cf (quasiquote \hyper{qq template})} is equivalent to the result of evaluating
{\cf (quote \hyper{qq template})}.

If an {\cf (unquote \hyper{expression} \dotsfoo)} form appears inside a
\hyper{qq template}, however, the \hyper{expression}s are evaluated
(``unquoted'') and their results are inserted into the structure instead
of the {\cf unquote} form.

If an {\cf (unquote-splicing \hyper{expression} \dotsfoo)} form
appears inside a \hyper{qq template}, then the \hyper{expression}s must
evaluate to lists; the opening and closing parentheses of the lists are
then ``stripped away'' and the elements of the lists are inserted in
place of the {\cf unquote-splicing} form.

Any {\cf unquote-splicing} or multi-operand {\cf unquote} form must
appear only within a list or vector \hyper{qq template}.

As noted in section~\ref{abbreviationsection},
{\cf (quasiquote \hyper{qq template})} may be abbreviated
\backquote\hyper{qq template},
{\cf (unquote \hyper{expression})} may be abbreviated
{\cf,}\hyper{expression}, and
{\cf (unquote-splicing \hyper{expression})} may be abbreviated
{\cf,}\atsign\hyper{expression}.

\begin{scheme}
`(list ,(+ 1 2) 4)  \ev  (list 3 4)
(let ((name 'a)) `(list ,name ',name)) %
          \lev  (list a (quote a))
`(a ,(+ 1 2) ,@(map abs '(4 -5 6)) b) %
          \lev  (a 3 4 5 6 b)
`(({\cf foo} ,(- 10 3)) ,@(cdr '(c)) . ,(car '(cons))) %
          \lev  ((foo 7) . cons)
`\#(10 5 ,(sqrt 4) ,@(map sqrt '(16 9)) 8) %
          \lev  \#(10 5 2 4 3 8)
(let ((name 'foo))
  `((unquote name name name)))%
          \lev (foo foo foo)
(let ((name '(foo)))
  `((unquote-splicing name name name)))%
          \lev (foo foo foo)
(let ((q '((append x y) (sqrt 9))))
  ``(foo ,,@q)) \lev `(foo
                 (unquote (append x y) (sqrt 9)))
(let ((x '(2 3))
      (y '(4 5)))
  `(foo (unquote (append x y) (sqrt 9)))) \lev (foo (2 3 4 5) 3)%
\end{scheme}

Quasiquote forms may be nested.  Substitutions are made only for
unquoted components appearing at the same nesting level
as the outermost {\cf quasiquote}.  The nesting level increases by one inside
each successive quasiquotation, and decreases by one inside each
unquotation.

\begin{scheme}
`(a `(b ,(+ 1 2) ,(foo ,(+ 1 3) d) e) f) %
          \lev  (a `(b ,(+ 1 2) ,(foo 4 d) e) f)
(let ((name1 'x)
      (name2 'y))
  `(a `(b ,,name1 ,',name2 d) e)) %
          \lev  (a `(b ,x ,'y d) e)%
\end{scheme}

A {\cf quasiquote} expression may return either fresh, mutable objects
or literal structure for any structure that is constructed at run time
during the evaluation of the expression.  Portions that do not need to
be rebuilt are always literal.  Thus,
%
\begin{scheme}
(let ((a 3)) `((1 2) ,a ,4 ,'five 6))%
\end{scheme}
%
may be equivalent to either of the following expressions:
%
\begin{scheme}
'((1 2) 3 4 five 6)
(let ((a 3)) 
  (cons '(1 2)
        (cons a (cons 4 (cons 'five '(6))))))%
\end{scheme}
%
However, it is not equivalent to this expression:
%
\begin{scheme}
(let ((a 3)) (list (list 1 2) a 4 'five 6))
\end{scheme}
%
It is a syntax violation if any of the identifiers
\ide{quasiquote}, \ide{unquote}, or \ide{unquote-splicing} appear in
positions within a \hyper{qq template} otherwise than as described above.

The following grammar for quasiquote expressions is not context-free.
It is presented as a recipe for generating an infinite number of
production rules.  Imagine a copy of the following rules for $D = 1, 2,
3, \ldots$.  $D$ keeps track of the nesting depth.

\begin{grammar}%
\meta{qq template} \: \meta{qq template 1}
\meta{qq template 0} \: \meta{expression}
\meta{quasiquotation $D$} \: (quasiquote \meta{qq template $D$})
\meta{qq template $D$} \: \meta{lexeme datum}
\>    \| \meta{list qq template $D$}
\>    \| \meta{vector qq template $D$}
\>    \| \meta{unquotation $D$}
\meta{list qq template $D$} \: (\arbno{\meta{qq template or splice $D$}})
\>    \| (\atleastone{\meta{qq template or splice $D$}} .\ \meta{qq template $D$})
\>    \| \meta{quasiquotation $D+1$}
\meta{vector qq template $D$} \: \#(\arbno{\meta{qq template or splice $D$}})
\meta{unquotation $D$} \: (unquote \meta{qq template $D-1$})
\meta{qq template or splice $D$} \: \meta{qq template $D$}
\>    \| \meta{splicing unquotation $D$}
\meta{splicing unquotation $D$} \:
\>\> (unquote-splicing \arbno{\meta{qq template $D-1$}})
\>    \| (unquote \arbno{\meta{qq template $D-1$}}) %
\end{grammar}

In \meta{quasiquotation}s, a \meta{list qq template $D$} can sometimes
be confused with either an \meta{un\-quota\-tion $D$} or a \meta{splicing
un\-quo\-ta\-tion $D$}.  The interpretation as an
\meta{un\-quo\-ta\-tion} or \meta{splicing
un\-quo\-ta\-tion $D$} takes precedence.

\end{entry}

\section{Binding constructs for syntactic keywords}
\label{bindsyntax}

The {\cf let-syntax} and {\cf letrec-syntax} forms 
bind keywords.
Like a {\cf begin} form, a {\cf let-syntax} or {\cf letrec-syntax} form
may appear in a definition context, in which case it is treated as a
definition, and the forms in the body must also be
definitions.
A {\cf let-syntax} or {\cf letrec-syntax} form may also appear in an
expression context, in which case the forms within their bodies must be
expressions.

\begin{entry}{%
\proto{let-syntax}{ \hyper{bindings} \hyper{form} \dotsfoo}{\exprtype}}

\syntax
\hyper{Bindings} must have the form
\begin{scheme}
((\hyper{keyword} \hyper{expression}) \dotsfoo)%
\end{scheme}
Each \hyper{keyword} is an identifier,
and each \hyper{expression} is 
an expression that evaluates, at macro-expansion
time, to a \textit{transformer}\index{transformer}\index{macro transformer}.
Transformers may be created by {\cf syntax-rules} or {\cf identifier-syntax}
(see section~\ref{syntaxrulessection}) or by one of the other mechanisms
described in library chapter~\extref{lib:syntaxcasechapter}{{\cf syntax-case}}.  It is a
syntax violation for \hyper{keyword} to appear more than once in the list of keywords
being bound.

\semantics
The \hyper{form}s are expanded in the syntactic environment
obtained by extending the syntactic environment of the
{\cf let-syntax} form with macros whose keywords are
the \hyper{keyword}s, bound to the specified transformers.
Each binding of a \hyper{keyword} has the \hyper{form}s as its region.

The \hyper{form}s of a {\cf let-syntax}
form are treated, whether in definition or expression context, as if
wrapped in an implicit {\cf begin}; see section~\ref{begin}.
Thus definitions in the result of expanding the \hyper{form}s have
the same region as any definition appearing in place of the {\cf
  let-syntax} form would have.

\implresp The implementation should detect if the value of
\hyper{expression} cannot possibly be a transformer.

\begin{scheme}
(let-syntax ((when (syntax-rules ()
                     ((when test stmt1 stmt2 ...)
                      (if test
                          (begin stmt1
                                 stmt2 ...))))))
  (let ((if \schtrue))
    (when if (set! if 'now))
    if))                           \ev  now

(let ((x 'outer))
  (let-syntax ((m (syntax-rules () ((m) x))))
    (let ((x 'inner))
      (m))))                       \ev  outer%

(let ()
  (let-syntax
    ((def (syntax-rules ()
            ((def stuff ...) (define stuff ...)))))
    (def foo 42))
  foo) \ev 42

(let ()
  (let-syntax ())
  5) \ev 5%
\end{scheme}

\end{entry}

\begin{entry}{%
\proto{letrec-syntax}{ \hyper{bindings} \hyper{form} \dotsfoo}{\exprtype}}

\syntax
Same as for {\cf let-syntax}.

\semantics
The \hyper{form}s are
expanded in the syntactic environment obtained by
extending the syntactic environment of the {\cf letrec-syntax}
form with macros whose keywords are the
\hyper{keyword}s, bound to the specified transformers.
Each binding of a \hyper{keyword} has the \hyper{bindings}
as well as the \hyper{form}s within its region,
so the transformers can
transcribe forms into uses of the macros
introduced by the {\cf letrec-syntax} form.

The \hyper{form}s of a {\cf letrec-syntax}
form are treated, whether in definition or expression context, as if
wrapped in an implicit {\cf begin}; see section~\ref{begin}.
Thus definitions in the result of expanding the \hyper{form}s have
the same region as any definition appearing in place of the {\cf
  letrec-syntax} form would have.

\implresp The implementation should detect if the value of
\hyper{expression} cannot possibly be a transformer.

\begin{scheme}
(letrec-syntax
  ((my-or (syntax-rules ()
            ((my-or) \schfalse)
            ((my-or e) e)
            ((my-or e1 e2 ...)
             (let ((temp e1))
               (if temp
                   temp
                   (my-or e2 ...)))))))
  (let ((x \schfalse)
        (y 7)
        (temp 8)
        (let odd?)
        (if even?))
    (my-or x
           (let temp)
           (if y)
           y)))        \ev  7%
\end{scheme}

The following example highlights how {\cf let-syntax}
and {\cf letrec-syntax} differ.

\begin{scheme}
(let ((f (lambda (x) (+ x 1))))
  (let-syntax ((f (syntax-rules ()
                    ((f x) x)))
               (g (syntax-rules ()
                    ((g x) (f x)))))
    (list (f 1) (g 1)))) \lev (1 2)

(let ((f (lambda (x) (+ x 1))))
  (letrec-syntax ((f (syntax-rules ()
                       ((f x) x)))
                  (g (syntax-rules ()
                       ((g x) (f x)))))
    (list (f 1) (g 1)))) \lev (1 1)%
\end{scheme}

The two expressions are identical except that the {\cf let-syntax} form
in the first expression is a {\cf letrec-syntax} form in the second.
In the first expression, the {\cf f} occurring in {\cf g} refers to
the {\cf let}-bound variable {\cf f}, whereas in the second it refers
to the keyword {\cf f} whose binding is established by the
{\cf letrec-syntax} form.
\end{entry}

\section{Macro transformers}
\label{syntaxrulessection}

\begin{entry}{%
\pproto{(syntax-rules (\hyper{literal} \dotsfoo) \hyper{syntax rule} \dotsfoo)}{\exprtype~({\cf expand})}
\litprotoexpandnoindex{\_}
\litprotoexpand{...}}
\mainschindex{syntax-rules}\schindex{\_}

\syntax Each \hyper{literal} must be an identifier.
Each \hyper{syntax rule} must have the following form:

\begin{scheme}
(\hyper{srpattern} \hyper{template})%
\end{scheme}

An \hyper{srpattern} is a restricted form of \hyper{pattern},
namely, a nonempty \hyper{pattern} in one of four parenthesized forms below
whose first subform is an identifier or an underscore {\cf \_}\schindex{\_}.
A \hyper{pattern} is an identifier, constant, or one of the following.

\begin{schemenoindent}
(\hyper{pattern} \ldots)
(\hyper{pattern} \hyper{pattern} \ldots . \hyper{pattern})
(\hyper{pattern} \ldots \hyper{pattern} \hyper{ellipsis} \hyper{pattern} \ldots)
(\hyper{pattern} \ldots \hyper{pattern} \hyper{ellipsis} \hyper{pattern} \ldots . \hyper{pattern})
\#(\hyper{pattern} \ldots)
\#(\hyper{pattern} \ldots \hyper{pattern} \hyper{ellipsis} \hyper{pattern} \ldots)%
\end{schemenoindent}

An \hyper{ellipsis} is the identifier ``{\cf ...}'' (three periods).\schindex{...}

A \hyper{template} is a pattern variable, an identifier that
is not a pattern
variable, a pattern datum, or one of the following.

\begin{scheme}
(\hyper{subtemplate} \ldots)
(\hyper{subtemplate} \ldots . \hyper{template})
\#(\hyper{subtemplate} \ldots)%
\end{scheme}

A \hyper{subtemplate} is a \hyper{template} followed by zero or more ellipses.

\semantics An instance of {\cf syntax-rules} evaluates, at
macro-expansion time, to a new macro
transformer by specifying a sequence of hygienic rewrite rules.  A use
of a macro whose keyword is associated with a transformer specified by
{\cf syntax-rules} is matched against the patterns contained in the
\hyper{syntax rule}s, beginning with the leftmost \hyper{syntax rule}.
When a match is found, the macro use is transcribed hygienically
according to the template.  It is a syntax violation when no match is found.

An identifier appearing within a \hyper{pattern} may be an underscore
(~{\cf \_}~), a literal identifier listed in the list of literals
{\cf (\hyper{literal} \dotsfoo)}, or an ellipsis (~{\cf ...}~).
All other identifiers appearing within a \hyper{pattern} are
\textit{pattern variables\mainindex{pattern variable}}.
It is a syntax violation if an ellipsis or underscore appears in {\cf (\hyper{literal} \dotsfoo)}.

While the first subform of \hyper{srpattern} may be an identifier, the
identifier is not involved in the matching and
is not considered a pattern variable or literal identifier.

Pattern variables match arbitrary input subforms and
are used to refer to elements of the input.
It is a syntax violation if the same pattern variable appears more than once in a
\hyper{pattern}.

Underscores also match arbitrary input subforms but are not pattern variables
and so cannot be used to refer to those elements.
Multiple underscores may appear in a \hyper{pattern}.

A literal identifier matches an input subform if and only if the input
subform is an identifier and either both its occurrence in the input
expression and its occurrence in the list of literals have the same
lexical binding, or the two identifiers have the same name and both have
no lexical binding.

A subpattern followed by an ellipsis can match zero or more elements of
the input.

More formally, an input form $F$ matches a pattern $P$ if and only if
one of the following holds:

\begin{itemize}
\item $P$ is an underscore (~{\cf \_}~).

\item $P$ is a pattern variable.

\item $P$ is a literal identifier
and $F$ is an identifier such that both $P$ and $F$ would refer to the
same binding if both were to appear in the output of the macro outside
of any bindings inserted into the output of the macro.
(If neither of two like-named identifiers refers to any binding, i.e., both
are undefined, they are considered to refer to the same binding.)

\item $P$ is of the form
{\cf ($P_1$ \dotsfoo{} $P_n$)}
and $F$ is a list of $n$ elements that match $P_1$ through
$P_n$.

\item $P$ is of the form
{\cf ($P_1$ \dotsfoo{} $P_n$ . $P_x$)}
and $F$ is a list or improper list of $n$ or more elements
whose first $n$ elements match $P_1$ through $P_n$
and
whose $n$th cdr matches $P_x$.

\item $P$ is of the form
{\cf ($P_1$ \dotsfoo{} $P_k$ $P_e$ \hyper{ellipsis} $P_{m+1}$ \dotsfoo{} $P_n$)},
where \hyper{ellipsis} is the identifier {\cf ...}
and $F$ is a list of $n$
elements whose first $k$ elements match $P_1$ through $P_k$,
whose next $m-k$ elements each match $P_e$,
and
whose remaining $n-m$ elements match $P_{m+1}$ through $P_n$.

\item $P$ is of the form
{\cf ($P_1$ \dotsfoo{} $P_k$ $P_e$ \hyper{ellipsis} $P_{m+1}$ \dotsfoo{} $P_n$ . $P_x$)},
where \hyper{ellipsis} is the identifier {\cf ...}
and $F$ is a list or improper list of $n$
elements whose first $k$ elements match $P_1$ through $P_k$,
whose next $m-k$ elements each match $P_e$,
whose next $n-m$ elements match $P_{m+1}$ through $P_n$,
and 
whose $n$th and final cdr matches $P_x$.

\item $P$ is of the form
{\cf \#($P_1$ \dotsfoo{} $P_n$)}
and $F$ is a vector of $n$ elements that match $P_1$ through
$P_n$.

\item $P$ is of the form
{\cf \#($P_1$ \dotsfoo{} $P_k$ $P_e$ \hyper{ellipsis} $P_{m+1}$ \dotsfoo{} $P_n$)},
where \hyper{ellipsis} is the identifier {\cf ...}
and $F$ is a vector of $n$ or more elements
whose first $k$ elements match $P_1$ through $P_k$,
whose next $m-k$ elements each match $P_e$,
and
whose remaining $n-m$ elements match $P_{m+1}$ through $P_n$.

\item $P$ is a pattern datum (any nonlist, nonvector, nonsymbol
datum) and $F$ is equal to $P$ in the sense of the
{\cf equal?} procedure.
\end{itemize}

When a macro use is transcribed according to the template of the
matching \hyper{syntax rule}, pattern variables that occur in the
template are replaced by the subforms they match in the input.

Pattern data and identifiers that are not pattern variables
or ellipses are copied into the output.
A subtemplate followed by an ellipsis expands
into zero or more occurrences of the subtemplate.
Pattern variables that occur in subpatterns followed by one or more
ellipses may occur only in subtemplates that are
followed by (at least) as many ellipses.
These pattern variables are replaced in the output by the input
subforms to which they are bound, distributed as specified.
If a pattern variable is followed by more ellipses in the subtemplate
than in the associated subpattern, the input form is replicated as
necessary.
The subtemplate must contain at least one pattern variable from a
subpattern followed by an ellipsis, and for at least one such pattern
variable, the subtemplate must be followed by exactly as many ellipses as
the subpattern in which the pattern variable appears.
(Otherwise, the expander would not be able to determine how many times the
subform should be repeated in the output.)
It is a syntax violation if the constraints of this paragraph are not met.

A template of the form
{\cf (\hyper{ellipsis} \hyper{template})} is identical to \hyper{template}, except that
ellipses within the template have no special meaning.
That is, any ellipses contained within \hyper{template} are
treated as ordinary identifiers.
In particular, the template {\cf (... ...)} produces a single
ellipsis, {\cf ...}.
This allows syntactic abstractions to expand into forms containing
ellipses.

\begin{scheme}
(define-syntax be-like-begin
  (syntax-rules ()
    ((be-like-begin name)
     (define-syntax name
       (syntax-rules ()
         ((name expr (... ...))
          (begin expr (... ...))))))))

(be-like-begin sequence)
(sequence 1 2 3 4) \ev 4%
\end{scheme}

As an example for hygienic use of auxiliary identifier,
if \ide{let} and \ide{cond} are defined as in
section~\ref{let} and appendix~\ref{derivedformsappendix} then they
are hygienic (as required) and the following is not an error.

\begin{scheme}
(let ((=> \schfalse))
  (cond (\schtrue => 'ok)))           \ev ok%
\end{scheme}

The macro transformer for {\cf cond} recognizes {\cf =>}
as a local variable, and hence an expression, and not as the
identifier {\cf =>}, which the macro transformer treats
as a syntactic keyword.  Thus the example expands into

\begin{scheme}
(let ((=> \schfalse))
  (if \schtrue (begin => 'ok)))%
\end{scheme}

instead of

\begin{scheme}
(let ((=> \schfalse))
  (let ((temp \schtrue))
    (if temp ('ok temp))))%
\end{scheme}

which would result in an assertion violation.
\end{entry}

\begin{entry}{%
\proto{identifier-syntax}{ \hyper{template}}{\exprtype~({\cf expand})}
\pproto{(identifier-syntax}{\exprtype~({\cf expand})}}\\
{\tt\obeyspaces
  (\hyperi{id} \hyperi{template})\\
  ((set! \hyperii{id} \hyper{pattern})\\
   \hyperii{template}))\\
\litprotoexpandnoindex{set!}}

\syntax The \hyper{id}s must be identifiers.  The \hyper{template}s
must be as for {\cf syntax-rules}.

\semantics
When a keyword is bound to a transformer produced by the first form of
{\cf identifier-syntax}, references to the keyword within the scope
of the binding are replaced by \hyper{template}.

\begin{scheme}
(define p (cons 4 5))
(define-syntax p.car (identifier-syntax (car p)))
p.car \ev 4
(set! p.car 15) \ev \exception{\&syntax}%
\end{scheme}

The second, more general, form of {\cf identifier-syntax} permits
the transformer to determine what happens when {\cf set!} is used.
In this case, uses of the identifier by itself are replaced by
\hyperi{template}, and uses of {\cf set!} with the identifier are
replaced by \hyperii{template}.

\begin{scheme}
(define p (cons 4 5))
(define-syntax p.car
  (identifier-syntax
    (\_ (car p))
    ((set! \_ e) (set-car! p e))))
(set! p.car 15)
p.car           \ev 15
p               \ev (15 5)%
\end{scheme}

\end{entry}

\section{Tail calls and tail contexts}
\label{basetailcontextsection}

A {\em tail call}\mainindex{tail call} is a procedure call that occurs
in a {\em tail context}.  Tail contexts are defined inductively.  Note
that a tail context is always determined with respect to a particular lambda
expression.

\begin{itemize}
\item The last expression within the body of a lambda expression,
  shown as \hyper{tail expression} below, occurs in a tail context.
%
\begin{scheme}
(l\=ambda \hyper{formals}
  \>\arbno{\hyper{definition}} 
  \>\arbno{\hyper{expression}} \hyper{tail expression})%
\end{scheme}
%
\item If one of the following expressions is in a tail context,
then the subexpressions shown as \hyper{tail expression} are in a tail context.
These were derived from specifications of the syntax of the forms described in
this chapter by replacing some occurrences of \hyper{expression}
with \hyper{tail expression}.  Only those rules that contain tail contexts
are shown here.
%
\begin{scheme}
(if \hyper{expression} \hyper{tail expression} \hyper{tail expression})
(if \hyper{expression} \hyper{tail expression})

(cond \atleastone{\hyper{cond clause}})
(cond \arbno{\hyper{cond clause}} (else \hyper{tail sequence}))

(c\=ase \hyper{expression}
  \>\atleastone{\hyper{case clause}})
(c\=ase \hyper{expression}
  \>\arbno{\hyper{case clause}}
  \>(else \hyper{tail sequence}))

(and \arbno{\hyper{expression}} \hyper{tail expression})
(or \arbno{\hyper{expression}} \hyper{tail expression})

(let \hyper{bindings} \hyper{tail body})
(let \hyper{variable} \hyper{bindings} \hyper{tail body})
(let* \hyper{bindings} \hyper{tail body})
(letrec* \hyper{bindings} \hyper{tail body})
(letrec \hyper{bindings} \hyper{tail body})
(let-values \hyper{mv-bindings} \hyper{tail body})
(let*-values \hyper{mv-bindings} \hyper{tail body})

(let-syntax \hyper{bindings} \hyper{tail body})
(letrec-syntax \hyper{bindings} \hyper{tail body})

(begin \hyper{tail sequence})%
\end{scheme}
%
A \hyper{cond clause} is 
%
\begin{scheme}
(\hyper{test} \hyper{tail sequence})\textrm{,}%
\end{scheme}
a \hyper{case clause} is
%
\begin{scheme}
((\arbno{\hyper{datum}}) \hyper{tail sequence})\textrm{,}%
\end{scheme}
%
a \hyper{tail body} is
\begin{scheme}
\arbno{\hyper{definition}} \hyper{tail sequence}\textrm{,}%
\end{scheme}
%
and a \hyper{tail sequence} is
%
\begin{scheme}
\arbno{\hyper{expression}} \hyper{tail expression}\textrm{.}%
\end{scheme}%

\item
If a {\cf cond} expression is in a tail context, and has a clause of
the form {\cf (\hyperi{expression} => \hyperii{expression})}
then the (implied) call to
the procedure that results from the evaluation of \hyperii{expression} is in a
tail context.  \hyperii{Expression} itself is not in a tail context.

\end{itemize}

Certain built-in procedures must also perform tail calls.
The first argument passed to {\cf apply} and to
{\cf call-\hp{}with-\hp{}current-continuation}, and the second argument passed to
{\cf call-with-values}, must be called via a tail call.

In the following example the only tail call is the call to {\cf f}.
None of the calls to {\cf g} or {\cf h} are tail calls.  The reference to
{\cf x} is in a tail context, but it is not a call and thus is not a
tail call.
\begin{scheme}%
(lambda ()
  (if (g)
      (let ((x (h)))
        x)
      (and (g) (f))))%
\end{scheme}%

\begin{note}
Implementations may
recognize that some non-tail calls, such as the call to {\cf h}
above, can be evaluated as though they were tail calls.
In the example above, the {\cf let} expression could be compiled
as a tail call to {\cf h}. (The possibility of {\cf h} returning
an unexpected number of values can be ignored, because in that
case the effect of the {\cf let} is explicitly unspecified and
implementation-dependent.)
\end{note}

%%% Local Variables: 
%%% mode: latex
%%% TeX-master: "r6rs"
%%% End: 

    \par
\clearchaptergroupstar{Appendices}
\appendix
\chapter{Formal semantics}
\label{formalsemanticschapter}
%!TEX root = r6rs.tex

This appendix presents a non-normative, formal, operational semantics for Scheme, that is based on an earlier semantics~\cite{mf:scheme-op-sem}. It does not cover the entire language. The notable missing features are the macro system, I/O, and the numerical tower. The precise list of features included is given in section~\ref{sec:semantics:grammar}.

The core of the specification is a single-step term rewriting relation that indicates how an (abstract) machine behaves. In general, the report is not a complete specification, giving implementations freedom to behave differently, typically to allow optimizations. This underspecification shows up in two ways in the semantics. 

The first is reduction rules that reduce to special ``\textbf{unknown:} \textit{string}'' states (where the string provides a description of the unknown state). The intention is that rules that reduce to such states can be replaced with arbitrary reduction rules. The precise specification of how to replace those rules is given in section~\ref{sec:semantics:underspecification}.

The other is that the single-step relation relates one program to
multiple different programs, each corresponding to a legal transition
that an abstract machine might take. Accordingly we use the transitive
closure of the single step relation $\rightarrow^*$ to define the
semantics, \calS, as a function from programs (\calP)
to sets of observable results (\calR):
\begin{center}
\begin{tabular}{l}
$\calS : \calP \longrightarrow 2^{\calR}$ \\
$\calS(\calP) = \{ \scrO(\calA) \mid \calP \rightarrow^* \calA \}$
\end{tabular}
\end{center}
where the function $\scrO$ turns an answer ($\calA$) from the semantics into an observable result. Roughly, $\scrO$ is the identity function on simple base values, and returns a special tag for more complex values, like procedure and pairs.

So, an implementation conforms to the semantics if, for every program $\calP$, the implementation produces one of the results in $\calS(\calP)$ or, if the implementation loops forever, then there is an infinite reduction sequence starting at $\calP$, assuming that the reduction relation $\rightarrow$ has been adjusted to replace the \textbf{unknown:} states.

The precise definitions of $\calP$, $\calA$, $\calR$, and $\scrO$ are also given in section~\ref{sec:semantics:grammar}.

To help understand the semantics and how it behaves, we have
implemented it in PLT Redex. The implementation is available at the
report's website: \url{http://www.r6rs.org/}. All of the reduction
rules and the metafunctions shown in the figures in this semantics
were generated automatically from the source code.

\section{Background}

We assume the reader has a basic familiarity with context-sensitive
reduction semantics. Readers unfamiliar with this system may wish to
consult Felleisen and Flatt's monograph~\cite{ff:monograph} or Wright
and Felleisen~\cite{wf:type-soundness} for a thorough introduction,
including the relevant technical background, or an introduction to PLT
Redex~\cite{mfff:plt-redex} for a somewhat lighter one.

As a rough guide, we define the operational semantics of a language
via a relation on program terms, where the relation corresponds to a
single step of an abstract machine. The relation is defined using
evaluation contexts, namely terms with a distinguished place in them,
called \emph{holes}\index{hole}, where the next step of evaluation
occurs. We say that a term $e$ decomposes into an evaluation
context $E$ and another term $e'$ if $e$ is the
same as $E$ but with the hole replaced by $e'$. We write
$E[e']$ to indicate the term obtained by replacing the hole in
$E$ with $e'$.

For example, assuming that we have defined a grammar containing
non-terminals for evaluation contexts ($E$), expressions
($e$), variables ($x$), and values ($v$), we
would write:
%
\begin{displaymath}
  \begin{array}{l}
    E_1[\texttt{((}\sy{lambda}~\texttt{(}x_1 \cdots{}\texttt{)}~e_1\texttt{)}~v_1~\cdots\texttt{)}] \rightarrow
    \\
    E_1[\{ x_1 \cdots \mapsto v_1 \cdots \} e_1] ~~~~~ (\#x_1 = \#v_1)
  \end{array}
\end{displaymath}
%
to define the $\beta_v$ rewriting rule (as a part of the $\rightarrow$
single step relation). We use the names of the non-terminals (possibly
with subscripts) in a rewriting rule to restrict the application of
the rule, so it applies only when some term produced by that grammar
appears in the corresponding position in the term. If the same
non-terminal with an identical subscript appears multiple times, the
rule only applies when the corresponding terms are structurally
identical (nonterminals without subscripts are not constrained to
match each other). Thus, the occurrence of $E_1$ on both the
left-hand and right-hand side of the rule above means that the context
of the application expression does not change when using this rule.
The ellipses are a form of Kleene star, meaning that zero or more
occurrences of terms matching the pattern proceeding the ellipsis may
appear in place of the the ellipsis and the pattern preceding it. We
use the notation $\{ x_1 \cdots \mapsto v_1 \cdots \} e_1$ for
capture-avoiding substitution; in this case it means that each
$x_1$ is replaced with the corresponding $v_1$ in
$e_1$. Finally, we write side-conditions in parentheses beside
a rule; the side-condition in the above rule indicates that the number
of $x_1$s must be the same as the number of $v_1$s.
Sometimes we use equality in the side-conditions; when we do it merely
means simple term equality, i.e., the two terms must have the
same syntactic shape.


\addtocounter{figure}{1} % get the figure counter in sync with the section counter
\subfigurestart{}
\beginfig

%

\caption{Grammar for programs and observables}\label{fig:grammar}
\endfig

Making the evaluation context $E$ explicit in the rule allows
us to define relations that manipulate their context. As a simple
example, we can add another rule that signals a violation when a
procedure is applied to the wrong number of arguments by discarding
the evaluation context on the right-hand side of a rule:
%
\begin{displaymath}
  \begin{array}{l}
    E[\texttt{((}\sy{lambda}~\texttt{(}x_1 \cdots\texttt{)}~e\texttt{)}~v_1~\cdots\texttt{)}] \rightarrow
    \\
    \textrm{\textbf{violation:} wrong argument count} ~~~~~ (\#x_1 \neq \#v_1)
  \end{array}
\end{displaymath}
%
Later we take advantage of the explicit evaluation context in more
sophisticated ways.



\section{Grammar}\label{sec:semantics:grammar}

\beginfig
\subfigureadjust{}

%

\caption{Grammar for evaluation contexts}\label{fig:ec-grammar}
\endfig
\subfigurestop{}

Figure~\ref{fig:grammar} shows the grammar for the subset of the
report this semantics models. Non-terminals are written in
\textit{italics} or in a calligraphic font ($\calP$
$\calA$, $\calR$, and $\calRv$) and literals are 
written in a \texttt{monospaced} font.

The $\calP$ non-terminal represents possible program states. The
first alternative is a program with a store and an expression. 
The second alternative is an uncaught exception, and the third is
used to indicate a place where the model does not completely specify
the behavior of the primitives it models (see section~\ref{sec:semantics:underspecification} for details of those situations). 
The $\calA$ non-terminal
represents a final result of a program. It is just like $\calP$
except that expression has been reduced to some sequence of values.

The $\calR$ and $\calRv$ non-terminals specify the observable results of a program. Each $\calR$ is either a sequence of values that correspond to the values produced by the program that terminates normally, or a tag indicating an uncaught exception was raised, or \sy{unknown} if the program encounters a situation the semantics does not cover. The $\calRv$ non-terminal specifies what the observable results are for a particular value: a pair, the empty list, a symbol, a self-quoting value (\schtrue, \schfalse, and numbers), a condition, or a procedure.

The \nt{sf} non-terminal generates individual elements of the
store. The store holds all of the mutable state of a program. It is
explained in more detail along with the rules that manipulate it.

Expressions ($\mathit{es}$) include quoted data, \sy{begin} expressions, \sy{begin0} expressions%
\footnote{ \sy{begin0} is not part of the standard, but we include it
  to make the rules for \va{dynamic-wind} and \va{letrec} easier to read. Although
  we model it directly, it can be defined in terms of other forms we
  model here that do come from the standard:
\begin{displaymath}
  \begin{array}{rcl}\tt
    \texttt{(}\sy{begin0}~e_1~e_2~\cdots\texttt{)} &=&
    \begin{array}{l}
      \texttt{(}\va{call\mbox{-}with\mbox{-}values}\\
      ~\texttt{(}\sy{lambda}~\texttt{()}~e_1\texttt{)}\\
      ~\texttt{(}\sy{lambda}~x\\
      ~~~e_2~\cdots\\
      ~~~\texttt{(}\va{apply}~\va{values}~x\texttt{)))}
    \end{array}
  \end{array}
\end{displaymath}
}, application expressions, \sy{if} expressions, \sy{set!}
expressions, variables, non-procedure values (\nt{nonproc}), primitive
procedures (\nt{pproc}), lambda expressions, \sy{letrec} and \sy{letrec*} expressions. 

The last few expression forms are only generated for intermediate states (\sy{dw} for \sy{dynamic-wind}, \sy{throw} for continuations, \sy{unspecified} for the result of the assignment operators, \sy{handlers} for exception handlers, and \sy{l!} and \sy{reinit} for \sy{letrec}), and should not appear in an initial program. Their use is described in the relevant sections of this appendix.

The \nt{f} non-terminal describes the formals for \sy{lambda} expressions. (The \sy{dot} is used instead of a period for procedures that accept an arbitrary number of arguments, in order to avoid meta-circular confusion in our PLT Redex model.) 

The \nt{s} non-terminal covers all datums, which can be either non-empty sequences (\nt{seq}), the empty sequence, self-quoting values (\nt{sqv}), or symbols. Non-empty sequences are either just a sequence of datums, or they are terminated with a dot followed by either a symbol or a self-quoting value. Finally the self-quoting values are numbers and the booleans \semtrue{} and \semfalse{}.

The \nt{p} non-terminal represents programs that have no quoted
data. Most of the reduction rules rewrite \nt{p} to \nt{p},
rather than $\calP$ to $\calP$, since quoted data is first
rewritten into calls to the list construction functions before
ordinary evaluation proceeds. In parallel to \nt{es}, \nt{e} represents
expressions that have no quoted expressions.

\beginfig
\begin{center}

%



%

\end{center}
\caption{Quote}\label{fig:quote}
\endfig

The values ($v$) are divided into four categories:
%
\begin{itemize}
\item Non-procedures (\nt{nonproc}) include pair pointers
  (\va{pp}), the empty list (\va{null}), symbols, self-quoting values
  (\nt{sqv}), and conditions. Conditions represent
  the report's condition values, but here just contain a message and
  are otherwise inert.
\item User procedures (\texttt{(}\sy{lambda} \nt{f} \nt{e} \nt{e} $\cdots$\texttt{)}) include multi-arity lambda expressions and lambda expressions with dotted parameter lists,
\item Primitive procedures (\nt{pproc}) include

\begin{itemize}
\item
 arithmetic procedures
  (\nt{aproc}): \va{+}, \va{-}, \va{/}, and \va{*}, 
\item 
  procedures of one
  argument (\nt{proc1}): \va{null?}, \va{pair?}, \va{car}, \va{cdr},
  \va{call/cc}, \va{procedure?}, \va{condition?}, \va{unspecified?}, \va{raise}, and \va{raise-continuable}, 
  \item
  procedures of
  two arguments (\nt{proc2}): \va{cons}, \va{set-car!}, \va{set-cdr!}, \va{eqv?},
  and \va{call-with-values}, 
  \item as well as \va{list}, \va{dynamic-wind},
  \va{apply}, \va{values}, and \va{with-exception-handler}.
\end{itemize}
\item Finally, continuations are represented as \sy{throw} expressions
  whose body consists of the context where the continuation was
  grabbed.
\end{itemize}
%
The next three set of non-terminals in figure~\ref{fig:grammar} represent pairs (\nt{pp}), which are divided into immutable pairs (\nt{ip}) and mutable pairs (\nt{mp}). The final set of non-terminals in figure~\ref{fig:grammar}, \nt{sym},
\nt{x}, and $n$ represent symbols, variables, and
numbers respectively. The non-terminals \nt{ip}, \nt{mp}, and \nt{sym} are all assumed to all be disjoint. Additionally, the variables $x$ are assumed not to include any keywords or primitive operations, so any program variables whose names coincide with them must be renamed before the semantics can give the meaning of that program.

\beginfig
\begin{center}

%

\end{center}
\caption{Multiple values and call-with-values}\label{fig:Multiple--values--and--call-with-values}
\endfig

The set of non-terminals for evaluation contexts is shown in
figure~\ref{fig:ec-grammar}. The \nt{P} non-terminal controls where
evaluation happens in a program that does not contain any quoted data.
The $E$ and $F$ evaluation contexts are for expressions.  They are factored in
that manner so that the \nt{PG}, \nt{G}, and \nt{H} evaluation contexts can
re-use \nt{F} and have fine-grained control over the context to support
exceptions and \va{dynamic-wind}. The starred and circled variants,
\Estar{}, \Eo{}, \Fstar{}, and \Fo{} dictate where a single value is
promoted to multiple values and where multiple values are demoted to a
single value. The \nt{U} context is used to manage the report's underspecification of the results of \sy{set!}, \va{set-car!}, and \va{set-cdr!} (see section~\ref{sec:semantics:underspecification} for details). Finally, the \nt{S} context is where quoted expressions can be simplified. The precise use of the evaluation contexts is explained along with the relevant rules.

To convert the answers ($\calA$)  of the semantics into observable results, we use these two functions:

%


%

They eliminate the store, and replace complex values with simple tags that indicate only the kind of value that was produced or, if no values were produced, indicates that either an uncaught exception was raised, or that the program reached a state that is not specified by the semantics.

\section{Quote}\label{sec:semantics:quote}

The first reduction rules that apply to any program is the 
rules in figure~\ref{fig:quote} that eliminate quoted expressions. 
The first two rules erase the quote for quoted expressions that do not introduce any pairs.
The last two rules lift quoted datums to the top of the expression so
they are evaluated only once, and turn the datums into calls to either \va{cons} or \va{consi}, via the metafunctions $\mathscr{Q}_i$ and $\mathscr{Q}_m$.

Note that the left-hand side of the \rulename{6qcons} and \rulename{6qconsi} rules are identical, meaning that if one rule applies to a term, so does the other rule. 
Accordingly, a quoted expression may be lifted out into a sequence of \va{cons} expressions, which create mutable pairs, or into a sequence of \va{consi} expressions, which create immutable pairs (see section~\ref{sec:semantics:lists} for the rules on how that happens).

These rules apply before any other because of the contexts in which they, and all of the other rules, apply. In particular, these rule applies in the
\nt{S} context. Figure~\ref{fig:ec-grammar} shows that the
\nt{S} context allows this reduction to apply in
any subexpression of an \nt{e}, as long as all of the
subexpressions to the left have no quoted expressions in them,
although expressions to the right may have quoted expressions.
Accordingly, this rule applies once for each quoted expression in the
program, moving out to the beginning of the program.
The rest of the rules apply in contexts that do not contain any quoted
expressions, ensuring that these rules convert all quoted data
into lists before those rules apply.

Although the identifier \nt{qp} does not have a subscript, the semantics of PLT Redex's ``fresh'' declaration takes special care to ensures that the \nt{qp} on the right-hand side of the rule is indeed the same as the one in the side-condition.

\beginfig
\begin{center}

%

\end{center}
\caption{Exceptions}\label{fig:Exceptions}
\endfig

\section{Multiple values}

The basic strategy for multiple values is to add a rule that demotes
$(\va{values}~v)$ to $v$ and another rule that promotes
$v$ to $(\va{values}~v)$. If we allowed these rules to apply
in an arbitrary evaluation context, however, we would get infinite
reduction sequences of endless alternation between promotion and
demotion. So, the semantics allows demotion only in a context
expecting a single value and allows promotion only in a context
expecting multiple values. We obtain this behavior with a small
extension to the Felleisen-Hieb framework (also present in the
operational model for R$^5$RS~\cite{mf:op-r5rs}).
We extend the notation so that
holes have names (written with a subscript), and the context-matching
syntax may also demand a hole of a particular name (also written with
a subscript, for instance $E[e]_{\star}$).  The extension
allows us to give different names to the holes in which multiple
values are expected and those in which single values are expected, and
structure the grammar of contexts accordingly.

To exploit this extension, we use three kinds of holes in the
evaluation context grammar in figure~\ref{fig:ec-grammar}. The
ordinary hole \hole{} appears where the usual kinds of
evaluation can occur. The hole \holes{} appears in contexts that
allow multiple values and \holeone{} appears in
contexts that expect a single value. Accordingly, the rule
\rulename{6promote} only applies in \holes{} contexts, and 
\rulename{6demote} only applies in \holeone{} contexts.

To see how the evaluation contexts are organized to ensure that
promotion and demotion occur in the right places, consider the \nt{F},
\Fstar{} and \Fo{} evaluation contexts. The \Fstar{} and \Fo{}
evaluation contexts are just the same as \nt{F}, except that they allow
promotion to multiple values and demotion to a single value,
respectively. So, the \nt{F} evaluation context, rather than being
defined in terms of itself, exploits \Fstar{} and \Fo{} to dictate
where promotion and demotion can occur. For example, \nt{F} can be
$\texttt{(}\sy{if}~\Fo{}~e~e\texttt{)}$ meaning that demotion from
$\texttt{(}\va{values}~v\texttt{)}$ to
$v$ can occur in the test of an \sy{if} expression.
Similarly, $F$ can be $\texttt{(}\sy{begin}~\Fstar{}~e~e~\cdots\texttt{)}$ meaning that
$v$ can be promoted to $\texttt{(}\va{values}~v\texttt{)}$ in the first subexpression of a \sy{begin}.

In general, the promotion and demotion rules simplify the definitions
of the other rules. For instance, the rule for \sy{if} does not
need to consider multiple values in its first subexpression.
Similarly, the rule for \sy{begin} does not need to consider the
case of a single value as its first subexpression.

\beginfig
\begin{center}

%


%

\end{center}
\caption{Arithmetic and basic forms}\label{fig:Arithmetic}
\endfig

The other two rules in
figure~\ref{fig:Multiple--values--and--call-with-values} handle
\va{call-\hp{}with-\hp{}values}. The evaluation contexts for
\va{call-with-values} (in the $F$ non-terminal) allow
evaluation in the body of a procedure that has been passed as the first
argument to \va{call-with-values}, as long as the second argument
has been reduced to a value. Once evaluation inside that procedure
completes, it will produce multiple values (since it is an \Fstar{}
position), and the entire \va{call-with-values} expression reduces
to an application of its second argument to those values, via the rule
\rulename{6cwvd}. Finally, in the
case that the first argument to \va{call-with-values} is a value,
but is not of the form $\texttt{(}\sy{lambda}~\texttt{()}~e\texttt{)}$, the rule
\rulename{6cwvw} wraps it in a thunk to trigger evaluation.

\beginfig
\begin{center}

%

\end{center}
\caption{Lists}\label{fig:Cons}
\endfig

\section{Exceptions}

The workhorses for the exception system are $$\texttt{(}\sy{handlers}~\nt{proc}~\cdots{}~\nt{e}\texttt{)}$$ expressions and the \nt{G} and \nt{PG} evaluation contexts (shown in figure~\ref{fig:ec-grammar}). 
The \sy{handlers} expression records the
active exception handlers (\nt{proc} $\cdots$) in some expression (\nt{e}). The
intention is that only the nearest enclosing \sy{handlers} expression
is relevant to raised exceptions, and the $G$ and \nt{PG} evaluation
contexts help achieve that goal. They are just like their counterparts
\nt{E} and \nt{P}, except that \sy{handlers} expressions cannot occur on the
path to the hole, and the exception system rules take advantage of
that context to find the closest enclosing handler.

To see how the contexts work together with \sy{handler}
expressions, consider the left-hand side of the \rulename{6xunee}
rule in figure~\ref{fig:Exceptions}.
It matches expressions that have a call to \va{raise} or
\va{raise-continuable} (the non-terminal \nt{raise*} matches
both exception-raising procedures) in a \nt{PG}
evaluation context. Since the \nt{PG} context does not contain any
\sy{handlers} expressions, this exception cannot be caught, so
this expression reduces to a final state indicating the uncaught
exception. The rule \rulename{6xuneh} also signals an uncaught
exception, but it covers the case where a \sy{handlers} expression
has exhausted all of the handlers available to it. The rule applies to
expressions that have a \sy{handlers} expression (with no
exception handlers) in an arbitrary evaluation context where a call to
one of the exception-raising functions is nested in the
\sy{handlers} expression. The use of the \nt{G} evaluation
context ensures that there are no other \sy{handler} expressions
between this one and the raise.

The next two rules cover call to the procedure \va{with-exception-handler}.
The \rulename{6xwh1} rule applies when there are no \sy{handler}
expressions. It constructs a new one and applies $\nt{v}_2$ as a
thunk in the \sy{handler} body. If there already is a handler
expression, the \rulename{6xwhn} applies. It collects the current
handlers and adds the new one into a new \sy{handlers} expression
and, as with the previous rule, invokes the second argument to
\va{with-exception-handlers}.

The next two rules cover exceptions that are raised in the context of
a \sy{handlers} expression. If a continuable exception is raised,
\rulename{6xrc} applies. It takes the most recently installed
handler from the nearest enclosing \sy{handlers} expression and
applies it to the argument to \va{raise-continuable}, but in a
context where the exception handlers do not include that latest
handler. The \rulename{6xr} rule behaves similarly, except it
raises a new exception if the handler returns. The new exception is
created with the \sy{make-cond} special form.

\beginfig
\begin{center}

%

\end{center}
\caption{Eqv}\label{fig:Eqv}
\endfig

The \sy{make-cond} special form is a stand-in for the report's
conditions. It does not evaluate its argument (note its absence from
the $E$ grammar in figure~\ref{fig:ec-grammar}). That argument
is just a literal string describing the context in which the exception
was raised. The only operation on conditions is \va{condition?},
whose semantics are given by the two rules \rulename{6ct} and
\rulename{6cf}.

Finally, the rule \rulename{6xdone} drops a \sy{handlers} expression
when its body is fully evaluated, and the rule \rulename{6weherr}
raises an exception when \va{with-exception-handler} is supplied with
incorrect arguments.

\section{Arithmetic and basic forms}

This model does not include the report's arithmetic, but does include
an idealized form in order to make experimentation with other features
and writing test suites for the model simpler.
Figure~\ref{fig:Arithmetic} shows the reduction rules for the
primitive procedures that implement addition, subtraction,
multiplication, and division. They defer to their mathematical
analogues. In addition, when the subtraction or divison operator are
applied to no arguments, or when division receives a zero as a
divisor, or when any of the arithmetic operations receive a
non-number, an exception is raised.

The bottom half of figure~\ref{fig:Arithmetic} shows the rules for
\sy{if}, \sy{begin}, and \sy{begin0}. The relevant
evaluation contexts are given by the $F$ non-terminal.

The evaluation contexts for \sy{if} only allow evaluation in its
test expression. Once that is a value, the rules reduce
an \sy{if} expression to its consequent if the test is not
\semfalse{}, and to its alternative if it is \semfalse{}.

The \sy{begin} evaluation contexts allow evaluation in the first
subexpression of a begin, but only if there are two or more
subexpressions. In that case, once the first expression has been fully
simplified, the reduction rules drop its value. If there is only a
single subexpression, the \sy{begin} itself is dropped.

\subfigurestart{}
\beginfig
\begin{center}

%

\end{center}
\caption{Procedures \& application}\label{fig:Procedure--application}
\endfig

Like the \sy{begin} evaluation contexts, the \sy{begin0}
evaluation contexts allow evaluation of the first subexpression of a
\sy{begin0} expression when there are two or more subexpressions.
The \sy{begin0} evaluation contexts also allow evaluation in the
second subexpression of a \sy{begin0} expression, as long as the first
subexpression has been fully simplified. The \rulename{6begin0n} rule for
\sy{begin0} then drops a fully simplified second subexpression.
Eventually, there is only a single expression in the \sy{begin0},
at which point the \rulename{begin01} rule fires, and removes the
\sy{begin0} expression.

\section{Lists}\label{sec:semantics:lists}

The rules in figure~\ref{fig:Cons} handle lists. The first two rules handle \va{list} by reducing it to a succession of calls to \va{cons}, followed by \va{null}.

The next two rules, \rulename{6cons} and \rulename{6consi}, allocate new \va{cons} cells.
They both move $\texttt{(}\va{cons}~v_1~v_2\texttt{)}$ into the store, bound to a fresh
pair pointer (see also section~\ref{sec:semantics:quote} for a description of ``fresh''). 
The \rulename{6cons} uses a \nt{mp} variable, to indicate the pair is mutable, and the \rulename{6consi} uses a \nt{ip} variable to indicate the pair is immutable.

The rules \rulename{6car} and \rulename{6cdr} extract the components of a pair from the store when presented with a pair pointer (the \nt{pp} can be either \nt{mp} or \nt{ip}, as shown in figure~\ref{fig:grammar}).

The rules \rulename{6setcar} and \rulename{6setcdr} handle assignment of mutable pairs. 
They replace the contents of the appropriate location in the store with the new value, and reduce to \va{unspecified}. See section~\ref{sec:semantics:underspecification} for an explanation of how \va{unspecified} reduces.

\beginfig
\subfigureadjust{}
\begin{center}

%

\end{center}
\caption{Variable-assignment relation}\label{fig:varsetd}
\endfig

The next four rules handle the \va{null?} predicate and the \va{pair?} predicate, and the final four rules raise exceptions when \va{car}, \va{cdr}, \va{set-car!} or \va{set-cdr!} receive non pairs.

\section{Eqv}

The rules for \va{eqv?} are shown in figure~\ref{fig:Eqv}. The first two rules cover most of the behavior of \va{eqv?}. 
The first says that when the two arguments to \va{eqv?} are syntactically identical, then \va{eqv?} produces \semtrue{} and the second says that when the arguments are not syntactically identical, then \va{eqv?} produces \semfalse{}. 
The structure of \nt{v} has been carefully designed so that simple term equality corresponds closely to \va{eqv?}'s behavior. 
For example, pairs are represented as pointers into the store and \va{eqv?} only compares those pointers.

The side-conditions on those first two rules ensure that they do not apply when simple term equality does not match the behavior of \va{eqv?}. There are two situations where it does not match: comparing two conditions and comparing two procedures. For the first, the report does not specify \va{eqv?}'s behavior, except to say that it must return a boolean, so the remaining two rules (\rulename{6eqct}, and \rulename{6eqcf}) allow such comparisons to return \semtrue{} or \semfalse{}. Comparing two procedures is covered in section~\ref{sec:semantics:underspecification}. 

\section{Procedures and application}

In evaluating a procedure call, the report leaves
unspecified the order in which arguments are evaluated. So, our reduction system allows multiple, different reductions to occur, one for each possible order of evaluation.

To capture unspecified evaluation order but allow only evaluation that
is consistent with some sequential ordering of the evaluation of an
application's subexpressions, we use non-deterministic choice to first pick
a subexpression to reduce only when we have not already committed to
reducing some other subexpression. To achieve that effect, we limit
the evaluation of application expressions to only those that have a
single expression that is not fully reduced, as shown in the
non-terminal $F$, in figure~\ref{fig:ec-grammar}. To evaluate
application expressions that have more than two arguments to evaluate,
the rule \rulename{6mark} picks one of the subexpressions of an
application that is not fully simplified and lifts it out in its own
application, allowing it to be evaluated. Once one of the lifted
expressions is evaluated, the \rulename{6appN} substitutes its value
back into the original application.

The \rulename{6appN} rule also handles other applications whose
arguments are finished by substituting the first argument for
the first formal parameter in the expression. Its side-condition uses
the relation in figure~\ref{fig:varsetd} to ensure that there are no
\sy{set!} expressions with the parameter $x_1$ as a target.
If there is such an assignment, the \rulename{6appN!} rule applies (see also section~\ref{sec:semantics:quote} for a description of ``fresh'').
Instead of directly substituting the actual parameter for the formal
parameter, it creates a new location in the store, initially bound the
actual parameter, and substitutes a variable standing for that
location in place of the formal parameter. The store, then, handles
any eventual assignment to the parameter. Once all of the parameters
have been substituted away, the rule \rulename{6app0} applies and
evaluation of the body of the procedure begins.

At first glance, the rule \rulename{6appN} appears superfluous, since it seems like the rules could just reduce first by \rulename{6appN!} and then look up the variable when it is evaluated. 
There are two reasons why we keep the \rulename{6appN}, however. 
The first is purely conventional: reducing applications via substitution is taught to us at an early age and is commonly used in rewriting systems in the literature.
The second reason is more technical:  the
\rulename{6mark} rule requires that \rulename{6appN} be applied once $\nt{e}_i$ has been reduced to a value. \rulename{6appN!} would
lift the value into the store and put a variable reference into the application, leading to another use of \rulename{6mark}, and another use of \rulename{6appN!}, which continues forever.

The rule \rulename{6$\mu$app} handles a well-formed application of a function with a dotted parameter lists. 
It such an application into an application of an
ordinary procedure by constructing a list of the extra arguments. Similarly, the rule \rulename{6$\mu$app1} handles an application of a procedure that has a single variable as its parameter list.

The rule \rulename{6var} handles variable lookup in the store and \rulename{6set} handles variable assignment.

The next two rules \rulename{6proct} and \rulename{6procf} handle applications of \va{procedure?}, and the remaining rules cover applications of non-procedures and arity violations.

\beginfig
\subfigureadjust{}
\begin{center}

%


%

\end{center}
\caption{Apply}\label{fig:Apply}
\endfig
\subfigurestop{}

The rules in figure~\ref{fig:Apply} 
cover \va{apply}. 
The first rule, \rulename{6applyf}, covers the case where the last argument to
\va{apply} is the empty list, and simply reduces by erasing the
empty list and the \va{apply}. The second rule, \rulename{6applyc}
covers a well-formed application of \va{apply} where \va{apply}'s final argument is a pair. It
reduces by extracting the components of the pair from the store and
putting them into the application of \va{apply}. Repeated
application of this rule thus extracts all of the list elements passed
to \va{apply} out of the store. 

The remaining five rules cover the
various violations that can occur when using \va{apply}. The first one covers the case where \va{apply} is supplied with a cyclic list. The next four cover applying a
non-procedure, passing a non-list as the last argument, and supplying
too few arguments to \va{apply}.

\section{Call/cc and dynamic wind}

\beginfig
\begin{center}

%
~
 \\

%

\end{center}
\caption{Call/cc and dynamic wind}\label{fig:Call-cc--and--dynamic-wind}
\endfig

The specification of \va{dynamic-wind} uses 
$\texttt{(}\sy{dw}~x~e~e~e\texttt{)}$
expressions to record which dynamic-wind \var{thunk}s are active at
each point in the computation. Its first argument is an identifier
that is globally unique and serves to identify invocations of
\va{dynamic-wind}, in order to avoid exiting and re-entering the
same dynamic context during a continuation switch. The second, third,
and fourth arguments are calls to some \var{before}, \var{thunk}, and
\var{after} procedures from a call to \va{dynamic-wind}. Evaluation only
occurs in the middle expression; the \sy{dw} expression only
serves to record which \var{before} and \var{after} procedures need to be run during a
continuation switch. Accordingly, the reduction rule for an
application of \va{dynamic-wind} reduces to a call to the
\var{before} procedure, a \sy{dw} expression and a call to the
\var{after} procedure, as
shown in rule \rulename{6wind} in
figure~\ref{fig:Call-cc--and--dynamic-wind}. The next two rules cover
abuses of the \va{dynamic-wind} procedure: calling it with
non-procedures, and calling it with the wrong number of arguments. The
\rulename{6dwdone} rule erases a \sy{dw} expression when its second
argument has finished evaluating.

The next two rules cover \va{call/cc}. The rule
\rulename{6call/cc} creates a new continuation. It takes the context
of the \va{call/cc} expression and packages it up into a
\sy{throw} expression that represents the continuation. The
\sy{throw} expression uses the fresh variable $x$ to record
where the application of \va{call/cc} occurred in the context for
use in the \rulename{6throw} rule when the continuation is applied.
That rule takes the arguments of the continuation, wraps them with a
call to \va{values}, and puts them back into the place where the
original call to \va{call/cc} occurred, replacing the current
context with the context returned by the $\mathscr{T}$ metafunction.

The $\mathscr{T}$ (for ``trim'') metafunction accepts two $D$ contexts and
builds a context that matches its second argument, the destination
context, except that additional calls to the \var{before} and
\var{after} procedures
from \sy{dw} expressions in the context have been added.

The first clause of the $\mathscr{T}$ metafunction exploits the
$H$ context, a context that contains everything except
\sy{dw} expressions. It ensures that shared parts of the
\va{dynamic-wind} context are ignored, recurring deeper into the
two expression contexts as long as the first \sy{dw} expression in
each have matching identifiers ($x_1$). The final rule is a
catchall; it only applies when all the others fail and thus applies
either when there are no \sy{dw}s in the context, or when the
\sy{dw} expressions do not match. It calls the two other
metafunctions defined in figure~\ref{fig:Call-cc--and--dynamic-wind} and
puts their results together into a \sy{begin} expression.

The $\mathscr{R}$ metafunction extracts all of the \var{before}
procedures from its argument and the $\mathscr{S}$ metafunction extracts all of the \var{after} procedures from its argument. They each construct new contexts and exploit
$H$ to work through their arguments, one \sy{dw} at a time.
In each case, the metafunctions are careful to keep the right
\sy{dw} context around each of the procedures in case a continuation
jump occurs during one of their evaluations. 
Since $\mathscr{R}$,
receives the destination context, it keeps the intermediate
parts of the context in its result.
In contrast
$\mathscr{S}$ discards all of the context except the \sy{dw}s,
since that was the context where the call to the
continuation occurred.

\section{Letrec}

\beginfig
\begin{center}

%

\end{center}
\caption{Letrec and letrec*}
\label{fig:Letrec}
\endfig

Figre~\ref{fig:Letrec} shows the rules that handle \sy{letrec} and \sy{letrec*} and the supplementary expressions that they produce, \sy{l!} and \sy{reinit}. As a first approximation, both \va{letrec} and \va{letrec*} reduce by allocating locations in the store to hold the values of the init expressions, initializing those locations to \sy{bh} (for ``black hole''), evaluating the init expressions, and then using \va{l!} to update the locations in the store with the value of the init expressions. They also use \va{reinit} to detect when an init expression in a letrec is reentered via a continuation.

Before considering how \sy{letrec} and \sy{letrec*} use \sy{l!} and \sy{reinit}, first consider how \sy{l!} and \sy{reinit} behave. The first two rules in figure~\ref{fig:Letrec} cover \sy{l!}. It behaves very much like \sy{set!}, but it initializes both ordinary variables, and variables that are current bound to the black hole (\sy{bh}).

The next two rules cover ordinary \sy{set!} when applied to a variable
that is currently bound to a black hole. This situation can arise when
the program assigns to a variable before letrec initializes it, eg
\verb|(letrec ((x (set! x 5))) x)|. The report specifies that either
an implementation should perform the assignment, as reflected in the
\rulename{6setdt} rule or it raise an exception, as reflected in the \rulename{6setdte} rule.

The \rulename{6dt} rule covers the case where a variable is referred
to before the value of a init expression is filled in, which must
always raise an exception.

A \va{reinit} expression is used to detect a program that captures a continuation in an initialization expression and returns to it, as shown in the three rules \rulename{6init}, \rulename{6reinit}, and \rulename{6reinite}. The \va{reinit} form accepts an identifier that is bound in the store to a boolean as its argument. Those are identifiers are initially \semfalse{}. When \va{reinit} is evaluated, it checks the value of the variable and, if it is still \semfalse{}, it changes it to \semtrue{}. If it is already \semtrue{}, then \va{reinit} either just does nothing, or it raises an exception, in keeping with the two legal behaviors of \va{letrec} and \va{letrec*}. 

The last two rules in figure~\ref{fig:Letrec} put together \sy{l!} and \sy{reinit}. The \rulename{6letrec} rule reduces a \sy{letrec} expression to an application expression, in order to capture the unspecified order of evaluation of the init expressions. Each init expression is wrapped in a \sy{begin0} that records the value of the init and then uses \sy{reinit} to detect continuations that return to the init expression. Once all of the init expressions have been evaluated, the procedure on the right-hand side of the rule is invoked, causing the value of the init expression to be filled in the store, and evaluation continues with the body of the original \sy{letrec} expression.

The \rulename{6letrec*} rule behaves similarly, but uses a \sy{begin} expression rather than an application, since the init expressions are evaluated from left to right. Moreover, each init expression is filled into the store as it is evaluated, so that subsequent init expressions can refer to its value.

\section{Underspecification}\label{sec:semantics:underspecification}

\beginfig
\begin{center}

%

\end{center}
\caption{Explicitly unspecified behavior}\label{fig:Underspecification}
\endfig

The rules in figure~\ref{fig:Underspecification} cover aspects of the
semantics that are explicitly unspecified. Implementations can replace
the rules \rulename{6ueqv}, \rulename{6uval} and with different rules that cover the left-hand sides and, as long as they follow the informal specification, any replacement is valid. Those three situations correspond to the case when \va{eqv?} applied to two procedures and when multiple values are used in a single-value context.

The remaining rules in figure~\ref{fig:Underspecification} cover the results from the assignment operations, \sy{set!}, \va{set-car!}, and \va{set-cdr!}. An implementation does not adjust those rules, but instead renders them useless by adjusting the rules that insert \va{unspecified}: \rulename{6setcar}, \rulename{6setcdr}, \rulename{6set}, and \rulename{6setd}. Those rules can be adjusted by replacing \va{unspecified} with any number of values in those rules.

So, the remaining rules just specify the minimal behavior that we know that a value or values must have and otherwise reduce to an \textbf{unknown:} state. The rule \rulename{6udemand} drops \va{unspecified} in the \sy{U} context. See figure~\ref{fig:ec-grammar} for the precise definition of \sy{U}, but intuitively it is a context that is only a single expression layer deep that contains expressions whose value depends on the value of their subexpressions, like the first subexpression of a \sy{if}. Following that are rules that discard \va{unspecified} in expressions that discard the results of some of their subexpressions. The \rulename{6ubegin} shows how \sy{begin} discards its first expression when there are more expressions to evaluate. The next two rules, \rulename{6uhandlers} and \rulename{6udw} propagate \va{unspecified} to their context, since they also return any number of values to their context. Finally, the two \va{begin0} rules preserve \va{unspecified} until the rule \rulename{6begin01} can return it to its context.

%\section*{Acknowledgments}
%Thanks to Michael Sperber for many helpful discussions of specific points in the semantics, for spotting many mistakes and places where the formal semantics diverged from the informal semantics, and for generally making it possible for us to keep up with changes to the informal semantics as it developed. Thanks also to Will Clinger for a careful reading of the semantics and its explanation.

%%% Local Variables: 
%%% mode: latex
%%% TeX-master: "r6rs"
%%% End: 
 \par
\chapter{Sample definitions for derived forms}
\label{derivedformsappendix}

This appendix contains sample definitions for some of the keywords
described in this report in terms of simpler forms:

\subsubsection*{{\tt cond}}
The {\cf cond} keyword (section~\ref{cond}) 
could be defined in terms of {\cf if}, {\cf let} and {\cf
  begin} using {\cf syntax-rules} as follows:

\begin{scheme}
(define-syntax \ide{cond}
  (syntax-rules (else =>)
    ((cond (else result1 result2 ...))
     (begin result1 result2 ...))
    ((cond (test => result))
     (let ((temp test))
       (if temp (result temp))))
    ((cond (test => result) clause1 clause2 ...)
     (let ((temp test))
       (if temp
           (result temp)
           (cond clause1 clause2 ...))))
    ((cond (test)) test)
    ((cond (test) clause1 clause2 ...)
     (let ((temp test))
       (if temp
           temp
           (cond clause1 clause2 ...))))
    ((cond (test result1 result2 ...))
     (if test (begin result1 result2 ...)))
    ((cond (test result1 result2 ...)
           clause1 clause2 ...)
     (if test
         (begin result1 result2 ...)
         (cond clause1 clause2 ...)))))%
\end{scheme}
\subsubsection*{{\tt case}}
The {\cf case} keyword (section~\ref{case}) could be defined in terms of {\cf let}, {\cf cond}, and
{\cf memv} (see library chapter~\extref{lib:listutilities}{List utilities}) using {\cf syntax-rules} as follows:

\begin{scheme}
(define-syntax \ide{case}
  (syntax-rules (else)
    ((case expr0
       ((key ...) res1 res2 ...)
       ...
       (else else-res1 else-res2 ...))
     (let ((tmp expr0))
       (cond
         ((memv tmp '(key ...)) res1 res2 ...)
         ...
         (else else-res1 else-res2 ...))))
    ((case expr0
       ((keya ...) res1a res2a ...)
       ((keyb ...) res1b res2b ...)
       ...)
     (let ((tmp expr0))
       (cond
         ((memv tmp '(keya ...)) res1a res2a ...)
         ((memv tmp '(keyb ...)) res1b res2b ...)
         ...)))))%
\end{scheme}

\subsubsection*{{\tt let*}}

The {\cf let*} keyword (section~\ref{let*})
could be defined in terms of {\cf let}
using {\cf syntax-rules} as follows:

\begin{scheme}
(define-syntax \ide{let*}
  (syntax-rules ()
    ((let* () body1 body2 ...)
     (let () body1 body2 ...))
    ((let* ((name1 expr1) (name2 expr2) ...)
       body1 body2 ...)
     (let ((name1 expr1))
       (let* ((name2 expr2) ...)
         body1 body2 ...)))))%
\end{scheme}

\subsubsection*{{\tt letrec}}
The {\cf letrec} keyword (section~\ref{letrec})
could be defined approximately in terms of {\cf let}
and {\cf set!} using {\cf syntax-rules}, using a helper
to generate the temporary variables
needed to hold the values before the assignments are made,
as follows:

\begin{scheme}
(define-syntax \ide{letrec}
  (syntax-rules ()
    ((letrec () body1 body2 ...)
     (let () body1 body2 ...))
    ((letrec ((var init) ...) body1 body2 ...)
     (letrec-helper
       (var ...)
       ()
       ((var init) ...)
       body1 body2 ...))))

(define-syntax letrec-helper
  (syntax-rules ()
    ((letrec-helper
       ()
       (temp ...)
       ((var init) ...)
       body1 body2 ...)
     (let ((var <undefined>) ...)
       (let ((temp init) ...)
         (set! var temp)
         ...)
       (let () body1 body2 ...)))
    ((letrec-helper
       (x y ...)
       (temp ...)
       ((var init) ...)
       body1 body2 ...)
     (letrec-helper
       (y ...)
       (newtemp temp ...)
       ((var init) ...)
       body1 body2 ...))))%
\end{scheme}

The syntax {\cf <undefined>} represents an expression that
returns something that, when stored in a location, causes an exception
with condition type {\cf\&assertion} to
be raised if an attempt to read from or write to the location occurs before the
assignments generated by the {\cf letrec} transformation take place.
(No such expression is defined in Scheme.)

A simpler definition using {\cf syntax-case} and {\cf
generate-\hp{}temporaries} is given in library
chapter~\extref{lib:syntaxcasechapter}{{\cf syntax-case}}.

\subsubsection*{{\tt letrec*}}

The {\cf letrec*} keyword could be defined approximately in terms of
{\cf let} and {\cf set!}  using {\cf syntax-rules} as follows:

\begin{scheme}
(define-syntax \ide{letrec*}
  (syntax-rules ()
    ((letrec* ((var1 init1) ...) body1 body2 ...)
     (let ((var1 <undefined>) ...)
       (set! var1 init1)
       ...
       (let () body1 body2 ...)))))%
\end{scheme}

The syntax {\cf <undefined>} is as in the definition of {\cf letrec} above.

\subsubsection*{{\tt let-values}}
The following definition of {\cf let-values} (section~\ref{let-values})
using {\cf syntax-rules}
employs a pair of helpers to
create temporary names for the formals.

\begin{scheme}
(define-syntax let-values
  (syntax-rules ()
    ((let-values (binding ...) body1 body2 ...)
     (let-values-helper1
       ()
       (binding ...)
       body1 body2 ...))))

(define-syntax let-values-helper1
  ;; map over the bindings
  (syntax-rules ()
    ((let-values
       ((id temp) ...)
       ()
       body1 body2 ...)
     (let ((id temp) ...) body1 body2 ...))
    ((let-values
       assocs
       ((formals1 expr1) (formals2 expr2) ...)
       body1 body2 ...)
     (let-values-helper2
       formals1
       ()
       expr1
       assocs
       ((formals2 expr2) ...)
       body1 body2 ...))))

(define-syntax let-values-helper2
  ;; create temporaries for the formals
  (syntax-rules ()
    ((let-values-helper2
       ()
       temp-formals
       expr1
       assocs
       bindings
       body1 body2 ...)
     (call-with-values
       (lambda () expr1)
       (lambda temp-formals
         (let-values-helper1
           assocs
           bindings
           body1 body2 ...))))
    ((let-values-helper2
       (first . rest)
       (temp ...)
       expr1
       (assoc ...)
       bindings
       body1 body2 ...)
     (let-values-helper2
       rest
       (temp ... newtemp)
       expr1
       (assoc ... (first newtemp))
       bindings
       body1 body2 ...))
    ((let-values-helper2
       rest-formal
       (temp ...)
       expr1
       (assoc ...)
       bindings
       body1 body2 ...)
     (call-with-values
       (lambda () expr1)
       (lambda (temp ... . newtemp)
         (let-values-helper1
           (assoc ... (rest-formal newtemp))
           bindings
           body1 body2 ...))))))%
\end{scheme}

\subsubsection*{{\tt let*-values}}

The following macro defines {\cf let*-values} in terms of {\cf let}
and {\cf let-values} using {\cf syntax-rules}:

\begin{scheme}
(define-syntax let*-values
  (syntax-rules ()
    ((let*-values () body1 body2 ...)
     (let () body1 body2 ...))
    ((let*-values (binding1 binding2 ...)
       body1 body2 ...)
     (let-values (binding1)
       (let*-values (binding2 ...)
         body1 body2 ...)))))%
\end{scheme}


\subsubsection*{{\tt let}}

The {\cf let} keyword could be defined in terms of {\cf lambda} and {\cf letrec}
using {\cf syntax-rules} as
follows:

\begin{scheme}
(define-syntax \ide{let}
  (syntax-rules ()
    ((let ((name val) ...) body1 body2 ...)
     ((lambda (name ...) body1 body2 ...)
      val ...))
    ((let tag ((name val) ...) body1 body2 ...)
     ((letrec ((tag (lambda (name ...)
                      body1 body2 ...)))
        tag)
      val ...))))%
\end{scheme}


%%% Local Variables: 
%%% mode: latex
%%% TeX-master: "r6rs"
%%% End: 
 \par
\chapter{Additional material}
\label{additionalmaterialappendix}

This report itself, as well as more material related to this report
such as reference implementations of some parts of Scheme and archives of
mailing lists discussing this report is at
\begin{center}
\url{http://www.r6rs.org/}
\end{center}

The Schemers web site at
\begin{center}
\url{http://www.schemers.org/}
\end{center}
as well as the Readscheme site at
\begin{center}
\url{http://library.readscheme.org/}
\end{center}
contain extensive Scheme bibliographies, as well as papers,
programs, implementations, and other material related to Scheme.

%%% Local Variables: 
%%% mode: latex
%%% TeX-master: "r6rs"
%%% End: 
 \par
\chapter{Example }
\label{exampleappendix}

\nobreak
This section describes an example consisting of the
\library{runge-kutta} library, which provides an {\cf integrate-system}
procedure that integrates the system 
$$y_k^\prime = f_k(y_1, y_2, \ldots, y_n), \; k = 1, \ldots, n$$
of differential equations with the method of Runge-Kutta.

As the \library{runge-kutta} library makes use of the \rsixlibrary{base}
library, its skeleton is as follows:

\begin{scheme}
\#!r6rs
(library (runge-kutta)
  (export integrate-system
          head tail)
  (import (rnrs base))
  \hyper{library body})
\end{scheme}

The procedure definitions described below go in the place of \hyper{library body}.

The parameter {\tt system-derivative} is a function that takes a system
state (a vector of values for the state variables $y_1, \ldots, y_n$)
and produces a system derivative (the values $y_1^\prime, \ldots,
y_n^\prime$).  The parameter {\tt initial-state} provides an initial
system state, and {\tt h} is an initial guess for the length of the
integration step.

The value returned by {\cf integrate-system} is an infinite stream of
system states.

\begin{schemenoindent}
(define integrate-system
  (lambda (system-derivative initial-state h)
    (let ((next (runge-kutta-4 system-derivative h)))
      (letrec ((states
                (cons initial-state
                      (lambda ()
                        (map-streams next states)))))
        states))))%
\end{schemenoindent}

The {\cf runge-kutta-4} procedure takes a function, {\tt f}, that produces a
system derivative from a system state.  The {\cf runge-kutta-4} procedure
produces a function that takes a system state and
produces a new system state.

\begin{schemenoindent}
(define runge-kutta-4
  (lambda (f h)
    (let ((*h (scale-vector h))
          (*2 (scale-vector 2))
          (*1/2 (scale-vector (/ 1 2)))
          (*1/6 (scale-vector (/ 1 6))))
      (lambda (y)
        ;; y {\rm{}is a system state}
        (let* ((k0 (*h (f y)))
               (k1 (*h (f (add-vectors y (*1/2 k0)))))
               (k2 (*h (f (add-vectors y (*1/2 k1)))))
               (k3 (*h (f (add-vectors y k2)))))
          (add-vectors y
            (*1/6 (add-vectors k0
                               (*2 k1)
                               (*2 k2)
                               k3))))))))
%|--------------------------------------------------|

(define elementwise
  (lambda (f)
    (lambda vectors
      (generate-vector
        (vector-length (car vectors))
        (lambda (i)
          (apply f
                 (map (lambda (v) (vector-ref  v i))
                      vectors)))))))

%|--------------------------------------------------|
(define generate-vector
  (lambda (size proc)
    (let ((ans (make-vector size)))
      (letrec ((loop
                (lambda (i)
                  (cond ((= i size) ans)
                        (else
                         (vector-set! ans i (proc i))
                         (loop (+ i 1)))))))
        (loop 0)))))

(define add-vectors (elementwise +))

(define scale-vector
  (lambda (s)
    (elementwise (lambda (x) (* x s)))))%
\end{schemenoindent}

The {\cf map-streams} procedure is analogous to {\cf map}: it applies its first
argument (a procedure) to all the elements of its second argument (a
stream).

\begin{schemenoindent}
(define map-streams
  (lambda (f s)
    (cons (f (head s))
          (lambda () (map-streams f (tail s))))))%
\end{schemenoindent}

Infinite streams are implemented as pairs whose car holds the first
element of the stream and whose cdr holds a procedure that delivers the rest
of the stream.

\begin{schemenoindent}
(define head car)
(define tail
  (lambda (stream) ((cdr stream))))%
\end{schemenoindent}

\bigskip
The following program illustrates the use of {\cf integrate-\hp{}system} in
integrating the system
$$ C {dv_C \over dt} = -i_L - {v_C \over R}$$\nobreak
$$ L {di_L \over dt} = v_C$$
which models a damped oscillator.

\begin{schemenoindent}
\#!r6rs
(import (rnrs base)
        (rnrs io simple)
        (runge-kutta))

(define damped-oscillator
  (lambda (R L C)
    (lambda (state)
      (let ((Vc (vector-ref state 0))
            (Il (vector-ref state 1)))
        (vector (- 0 (+ (/ Vc (* R C)) (/ Il C)))
                (/ Vc L))))))

(define the-states
  (integrate-system
     (damped-oscillator 10000 1000 .001)
     '\#(1 0)
     .01))

(letrec ((loop (lambda (s)
                 (newline)
                 (write (head s))
                 (loop (tail s)))))
  (loop the-states))%
\end{schemenoindent}

This prints output like the following:

\begin{scheme}
\#(1 0)
\#(0.99895054 9.994835e-6)
\#(0.99780226 1.9978681e-5)
\#(0.9965554 2.9950552e-5)
\#(0.9952102 3.990946e-5)
\#(0.99376684 4.985443e-5)
\#(0.99222565 5.9784474e-5)
\#(0.9905868 6.969862e-5)
\#(0.9888506 7.9595884e-5)
\#(0.9870173 8.94753e-5)
\end{scheme}

%%% Local Variables: 
%%% mode: latex
%%% TeX-master: "r6rs"
%%% End: 
 \par
\chapter{Language changes}
\label{languagechangesappendix}

This chapter describes most of the changes that have been made to
Scheme since the ``Revised$^5$ Report''~\cite{R5RS} was published:

\begin{itemize}
\item Scheme source code now uses the Unicode character set.
  Specifically, the character set that can be used for identifiers has
  been greatly expanded.
\item Identifiers can now start with the characters {\cf ->}.
\item Identifiers and symbol literals are now case-sensitive.
\item Identifiers and representations of characters, booleans,
  number objects, and {\cf .} must be explicitly delimited.
\item {\cf \sharpsign} is now a delimiter.
\item Bytevector literal syntax has been added.
\item Matched square brackets can be used synonymously with parentheses.
\item The read-syntax abbreviations {\cf \sharpsign{}'} (for {\cf
    syntax}), {\cf \sharpsign\backquote} (for {\cf quasisyntax}), {\cf
    \sharpsign{},} (for {\cf unsyntax}), and {\cf \sharpsign{},@}
  (for {\cf unsyntax-splicing} have been added; see section~\ref{abbreviationsection}.)
\item {\cf \sharpsign} can no longer be used in place of digits in number
  representations.
\item The external representation of number objects can now include a
  mantissa width.
\item Literals for NaNs and infinities were added.
\item String and character literals can now use a variety of escape
  sequences.
\item Block and datum comments have been added.
\item The {\cf \sharpsign{}!r6rs} comment for marking report-compliant
  lexical syntax has been added.
\item Characters are now specified to correspond to Unicode scalar
  values.
\item Many of the procedures and syntactic forms of the language are
  now part of the \rsixlibrary{base} library.  Some procedures and
  syntactic forms have been moved to other libraries; see figure~\ref{r5rsmovedfigure}.

  \begin{figure*}[tb]
    \centering
    \small
    \begin{tabular}[t]{ll}
      identifier & moved to \\\hline
      {\cf assoc} & \rsixlibrary{lists} \\
      {\cf assv} & \rsixlibrary{lists} \\
      {\cf assq} & \rsixlibrary{lists} \\
      {\cf call-with-input-file} & \rsixlibrary{io simple} \\
      {\cf call-with-output-file} & \rsixlibrary{io simple} \\
      {\cf char-upcase} & \rsixlibrary{unicode} \\
      {\cf char-downcase} & \rsixlibrary{unicode} \\
      {\cf char-ci=?} & \rsixlibrary{unicode} \\
      {\cf char-ci<?} & \rsixlibrary{unicode} \\
      {\cf char-ci>?} & \rsixlibrary{unicode} \\
      {\cf char-ci<=?} & \rsixlibrary{unicode} \\
      {\cf char-ci>=?} & \rsixlibrary{unicode} \\
      {\cf char-alphabetic?} & \rsixlibrary{unicode} \\
      {\cf char-numeric?} & \rsixlibrary{unicode} \\
      {\cf char-whitespace?} & \rsixlibrary{unicode} \\
      {\cf char-upper-case?} & \rsixlibrary{unicode} \\
      {\cf char-lower-case?} & \rsixlibrary{unicode} \\
      {\cf close-input-port} & \rsixlibrary{io simple} \\
      {\cf close-output-port} & \rsixlibrary{io simple} \\
      {\cf current-input-port} & \rsixlibrary{io simple} \\
      {\cf current-output-port} & \rsixlibrary{io simple} \\
      {\cf display} & \rsixlibrary{io simple} \\
      {\cf do} & \rsixlibrary{control} \\
      {\cf eof-object?} & \rsixlibrary{io simple} \\
      {\cf eval} & \rsixlibrary{eval} \\
      {\cf delay} & \rsixlibrary{r5rs}\\
      {\cf exact->inexact} & \rsixlibrary{r5rs}\\
      {\cf force} & \rsixlibrary{r5rs}
\htmlonly \\ \endhtmlonly
\texonly
    \end{tabular}
    \qquad
    \begin{tabular}[t]{ll}
      identifier & moved to \\\hline
\endtexonly
      {\cf inexact->exact} & \rsixlibrary{r5rs}\\
      {\cf member} & \rsixlibrary{lists} \\
      {\cf memv} & \rsixlibrary{lists} \\
      {\cf memq} & \rsixlibrary{lists} \\
      {\cf modulo} & \rsixlibrary{r5rs} \\
      {\cf newline} & \rsixlibrary{io simple} \\
      {\cf null-environment} & \rsixlibrary{r5rs} \\
      {\cf open-input-file} & \rsixlibrary{io simple} \\
      {\cf open-output-file} & \rsixlibrary{io simple} \\
      {\cf peek-char} & \rsixlibrary{io simple} \\
      {\cf quotient} & \rsixlibrary{r5rs} \\
      {\cf read} & \rsixlibrary{io simple} \\
      {\cf read-char} & \rsixlibrary{io simple} \\
      {\cf remainder} & \rsixlibrary{r5rs} \\
      {\cf scheme-report-environment} & \rsixlibrary{r5rs} \\
      {\cf set-car!} & \rsixlibrary{mutable-pairs} \\
      {\cf set-cdr!} & \rsixlibrary{mutable-pairs} \\
      {\cf string-ci=?} & \rsixlibrary{unicode} \\
      {\cf string-ci<?} & \rsixlibrary{unicode} \\
      {\cf string-ci>?} & \rsixlibrary{unicode} \\
      {\cf string-ci<=?} & \rsixlibrary{unicode} \\
      {\cf string-ci>=?} & \rsixlibrary{unicode} \\
      {\cf string-set!} & \rsixlibrary{mutable-strings} \\
      {\cf string-fill!} & \rsixlibrary{mutable-strings} \\
      {\cf with-input-from-file} & \rsixlibrary{io simple} \\
      {\cf with-output-to-file} & \rsixlibrary{io simple} \\
      {\cf write} & \rsixlibrary{io simple} \\
      {\cf write-char} & \rsixlibrary{io simple}
    \end{tabular}
    \caption{Identifiers moved to libraries}
    \label{r5rsmovedfigure}
  \end{figure*}
\item The base language has the following new procedures and syntactic
  forms: {\cf letrec*}, {\cf let-values}, {\cf let*-\hp{}values}, {\cf
    real-valued?}, {\cf rational-valued?}, {\cf integer-valued?}, {\cf
    exact}, {\cf inexact}, {\cf finite?}, {\cf infinite?}, {\cf nan?},
  {\cf div}, {\cf mod}, {\cf
    div-and-mod}, {\cf div0}, {\cf mod0}, {\cf div0-and-mod0}, {\cf
    exact-integer-sqrt}, {\cf boolean=?}, {\cf symbol=?}, {\cf
    string-for-each}, {\cf vector-map}, {\cf vector-\hp{}for-\hp{}each}, {\cf
    error}, {\cf assertion-violation}, {\cf assert}, {\cf call/cc},
  {\cf identifier-syntax}.
\item The following procedures have been removed: {\cf
    char-\hp{}ready?}, {\cf transcript-on}, {\cf transcript-off},
  {\cf load}.
\item The case-insensitive string comparisons ({\cf string-\hp{}ci=?}, {\cf
    string-\hp{}ci<?}, {\cf string-ci>?}, {\cf string-ci<=?}, {\cf
    string-ci>=?}) operate on the case-folded versions of the strings
  rather than as the simple lexicographic ordering induced by the
  corresponding character comparison procedures.
\item Libraries have been added to the language.
\item A number of standard libraries are described in a separate
  report~\cite{R6RS-libraries}.
\item Many situations that ``were an error'' now have defined or
  constrained behavior.  In particular, many are now specified in
  terms of the exception system.
\item The full numerical tower is now required.
\item The semantics for the transcendental functions has been
  specified more fully.
\item The semantics of {\cf expt} for zero bases has been refined.
\item In {\cf syntax-rules} forms, a {\cf\_} may be used in place of
  the keyword.
\item The {\cf let-syntax} and {\cf letrec-syntax} no longer introduce a
  new environment for their bodies.
\item For implementations that support NaNs or infinities,
  many arithmetic operations have been specified on
  these values consistently with IEEE~754.
\item For implementations that support a distinct -0.0, the semantics
  of many arithmetic operations with regard to -0.0 has been specified
  consistently with IEEE~754.
\item Scheme's real number objects now have an exact zero as their imaginary part.
\item The specification of {\cf quasiquote} has been extended.  Nested
  quasiquotations work correctly now, and {\cf unquote} and {\cf
    unquote-splicing} have been extended to several operands.
\item Procedures now may or may not refer to
  locations.  Consequently, {\cf eqv?} is now unspecified in a few
  cases where it was specified before.
\item The mutability of the values of {\cf quasiquote} structures has
  been specified to some degree.
\item The dynamic environment of the \var{before} and \var{after}
  procedures of {\cf dynamic-wind} is now specified.
\item Various expressions that have only side effects are now allowed
  to return an arbitrary number of values.
\item The order and semantics for macro expansion has been more fully
  specified.
\item Internal definitions are now defined in terms of {\cf letrec*}.
\item The old notion of program structure and Scheme's top-level
  environment has been replaced by top-level programs and libraries.
\item The denotational semantics has been replaced by an operational
  semantics based on an earlier semantics for the language of the
  ``Revised$^5$ Report''~\cite{R5RS,mf:scheme-op-sem}.
\end{itemize}

%%% Local Variables: 
%%% mode: latex
%%% TeX-master: "r6rs"
%%% End: 
 \par
\newpage
\renewcommand{\bibname}{References}

\bibliographystyle{plain}
\bibliography{abbrevs,rrs}

\vfill\eject


\newcommand{\indexheading}{Alphabetic index of definitions of
  concepts, keywords, and procedures}
\texonly
\newcommand{\indexintro}{The index includes entries from the library
  document; the entries are marked with ``(library)''.}
\endtexonly

\printindex

\end{document}
