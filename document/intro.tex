%\clearextrapart{Introduction}
\clearextrapart{Введение}

\label{historysection}

%Programming languages should be designed not by piling feature on top of
%feature, but by removing the weaknesses and restrictions that make additional
%features appear necessary.  Scheme demonstrates that a very small number
%of rules for forming expressions, with no restrictions on how they are
%composed, suffice to form a practical and efficient programming language
%that is flexible enough to support most of the major programming
%paradigms in use today.
Языки программирования должны разрабатываться не путём накопления функциональных возможностей, а
путём удаления слабостей и ограничений, приводящих к мнимой необходимости дополнительных
средств. Scheme демонстрирует, что очень небольшого количества правил формирования выражений
без ограничений на их составление достаточно для формирования практического и эффективного языка
программирования, достаточно гибкого для поддержки большинства основных парадигм
программирования, используемых в настоящее время.

%Scheme
%was one of the first programming languages to incorporate first-class
%procedures as in the lambda calculus, thereby proving the usefulness of
%static scope rules and block structure in a dynamically typed language.
%Scheme was the first major dialect of Lisp to distinguish procedures
%from lambda expressions and symbols, to use a single lexical
%environment for all variables, and to evaluate the operator position
%of a procedure call in the same way as an operand position.  By relying
%entirely on procedure calls to express iteration, Scheme emphasized the
%fact that tail-recursive procedure calls are essentially gotos that
%pass arguments.  Scheme was the first widely used programming language to
%embrace first-class escape procedures, from which all previously known
%sequential control structures can be synthesized.  A subsequent
%version of Scheme introduced the concept of exact and inexact number objects,
%an extension of Common Lisp's generic arithmetic.
%More recently, Scheme became the first programming language to support
%hygienic macros, which permit the syntax of a block-structured language
%to be extended in a consistent and reliable manner.
Scheme был одним из первых языков программирования, включающим полноправные процедуры, как в
лямбда-исчислении, тем самым доказывая полноценность статических правил области видимости и
блочной структуры в языке с динамической типизацией. Scheme был первым основным диалектом Lisp,
различающим процедуры из лямбда-выражений и символы, использующий единственное лексическое
окружение для всех переменных и вычисляющий позицию оператора вызова процедуры таким же образом,
как и позицию операнда. Полностью основыванный на вызовах процедур для отражения итерации,
Scheme подчеркивает факт, что хвост-рекурсивные вызовы процедур являются по существу командами
перехода, передающими аргументы. Scheme был первым широко применяемым языком программирования,
использующим полноправные управляющие процедуры, из которых могут синтезироваться все известные
ранее последовательные управляющие структуры. В последующую версию Scheme введено понятие точных
и неточных числовых объектов, расширение обобщённой арифметики Common Lisp. Позже Scheme стал
первым языком программирования, поддерживающим гигиенические макросы, предоставляющие
возможность расширения синтаксиса языка с блочной структурой логичным и безопасным способом.\vspace{-3mm}

%\subsection*{Guiding principles}
\subsection*{Руководящие принципы}

%To help guide the standardization effort, the editors have adopted a
%set of principles, presented below.
%Like the Scheme language defined in \rrs{5}~\cite{R5RS}, the language described
%in this report is intended to:
Для содействия в проведении работы по стандартизации редакторы приняли ряд принципов, представленных
ниже. Как и язык Scheme, определённый в \rrs{5}~\cite{R5RS}, язык, описанный в данном стандарте,
предназначен для:\vspace{-1mm}

\begin{itemize}
%\item allow programmers to read each other's code, and allow
%  development of portable programs that can be executed in any
%  conforming implementation of Scheme;
\item предоставления программистам возможности чтения кода друг друга и разработки
  переносимых программ, которые могут быть выполнены в любой соответствующей реализации Scheme;

%\item derive its power from simplicity, a small number of generally
%  useful core syntactic forms and procedures, and no unnecessary
%  restrictions on how they are composed;
\item извлечения пользы из его эффективности за счёт простоты, небольшого количества основных
  синтаксических форм и процедур общего применения и отсутствия ненужных ограничений на их
  составление;

%\item allow programs to define new procedures and new hygienic
%  syntactic forms;
\item предоставления возможности определения в программах новых процедур и гигиенических
синтаксических форм;

%\item support the representation of program source code as data;
\item поддержки представления исходного текста программы в качестве данных;

%\item make procedure calls powerful enough to express any form of
%  sequential control, and allow programs to perform non-local control
%  operations without the use of global program transformations;
\item достижения эффективности вызовов процедур, достаточной для выражения любой формы
  последовательного управления и предоставления возможности выполнения в программах нелокальных
  управляющих операций без использования глобальных преобразований программы;\vspace{-1.2mm}

%\item allow interesting, purely functional programs to run indefinitely
%  without terminating or running out of memory on finite-memory
%  machines;
\item предоставления возможности неопределённо долгой работы интересных, чисто функциональных
  программ без остановки или выхода за границы памяти на машинах с конечной памятью;\vspace{-1.2mm}

%\item allow educators to use the language to teach programming
%  effectively, at various levels and with a variety of pedagogical
%  approaches; and
\item предоставления педагогам возможности применения языка для эффективного преподавания
  программирования на различных уровнях и с разнообразными педагогическими подходами; и\vspace{-1.2mm}

%\item allow researchers to use the language to explore the design,
%  implementation, and semantics of programming languages.
\item предоставления научным работникам возможности применения языка для изучения разработки,
  реализации и семантики языков программирования.
\end{itemize}\vspace{-1.2mm}

%In addition, this report is intended to:
Кроме того, данный стандарт предназначен для:\vspace{-1.2mm}

\begin{itemize}
%\item allow programmers to create and distribute substantial programs
%  and libraries, e.g., implementations of Scheme Requests for
%  Implementation, that run without
%  modification in a variety of Scheme implementations;
\item предоставления программистам возможности создания и распространения фундаментальных
  программ и библиотек, например, реализаций Scheme Requests for Implementation, выполняемых без
  модификации в разнообразных реализациях Scheme;\vspace{-1.2mm}

%\item support procedural, syntactic, and data abstraction more fully
%  by allowing programs to define hygiene-bending and hygiene-breaking
%  syntactic abstractions and new unique datatypes along with
%  procedures and hygienic macros in any scope;
\item максимально полной поддержки абстракций -- процедурных, синтаксических и данных, что
  позволяет определять в программах (сгибающие?) гигиену и (нарушающие?) гигиену синтаксические
  абстракции и новые уникальные типы данных наряду с процедурами и гигиеническими макросами в
  любых областях видимости;\vspace{-1.2mm}

%\item allow programmers to rely on a level of automatic run-time type
%  and bounds checking sufficient to ensure type safety; and
\item предоставления программистам возможности руководствоваться уровнем
  автоматической проверки типов и границ этапа выполнения,
  достаточным для гарантии безопасности типов; и\vspace{-1.2mm}

%\item allow implementations to generate efficient code, without
%  requiring programmers to use implementation-specific operators or
%  declarations.
\item предоставления реализациям возможности генерации эффективного кода, не требуя от
  программистов применения операторов или деклараций конкретной реализации.

\end{itemize}\vspace{-1.2mm}

%While it was possible to write portable programs in Scheme as
%described in \rrs{5}, and indeed portable Scheme programs were written
%prior to this report, many Scheme programs were not, primarily because
%of the lack of substantial standardized libraries and the
%proliferation of implementation-specific language additions.
В то время как существовала возможность написания переносимых программ на Scheme согласно
\rrs{5}, и безусловно переносимые программы Scheme были написаны до данного стандарта, множество
программ не было написано на Scheme прежде всего из-за нехватки фундаментальных
стандартизированных библиотек и чрезмерного количества зависящих от реализации языковых
дополнений.

%In general, Scheme should include building blocks that allow a wide
%variety of libraries to be written, include commonly used user-level
%features to enhance portability and readability of library and
%application code, and exclude features that are less commonly used and
%easily implemented in separate libraries.
В общем случае Scheme должен содержать стандартные блоки, позволяющие создавать широкий спектр
библиотек, включая универсальные средства пользовательского уровня для улучшения переносимости и
удобочитаемости библиотек и прикладного кода, и исключать средства, реже используемые и легко
реализуемые в отдельных библиотеках.

%The language described in this report is intended to also be backward
%compatible with programs written in Scheme as described in \rrs{5} to
%the extent possible without compromising the above principles and
%future viability of the language.  With respect to future viability,
%the editors have operated under the assumption that many more Scheme
%programs will be written in the future than exist in the present, so
%the future programs are those with which we should be most concerned.
Язык, описанный в данном стандарте, также подразумевает обратную совместимость и с программами,
написанными на Scheme согласно \rrs{5} при условии отсутствия компромисса между вышеописанными
основными положениями и будущей эффективностью языка. Принимая во внимание будущую эффективность,
редакторы исходили из предположения, что в будущем на Scheme будет написано больше программ,
чем существует в настоящее время, таким образом, больше всего мы должны быть
заинтересованы в будущих программах.\vspace{3mm}

%\subsection*{Acknowledgements}
\subsection*{Благодарности}\vspace{4mm}

%Many people contributed significant help to this revision of the
%report.  Specifically, we thank Aziz Ghuloum and Andr\'e van Tonder for
%contributing reference implementations of the library system.  We
%thank Alan Bawden, John Cowan, Sebastian Egner, Aubrey Jaffer, Shiro
%Kawai, Bradley Lucier, and Andr\'e van Tonder for contributing insights on
%language design.  Marc Feeley, Martin Gasbichler, Aubrey Jaffer, Lars T Hansen,
%Richard Kelsey, Olin Shivers, and Andr\'e van Tonder wrote SRFIs that
%served as direct input to the report.  Marcus Crestani, David Frese,
%Aziz Ghuloum, Arthur A.\ Gleckler, Eric Knauel, Jonathan Rees, and Andr\'e
%van Tonder thoroughly proofread early versions of the report.
Множество людей внесли существенный вклад в данную редакцию стандарта. В связи с этим мы
благодарим Aziz Ghuloum и Andr\'e van Tonder за предоставление эталонной реализации системы
библиотек. Мы благодарим Alan Bawden, John Cowan, Sebastian Egner, Aubrey Jaffer, Shiro Kawai,
Bradley Lucier и Andr\'e van Tonder за предоставление уникальной информации о разработке языка.
Marc Feeley, Martin Gasbichler, Aubrey Jaffer, Lars T Hansen, Richard Kelsey, Olin Shivers и
Andr\'e van Tonder писали SRFI, которые помогали непосредственным вкладом в стандарт. Marcus
Crestani, David Frese, Aziz Ghuloum, Arthur A.\ Gleckler, Eric Knauel, Jonathan Rees и Andr\'e
van Tonder тщательно вычитывали ранние версии стандарта.\vspace{4mm}

%We would also like to thank the following people for their
%help in creating this report: Lauri Alanko,
%Eli Barzilay, Alan Bawden, Brian C.\ Barnes, Per Bothner, Trent Buck,
%Thomas Bushnell, Taylor Campbell, Ludovic Court\`es, Pascal Costanza,
%John Cowan, Ray Dillinger, Jed Davis, J.A.\ ``Biep'' Durieux, Carl Eastlund,
%Sebastian Egner, Tom Emerson, Marc Feeley, Matthias Felleisen, Andy
%Freeman, Ken Friedenbach, Martin Gasbichler, Arthur A.\ Gleckler, Aziz
%Ghuloum, Dave Gurnell, Lars T Hansen, Ben Harris, Sven Hartrumpf, Dave
%Herman, Nils M.\ Holm, Stanislav Ievlev, James Jackson, Aubrey Jaffer,
%Shiro Kawai, Alexander Kjeldaas, Eric Knauel, Michael Lenaghan, Felix Klock,
%Donovan Kolbly, Marcin Kowalczyk, Thomas Lord, Bradley Lucier, Paulo
%J.\ Matos, Dan Muresan, Ryan Newton, Jason Orendorff, Erich Rast, Jeff
%Read, Jonathan Rees, Jorgen Sch\"afer, Paul Schlie, Manuel Serrano,
%Olin Shivers, Jonathan Shapiro, Jens Axel S\o{}gaard, Jay Sulzberger,
%Pinku Surana, Mikael Tillenius, Sam Tobin-Hochstadt, David Van Horn,
%Andr\'e van Tonder, Reinder Verlinde, Alan Watson, Andrew Wilcox, Jon
%Wilson, Lynn Winebarger, Keith Wright, and Chongkai Zhu.
Мы также хотели бы поблагодарить следующих людей за их помощь в создании данного стандарта: Lauri Alanko,
Eli Barzilay, Alan Bawden, Brian C.\ Barnes, Per Bothner, Trent Buck, Thomas Bushnell, Taylor
Campbell, Ludovic Court\`es, Pascal Costanza, John Cowan, Ray Dillinger, Jed Davis,
J.A.\ ``Biep'' Durieux, Carl Eastlund, Sebastian Egner, Tom Emerson, Marc Feeley, Matthias
Felleisen, Andy Freeman, Ken Friedenbach, Martin Gasbichler, Arthur A.\ Gleckler, Aziz Ghuloum,
Dave Gurnell, Lars T Hansen, Ben Harris, Sven Hartrumpf, Dave Herman, Nils M.\ Holm, Stanislav
Ievlev, James Jackson, Aubrey Jaffer, Shiro Kawai, Alexander Kjeldaas, Eric Knauel, Michael
Lenaghan, Felix Klock, Donovan Kolbly, Marcin Kowalczyk, Thomas Lord, Bradley Lucier, Paulo
J.\ Matos, Dan Muresan, Ryan Newton, Jason Orendorff, Erich Rast, Jeff Read, Jonathan Rees,
Jorgen Sch\"afer, Paul Schlie, Manuel Serrano, Olin Shivers, Jonathan Shapiro, Jens Axel
S\o{}gaard, Jay Sulzberger, Pinku Surana, Mikael Tillenius, Sam Tobin-Hochstadt, David Van Horn,
Andr\'e van Tonder, Reinder Verlinde, Alan Watson, Andrew Wilcox, Jon Wilson, Lynn Winebarger,
Keith Wright, and Chongkai Zhu.\vspace{4mm}

%We would like to thank the following people for their help in creating
%the previous revisions of this report: Alan Bawden, Michael
%Blair, George Carrette, Andy Cromarty, Pavel Curtis, Jeff Dalton, Olivier Danvy,
%Ken Dickey, Bruce Duba, Marc Feeley,
%Andy Freeman, Richard Gabriel, Yekta G\"ursel, Ken Haase, Robert
%Hieb, Paul Hudak, Morry Katz, Chris Lindblad, Mark Meyer, Jim Miller, Jim Philbin,
%John Ramsdell, Mike Shaff, Jonathan Shapiro, Julie Sussman,
%Perry Wagle, Daniel Weise, Henry Wu, and Ozan Yigit.
Мы хотели бы благодарить следующих людей за их помощь в создании предыдущих редакций данного
стандарта: Alan Bawden, Michael Blair, George Carrette, Andy Cromarty, Pavel Curtis, Jeff
Dalton, Olivier Danvy, Ken Dickey, Bruce Duba, Marc Feeley, Andy Freeman, Richard Gabriel, Yekta
G\"ursel, Ken Haase, Robert Hieb, Paul Hudak, Morry Katz, Chris Lindblad, Mark Meyer, Jim
Miller, Jim Philbin, John Ramsdell, Mike Shaff, Jonathan Shapiro, Julie Sussman, Perry Wagle,
Daniel Weise, Henry Wu и Ozan Yigit.\vspace{4mm}

%We thank Carol Fessenden, Daniel
%Friedman, and Christopher Haynes for permission to use text from the Scheme 311
%version 4 reference manual.  We thank Texas Instruments, Inc.~for permission to
%use text from the {\em TI Scheme Language Reference Manual}~\cite{TImanual85}.
%We gladly acknowledge the influence of manuals for MIT Scheme~\cite{MITScheme},
%T~\cite{Rees84}, Scheme 84~\cite{Scheme84}, Common Lisp~\cite{CLtL},
%Chez Scheme~\cite{csug7}, PLT~Scheme~\cite{mzscheme352},
%and Algol 60~\cite{Naur63}.
Мы благодарим Carol Fessenden, Daniel
Friedman и Christopher Haynes за разрешение использовать текст из Scheme 311
version 4 reference manual.  Мы благодарим Texas Instruments, Inc.~за разрешение
использовать текст из {\em TI Scheme Language Reference Manual}~\cite{TImanual85}.
Мы охотно признаём влияние руководств MIT Scheme~\cite{MITScheme},
T~\cite{Rees84}, Scheme 84~\cite{Scheme84}, Common Lisp~\cite{CLtL},
Chez Scheme~\cite{csug7}, PLT~Scheme~\cite{mzscheme352},
и Algol 60~\cite{Naur63}.\vspace{4mm}

%\vest We also thank Betty Dexter for the extreme effort she put into
%setting this report in \TeX, and Donald Knuth for designing the program
%that caused her troubles.
\vest Мы также благодарим Betty Dexter за титанические усилия, которые она вложила при
вводе данного стандарта в \TeX и Donald Knuth за разработку программы,
вызвавшей эти трудности.\vspace{4mm}

\vest The Artificial Intelligence Laboratory of the
Massachusetts Institute of Technology, the Computer Science
Department of Indiana University, the Computer and Information
Sciences Department of the University of Oregon, and the NEC Research
Institute supported the preparation of this report.  Support for the MIT
work was provided in part by
the Advanced Research Projects Agency of the Department of Defense under Office
of Naval Research contract N00014-80-C-0505.  Support for the Indiana
University work was provided by NSF grants NCS 83-04567 and NCS
83-03325.


%%% Local Variables:
%%% mode: latex
%%% TeX-master: "r6rs"
%%% End:
