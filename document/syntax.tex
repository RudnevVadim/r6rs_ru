%\chapter{Primitive syntax}
\chapter{Синтаксис примитивов}\vspace{2mm}

%After the {\cf import} form within a {\cf library} form or a top-level
%program, the forms
%that constitute the body of the library or the top-level program
%depend on the libraries that are
%imported. In particular, imported syntactic keywords determine
%the available syntactic abstractions and whether each form is a
%definition or expression. A few form types are
%always available independent of imported libraries, however,
%including constant literals, variable references, procedure calls,
% and macro uses.
После формы {\bfseries\cf import} внутри формы {\bfseries\cf library} или программы верхнего уровня,
формы, составляющие тело библиотеки или программы верхнего уровня, зависят от
импортированных библиотек.
В частности, импортированные синтаксические ключевые слова определяют
доступные синтаксические абстракции и является ли каждая форма определением или
выражением. Некоторые типы форм всегда доступны независимо от импортированных библиотек,
однако, включая константные литералы, обращения к переменным, вызовы
процедур и применения макросов.

%\section{Primitive expression types}
\section{Типы примитивных выражений}
\label{primitiveexpressionsection}

%The entries in this section all describe expressions, which may occur
%in the place of \hyper{expression} syntactic variables.  See
%also section~\ref{expressionsection}.
Все записи в этой секции описывают выражения, которые могут находиться на месте синтаксических
переменных \hyper{expression}. См. также секцию~\ref{expressionsection}.

%\subsection*{Constant literals}\unsection
\subsection*{Константные литералы}\unsection

\begin{entry}{%
\pproto{\hyper{number}}{\exprtype}
\pproto{\hyper{boolean}}{\exprtype}
\pproto{\hyper{character}}{\exprtype}
\pproto{\hyper{string}}{\exprtype}
\pproto{\hyper{bytevector}}{\exprtype}}\mainindex{literal}\vspace{1mm}

%An expression consisting of a representation of a number object, a
%boolean, a character, a string, or a bytevector, evaluates ``to
%itself''.
Выражение, состоящее из представления числового объекта, булевого значения, символа, строки или
байт-вектора, вычисляется ``как есть''.\vspace{1mm}

\begin{scheme}
\bfseries 145932     \ev  \bfseries 145932
\bfseries \schtrue   \ev  \bfseries \schtrue
\bfseries "abc"      \ev  \bfseries "abc"
\bfseries \#vu8(2 24 123) \ev \bfseries\#vu8(2 24 123)%
\end{scheme}\vspace{1mm}

%As noted in section~\ref{storagemodel}, the value of a literal
%expression is immutable.
Как отмечено в секции ~\ref{storagemodel}, значение литерального выражения является неизменяемым.
\end{entry}

%\subsection*{Variable references}\unsection
\subsection*{Обращения к переменным}\unsection
\begin{entry}{%
\pproto{\hyper{variable}}{\exprtype}}\vspace{1mm}

%An expression consisting of a variable\index{variable}
%(section~\ref{variablesection}) is a variable reference if it is not a
%macro use (see below).  The value of
%the variable reference is the value stored in the location to which the
%variable is bound.  It is a syntax violation to reference
%an unbound\index{unbound} variable.
Выражение, состоящее из переменной\index{variable} (секция~\ref{variablesection}), является
обращением к переменной в случае, если оно не является применением макроса (см. ниже). Значением
обращения к переменной является значение,
хранящееся в области памяти, с которой связана переменная.
Обращение к непривязанной \index{unbound} переменной является нарушением синтаксиса.

%The following example examples assumes the base library
%has been imported:
В следующих примерах предполагается, что основная библиотека импортирована:\vspace{1mm}
%
\begin{scheme}
\bfseries(define x 28)
\bfseries x   \ev  \bfseries 28%
\end{scheme}\vspace{1mm}
\end{entry}

%\subsection*{Procedure calls}\unsection
\subsection*{Вызовы процедур}\unsection

\begin{entry}{%
\pproto{\textbf{(}\hyper{operator} \hyperi{operand} \dotsfoo\textbf{)}}{\exprtype}}\vspace{1mm}

%A procedure call consists of expressions for the procedure to be
%called and the arguments to be passed to it, with enclosing
%parentheses.  A form in an expression context is a procedure call if
%\hyper{operator} is not an identifier bound as a syntactic keyword
%(see section~\ref{macrosection} below).
Вызов процедуры состоит из заключённых в круглые скобки выражений вызова процедуры и передаваемых ей
аргументов. Форма в контексте выражения является вызовом процедуры, если \hyper{operator} не
является идентификатором, привязанным к синтаксическому ключевому слову
(см. секцию~\ref{macrosection} ниже).

%When a procedure call is evaluated, the operator and operand
%expressions are evaluated (in an unspecified order) and the resulting
%procedure is passed the resulting
%arguments.\mainindex{call}\mainindex{procedure call}
При вычислении вызова процедуры вычисляются (в произвольном порядке) выражения оператора и операндов
и полученной процедуре передаются полученные аргументы.\mainindex{call}\mainindex{procedure
  call}

%The following examples assume the \rsixlibrary{base} library
%has been imported:
В следующих примерах предполагается, что библиотека \textbf{\rsixlibrary{base}} импортирована.\vspace{1mm}
%
\begin{scheme}%
\bfseries (+ 3 4)                          \ev\bfseries 7
\bfseries ((if \schfalse + *) 3 4)         \ev\bfseries 12%
\end{scheme}\vspace{1mm}
%
%If the value of \hyper{operator} is not a procedure, an exception with
%condition type {\cf\&assertion} is raised.  Also, if \hyper{operator}
%does not accept as many arguments as there are \hyper{operand}s, an
%exception with condition type {\cf\&assertion} is raised.

Если значение \hyper{operator} не является процедурой, возбуждается исключение с типом состояния
{\cf\bfseries\&assertion}. Если количество \hyper{operand} превышает количество принимаемых
\hyper{operand} аргументом, также возбуждается исключение с типом состояния {\cf\bfseries\&assertion}.

\begin{note} %In contrast to other dialects of Lisp, the order of
%evaluation is unspecified, and the operator expression and the operand
%expressions are always evaluated with the same evaluation rules.
В отличие от других диалектов Lisp порядок вычисления не определён, и выражение оператора, и
выражения операнда всегда вычисляются с теми же правилами вычисления.

%Although the order of evaluation is otherwise unspecified, the effect of
%any concurrent evaluation of the operator and operand expressions is
%constrained to be consistent with some sequential order of evaluation.
%The order of evaluation may be chosen differently for each procedure call.
Хотя порядок вычисления иначе не определён, эффект любого параллельного вычисления выражений
оператора и операнда ограничен быть совместимым с некоторым последовательным порядком
вычисления. Порядок вычисления может быть выбран другим при каждом вызове процедуры.
\end{note}

%\newpage

\begin{note} %In many dialects of Lisp, the form {\tt
%()} is a legitimate expression.  In Scheme, expressions written as
%list/pair forms must have at
%least one subexpression, so {\tt ()} is not a syntactically valid
%expression.
Во многих диалектах Lisp форма {\tt\textbf{()}} является допустимым выражением. В Scheme
выражения, записанные как формы списков/пар, должны иметь по крайней мере одно подвыражение, таким
образом, {\tt\textbf{()}} не является синтаксически допустимым выражением.
\end{note}

\end{entry}

%\section{Macros}
\section{Макросы}\vspace{1mm}
\label{macrosection}

%Libraries and top-level programs can define and use new kinds of derived expressions and
%definitions called {\em syntactic abstractions} or
%{\em macros}.\mainindex{syntactic abstraction}\mainindex{macro}
%A syntactic abstraction is created by binding a keyword to a
%{\em macro transformer} or, simply, {\em transformer}.
%\index{macro transformer}\index{transformer}
%The transformer determines
%how a use of the macro (called a \defining{macro use})
%is transcribed into a more primitive form.
В библиотеках и программах верхнего уровня могут определяться и использоваться новые виды
производных выражений и определений, называемых {\em синтаксическими абстракциями} или {\em
  макросами}.\mainindex{syntactic abstraction} \mainindex {macro} Синтаксическая абстракция
создаётся путём привязки ключевого слова к {\em макротрансформеру}, или просто {\em
  трансформеру}.\index{macro transformer}\index{macro transformer} Трансформер
определяет, как применение макроса (называемое \defining{макроприменением}) расшифровывается в
более примитивную форму.\vspace{1mm}

%Most macro uses have the form:
Большинство макросов имеют форму:\vspace{1mm}
\begin{scheme}
\textbf{(}\hyper{keyword} \hyper{datum} \dotsfoo\textbf{)}%
\end{scheme}\vspace{1mm}%
%where \hyper{keyword} is an identifier that uniquely determines the
%kind of form.  This identifier is called the {\em syntactic
%keyword}\index{syntactic keyword}, or simply {\em
%keyword}\index{keyword}, of the macro\index{macro keyword}.
%The number of \hyper{datum}s and the syntax
%of each depends on the syntactic abstraction.
где \hyper{keyword} -- идентификатор, уникально определяющий вид формы. Этот
идентификатор называется {\em синтаксическим ключевым словом}\index{syntactic keyword},
или просто {\em ключевым словом}\index{keyword}, макроса\index{macro keyword}. Количество
\hyper{datum} и синтаксис каждого из них зависит от синтаксической абстракции.\vspace{1mm}

%Macro uses can also take the form of improper lists, singleton
%identifiers, or {\cf set!} forms, where the second subform of the
%{\cf set!} is the keyword (see section~\ref{identifier-syntax})
%library section~\extref{lib:make-variable-transformer}{{\cf make-variable-transformer}}):
Макроприменения могут также принимать форму нестрогих списков, еденичных идентификаторов
или форм {\cf\bfseries set!}, где вторая подформа {\cf\bfseries set!} является
ключевым словом (см. секцию~\ref{identifier-syntax}), библиотечную
секцию~\extref{lib:make-variable-transformer} {{\cf\bfseries make-variable-transformer}}):\vspace{1mm}
\begin{scheme}
\textbf{(}\hyper{keyword} \hyper{datum} \dotsfoo . \hyper{datum}\textbf{)}
\hyper{keyword}
\textbf{(}set! \hyper{keyword} \hyper{datum}\textbf{)}%
\end{scheme}\vspace{1mm}

%The {\cf define-syntax}, {\cf let-syntax} and {\cf letrec-syntax}
%forms, described in sections~\ref{define-syntax} and \ref{let-syntax},
%create bindings for keywords, associate them with macro transformers,
%and control the scope within which they are visible.
Формы {\cf\bfseries define-syntax}, {\cf\bfseries let-syntax} и {\cf\bfseries letrec-syntax},
описанные в секциях~\ref{define-syntax} и \ref{letrec-syntax}, создают привязки для ключевых
слов, связывают их с макротрансформерами, и задают область, внутри которых они видимы.\vspace{1mm}

%The {\cf syntax-rules} and {\cf identifier-syntax} forms, described in
%section~\ref{syntaxrulessection}, create transformers via a pattern
%language.  Moreover, the {\cf syntax-case} form, described in library
%chapter~\extref{lib:syntaxcasechapter}{{\cf syntax-case}},
%allows creating transformers via arbitrary Scheme code.
Формы {\cf\bfseries syntax-rules} и {\cf\bfseries identifier-syntax}, описанные в секции
~\ref{syntaxrulessection}, создают трансформеры посредством языка шаблонов. Кроме того, форма
{\cf\bfseries syntax-case}, описанная в библиотечной главе~\extref{lib:syntaxcasechapter}
{{\cf\bfseries syntax-case}}, позволяет создавать трансформеры посредством произвольного кода
Scheme.\vspace{1mm}

%Keywords occupy the same name space as variables.
%That is, within the same
%scope, an identifier can be bound as a variable or keyword, or neither, but
%not both, and local bindings of either kind may shadow other bindings of
%either kind.
Ключевые слова занимают то же пространство имён, что и переменные. Таким образом,
внутри той же области видимости, идентификатор может привязываться как переменная или
ключевое слово, или ни один, но не оба, и локальные привязки любого вида могут маскировать другие
привязки любого вида.\vspace{1mm}

%Macros defined using {\cf syntax-rules} and {\cf identifier-\hp{}syntax}
%are ``hygienic'' and ``referentially transparent'' and thus preserve
%Scheme's lexical scoping~\cite{Kohlbecker86,
%  hygienic,Bawden88,macrosthatwork,syntacticabstraction}:
%\mainindex{hygienic} \mainindex{referentially transparent}
Макрос, определённый с помощью {\cf\bfseries syntax-rules} и {\cf\bfseries
  identifier-\hp{}syntax}, является ``гигиеничным'' и ``референциально прозрачным'' и, таким
образом, предохраняет лексическую сферу действия
Scheme~\cite{Kohlbecker86,hygienic,Bawden88,macrosthatwork,syntacticabstraction}:\mainindex{hygienic}
\mainindex{referentially transparent}

\begin{itemize}
%\item If a macro transformer inserts a binding for an identifier
%(variable or keyword) not appearing in the macro use, the identifier is in effect renamed
%throughout its scope to avoid conflicts with other identifiers.
\item Если макротрансформер вносит привязку для идентификатора (переменной или ключевого
слова), отсутствующего в макроприменении, идентификатор используется переименованным
по всей своей области видимости для предотвращения конфликтов с другими идентификаторами.

%\item If a macro transformer inserts a free reference to an
%identifier, the reference refers to the binding that was visible
%where the transformer was specified, regardless of any local
%bindings that may surround the use of the macro.
\item Если макротрансформер вносит свободное обращение к идентификатору, обращение относится к
  привязке, которая была видима в месте определения трансформера, независимо от всех
  локальных привязкок, которые могут окружать применение макроса.
\end{itemize}

%Macros defined using the {\cf syntax-case} facility are also
%hygienic unless {\cf datum\coerce{}syntax}
%(see library section~\extref{lib:conversionssection}{Syntax-object and datum conversions}) is
%used.
Макрос, определённый с помощью средства {\cf\bfseries syntax-case},
также является гигиеничным, кроме случаев, когда используется {\cf\bfseries
  datum\coerce{}syntax} (см. библиотечную секцию~\extref{lib:conversionssection}{Syntax-object
  and datum conversions}).

%%% Local Variables:
%%% mode: latex
%%% TeX-master: "r6rs"
%%% End:
