%\vfill\eject
%\chapter{Entry format}
\chapter{Формат записи}
\label{entryformatchapter}

%The chapters that describe bindings in the base library and the standard
%libraries are organized
%into entries.  Each entry describes one language feature or a group of
%related features, where a feature is either a syntactic construct or a
%built-in procedure.  An entry begins with one or more header lines of the form
Главы, описывающие привязки в основной и стандартных библиотеках,
упорядочены записями. Каждая запись описывает одну языковую характеристику или группу
связанных характеристик, где характеристика является или синтаксической конструкцией, или
встроенной процедурой. Запись начинается с одной или более строки заголовка вида

\newpage

\noindent\pproto{\var{template}}{\var{category}}\unpenalty

%The \var{category} defines the kind of binding described by the entry,
%typically either ``\exprtype'' or ``procedure''.
%An entry may specify various restrictions on subforms or arguments.
%For background on this, see section~\ref{argumentcheckingsection}.
\var{category} определяет вид привязки, описываемой записью, обычно или ``\exprtype'', или
``procedure''. В записи могут указываться различные ограничения подформ или аргументов. Вводную
информацию о них см. в секции~\ref{argumentcheckingsection}.\vspace{2mm}

%\section{Syntax entries}
\section{Запись синтаксиса}\vspace{2mm}

%If \var{category} is ``\exprtype'', the entry describes a
%special syntactic construct, and the template gives the syntax of the
%forms of the construct.
%The template is written in a notation similar to a right-hand
%side of the BNF rules in chapter~\ref{readsyntaxchapter}, and describes
%the set of forms equivalent to the forms matching the
%template as syntactic data.  Some ``\exprtype'' entries carry a
%suffix ({\cf expand}), specifying that the syntactic keyword of the
%construct is exported with level
%$1$.  Otherwise, the syntactic keyword is exported with level $0$; see
%section~\ref{phasessection}.
Если \var{category} -- ``\exprtype'', запись описывает специальную синтаксическую
конструкцию, а шаблон демонстрирует синтаксис форм конструкции. Шаблон записывается в
нотации, аналогичной правой часть правил BNF в главе~\ref{readsyntaxchapter}, и описывает
набор форм, эквивалентных формам, соответствующим шаблону в качестве синтаксических
данных. Некоторые записи ``\exprtype'' содержат суффикс ({\bfseries\cf expand}), что указывает
на экспорт синтаксического ключевого слова конструкции с уровнем $1$. В противном случае
синтаксическое ключевое слово экспортируется с уровнем $0$; см. секцию~\ref{phasessection}.\vspace{2mm}

%Components of the form described by a template are designated
%by syntactic variables, which are written using angle brackets, for
%example, \hyper{expression}, \hyper{variable}.  Case is insignificant
%in syntactic variables.  Syntactic variables
%stand for other forms, or
%sequences of them.  A syntactic variable may refer to a non-terminal
%in the grammar for syntactic data (see section~\ref{datumsyntax}),
%in which case only forms matching
%that non-terminal are permissible in that position.
%For example, \hyper{identifier} stands for a form which must be an
%identifier.
%Also,
%\hyper{expression} stands for any form which is a
%syntactically valid expression.  Other non-terminals that are used in
%templates are defined as part of the specification.
Компоненты описываемой шаблоном формы устанавливаются синтаксическими переменными, заключёнными
в угловые скобки, например, \hyper{expression}, \hyper{variable}. В синтаксических переменных
регистр является незначащим. Синтаксические переменные представляют другие формы или их
последовательности. Синтаксическая переменная может обращаться к нетерминальному символу в
грамматике синтаксических данных (см. секцию~\ref{datumsyntax}) только в случае, если формы,
соответствующие нетерминальному символу, допустимы в этой позиции. Например, \hyper{identifier}
представляет форму, которая должна быть идентификатором. Кроме того, \hyper{expression}
представляет любую форму, являющуюся синтаксически верным выражением. Другие используемые
в шаблонах нетерминальные символы определены как часть спецификации.\vspace{2mm}

%The notation
Нотация
\begin{tabbing}
\qquad \hyperi{thing} $\ldots$
\end{tabbing}
%indicates zero or more occurrences of a \hyper{thing}, and
означает ноль или более вхождений \hyper{thing}, а
\begin{tabbing}
\qquad \hyperi{thing} \hyperii{thing} $\ldots$
\end{tabbing}
%indicates one or more occurrences of a \hyper{thing}.
означает одно или более вхождение \hyper{thing}\vspace{2mm}

%It is the programmer's responsibility to ensure that each component of
%a form has the shape specified by a template.  Descriptions of syntax
%may express other restrictions on the components of a form.
%Typically, such a restriction is formulated as a phrase of the form
%``\hyper{x} must be\mainindex{must be} a \ldots''.  Again, these
%specify the programmer's responsibility.  It is the implementation's
%responsibility to check that these restrictions are satisfied, as long
%as the macro transformers involved in expanding the form terminate.
%If the implementation detects that a component does not meet the
%restriction, an exception with condition type {\cf\&syntax} is raised.
Обеспечение каждого компонента формы шейпом, определённым в шаблоне, является компетенцией
программиста. В описании синтаксиса могут формулироваться другие ограничения на компоненты формы.
Обычно такое ограничение формулируется фразой вида ``\hyper{x} должен быть\mainindex{must be}
\ldots''. Опять же, это подразумевает компетенцию программиста. Обязанностью реализации является
проверка удовлетворения этих ограничений при условии завершения работы участвующих в
разворачивании формы макротрансформеров. При обнаружении реализацией невыполнения компонентом
ограничения генерируется исключение с типом состояния {\bfseries\cf\&syntax}.\vspace{2mm}

%\newpage

%\section{Procedure entries}
\section{Запись процедур}

%If \var{category} is ``procedure'', then the entry describes a procedure, and
%the header line gives a template for a call to the procedure.  Parameter
%names in the template are \var{italicized}.  Thus the header line
Если \var{category} -- ``procedure'', запись описывает процедуру, а строка заголовка
демонстрирует шаблон вызова процедуры. Имена параметров в шаблоне выделены \var{курсивом}.
Таким образом, строка заголовка\vspace{1mm}

%\noindent\pproto{(vector-ref \var{vector} \var{k})}{procedure}\unpenalty
\noindent\pproto{\textbf{(vector-ref} \var{vector} \var{k}\textbf{)}}{procedure}\unpenalty

%indicates that the built-in procedure {\tt vector-ref} takes
%two arguments, a vector \var{vector} and an exact non-negative integer
%object \var{k} (see below).  The header lines
указывает, что встроенная процедура {\bfseries\tt vector-ref} принимает два аргумента, вектор
\var{vector} и точный неотрицательный целый объект \var{k} (см. ниже). Строки заголовка

\noindent%
%\pproto{(make-vector \var{k})}{procedure}
%\pproto{(make-vector \var{k} \var{fill})}{procedure}\unpenalty
\pproto{\textbf{(make-vector} \var{k}\textbf{)}}{procedure}
\pproto{\textbf{(make-vector} \var{k} \var{fill}\textbf{)}}{procedure}\unpenalty

%indicate that the {\tt make-vector} procedure takes
%either one or two arguments.  The parameter names are
%case-insensitive: \var{Vector} is the same as \var{vector}.
указывает, что процедура {\bfseries\tt make-vector} принимает один или два
аргумента. Имена параметров нечувствительны к регистру: \var{Vector} тождественен
\var{vector}.

%As with syntax templates, an ellipsis \dotsfoo{} at the end of a header
%line, as in
Как и в случае синтаксических шаблонов, многоточие \dotsfoo{} в конце строки заголовка, как например

\noindent\pproto{\textbf{(=} \vari{z} \varii{z} \variii{z} \dotsfoo\textbf{)}}{procedure}\unpenalty

%indicates that the procedure takes arbitrarily many arguments of the
%same type as specified for the last parameter name.  In this case,
%{\cf =} accepts two or more arguments that must all be complex
%number objects.
указывает, что процедура принимает произвольное количество аргументов того же типа, который
указан для последнего имени параметра. В данном случае {\bfseries\cf =} принимает два или более
аргумента, которые должны быть комплексными числовыми объектами.\vspace{1mm}

\label{typeconventions}
%A procedure that detects an argument that it is not specified to
%handle must raise an exception with condition type
%{\cf\&assertion}.  Also, the argument specifications are exhaustive: if the
%number of arguments provided in a procedure call does not match
%any number of arguments accepted by the procedure, an exception with
%condition type {\cf\&assertion} must be raised.
Процедура, обнаружившая не определённый для обработки аргумент, должна возбудить
исключение с типом состояния {\bfseries\cf\&assertion}. Кроме того, спецификации аргументов
являются полными: если количество аргументов, предоставляемых в вызове процедуры, не
соответствует любому принимаемому процедурой количеству аргументов, должно быть возбуждено
исключение с типом состояния {\bfseries\cf\&assertion}.\vspace{1mm}

%For succinctness, the report follows the convention
%that if a parameter name is also the name of a type, then the corresponding argument must be of the named type.
%For example, the header line for {\tt vector-ref} given above dictates that the
%first argument to {\tt vector-ref} must be a vector.  The following naming
%conventions imply type restrictions:
Для краткости в работе выполняется соглашение, что если имя параметра является также и
именем типа, соответствующий аргумент должен иметь именованный тип. Например, строка
заголовка для {\bfseries\tt vector-ref}, приведённая выше, предписывает, что первый аргумент
{\bfseries\tt vector-ref} должен быть вектором. Следующие соглашения именования
подразумевают ограничения типа:\vspace{1mm}

%\texonly\begin{center}\endtexonly
%  \begin{tabular}{ll}
%    \var{obj}&any object\\
%    \var{z}&complex number object\\
%    \var{x}&real number object\\
%    \var{y}&real number object\\
%    \var{q}&rational number object\\
%    \var{n}&integer object\\
%    \var{k}&exact non-negative integer object\\
%    \var{bool}&boolean (\schfalse{} or \schtrue{})\\
%    \var{octet}&exact integer object in $\{0, \ldots, 255\}$\\
%    \var{byte}&exact integer object in $\{-128, \ldots, 127\}$\\
%    \var{char}&character (see section~\ref{charactersection})\\
%    \var{pair}&pair (see section~\ref{listsection})\\
%    \var{vector}&vector (see section~\ref{vectorsection})\\
%    \var{string}&string (see section~\ref{stringsection})\\
%    \var{condition}&condition (see library section~\extref{lib:conditionssection}{Conditions})\\
%    \var{bytevector}&bytevector (see library chapter~\extref{lib:bytevectorschapter}{Bytevectors})\\
%    \var{proc}&procedure (see section~\ref{proceduressection})
%  \end{tabular}
%\texonly\end{center}\endtexonly
\texonly\begin{center}\endtexonly
  \begin{tabular}{ll}
    \var{obj}&любой объект\\
    \var{z}&комплексный числовой объект\\
    \var{x}&действительный числовой объект\\
    \var{y}&действительный числовой объект\\
    \var{q}&рациональный числовой объект\\
    \var{n}&целый объект\\
    \var{k}&точный неотрицательный целый объект\\
    \var{bool}&булево значение ({\bfseries\schfalse{}} или \bfseries\schtrue{})\\
    \var{octet}&точный целый объект диапазона $\{0, \ldots, 255\}$\\
    \var{byte}&точный целый объект диапазона $\{-128, \ldots, 127\}$\\
    \var{char}&символ (см. секцию~\ref{charactersection})\\
    \var{pair}&пара (см. секцию~\ref{listsection})\\
    \var{vector}&вектор (см. секцию~\ref{vectorsection})\\
    \var{string}&строка (см. секцию~\ref{stringsection})\\
    \var{condition}&условие (см. библиотечную секцию~\extref{lib:conditionssection}{Conditions})\\
    \var{bytevector}&байт-вектор (см. библиотецный раздел~\extref{lib:bytevectorschapter}{Bytevectors})\\
    \var{proc}&процедура (см. секцию~\ref{proceduressection})
  \end{tabular}
\texonly\end{center}\endtexonly

%Other type restrictions are expressed through parameter-naming
%conventions that are described in specific chapters.  For example,
%library chapter~\extref{lib:numberchapter}{Arithmetic} uses a number of special
%parameter variables for the various subsets of the numbers.
Другие ограничения типа выражаются посредством соглашений именований параметров, описанных в
конкретных главах. Например, в библиотечной главе~\extref{lib:numberchapter}{Arithmetic}
используется множество специальных переменных параметра для различных подмножеств чисел.\vspace{1.5mm}

%With the listed type restrictions, it is the programmer's responsibility to
%ensure that the corresponding argument is of the specified type.
%It is the implementation's responsibility to check for
%that type.
При перечисленных ограничениях типа обязанностью программиста является гарантия, что
соответствующий аргумент имеет указанный тип. Проверка этого типа является обязанностью
реализации.\vspace{1.6mm}

%A parameter called \var{list} means that it is the
%programmer's responsibility to pass an argument that is a list (see
%section~\ref{listsection}).  It is the implementation's responsibility
%to check that the argument is appropriately structured for the
%operation to perform its function, to the extent that this is possible
%and reasonable.  The implementation must at least check that the
%argument is either an empty list or a pair.
Параметр, называемый \var{list}, означает, что обязанностью программиста является передача
аргумента, который является списком (см. секцию~\ref{listsection}). Обязанностью реализации
является проверка, что аргумент должным образом структурирован для операции, чтобы выполнить её
функцию, до степени, что это возможно и разумно. Реализация должна по крайней мере проверить,
что аргумент является или пустым списком или парой.\vspace{1.6mm}

%Descriptions of procedures may express other restrictions on the
%arguments of a procedure.  Typically, such a restriction is formulated
%as a phrase of the form ``\var{x} must be a \ldots'' (or otherwise
%using the word ``must'').
Описания процедур могут выражать другие ограничения аргументов процедур. Обычно такое
ограничение формулируется фразой вида ``\var{x} должна быть \ldots'' (или иначе
с помощью слова ``должна'').\vspace{1.6mm}

%\section{Implementation responsibilities}
\section{Обязанности реализации}\vspace{1.6mm}

%In addition to the restrictions implied by naming conventions, an
%entry may list additional explicit restrictions.
%These explicit restrictions usually describe both the
%programmer's responsibilities, who must ensure that the subforms of a
%form are appropriate, or that an appropriate
%argument is passed, and the implementation's responsibilities, which
%must check that subform adheres to the specified restrictions (if
%macro expansion terminates), or if the argument is appropriate.  A description
%may explicitly list the implementation's responsibilities for some
%arguments or subforms in a paragraph labeled ``\textit{Implementation
%  responsibilities}''.  In this case, the responsibilities specified
%for these subforms or arguments in the rest of the description are only for the
%programmer.  A paragraph describing implementation responsibility does not
%affect the implementation's responsibilities for checking subforms or arguments not
%mentioned in the paragraph.
Кроме ограничений, подразумеваемых соглашениями именования, в записи могут перечисляться
дополнительные явные ограничения. Эти явные ограничения обычно описывают и обязанности
программиста, который должен гарантировать, что подформы формы являются соответствующими, или
что передаётся соответствующий аргумент, и обязанности реализаций, которые должны проверить, что
подформа соблюдает указанные ограничения (если разворачивание макросов завершено), или если
аргумент является соответствующим. В описании могут явно перечисляться обязанности реализации
для некоторых аргументов или подформ в параграфе, помеченном ``\textit {Обязанности
  реализации}''. В этом случае обязанности, указанные для этих подформ или аргументов в
остальной части описания, касаются только программиста. Параграф, описывающий обязанность
реализации, не затрагивает обязанности реализации по проверке подформ или аргументов, не
упомянутых в параграфе.\vspace{1.6mm}

%\section{Other kinds of entries}
\section{Другие виды записей}\vspace{1.6mm}

%If \var{category} is something other than ``syntax'' and
%``procedure'', then the entry describes a non-procedural value, and
%the \var{category} describes the type of that value.  The header line
Если \var{category} -- нечто иное, чем ``syntax'' и ``procedure'', запись описывает
непроцедурное значение, и \var{category} описывает тип этого значения. Строка
заголовка\vspace{1.6mm}

\noindent\rvproto{\bfseries\&who}{condition type}\\
%indicates that {\cf\&who} is a condition type.  The header line
указывает, что {\bfseries\cf\&who} является типом условий. Строка заголовка

\noindent\rvproto{\bfseries unquote}{auxiliary syntax}\\
%indicates that {\cf unquote} is a syntax binding that may occur
%only as part of specific surrounding expressions.  Any use as an
%independent syntactic construct or identifier is a syntax violation.
%As with ``\exprtype'' entries, some ``auxiliary syntax'' entries  carry a
%suffix ({\cf expand}), specifying that the syntactic keyword of the
%construct is exported with level $1$.
%\section{Equivalent entries}
указывает, что {\bfseries\cf unquote} является синтаксической привязкой, которая может
фигурировать только как часть конкретных окружающих выражений. Любое применение
в качестве независимой синтаксической конструкции или идентификатора является нарушением
синтаксиса. Как и в случае с записями ``\exprtype'', некоторые записи ``вспомогательного синтаксиса''
содержат суффикс ({\bfseries\cf expand}), что указывает на экспорт синтаксического ключевого слова
конструкции с уровнем $1$.\vspace{1mm}

\section {Эквивалентные записи}\vspace{1mm}

%The description of an entry occasionally states that it is \textit{the
%  same} as another entry.  This means that both entries are
%equivalent.  Specifically, it means that if both entries have the same
%name and are thus exported from different libraries, the entries from
%both libraries can be imported under the same name without conflict.
В описании записи иногда утверждается, что она \textit{аналогична} другой записи. Это
означает, что обе записи эквивалентны. Реально это означает, что, если обе записи имеют
одинаковые имена и при этом экспортируются из различных библиотек, записи
могут импортироваться из обеих библиотек под одинаковыми именами без конфликта.\vspace{1mm}

%\section{Evaluation examples}
\section{Примеры вычислений}\vspace{1mm}

%The symbol ``\evalsto'' used in program examples can be read
%``evaluates to''.  For example,
Символ ``\evalsto'' применяется в примерах программ и может быть прочитан ``вычисляется
как''. Например,\vspace{1mm}

\begin{scheme}
\bfseries(* 5 8)      \ev\bfseries  40%
\end{scheme}\vspace{1mm}

%means that the expression {\tt(* 5 8)} evaluates to the object {\tt 40}.
%Or, more precisely:  the expression given by the sequence of characters
%``{\tt(* 5 8)}'' evaluates, in an environment that imports the relevant library, to an object
%that may be represented externally by the sequence of characters ``{\tt
%40}''.  See section~\ref{datumsyntaxsection} for a discussion of external
%representations of objects.
означает, что выражение {\bfseries\tt (* 5 8)} вычисляется как объект {\bfseries\tt 40}. Или,
точнее: заданное последовательностью символов выражение ``{\bfseries\tt (* 5 8)}'' в окружении,
импортирующем релевантную библиотеку, вычисляется как объект, который может быть представлен
внешне последовательностью символов ``{\bfseries\tt 40}''. См. секцию ~\ref{datumsyntaxsection},
где рассматриваются внешние представления объектов.\vspace{1mm}

%The ``\evalsto'' symbol is also used when the evaluation of an
%expression causes a violation.  For example,
Символ ``\evalsto'' также применяется, если вычисление выражения приводит к нарушению. Например,\vspace{1mm}

\begin{scheme}
\bfseries(integer->char \sharpsign{}xD800) \xev \exception{\bfseries\&assertion}%
\end{scheme}\vspace{1mm}
%
%means that the evaluation of the expression {\cf (integer->char
%  \sharpsign{}xD800)} must raise an exception with condition type
%{\cf\&assertion}.
означает, что вычисление выражения {\bfseries\cf (integer->char \sharpsign{}xD800)}
должно возбудить исключение с типом состояния {\bfseries\cf\&assertion}.\vspace{1mm}

%Moreover, the ``\evalsto'' symbol is also used to explicitly say that
%the value of an expression in unspecified.  For example:
Кроме того, символ ``\evalsto'' также используется для явного указания, что значение выражения
не определено. Например:\vspace{1mm}
%
\begin{scheme}
\bfseries(eqv? "" "")             \ev  \unspecified%
\end{scheme}\vspace{1mm}

%Mostly, examples merely illustrate the behavior specified in the
%entry.  In some cases, however, they disambiguate otherwise ambiguous
%specifications and are thus normative.  Note that, in some cases,
%specifically in the case of inexact number objects, the return value is only
%specified conditionally or approximately.  For example:
В основном примеры просто демонстрируют определённую записью функциональность. Однако в
некоторых случаях они устраняют неоднозначность двусмысленных в ином случае спецификаций и,
таким образом, нормативны. Необходимо отметить, что в некоторых случаях, как правило в случае
неточных числовых объектов, возвращаемое значение указывается только условно или
приблизительно. Например:\vspace{1mm}
%
\begin{scheme}
\bfseries(atan -inf.0)                  \lev \textbf{-1.5707963267948965} ; \textrm{approximately}%
\end{scheme}

%\newpage

%\section{Naming conventions}
\section{Соглашения по именованию}

%By convention, the names of procedures that store values into previously
%allocated locations (see section~\ref{storagemodel}) usually end in
%``\ide{!}''.
В соответствии с соглашением имена процедур, сохраняющих значения в отведённых ранее
областях памяти (см. секцию ~\ref{storagemodel}), обычно завершаются ``\ide{\bfseries !}''.

%By convention, ``\ide{->}'' appears within the names of procedures that
%take an object of one type and return an analogous object of another type.
%For example, {\cf list->vector} takes a list and returns a vector whose
%elements are the same as those of the list.
В соответствии с соглашением ``\ide{\bfseries ->}'' присутствует внутри имён процедур,
принимающих объект одного типа и возвращающих аналогичный объект другого типа. Например,
{\cf\bfseries list->vector} принимает список и возвращает вектор, элементы которого совпадают с
таковыми из списка.

%By convention, the names of predicates---procedures that always return
%a boolean value---end in ``\ide{?}'' when the name contains any
%letters; otherwise, the predicate's name does not end with a question
%mark.
В соответствии с соглашением имена предикатов --- процедур, всегда возвращающих
булево значение --- завершаются ``\ide{\bfseries ?}'', когда имя содержит произвольные буквы; в
противном случае имя предиката не завершается вопросительным знаком.

%By convention, the components of compound names are separated by ``\ide{-}''
%In particular, prefixes that are actual words or can be pronounced as
%though they were actual words are followed by a hyphen, except when
%the first character following the hyphen would be something other than
%a letter, in which case the hyphen is omitted.  Short,
%unpronounceable prefixes (``\ide{fx}'' and ``\ide{fl}'') are not
%followed by a hyphen.
В соответствии с соглашением компоненты составных имён разделяются ``\ide{\bfseries -}''. В
частности, после приставок, являющихся фактическими словами или которые могут быть произнесены,
как если бы они были фактическими словами, ставится дефис, кроме тех случаев, когда первый
символ после дефиса не является буквой, в этом случае дефис не ставится. После коротких,
труднопроизносимых приставок (``\ide{\bfseries fx}'' и ``\ide{\bfseries fl}'') дефис не
ставится.

%By convention, the names of condition types start with
В соответствии с соглашением имена типов состояния начинаются с
``{\bfseries\cf\&}''\index{&@\texttt{\&}}.

%%% Local Variables:
%%% mode: latex
%%% TeX-master: "r6rs"
%%% End:
