%\chapter{Top-level programs}
\chapter{Программы верхнего уровня}
\label{programchapter}

%A \defining{top-level program} specifies an entry point for defining and running
%a Scheme program.  A top-level program specifies a set of libraries to import and
%code to run.  Through the imported libraries, whether directly or through the
%transitive closure of importing, a top-level program defines a complete Scheme
%program.
\defining{Программа верхнего уровня} задаёт точку входа к определению и выполнению программы
Scheme. В программе верхнего уровня указывается набор импортируемых библиотек и исполняемый код.
Через импортированные библиотеки непосредственно, или через транзитивное замыкание
импортирования, программа верхнего уровня определяет полную программу Scheme.

%\section{Top-level program syntax}
\section{Синтаксис программ верхнего уровня}
\label{programsyntaxsection}

%A top-level program is a delimited piece of text, typically a file, that
%has the following form:
Программа верхнего уровня является ограниченной частью текста, обычно файлом, имеющим следующую
форму:
%
\begin{scheme}
\hyper{import form} \hyper{top-level body}%
\end{scheme}
%
%An \hyper{import form} has the following form:
\hyper{import form} имеет следующую форму:
%
\begin{scheme}
(\textbf{import} \hyper{import spec} \dotsfoo)%
\end{scheme}
%
%A \hyper{top-level body} has the following form:
\hyper{top-level body} имеет следующую форму:
\begin{scheme}
\hyper{top-level body form} \dotsfoo%
\end{scheme}
%
%A \hyper{top-level body form} is either a \hyper{definition} or an
%\hyper{expression}.
\hyper{top-level body form} является или \hyper{definition} или
\hyper{expression}.

%The \hyper{import form} is identical to the import clause in
%libraries (see section~\ref{librarysyntaxsection}),
%and specifies a set of libraries to import.  A \hyper{top-level
% body} is like a \hyper{library body} (see
%section~\ref{librarybodysection}), except that
%definitions and expressions may occur in any order.  Thus, the syntax
%specified by \hyper{top-level body form} refers to the result of macro
%expansion.
\hyper{import form} идентичен разделу import в библиотеках
(см. секцию~\ref{librarysyntaxsection}) и задаёт набор импортируемых библиотек. \hyper{top-level body}
похож на \hyper{library body} (см. секцию~\ref{librarybodysection}), за исключением того, что
определения и выражения могут находиться в произвольном порядке. Таким образом, синтаксис,
задаваемый \hyper{top-level body form} относится к результату разворачивания макросов.

%When uses of {\cf begin}, {\cf let-syntax}, or {\cf letrec-syntax}
%from the \rsixlibrary{base} library
%occur in a top-level body prior to the first
%expression, they are spliced into the body; see section~\ref{begin}.
%Some or all of the body, including portions wrapped in {\cf begin},
%{\cf let-syntax}, or {\cf letrec-syntax}
%forms, may be specified by a syntactic abstraction
%(see section~\ref{macrosection}).
Если используются {\bfseries\cf begin}, {\bfseries\cf let-syntax} или {\bfseries\cf
  letrec-syntax} из библиотеки \textbf{\rsixlibrary{base}}, находящиеся в теле верхнего уровня
до первого выражения, они соединяются с телом; см. секцию~\ref{begin}. Часть или всё тело,
включая части, обернутые в формы {\bfseries\cf begin}, {\bfseries\cf let-syntax} или
{\bfseries\cf letrec-syntax}, может быть определено синтаксической абстракцией
(см. секцию~\ref{macrosection}).

%\section{Top-level program semantics}
\section{Семантика программы верхнего уровня}\vspace{1mm}

%A top-level program is executed by treating the program similarly to a library, and
%evaluating its definitions and expressions.
%The semantics of a top-level body may be roughly explained by
%a simple translation into a library body:
%Each \hyper{expression} that appears before a
%definition in
%the top-level body is converted into a dummy definition
Программа верхнего уровня выполняется путём обработки программы аналогично библиотеке, и вычисления её
определений и выражений. Семантику тела верхнего уровня можно грубо интерпретировать простым
переводом в тело библиотеки: Каждый \hyper{expression}, находящееся перед определением
в теле верхнего уровня, преобразовывается в фиктивное определение\vspace{2mm}
%
\begin{scheme}
\textbf{(define} \hyper{variable} \textbf{(begin} \hyper{expression} \hyper{unspecified}\textbf{))}%
\end{scheme}\vspace{2mm}
%
%where \hyper{variable} is a fresh identifier and \hyper{unspecified}
%is a side-effect-free expression returning an unspecified value.
%(It is generally impossible to determine which forms are
%definitions and expressions without concurrently expanding the body, so
%the actual translation is somewhat more complicated; see
%chapter~\ref{expansionchapter}.)
где \hyper{variable} -- новый идентификатор, а \hyper{unspecified} -- выражение без
побочного эффекта, возвращающее неопределённое значение. (В ообщем случае невозможно определить, какие формы
являются определениями и выражениями, одновременно не разворачивая тело, таким образом, фактическая
трансляция несколько более сложна; см. главу~\ref{expansionchapter}.)\vspace{1mm}

%On platforms that support it, a top-level program may access its command line
%by calling the {\cf command-line} procedure (see library
%section~\extref{lib:command-line}{Command-line access and exit values}).
На платформах, которые поддерживают это, программа верхнего уровня может получить доступ к её
командной строке вызовом процедуры {\bfseries\cf command-line} (см. библиотечную
секцию~\extref{lib:command-line}{Command-line access and exit values}).


%%% Local Variables:
%%% mode: latex
%%% TeX-master: "r6rs"
%%% End:
