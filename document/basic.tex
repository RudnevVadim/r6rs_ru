%\vfill\eject
%\chapter{Semantic concepts}
\chapter{Семантические концепции}
\label{basicchapter}

%\section{Programs and libraries}
\section{Программы и библиотеки}

%A Scheme program consists of a \textit{top-level program\index{top-level program}}
%together with a set of \textit{libraries\index{library}}, each
%of which defines a part of the program connected to the others through
%explicitly specified exports and imports.  A library consists of a set
%of export and import specifications and a body, which consists of
%definitions, and expressions.
%A top-level program is similar to a library, but
%has no export specifications.
%Chapters~\ref{librarychapter} and \ref{programchapter}
%describe the syntax and semantics of libraries and top-level programs,
%respectively.
%Chapter~\ref{baselibrarychapter} describes a base
%library that defines many of the constructs traditionally associated with
%Scheme.
%A separate report~\cite{R6RS-libraries}
%describes the various \textit{standard libraries}\index{standard
%  library} provided by a Scheme system.
Программа Scheme состоит из \textit{программы верхнего уровня\index{top-level program}} совместно с
набором \textit{библиотек\index{library}}, каждая из которых определяет часть программы, связанную
с другими частями посредством явно указываемых экспорта и импорта. Библиотека состоит из ряда
спецификаций экспорта и импорта, а также тела, состоящего из определений и выражений. Программа
верхнего уровня похожа на библиотеку, но не имеет спецификаций
экспорта. В главах~\ref{librarychapter} и \ref{programchapter} описаны синтаксис и семантика
библиотек и программ верхнего уровня соответственно. В главе~\ref {baselibrarychapter} описана
основная библиотека, в которой определено большинство конструкций, традиционно ассоциированных со
Scheme. В отдельной работе~\cite{R6RS-libraries} описаны различные \textit{стандартные
библиотеки}, \index{standard library} предоставляемые системой Scheme.

%The division between the base library and the other standard libraries is
%based on use, not on construction.  In particular, some facilities
%that are typically implemented as ``primitives'' by a compiler or the
%run-time system rather than in terms of other standard procedures
% or syntactic forms are not part of the base library, but are defined in
%separate libraries.  Examples include the fixnums and flonums libraries,
%the exceptions and conditions libraries, and the libraries for
%records.
Деление на основную библиотеку и прочие стандартные библиотеки является прикладным, а не
конструктивным. В частности, некоторые средства, обычно реализуемые как ``примитивы''
компилятором или системой во время выполнения, а не в терминах других стандартных процедур или
синтаксических форм, не являются частью основной библиотеки, а определены в отдельных
библиотеках. Примеры включают библиотеки fixnums и flonums, библиотеки исключений и условий, а
также библиотеки для записей.

%\section{Variables, keywords, and regions}
\section{Переменные, ключевые слова и регионы}
\label{specialformsection}
\label{variablesection}

%Within the body of a library or top-level program,
%an identifier\index{identifier} may name a kind of syntax, or it may name
%a location where a value can be stored.  An identifier that names a kind
%of syntax is called a {\em keyword}\mainindex{keyword}, or {\em syntactic keyword}\mainindex{syntactic keyword},
%and is said to be {\em bound} to that kind of syntax (or, in the case of a
%syntactic abstraction, a {\em transformer} that translates the syntax into more
%primitive forms; see section~\ref{macrosection}).  An identifier that names a
%location is called a {\em variable}\mainindex{variable} and is said to be
%{\em bound} to that location.
%At each point within a top-level program or a library, a specific, fixed set
%of identifiers is bound.  The set of these identifiers, the set of \textit{visible
%bindings}\mainindex{binding}, is
%known as the {\em environment} in effect at that point.
В теле библиотеки или программы верхнего уровня идентификатор\index{identifier} может именовать
или вид синтаксиса, или ячейку памяти, где может храниться
значение. Идентификатор, именующий вид синтаксиса, называется {\em ключевым
  словом}\mainindex{keyword}, или {\em синтаксическим ключевым словом}\mainindex{syntactic
  keyword}, и считается {\em привязанным} к этому виду синтаксиса (или, в случае
синтаксической абстракции, {\em преобразователем}, транслирующим синтаксис в более
примитивные формы; см. секцию~\ref{macrosection}). Идентификатор, именующий ячейку памяти,
называется {\em переменной}\mainindex{variable} и считается {\em привязанным} к этой
ячейке памяти. К каждой точке программы верхнего уровня или библиотеке привязан конкретный,
постоянный набор идентификаторов. Набор таких идентификаторов, набор
\textit{видимого связывания}\mainindex{binding}, называется {\em окружением}, действующим в
данной точке.

%Certain forms are used to create syntactic abstractions
%and to bind keywords to transformers for those new syntactic abstractions, while other
%forms create new locations and bind variables to those
%locations.  Collectively, these forms are called {\em binding
%  constructs}.\mainindex{binding construct}
%Some binding constructs take the form of
%\textit{definitions}\index{definition}, while others are
%expressions.
%With the exception of exported library bindings, a binding created
%by a definition is visible only within the body in which the
%definition appears, e.g., the body of a library, top-level program,
%or {\cf lambda} expression.
%Exported library bindings are also visible within the bodies of
%the libraries and top-level programs that import them (see
%chapter~\ref{librarychapter}).
Одни формы используются для создания синтаксических абстракций и связывания ключевых
слов с преобразователями для этих новых синтаксических абстракций, в то время как другие формы
создают новые ячейки памяти и связывают переменные с этими ячейками. Все эти
формы обобщённо называются {\em конструкциями привязки}.\mainindex{binding construct}, Некоторые
конструкции привязки принимают форму \textit{определений}\index{definition}, в то время как
другие являются выражениями. За исключением экспортируемых библиотечных привязок, привязка,
созданная определением, видима только внутри тела, в котором находится определение,
например, в теле библиотеки, в программе верхнего уровня или в выражении {\cf\bfseries
  lambda}. Экспортируемые библиотечные привязки также видимы внутри тел библиотек и программ
верхнего уровня, импортирующих их (см. главу~\ref{librarychapter}).

%Expressions that bind variables include the {\cf lambda},
%{\cf let}, {\cf let*}, {\cf letrec}, {\cf letrec*}, {\cf let-values},
%and {\cf let*-values} forms from the base library (see
%sections~\ref{lambda}, \ref{letrec}).
%Of these, {\cf lambda} is the most fundamental.
%Variable definitions appearing within the body of
%such an expression, or within the bodies of a library or top-level
%program, are treated as a set of
%{\cf letrec*} bindings.
%In addition, for library bodies,
%the variables exported from the library can be referenced by
%importing libraries and top-level programs.
Выражения, связывающие переменные, включают формы {\cf\bfseries lambda}, {\cf\bfseries let},
{\cf\bfseries let*}, {\cf\bfseries letrec}, {\cf\bfseries letrec*}, {\cf\bfseries let-values}, и
{\cf\bfseries let*-values} из основной библиотеки (см. секции~\ref{lambda}, \ref {letrec}). Из
них {\cf\bfseries lambda} является самой фундаментальной. Определения переменных, находящиеся внутри
тела такого выражения, или внутри тела библиотеки или программы верхнего уровня,
интерпретируются как набор привязок {\cf\bfseries letrec*}. Кроме того, для тел библиотеки, к переменным,
экспортируемым из библиотеки, можно обратиться, импортируя программы верхнего уровня и
библиотеки.

%Expressions that bind keywords include the {\cf
%  let-syntax} and {\cf letrec-syntax} forms (see
%section~\ref{bindsyntax}).  A {\cf define} form (see section~\ref{define}) is a
%definition that creates a variable binding (see
%section~\ref{defines}), and a {\cf define-syntax} form is
%a definition that creates a keyword binding (see
%section~\ref{syntaxdefinitionsection}).
Выражения, привязывающие ключевые слова, включают формы {\cf\bfseries let-syntax} и
{\cf\bfseries letrec-syntax} (см. секцию~\ref{bindsyntax}). Форма {\cf\bfseries define}
(см. секцию~\ref{define}) является определением, создающим привязку переменной
(см. секцию~\ref{define}), а форма {\cf\bfseries define-syntax} - определением, создающим
привязку ключевого слова (см. секцию~\ref{syntaxdefinitionsection}).

%\vest Scheme is a statically scoped language with
%block structure.  To each place in a top-level program or library body where an identifier is bound
%there corresponds a \defining{region} of code within which
%the binding is visible.  The region is determined by the particular
%binding construct that establishes the binding; if the binding is
%established by a {\cf lambda} expression, for example, then its region
%is the entire {\cf lambda} expression.  Every mention of an identifier
%refers to the binding of the identifier that establishes the
%innermost of the regions containing the use.  If a use of an
%identifier appears in a place where none of the surrounding expressions
%contains a binding for the identifier, the use may refer to a
%binding established by a definition or import at the top of the
%enclosing library or top-level program
%(see chapter~\ref{librarychapter}).
%If there is no binding for the identifier,
%it is said to be \defining{unbound}.\mainindex{bound}
\vest Scheme -- язык со статическими областями видимости и блочной структурой. Каждой позиции в
программе верхнего уровня или тела библиотеки, где привязан идентификатор, соответствует
\defining{регион} кода, внутри которого видима привязка. Регион определяется конкретной
конструкцией привязки, устанавливающей привязку; если привязка установлена, например, выражением
{\cf\bfseries lambda}, её регионом является всё выражение {\cf\bfseries lambda}. Каждое
упоминание идентификатора обращается к привязке идентификатора, установленной самым внутренним
из регионов, содержащих её применение. Если применение идентификатора находится в позиции, где
ни одно из близлежащих выражений не содержит привязки идентификатора, применение может
обратиться к привязке, установленной определением или импортом из верхнего уровня окружающей
библиотеки, или программой верхнего уровня (см. главу~\ref {librarychapter}). Если для
идентификатора не существует привязки, он называется \defining{несвязанным}.\mainindex{bound}

%\section{Exceptional situations}
\section{Исключительные ситуации}
\label{exceptionalsituationsection}

%\mainindex{exceptional situation}A variety of exceptional situations
%are distinguished in this report, among them violations of syntax,
%violations of a procedure's specification, violations of
%implementation restrictions, and exceptional situations in the
%environment.  When an exceptional situation is detected by the
%implementation, an \textit{exception is raised}\mainindex{raise},
%which means that a special procedure called the \textit{current
%  exception handler} is called.  A program can also raise an
%exception, and override the current exception handler; see
%library section~\extref{lib:exceptionssection}{Exceptions}.
\mainindex{exceptional situation}В данной работе рассматривается разнообразие исключительных
ситуаций, среди которых нарушения синтаксиса, нарушения спецификаций процедур, нарушения
ограничений реализации и исключительные ситуации в окружении. При обнаружении реализацией
исключительной ситуации \textit{возбуждается исключение}\mainindex{raise}, что
означает вызов специальной процедуры, называемой \textit{текущим обработчиком исключений}.
Программа может также возбудить исключение и переопределить текущий обработчик
исключений; см. библиотечную секцию~\extref{lib:exceptionssection}{Exceptions}.

%When an exception is raised, an object is provided that
%describes the nature of the exceptional situation.  The report uses
%the condition system described in library section~\extref{lib:conditionssection}{Conditions} to
%describe exceptional situations, classifying them by condition types.
При возбуждении исключения порождается объект, описывающий характер исключительной ситуации. В
работе используется система состояний, описанная в библиотечной
секции~\extref{lib:conditionssection}{Conditions}, для описания исключительных ситуаций
классификацией их типами состояний.

%Some exceptional situations allow continuing the program if the
%exception handler takes appropriate action.  The corresponding
%exceptions are called \textit{continuable}\index{continuable exception}.
%For most of the exceptional situations described in this report,
%portable programs cannot rely upon the exception being continuable
%at the place where the situation was detected.
%For those exceptions, the exception handler that is invoked by the
%exception should not return.
%In some cases, however, continuing is permissible, and the
%handler may return.  See library section~\extref{lib:exceptionssection}{Exceptions}.
Некоторые исключительные ситуации допускают продолжение программы, если обработчик исключений
предпримет соответствующее действие. Соответствующие исключения называются
\textit{продолжаемыми}\index{continuable exception}. В большинстве исключительных ситуаций,
описанных в данной работе, переносимые программы не могут зависеть от продолжаемых в месте
обнаружения ситуации исключений. Для таких исключений обработчик исключений, вызванный
исключением, не должен возвращаться. В некоторых случаях, однако, продолжение допустимо, и
обработчик может возвращаться. См. библиотечную
секцию~\extref{lib:exceptionssection}{Exceptions}.

%Implementations must raise an exception
%when they are unable to continue correct execution
%of a correct program due to some \defining{implementation restriction}.  For
%example, an implementation that does not support infinities
%must raise an exception with condition type
%{\cf\&implementation-restriction} when it evaluates an expression
%whose result would be an infinity.
Реализации должны возбудить исключение в случае невозможности продолжения корректного выполнения
корректной программы из-за некоторого \defining{ограничения реализации}. Например, реализация,
не поддерживающая бесконечность, должна возбудить исключение с типом состояния
{\cf\bfseries\&implementation-restriction}, при вычислениии выражения, результатом которого
может быть бесконечность.

%Some possible implementation restrictions such as the lack of
%representations for NaNs and infinities (see
%section~\ref{infinitiesnanssection}) are anticipated by this report,
%and implementations typically must raise an exception of the
%appropriate condition type if they encounter such a situation.
Некоторые возможные ограничения реализации, типа нехватки представлений для NaN и
бесконечностей (см. секцию~\ref{infinitiesnanssection}) предугадываются в соответствии с данной
работой, и реализации, как правило, должны возбуждать исключение соответствующего типа
состояния, если они сталкиваются с такой ситуацией.

%This report uses the phrase ``an exception is raised'' synonymously
%with ``an exception must be raised''.
%This report uses the phrase ``an exception with condition type \var{t}''
%to indicate that the object provided with the
%exception is a condition object of the specified type.
%The phrase ``a continuable exception is raised'' indicates an
%exceptional situation that permits the exception handler to return.
В данной работе применение фразы ``исключение возбуждено'' синонимично с "исключение должно быть
возбуждено". В данной работе используется фраза ``исключение с типом состояния \var{t}'', для указания,
что объект, порождаемый с исключением, является объектом состояния указанного типа. Фраза
``продолжаемое исключение возбуждено'', указывает исключительную ситуацию, разрешающую возврат обработчика
исключений.

%\section{Argument checking}
\section{Проверка аргументов}
\label{argumentcheckingsection}

\mainindex{argument checking}
%Many procedures specified in this report or as part of a
%standard library restrict the arguments they accept.
%Typically, a procedure accepts only specific numbers and types of arguments.
%Many syntactic forms similarly restrict the values to which one or
%more of their subforms can evaluate.
%These restrictions imply responsibilities\mainindex{responsibility} for
%both the programmer and the implementation.
%Specifically, the programmer is responsible for ensuring
%that the values indeed adhere to the restrictions described
%in the specification.  The implementation must check
%that the restrictions in the specification are indeed met, to the
%extent that it is reasonable, possible, and necessary to allow the
%specified operation to complete successfully.  The implementation's
%responsibilities are specified in more detail in
%chapter~\ref{entryformatchapter} and throughout the report.
Многие процедуры, определённые в данной работе или в качестве части стандартной библиотеки,
накладывают ограничения на принимаемые ими аргументы. Обычно процедура принимает только
конкретное количество и конкретные типы аргументов. Аналогично, многие синтаксические формы
накладывают ограничения на значения результатов вычислений одной или более своих подформ. Эти
ограничения подразумевают обязанности\mainindex{responsibility} и программиста, и реализации. А
именно, программист обязан гарантировать, что значения действительно удовлетворяют описанным в
спецификации ограничениям. Реализация должна проверить, что ограничения спецификации
действительно выполнены в том смысле, что предоставление возможности успешного завершения
указанной операции является разумным, возможным и необходимым. Обязанности реализации определены
более подробно в главе~\ref{entryformatchapter} и повсюду в данной работе.

%Note that it is not always possible for an implementation to completely check
%the restrictions set forth in a specification.  For example, if an
%operation is specified to accept a procedure with specific properties,
%checking of these properties is undecidable in general.  Similarly,
%some operations accept both lists and procedures that are
%called by these operations.  Since lists can be mutated by the procedures
%through the \rsixlibrary{mutable-pairs} library (see library
%chapter~\extref{lib:pairmutationchapter}{Mutable pairs}), an argument that is a list
%when the operation starts may become a non-list during the execution of the operation.
%Also, the procedure might escape to a different continuation,
%preventing the operation from performing more checks.
%Requiring the operation to check that the argument is a list after
%each call to such a procedure would be impractical.  Furthermore, some
%operations that accept lists only need to traverse these lists
%partially to perform their function; requiring the implementation to
%traverse the remainder of the list to verify that all specified
%restrictions have been met might
%violate reasonable performance assumptions.  For these reasons, the
%programmer's obligations may exceed the checking obligations of the
%implementation.
Необходимо отметить, что у реализации не всегда имеется возможность полной проверки
сформулированных в спецификации ограничений. Например, если операция определена для приёма
процедуры с конкретными свойствами, проверка этих свойств неразрешима в принципе. Аналогично,
некоторые операции принимают как списки, так и процедуры, которые вызывают эти операции. Так как
списки могут быть видоизменены процедурами с помощью библиотеки \textbf{\rsixlibrary{mutable-pairs}}
(см. библиотечную главу~\extref{lib:pairmutationchapter}{Изменяемые пары}), аргумент, являвшийся
списком при запуске операции, может перестать быть списком во время выполнения операции. К тому
же, процедура может перейти к другому продолжению, что не позволит операции произвести больше
проверок. Требовать от операции проверки, что аргумент является списком после каждого вызова
такой процедуры было бы непрактично. Кроме того, некоторым операциям, принимающим только
списки, для выполнения своей функции необходимо частичное прохождение по данным спискам; требование
реализации проходить остаток списка для проверки выполнения всех указанных ограничений
может нарушить разумные условия производительности. По этим причинам обязанности программистов
могут превышать обязанности проверки реализацией.

%When an implementation detects a violation of a restriction for an
%argument, it must raise an exception with condition type
%{\cf\&assertion} in a way consistent with the safety of execution as
%described in section~\ref{safetysection}.
При обнаружении реализацией нарушения ограничения аргумента она должна возбудить исключение
с типом состояния {\cf\&assertion} совместимым с безопасностью выполнения образом, как описано в
секции ~\ref{safetysection}.\vspace{3mm}

%\section{Syntax violations}
\section{Нарушения синтаксиса}\vspace{3mm}

%The subforms of a special form usually need to obey certain syntactic
%restrictions.  As forms may be subject to macro expansion, which may
%not terminate, the question of whether they obey the specified
%restrictions is undecidable in general.
Подформы специальной формы обычно должны подчиняться некоторым синтаксическим
ограничениям. Поскольку формы могут быть представлены макро-разворачиванием, возможно, не
завершённым, вопрос их подчинения указанным ограничениям неразрешим в принципе.

%When macro expansion terminates, however, implementations must detect
%violations of the syntax.  A \defining{syntax violation} is an error
%with respect to the syntax of library bodies, top-level bodies,
%or the ``\exprtype'' entries in the
%specification of the base library or the standard libraries.
%Moreover, attempting to assign to an immutable variable (i.e., the
%variables exported by a library; see
%section~\ref{importsareimmutablesection}) is also
%considered a syntax violation.
При окончании макро-разворачивания, однако, реализации должны обнаруживать нарушения
синтаксиса. \defining{Нарушение синтаксиса} - ошибка относительно синтаксиса библиотечных тел,
тел верхнего уровня, или ``\exprtype'' записей в спецификации базовой библиотеки или
стандартных библиотек. Кроме того, попытка присваивания неизменяемой переменной (то есть,
переменным, экспортируемым библиотекой; см. секцию~\ref{importsareimmutablesection}) также
считается нарушением синтаксиса.

\newpage

%If a top-level or library form in a program is not syntactically
%correct, then the implementation must raise an exception with
%condition type {\cf\&syntax}, and execution of that top-level program
%or library must not be allowed to begin.
Если форма верхнего уровня или библиотеки в программе не является синтаксически корректной,
реализация должна возбудить исключение с типом состояния {\bfseries\cf\&syntax}, и не должна
позволить начать выполнение такой программы или библиотеки верхнего уровня.\vspace{-2mm}

%\section{Safety}
\section{Безопасность}\vspace{-1mm}
\label{safetysection}

%The standard libraries whose exports are described by this document
%are said to be \defining{safe libraries}.  Libraries and top-level
%programs that import only from safe libraries are also said to be safe.
Стандартные библиотеки, экспорт которых описан в данном документе, называются
\defining{безопасными библиотеками}. Библиотеки и программы верхнего уровня,
импортирующие только из безопасных библиотек, также называются безопасными.

%As defined by this document, the Scheme programming language
%is safe in the following sense:
%The execution of a safe top-level program
%cannot go so badly wrong as to crash or to continue to
%execute while behaving in ways that are
%inconsistent with the semantics described in this document,
%unless an exception is raised.
Как определено данным документом, язык программирования Scheme безопасен в
следующем смысле: выполнение безопасной программы верхнего уровня не может происходить
настолько неверно, чтобы привести к аварийному завершению или продолжению выполнения,
функционирование которого противоречит семантике, описанной в данном документе, если исключение
не возбуждено.

%Violations of an implementation restriction must raise an
%exception with condition type {\cf\&implementation-\hp{}restriction},
%as must all
%violations and errors that would otherwise threaten system
%integrity in ways that might result in execution that is
%inconsistent with the semantics described in this document.
Нарушения ограничений реализации должны возбуждать исключение с типом состояния
{\bfseries\cf\&implementation-\hp{}restriction}, как должны все нарушения и ошибки, которые в
противном случае угрожали бы целостности системы таким образом, что это могло бы привести к
выполнению, противоречащему семантике, описанной в данном документе.

%The above safety properties are guaranteed only for top-level programs
%and libraries that are said to be safe.  In particular,
%implementations may provide access to unsafe libraries in ways that
%cannot guarantee safety.
Вышеупомянутые свойства безопасности гарантируются только для программ верхнего уровня и
библиотек, которые называются безопасными. В частности, реализации могут обеспечивать
доступ к опасным библиотекам способами, которые не могут гарантировать безопасность.\vspace{-2mm}

%\section{Boolean values}
\section{Булевы значения}\vspace{-1mm}
\label{booleanvaluessection}

%Although there is a separate boolean type, any Scheme value can be
%used as a boolean value for the purpose of a conditional test.  In a
%conditional test, all values count as true in such a test except for
%\schfalse{}.  This report uses the word ``true'' to refer to any
%Scheme value except \schfalse{}, and the word ``false'' to refer to
%\schfalse{}. \mainindex{true} \mainindex{false}
Хотя существует отдельный булевый тип, любое значение Scheme может использоваться в качестве
булевого для проверки условия. В проверке условия все значения считаются истинными в таком тесте,
за исключением {\bfseries\schfalse{}}. В данной работе используется слово ``true'' для обращения
к любому значению Scheme, кроме {\bfseries\schfalse{}}, и слово ``false'' для обращения к
{\bfseries\schfalse{}}. \mainindex{true}\mainindex{false}\vspace{-2mm}

%\section{Multiple return values}
\section{Несколько возвращаемых значений}\vspace{-1mm}
\label{multiplereturnvaluessection}

%A Scheme expression can evaluate to an arbitrary finite number of
%values.  These values are passed to the expression's continuation.
Выражение Scheme может вычисляться, как произвольное конечное количество значений. Эти значения
передаются продолжению выражения.

%Not all continuations accept any number of values. For example, a continuation that
%accepts the argument to a procedure call is guaranteed to accept
%exactly one value.  The effect of passing some other number of values
%to such a continuation is unspecified.  The {\cf call-with-values}
%procedure
%described in section~\ref{controlsection} makes it possible to create
%continuations that accept specified numbers of return values.
%If the number of
%return values passed to a continuation created by a call to
%{\cf call-with-values} is not accepted by its consumer
%that was passed in that call, then an exception is raised.
%A more complete description of the number of values accepted by
%different continuations and the consequences of passing an unexpected
%number of values is given in the description of the {\cf values}
%procedure in section~\ref{values}.
Не все продолжения принимают несколько значений. Например, продолжение, принимающее аргумент
вызова процедуры, гарантированно примет ровно одно значение. Результат передачи любого другого
количества значений такому продолжению не определён. Процедура {\bfseries\cf call-with-values},
описанная в секции~\ref{controlsection}, позволяет создать продолжения, принимающие указанное
количество возвращаемых значений. Если количество возвращаемых значений, переданных продолжению,
созданному вызовом {\bfseries\cf call-with-values}, не принято его потребителем, переданным в
этом вызове, возбуждается исключение. Более полное описание количества значений, принимаемых
различными продолжениями и последствия передачи неожидаемого количества значений приведено в
описании процедуры {\bfseries\cf values} в секции~\ref{values}.

%A number of forms in the base library have sequences of expressions
%as subforms that are evaluated sequentially, with the return values of
%all but the last expression being discarded.  The continuations
%discarding these values accept any number of values.
Во множестве форм основной библиотеки содержатся последовательности выражений в качестве подформ,
вычисляемых последовательно, со всеми возвращаемыми ими значениями, кроме последнего, не учитывемого
выражения. Продолжения, не учитывающие эти значения, принимают любое количество значений.\vspace{-4mm}

%\section{Unspecified behavior}
\section{Неопределённое поведение}\vspace{-2mm}

%\vest If an expression is said to ``return unspecified values'',
%then the expression must evaluate without raising an exception, but
%the values returned depend on the implementation; this report
%explicitly does not say how many or what values should be returned.
%Programmers should not rely on a specific number of return values or
%the specific values themselves.
%\mainindex{unspecified behavior}\mainindex{unspecified values}
\vest Если сообщается, что выражение ``возвращает неопределённые значения'', выражение должно
вычисляться без возбуждения исключения, но возвращаемые значения зависят от реализации; в данной
работе явно не оговаривается, сколько или какие значения должны возвращаться. Программисты не
должны полагаться на конкретное количество возвращаемых значений или непосредственно конкретных
значений. \mainindex{unspecified behavior}\mainindex{unspecified values}\vspace{-4mm}

%\section{Storage model}
\section{Модель памяти}\vspace{-2mm}
\label{storagemodel}

%Variables and objects such as pairs, vectors, bytevectors, strings,
%hashtables, and records implicitly
%refer to locations\mainindex{location} or sequences of locations.  A string, for
%example, contains as many locations as there are characters in the string.
%(These locations need not correspond to a full machine word.) A new value may be
%stored into one of these locations using the {\tt string-set!} procedure, but
%the string contains the same locations as before.
Переменные и объекты типа пар, векторов, байтовых векторов, строк, хэш-таблиц и записей, неявно
адресуются к областям памяти\mainindex{location} или последовательностям областей памяти. Строка,
например, содержит столько областей памяти, сколько символов в строке. (Эти области памяти не
обязаны соответствовать полному машинному слову.) Новое значение может быть сохранено в одной из
этих областей памяти с помощью процедуры {\bfseries\tt string-set!}, но строка содержит те же
области памяти, как и прежде.

%An object fetched from a location, by a variable reference or by
%a procedure such as {\cf car}, {\cf vector-ref}, or {\cf string-ref}, is
%equivalent in the sense of \ide{eqv?} % and \ide{eq?} ??
%(section~\ref{equivalencesection})
%to the object last stored in the location before the fetch.
Объект, считанный из области памяти обращением к переменной или процедурой типа {\bfseries\cf
  car}, {\bfseries\cf vector-ref} или {\bfseries\cf string-ref}, эквивалентен в смысле
\textbf{\ide{eqv?}} (секция~\ref{equivalencesection}) последнему сохранённому в области памяти перед
считыванием объекту.

%Every location is marked to show whether it is in use.
%No variable or object ever refers to a location that is not in use.
%Whenever this report speaks of storage being allocated for a variable
%or object, what is meant is that an appropriate number of locations are
%chosen from the set of locations that are not in use, and the chosen
%locations are marked to indicate that they are now in use before the variable
%or object is made to refer to them.
Каждая область памяти помечается признаком её использования. Ни переменная, ни объект, никогда
не адресуются к неиспользуемой области памяти. При каждом упоминании в данной работе памяти,
выделяемой для переменной или объекта, имеется в виду, что соответствующее количество областей
памяти выбрано из множества неиспользуемых областей памяти, и выбранные области памяти
помечаются признаком их использования перед обращением к ним переменной или объекта.

%It is desirable for constants\index{constant} (i.e. the values of
%literal expressions) to reside in read-only memory.  To express this,
%it is convenient to imagine that every object that refers to locations
%is associated with a flag telling whether that object is
%mutable\index{mutable} or immutable\index{immutable}.  Literal
%constants, the strings returned by \ide{symbol->string}, records with
%no mutable fields, and other values explicitly designated as immutable
%are immutable objects, while all objects created by the other
%procedures listed in this report are mutable.  An attempt to store a
%new value into a location referred to by an immutable object
%should raise an exception with condition type {\cf\&assertion}.
Хранить константы\index{constant} (то есть значения литеральных выражений) целесообразно в
памяти только для чтения. Чтобы выразить это, удобно представить, что каждый объект,
адресующийся к области памяти, связан с флагом, указывающим, является ли объект
изменяемым\index{mutable} или неизменяемым\index{immutable}. Литеральные константы, строки,
возвращаемые {\bfseries\ide{symbol->string}}, записи без изменяемых полей и другие значения,
явно определённые неизменяемыми являются неизменяемыми объектами, в то время как все объекты,
созданные другими процедурами, перечисленными в данной работе - изменяемыми. Попытка сохранить
новое значение в области памяти, адресуемое неизменяемым объектом, должна возбудить исключение с
типом состояния {\bfseries\cf\&assertion}.

%\section{Proper tail recursion}
\section{Чистая хвостовая рекурсия}
\label{proper tail recursion}

%Implementations of Scheme must be
%{\em properly tail-recursive}\mainindex{proper tail recursion}.
%Procedure calls that occur in certain syntactic
%contexts called \textit{tail contexts}\index{tail context}
%are \textit{tail calls}\mainindex{tail call}.  A Scheme implementation is
%properly tail-recursive if it supports an unbounded number of active
%tail calls.  A call is {\em active} if the called procedure may still
%return.  Note that this includes regular returns as well as returns
%through continuations captured earlier by
%{\cf call-with-current-continuation} that are later invoked.
%In the absence of captured continuations, calls could
%return at most once and the active calls would be those that had not
%yet returned.
%A formal definition of proper tail recursion can be found
%in Clinger's paper~\cite{propertailrecursion}.  The rules for identifying tail calls
%in constructs from the \rsixlibrary{base} library are described in
%section~\ref{basetailcontextsection}.
Реализации Scheme должны быть {\em чистыми хвост-рекурсивными}\mainindex{proper tail
  recursion}. Вызовы процедур, находящихся в определённых синтаксических контекстах, называемых
\textit{хвостовыми контекстами}\index{tail context}, являются \textit{хвостовыми
  вызовами}\mainindex{tail call}. Реализация Scheme обладает свойством чистой хвостовой
рекурсии, если она поддерживает неограниченное количество активных хвостовых вызовов. Вызов
является {\em активным}, если вызываемая процедура может все еще возвращаться. Отметьте, что это
включает регулярные возвращения так же как возвращения через продолжения, захваченные ранее
{\bfseries\cf call-with-current-continuation}, вызываемые позже. В отсутствии захваченных
продолжений, вызовы могли возвратиться максимум один раз, и активными будут вызовы, которые еще
не возвратились. Формальное определение чистой хвостовой рекурсии может быть найдено в газете
Клайнджера~\cite{propertailrecursion}. Правила идентификации хвостовых вызовов в конструкциях из
библиотеки \textbf{\rsixlibrary{base}} описаны в секции ~\ref{basetailcontextsection}.

%\section{Dynamic extent and the dynamic environment}
\section{Динамический экстент и динамическое окружение}
\label{dynamicenvironmentsection}

%For a procedure call, the time between when it is initiated and when
%it returns is called its \defining{dynamic extent}.  In Scheme, {\cf
%  call-with-current-continuation}
%(section~\ref{call-with-current-continuation}) allows reentering a
%dynamic extent after its procedure call has returned.  Thus, the
%dynamic extent of a call may not be a single, connected time period.
Для вызова процедуры, интервал между её началом и возвращением,
называют её \defining {динамическим экстентом}. В Scheme, {\cf\bfseries call-with-current-continuation}
(секция~\ref{call-with-current-continuation}) позволяет повторно войти динамическую экстент после
того, как ее вызов процедуры возвратился. Таким образом, динамический экстент вызова может не быть
единственным, связанным периодом времени.

Some operations described in the report acquire information in
addition to their explicit arguments from the \defining{dynamic
  environment}.  For example, {\cf call-\hp{}with-\hp{}current-\hp{}continuation}
accesses an implicit context established
by {\cf dynamic-wind} (section~\ref{dynamic-wind}), and the {\cf
  raise} procedure (library
section~\extref{lib:exceptionssection}{Exceptions}) accesses the
current exception handler.  The operations that modify the dynamic
environment do so dynamically, for the dynamic extent of a call to a
procedure like {\cf dynamic-wind} or {\cf with-exception-handler}.
When such a call returns, the previous dynamic environment is
restored.  The dynamic environment can be thought of as part of the
dynamic extent of a call.  Consequently, it is captured by {\cf
  call-with-current-continuation}, and restored by invoking the escape
procedure it creates.

%%% Local Variables:
%%% mode: latex
%%% TeX-master: "r6rs"
%%% End:
