%\chapter{Numbers}
\chapter{Числа}
\label{numbertypeschapter}
\mainindex{number}

%This chapter describes Scheme's model for numbers.  It is important to
%distinguish between the mathematical numbers, the Scheme objects that
%attempt to model them, the machine representations used to implement
%the numbers, and notations used to write numbers.  In this report, the
%term \textit{number} refers to a mathematical number, and the term
%\textit{number object} refers to a Scheme object representing a
%number.  This report uses the types \type{complex}, \type{real},
%\type{rational}, and \type{integer} to refer to both mathematical
%numbers and number objects.  The \type{fixnum} and \type{flonum} types
%refer to special subsets of the number objects, as determined by
%common machine representations, as explained below.
В этой главе описывается модель Scheme для чисел. Важно различать математические числа, объекты
Scheme, стремящиеся представить их, машинные представления, используемые для реализации чисел, и
используемые для записи чисел нотации. В данной работе термин \textit{число} относится к
математическому числу, а термин \textit{числовой объект} - к объекту Scheme, представляющему
число. В данной работе используются типы \type{complex}, \type{real}, \type{rational} и
\type{integer} для обращения и к математическим числам, и к числовым объектам. Типы
\type{fixnum} и \type{flonum} относятся к специальным подмножествам числовых объектов, как
определяется общими машинными представлениями и объясняется ниже.

%\section{Numerical tower}
\section{Числовая башня}
\label{numericaltypes}
\index{numerical types}

%Numbers may be arranged into a tower of subsets in which each level
%is a subset of the level above it:
Числа могут быть квалифицированы башней подмножеств, в которой каждый уровень является
подмножеством уровня выше него:
\begin{tabbing}
\ \ \ \ \ \ \ \ \ \=\tupe{number} \\
\> \tupe{complex} \\
\> \tupe{real} \\
\> \tupe{rational} \\
\> \tupe{integer}
\end{tabbing}

%For example, 5 is an integer.  Therefore 5 is also a rational,
%a real, and a complex.  The same is true of the number objects
%that model 5.
Например, 5 является целым числом. Поэтому 5 также является рациональным, действительным и
комплексным. То же верно и для числовых объектов, представляющих 5.


%Number objects are organized as a corresponding tower of subtypes
%defined by the predicates {\cf number?}, {\cf complex?}, {\cf real?},
%{\cf rational?}, and {\cf integer?}; see section~\ref{number?}.
%Integer number objects are also called \textit{integer
%  objects}\mainindex{integer object}.
Числовые объекты организованы как соответствующая башня подтипов, определённых предикатами
{\cf\bfseries number?}, {\cf\bfseries complex?}, {\cf\bfseries real?}, {\cf\bfseries racional?},
и {\cf\bfseries integer?}; см. секцию~\ref{number?}. Числовые объекты целого также
называются \textit{объектами целого}\mainindex{integer objects}.

%There is no simple relationship between the subset that contains a
%number and its representation inside a computer.  For example, the
%integer 5 may have several representations.  Scheme's numerical
%operations treat number objects as abstract data, as independent of
%their representation as possible.  Although an implementation of
%Scheme may use many different representations for numbers, this should
%not be apparent to a casual programmer writing simple programs.
Не существует очевидной взаимосвязи между подмножеством, содержащим число, и его представлением
в компьютере. Например, целое число 5 может иметь несколько представлений. Операции с числами в
Scheme интерпретируют числовые объекты как абстрактные данные, столь же независимые от их
представления, насколько возможно. Хотя реализация Scheme может использовать множество различных
представлений чисел, это не должно быть очевидно случайному программисту, пишущему простые
программы.

%\section{Exactness}
\section{Точность}
\label{exactly}

%\mainindex{exactness}It is useful to distinguish between number objects
%that are known to correspond to a number exactly, and those number
%objects whose computation involved rounding or other errors.  For
%example, index operations into data structures may need to know the index
%exactly, as may some operations on polynomial coefficients in a symbolic algebra
%system.  On the other hand, the results of measurements are inherently
%inexact, and irrational numbers may be approximated by rational and
%therefore inexact approximations.  In order to catch uses of numbers
%known only inexactly where exact numbers are required, Scheme
%explicitly distinguishes \defining{exact} from \defining{inexact} number objects.  This
%distinction is orthogonal to the dimension of type.
\mainindex{exactness}Целесообразно различать числовые объекты, о которых известно, что они точно
соответствуют числу, и числовые объекты, вычисление которых повлекло за собой округление или
другие ошибки. Например, операциям индексации структур данных может потребоваться знание точного
индекса, как и некоторым операциям с коэффициентами полинома в системе символьной алгебры. С
другой стороны, результаты измерений являются неточными по определению, и иррациональные
числа могут быть апроксимированы рациональной, и поэтому неточной, апроксимацией. Для обнаружения
использования чисел, известных только приблизительно, там, где требуются точные числа,
Scheme явно различает \defining{точные} числовые объекты от \defining{неточных}. Это
различие независимо от измерения типа.

%A
%number object is exact if it is the value of an exact numerical
%literal or was derived from exact number objects using only exact
%operations.  Exact number objects correspond to mathematical numbers
%in the obvious way.
Числовой объект является точным, если он является значением точного числового литерала или был
получен из точных числовых объектов с помощью только точных операций. Точные числовые объекты
соответствуют математическим числам явным образом.

%Conversely, a number object is inexact if it is the value of an
%inexact numerical literal, or was derived from inexact number objects,
%or was derived using inexact operations.  Thus inexactness is
%contagious.
В свою очередь, числовой объект является неточным, если он является значением неточного
числового литерала или был получен из неточных числовых объектов числа, или с помощью неточных
операций. Таким образом неточность передаётся непосредственно.

%Exact arithmetic is reliable in the following sense: If exact number
%objects are passed to any of the arithmetic procedures described in
%section~\ref{propagationsection}, and an exact number object is
%returned, then the result is mathematically correct.  This is
%generally not true of computations involving inexact number objects
%because approximate methods such as floating-point arithmetic may be
%used, but it is the duty of each implementation to make the result as
%close as practical to the mathematically ideal result.
Точная арифметика безопасна в следующем смысле: Если точные числовые объекты передаются любой из
арифметических процедур, описанных в секции~\ref{propagationsection}, и возвращается точный
числовой объект, результат математически корректен. Это в общем случае не верно для вычислений с
использованием неточных числовых объектов, так как могут использоваться методы аппроксимации,
типа арифметики с плавающей запятой, но это уже является обязанностью каждой реализации -
выдать результат как можно ближе практически к математически идеальному результату.

\section{Fixnums and flonums}

%A \defining{fixnum} is an exact integer object that lies
%within a certain implementation-dependent subrange of the
%exact integer objects. (Library section \extref{lib:fixnumssection}{Fixnums} describes a
%library for computing with fixnums.)
%Likewise, every implementation must
%designate a subset of its inexact real number objects as \defining{flonum}s, and
%to convert certain external representations into flonums.
%(Library section \extref{lib:flonumssection}{Flonums} describes a library for
%computing with flonums.)  Note that
%this does not imply that an implementation must use
%floating-point representations.
\defining{fixnum} является точным объектом целого, который лежит в пределах конкретного
зависимого от реализации поддиапазона точного объекта целого. (В секции
\extref{lib:fixnumssection}{Fixnums} описана библиотека для вычислений с fixnums.) Аналогично,
каждая реализация должно обозначать подмножество своих неточных вещественных числовых объектов
как \defining {flonum}s, и преобразовывать конкретные внешние представления в flonums. (В секции
\extref{lib:flonumssection}{Flonums} описана библиотека для вычислений с flonums.) Заметьте, что
не подразумевается, что реализация должна использовать представления с плавающей запятой.

%\section{Implementation requirements}
\section{Требования к реализациям}

\index{implementation restriction}\label{restrictions}

%Implementations of Scheme must support number objects for
%the entire tower of subtypes given in section~\ref{numericaltypes}.
%Moreover, implementations must support exact integer
%objects and exact rational number objects of practically unlimited
%size and precision, and to implement certain procedures (listed in
%\ref{propagationsection}) so they always return exact results when
%given exact arguments.  (``Practically unlimited'' means that the size
%and precision of these numbers should only be limited by the size of
%the available memory.)
Реализации Scheme должны поддерживать числовые объекты для всей башни подтипов, приведённых в
секции~\ref{numericaltypes}. Кроме того, реализации должны поддерживать точные объекты целого
точные рациональные числовые объекты практически неограниченного размера и точности,
и реализовывать конкретные процедуры (перечисленные в \ref{propagationsection}) с тем, чтобы
они всегда возвращали точные результаты при предоставлении точных аргументов. (``Практически
неограниченный'' означает, что размер и точность таких чисел должны
ограничиваться только размером доступной памяти.)

\newpage

%Implementations may support only a limited range of inexact number
%objects of any type, subject to the requirements of this section.  For
%example, an implementation may limit the range of the inexact real
%number objects (and therefore the range of inexact integer and
%rational number objects) to the dynamic range of the flonum format.
%Furthermore the gaps between the inexact integer objects and
%rationals are likely to be very large in such an implementation as the
%limits of this range are approached.
Реализации могут поддерживать только ограниченный диапазон неточных числовых объектов любого
типа, подчининяясь требованиям данной секции. Например, реализация может ограничить диапазон
неточных вещественных числовых объектов (и, следовательно, диапазон неточных целых и
рациональных числовых объектов) динамическим диапазоном формата flonum. Кроме того, зазоры
между неточными целыми объектами и rationals, вероятно, будут очень большими в такой
реализации при приближении к границам этого диапазона.

%An implementation may use floating point and other approximate
%representation strategies for \tupe{inexact} numbers.
%This report recommends, but does not require, that the IEEE
%floating-point standards be followed by implementations that use
%floating-point representations, and that implementations using
%other representations should match or exceed the precision achievable
%using these floating-point standards~\cite{IEEE}.
Реализация может использовать плавающую запятую и другие неточные способы представления
\tupe{неточных} чисел. В данной работе рекомендуется, но не требуется, чтобы реализации,
использующие представления с плавающей запятой, следовали стандартам плавающей запятой IEEE, а
реализации, использующие другие представления, должны иметь точность, соответствовующую или
превышающую точность, достигаемую этими стандартами с плавающей запятой~\cite{IEEE}.

%In particular, implementations that use floating-point representations
%must follow these rules: A floating-point result must be represented
%with at least as much precision as is used to express any of the
%inexact arguments to that operation.
%Potentially inexact operations such as {\cf sqrt}, when
%applied to exact arguments, should produce exact answers whenever possible
%(for example the square root of an exact 4 ought to be an exact 2).
%However, this is not required.
%If, on the other hand, an exact number object is operated upon so as to produce an
%inexact result (as by {\cf sqrt}), and if the result is represented in
%floating point, then the most precise floating-point format available
%must be used; but if the result is represented in some other way then
%the representation must have at least as much precision as the most
%precise floating-point format available.
В частности, реализации, использующие представления с плавающей запятой, должны соблюдать
следующие правила: результат с плавающей запятой должен быть представлен с точностью, по крайней
мере не менее используемой для выражения любого неточного аргумента в данной операции. При
применении потенциально неточных операций, типа {\cf\bfseries sqrt} к точным аргументам должны
выдаваться точные ответы всегда, когда возможно (например, квадратный корень точного 4 должен
быть точным 2). Однако это не является требованием. Если, с другой стороны, точным числовым
объектом управляют с целью получения неточного результата (как {\cf\bfseries sqrt}), и если
результат представляется с плавающей запятой, должен использоваться самый точный доступный
формат с плавающей запятой; но если результат представлен неким иным способом, тогда,
представление должно иметь точность, по крайней мере не менее самого точного доступного формата с
плавающей запятой.

%It is the programmer's responsibility to avoid using inexact number
%objects with magnitude or significand too large to be represented in
%the implementation.
Недопущение использования неточных числовых объектов со слишком большим для представления в
реализации значением или значащей частью является заботой программистов.\vspace{-5mm}

%\section{Infinities and NaNs}
\section{Бесконечность и NaN}\vspace{-2mm}

%Some Scheme implementations, specifically those that follow the IEEE
%floating-point standards, distinguish special number objects called
%\mainindex{infinity}\defining{positive infinity}, \defining{negative
%  infinity}, and \defining{NaN}.
Некоторое реализации Scheme, например, придерживающиеся стандартов IEEE с плавающей
точкой, различают специальные числовые объекты, называемые \mainindex
{infinity}\defining{положительная бесконечность}, \defining {отрицательная бесконечность} и
\defining{NaN}.

%Positive infinity is regarded as an inexact real (but not rational) number
%object that represents an indeterminate number greater than the
%numbers represented by all rational number objects.  Negative infinity
%is regarded as an inexact real (but not rational) number object that represents
%an indeterminate number less than the numbers represented by all
%rational numbers.
Положительная бесконечность рассматривается как неточный вещественный (но не рациональный)
числовой объект, представляющий неопределённое число, большее, чем числа, представляемые всеми
рациональными числовыми объектами. Отрицательная бесконечность рассматривается как неточный
вещественный (но не рациональный) числовой объект, представляющий неизвестное число, меньшее,
чем числа, представляемые всеми рациональными числами.

%A NaN is regarded as an inexact real (but not rational) number object so
%indeterminate that it might represent any real number, including
%positive or negative infinity, and might even be greater than positive
%infinity or less than negative infinity.
NaN рассматривается как неточный вещественный (но не рациональный) числовой объект, настолько
неопределённый, что он может представлять любое вещественное число, включая положительную или
отрицательную бесконечность, и даже может быть больше положительной бесконечности или
меньше отрицательной бесконечности.\vspace{-5mm}

%\section{Distinguished -0.0}
\section{Распознавание -0.0}\vspace{-2mm}

%\index{-0.0}
%Some Scheme implementations, specifically those that follow the IEEE
%floating-point standards, distinguish between number objects for $0.0$
%and $-0.0$, i.e., positive and negative inexact zero.  This report
%will sometimes specify the behavior of certain arithmetic operations
%on these number objects.  These specifications are marked with ``if
%$-0.0$ is distinguished'' or ``implementations that distinguish
%$-0.0$''.
\index{-0.0} Некоторые реализации Scheme, например, придерживающиеся стандартов IEEE с
плавающей точкой, различают числовые объекты $0.0$ и $-0.0$, то есть, положительный и
отрицательный неточный ноль. В данной работе в ряде случаев специфицируется функционирование
конкретных арифметических операций с такими числовыми объектами. Такие спецификации записываются
вместе с ``если $-0.0$ распознан'' или ``реализация, различающая $-0.0$''.\vspace{-4mm}

%%% Local Variables:
%%% mode: latex
%%% TeX-master: "r6rs"
%%% End:
