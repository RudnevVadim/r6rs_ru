%\chapter{Expansion process}
\chapter{Процесс разворачивания}
\label{expansionchapter}

%Macro uses (see section~\ref{macrosection}) are expanded into \textit{core
%forms}\mainindex{core form} at the start of evaluation (before compilation or interpretation)
%by a syntax \emph{expander}.
%The set of core forms is implementation-dependent, as is the
%representation of these forms in the expander's output.
%If the expander encounters a syntactic abstraction, it invokes
%the associated transformer to expand the syntactic abstraction, then
%repeats the expansion process for the form returned by the transformer.
%If the expander encounters a core form, it recursively
%processes its subforms that are in expression or definition context,
%if any, and reconstructs the form from the
%expanded subforms.
%Information about identifier bindings is maintained during expansion
%to enforce lexical scoping for variables and keywords.
Макроприменения (см. секцию~\ref{macrosection}) разворачиваются в \textit{основные формы}
\mainindex{core form} в начале вычисления (до компиляции или интерпретации)
синтаксическим \emph{экспандером}. Набор основных форм зависит от реализации, как и
представление этих форм на выходе экспандера. Если экспандер
встречает синтаксическую абстракцию, он запускает ассоциированный трансформер для
разворачивания синтаксической абстракции, затем повторяет процесс разворачивания формы, возвращённой
трансформером. Если экспандер встречает основную форму, он рекурсивно
обрабатывает её подформы, находящиеся в выражении или контексте определения, если таковые
имеются, и восстанавливает форму из развёрнутых подформ. Информация о привязках
идентификаторов сохраняется в течение разворачивания для обеспечения лексической сферы действия
переменных и ключевых слов.

%To handle definitions, the expander processes the initial
%forms in a \hyper{body} (see section~\ref{bodiessection}) or
%\hyper{library body} (see section~\ref{librarybodysection})
%from left to
%right.  How the expander processes each form encountered
%depends upon the kind of form.
Для обработки определений экспандер обрабатывает начальные формы в \hyper{body}
(см. секцию~\ref{bodiessection}) или \hyper{library body} (см. секцию~\ref{librarybodysection})
слева направо. То, как экспандер обрабатывает каждую обнаруженную форму, зависит от
вида формы.

\begin{description}
%\item[macro use]
%The expander invokes the associated transformer to transform the macro
%use, then recursively performs whichever of these actions are appropriate
%for the resulting form.
\item[макроприменение]
  Экспандер активизирует ассоциированный трансформер для преобразования макроприменения, затем
  рекурсивно выполняет одно из этих действий, соответствующее полученной форме.

%\item[{\cf define-syntax} form]
%The expander expands and evaluates the right-hand-side expression and binds the
%keyword to the resulting transformer.
\item[{\cf форма define-syntax}]
  Экспандер разворачивает и вычисляет выражение правой части и связывает ключевое слово
  с полученным трансформером.

%\item[{\cf define} form]
%The expander records the fact that the defined identifier is a variable but defers
%expansion of the right-hand-side expression until after all of the
%definitions have been processed.
\item[{\cf форма define}]
  Экспандер фиксирует факт, что определённый идентификатор является переменной, но
  откладывает разворачивание выражения правой части до обработки всех определений.

%\item[{\cf begin} form]
%The expander splices the subforms into the list of body forms it is
%processing.  (See section~\ref{begin}.)
\item[{\cf форма begin}]
  Экспандер соединяет подформы в обрабатываемый список форм тела. (См. секцию~\ref{begin}.)

%\item[{\cf let-syntax} or {\cf letrec-syntax} form]
%The expander splic\-es the inner body forms into the list of (outer) body forms it is
%processing, arranging for the keywords bound by the {\cf let-syntax}
%and {\cf letrec-syntax} to be visible only in the inner body forms.
\item[{\cf формы let-syntax} or {\cf letrec-syntax}]
  Экспандер соединяет внутренние формы тела в обрабатываемый список (внешних) форм тела,
  обеспечивая видимость ключевых слов, привязанных посредством {\cf\bfseries let-syntax} и
  {\cf\bfseries letrec-syntax}, только во внутренних формах тела.

%\newpage

%\item[expression, i.e., nondefinition]
%The expander completes the expansion of the deferred right-hand-side expressions
%and the current and remaining expressions in the body, and then
%creates the equivalent of a {\cf letrec*} form from the defined variables,
%expanded right-hand-side expressions, and expanded body expressions.
\item[выражение, т.е. не определение]
  Экспандер завершает разворачивание отложеннных выражений правой части, а также текущих и
  оставшихся выражений в теле, и затем создает эквивалент формы {\cf\bfseries letrec*} из
  определённых переменных, развёрнутых выражений правой части и развёрнутых выражений тела.
\end{description}\vspace{1mm}

%For the right-hand side of the definition of a variable, expansion is
%deferred until after all of the definitions have been seen.  Consequently,
%each keyword and variable reference within the right-hand side
%resolves to the local binding, if any.
Разворачивание правой части определения переменной откладывается до обнаружения всех
определений. Следовательно, каждое ключевое слово и обращение к переменной внутри правой части
разрешается локальной привязкой, если таковые имеются.\vspace{1mm}

%A definition in the sequence of forms must not define any identifier whose
%binding is used to determine the meaning of the undeferred portions of the
%definition or any definition that precedes it in the sequence of forms.
%For example, the bodies of the following expressions violate this
%restriction.
Определение в последовательности форм не должно определять любой идентификатор, привязка
которого используется для задания смысла неотложенных частей определения или любого
определения, предшествующего ему в последовательности форм. Например, в телах следующих
выражений это ограничение нарушается.\vspace{1mm}

\begin{scheme}
\bfseries (let ()
\bfseries   (define define 17)
\bfseries   (list define))
\bfseries
\bfseries (let-syntax ([def0 (syntax-rules ()
\bfseries                      [(\_ x) (define x 0)])])
\bfseries   (let ([z 3])
\bfseries     (def0 z)
\bfseries     (define def0 list)
\bfseries     (list z)))

\bfseries (let ()
\bfseries   (define-syntax foo
\bfseries     (lambda (e)
\bfseries       (+ 1 2)))
\bfseries   (define + 2)
\bfseries   (foo))%
\end{scheme}\vspace{2mm}

%The following do not violate the restriction.
Ниже ограничение не нарушается.\vspace{1mm}

\begin{scheme}
\bfseries (let ([x 5])
\bfseries   (define lambda list)
\bfseries   (lambda x x))         \ev\bfseries  (5 5)

\bfseries (let-syntax ([def0 (syntax-rules ()
\bfseries                      [(\_ x) (define x 0)])])
\bfseries   (let ([z 3])
\bfseries     (define def0 list)
\bfseries     (def0 z)
\bfseries     (list z)))          \ev\bfseries  (3)

\bfseries (let ()
\bfseries   (define-syntax foo
\bfseries     (lambda (e)
\bfseries       (let ([+ -]) (+ 1 2))))
\bfseries   (define + 2)
\bfseries   (foo))                \ev\bfseries  -1%
\end{scheme}\vspace{2mm}

%The implementation should treat a violation of the restriction as a
%syntax violation.
Реализация должна трактовать нарушение ограничения как нарушение синтаксиса.

% Andre's proposed implementation:
% To detect this violation, the expander can record each
% identifier whose denotation is determined during expansion
% of the body, together with the denotation.
% Before an identifier is bound, its current denotation is compared
% against denotations already used for the same (in the sense of
% bound-identifier=?) identifier in the scope of the intended binding,
% to determine if its current denotation has already been used
% during the expansion of the body.

%Note that this algorithm does not directly reprocess any form.
%It requires a single left-to-right pass over the definitions followed by a
%single pass (in any order) over the body expressions and deferred
%right-hand sides.
Следует отметить, что данный алгоритм непосредственно не обрабатывает повторно какую-либо
форму. Он требует единственного прохода слева направо по определениям и следующего за ним
отдельного прохода (в любом порядке) по выражениям тела и отложенным правым частям.

%Example:
Примеры:\vspace{2.4mm}

\begin{scheme}
\bfseries (lambda (x)
\bfseries   (define-syntax defun
\bfseries     (syntax-rules ()
\bfseries       [(\_ x a e) (define x (lambda a e))]))
\bfseries   (defun even? (n) (or (= n 0) (odd? (- n 1))))
\bfseries   (define-syntax odd?
\bfseries     (syntax-rules () [(\_ n) (not (even? n))]))
\bfseries   (odd? (if (odd? x) (* x x) x)))%
\end{scheme}\vspace{2.4mm}

%In the example, the definition of {\cf defun} is encountered first, and the keyword
%{\cf defun} is associated with the transformer resulting from
%the expansion and evaluation of the corresponding right-hand side.
%A use of {\cf defun} is encountered next and expands into a
%{\cf define} form.
%Expansion of the right-hand side of this define form is deferred.
%The definition of {\cf odd?} is next and results in the association
%of the keyword {\cf odd?} with the transformer resulting from
%expanding and evaluating the corresponding right-hand side.
%A use of {\cf odd?} appears next and is expanded; the resulting
%call to {\cf not} is recognized as an expression
%because {\cf not} is bound as a variable.
%At this point, the expander completes the expansion of the current
%expression (the call to {\cf not}) and the deferred right-hand side of the
%{\cf even?} definition;
%the uses of {\cf odd?} appearing in these expressions are expanded
%using the transformer associated with the keyword {\cf odd?}.
%The final output is the equivalent of
В начале примера обнаруживается определение {\cf\bfseries defun}, и ключевое слово {\cf\bfseries
  defun} ассоциируется с трансформером, являющимся результатом разворачивания и вычисления
соответствующей правой части. Далее обнаруживается применение {\cf\bfseries defun} и
разворачивается в форму {\cf\bfseries define}. Разворачивание правой части этой формы define
откладывается. Определение {\cf\bfseries odd?} является следующим и приводит к ассоциации
ключевого слова {\cf\bfseries odd?} с трансформером, полученным в результате разворачивания и
вычисления соответствующей правой части. Далее обнаруживается и разворачивается применение
{\cf\bfseries odd?}; полученный в результате вызов {\cf\bfseries not} распознаётся как
выражение, так как {\cf\bfseries not} привязан как переменная. В этой точке экспандер
завершает разворачивание текущего выражения (вызов {\cf\bfseries not}) и отложенной правой части
определения {\cf\bfseries even?}; применения {\cf\bfseries odd?}, фигурирующие в этих
выражениях, разворачиваются с помощью трансформера, связанного с ключевым словом {\cf\bfseries
  odd?}. Окончательный вывод эквивалентен\vspace{2.4mm}

\begin{scheme}
\bfseries (lambda (x)
\bfseries   (letrec* ([even?
\bfseries               (lambda (n)
\bfseries                 (or (= n 0)
\bfseries                     (not (even? (- n 1)))))])
\bfseries     (not (even? (if (not (even? x)) (* x x) x)))))%
\end{scheme}\vspace{2.4mm}

%although the structure of the output is implementation-dependent.
хоть и структура выхода зависит от реализации.\vspace{2.4mm}

%Because definitions and expressions can be interleaved in a
%\hyper{top-level body} (see chapter~\ref{programchapter}),
%the expander's processing of a \hyper{top-level body} is somewhat
%more complicated.
%It behaves as described above for a \hyper{body} or
%\hyper{library body} with the following exceptions:
%When the expander finds a nondefinition,
%it defers its expansion and continues scanning for definitions.
%Once it reaches the end of the set of forms, it processes the
%deferred right-hand-side and body expressions, then
%generates the equivalent of a {\cf letrec*} form from the defined variables,
%expanded right-hand-side expressions, and expanded body expressions.
%For each body expression \hyper{expression} that appears before a variable definition
%in the body, a dummy binding is created at the corresponding place within
%the set of {\cf letrec*} bindings, with a fresh temporary variable on the
%left-hand side and the equivalent of {\cf (begin \hyper{expression}
%  \hyper{unspecified})},
%where \hyper{unspecified} is a side-effect-free expression returning
%an unspecified value,
%on the right-hand side, so that
%left-to-right evaluation order is preserved.
%The {\cf begin} wrapper allows \hyper{expression} to evaluate to an
%arbitrary number of values.
Поскольку определения и выражения в \hyper{top-level body} могут чередоваться
(см. главу~\ref{programchapter}), обработка экспандером \hyper{top-level body} несколько
сложнее. Она происходит так, как описано выше для \hyper{body} или \hyper{library body} со
следующими исключениями: При нахождении экспандером неопределения, он откладывает его
разворачивание и продолжает сканировать определения. Как только он достигает конца набора форм,
он обрабатывает отложенные выражения правой части и тела, затем генерирует эквивалент формы
{\cf\bfseries letrec*} из определённых переменных, развёрнутых выражений правой части и
развёрнутых выражений тела. Для каждого выражения тела \hyper{expression}, находящегося в теле
перед определением переменной, создаётся фиктивная привязка в соответствующем месте внутри
набора привязок {\cf\bfseries letrec*}, с новой временной переменной в левой части и
эквивалентом {\cf \textbf{(begin} \hyper{expression} \hyper{unspecified}\textbf{)}}, где
\hyper{unspecified} -- выражение без побочного эффекта, возвращающее неопределённое значение
правой части, так, чтобы порядок вычисления слева направо сохранялся. Обёртка {\cf\bfseries
  begin} позволяет \hyper{expression} вычисляться к произвольному количеству значений.

%%% Local Variables:
%%% mode: latex
%%% TeX-master: "r6rs"
%%% End:
